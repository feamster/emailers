\documentclass[11pt]{article}

\usepackage{epsf}
\usepackage{epsfig}
\usepackage{url}
\usepackage{6829hw}

\newcommand{\newc}{\newcommand}

\newc{\code}[1]{{\tt #1}}
\newc{\func}[1]{{\em #1\/}}

\newc{\be}{\begin{enumerate}}
\newc{\ee}{\end{enumerate}}

\newc{\bi}{\begin{itemize}}
\newc{\ei}{\end{itemize}}

\newc{\bd}{\begin{description}}
\newc{\ed}{\end{description}}

\newc{\ov}[1]{$\overline{#1}$}
\newc{\instr}{\tt}

\newc{\doublespace}{\renewcommand{\baselinestretch}{1.5}}

\newcommand{\figref}[1]{Figure~\ref{#1}}
\newcommand{\tref}[1]{Table~\ref{#1}}

% Captioned table
\newc{\tbl}[3]{
        \begin{table}[htb]
                \centering
                #1
                \caption{#3}
                \label{#2}
        \end{table}
}

% Input a table.
\newcommand{\dblfig}[3]{
        \begin{figure}[htb]
		\centering
                \input{#1}
                \caption{#3}
                \label{#2}
        \end{figure}
}

\newcommand{\ddblfig}[4]{
        \begin{figure}[htb]
		\hspace{-0.1in}
                \psfig{figure=#1,width=0.45\textwidth}
                \caption{#3}
                \label{#2}
        \end{figure}
}

% Figure with no caption
\newcommand{\nofig}[2]{
        \begin{figure}[htb]
                \centering
                \psfig{figure=#1}
                \label{#2}
        \end{figure}
}

% Whole page figure
\newcommand{\schfig}[3]{
        \begin{figure}[p]
                \centering
                \psfig{figure=#1,height=7in}
                \caption{#3}
                \label{#2}
        \end{figure}
}

% Small figure
\newcommand{\sfig}[3]{
        \begin{figure}[ltb]
                \centering
               \hspace*{\fill}\rule{\linewidth}{.5mm}\hspace*{\fill}\vspace{3mm}
                \psfig{figure=#1,width=0.4\textwidth}
                \caption{#3}
                \label{#2}
               \vspace{3mm}\hspace*{\fill}\rule{\linewidth}{.5mm}\hspace*{\fill}
        \end{figure}
}

% Medium figure
\newcommand{\mfig}[3]{
        \begin{figure}[ltb]
		\centering
               \hspace*{\fill}\rule{\linewidth}{.5mm}\hspace*{\fill}\vspace{1mm}
                \psfig{figure=#1,height=2.5in}
                \caption{#3}
                \label{#2}
               \vspace{0mm}\hspace*{\fill}\rule{\linewidth}{.5mm}\hspace*{\fill}
        \end{figure}
}

\newcommand{\widefig}[4]{
        \begin{figure*}[htb]
                \centering
               \hspace*{\fill}\rule{\linewidth}{.5mm}\hspace*{\fill}\vspace{5mm}
                \psfig{figure=#1,width=#3}
                \caption{#4}
                \label{#2}
               \vspace{5mm}\hspace*{\fill}\rule{\linewidth}{.5mm}\hspace*{\fill}
        \end{figure*}
}

\newcommand{\mcfig}[4]{
        \begin{figure}[htbp]
                \centering
               \hspace*{\fill}\rule{\linewidth}{.5mm}\hspace*{\fill}\vspace{5mm}
                \psfig{figure=#1,width=#3}
                \caption{#4}
                \label{#2}
               \vspace{5mm}\hspace*{\fill}\rule{\linewidth}{.5mm}\hspace*{\fill}
        \end{figure}
}

\newcommand{\docfig}[3]{
        \begin{figure}[htbp]
               \hspace*{\fill}\rule{\linewidth}{.5mm}\hspace*{\fill}\vspace{5mm}
                \centering
                \psfig{figure=#1,width=#3}
                \label{#2}
               \vspace{5mm}\hspace*{\fill}\rule{\linewidth}{.5mm}\hspace*{\fill}
        \end{figure}
}

% Medium-large figure
\newcommand{\mlfig}[3]{
        \begin{figure}[htb]
                \centering
               \hspace*{\fill}\rule{\linewidth}{.5mm}\hspace*{\fill}\vspace{5mm}
                \psfig{figure=#1,height=3.25in}
                \caption{#3}
                \label{#2}
               \vspace{5mm}\hspace*{\fill}\rule{\linewidth}{.5mm}\hspace*{\fill}
        \end{figure}
}

% Large figure
\newcommand{\lfig}[3]{
        \begin{figure}[p]
                \centering
               \hspace*{\fill}\rule{\linewidth}{.5mm}\hspace*{\fill}\vspace{5mm}
                \psfig{figure=#1,height=5in}
                \caption{#3}
                \label{#2}
               \vspace{5mm}\hspace*{\fill}\rule{\linewidth}{.5mm}\hspace*{\fill}
        \end{figure}
}

% 'gg' figures are the double column versions of the 'g' figures above.
\newcommand{\sfigg}[3]{
        \begin{figure*}[htb]
                \centering
               \hspace*{\fill}\rule{\linewidth}{.5mm}\hspace*{\fill}\vspace{5mm}
                \psfig{figure=#1,height=1.5in}
                \caption{#3}
                \label{#2}
               \vspace{5mm}\hspace*{\fill}\rule{\linewidth}{.5mm}\hspace*{\fill}
        \end{figure*}
}

% Medium figure
\newcommand{\mfigg}[3]{
        \begin{figure*}
                \centering
               \hspace*{\fill}\rule{\linewidth}{.5mm}\hspace*{\fill}\vspace{5mm}
                \psfig{figure=#1,width=\linewidth}
                \caption{#3}
                \label{#2}
               \vspace{0mm}\hspace*{\fill}\rule{\linewidth}{.5mm}\hspace*{\fill}
        \end{figure*}
}

% Medium-large figure
\newcommand{\mlfigg}[3]{
        \begin{figure*}[htb]
                \centering
               \hspace*{\fill}\rule{\linewidth}{.5mm}\hspace*{\fill}\vspace{5mm}
                \psfig{figure=#1,height=3.25in}
                \caption{#3}
                \label{#2}
               \vspace{5mm}\hspace*{\fill}\rule{\linewidth}{.5mm}\hspace*{\fill}
        \end{figure*}
}

% Large figure
\newcommand{\lfigg}[3]{
        \begin{figure*}[p]
                \centering
               \hspace*{\fill}\rule{\linewidth}{.5mm}\hspace*{\fill}\vspace{5mm}
                \psfig{figure=#1,height=5in}
                \caption{#3}
                \label{#2}
               \vspace{5mm}\hspace*{\fill}\rule{\linewidth}{.5mm}\hspace*{\fill}
        \end{figure*}
}

% Variable size figure
\newcommand{\vfigg}[4]{
        \begin{figure*}[htb]
                \centering
               \hspace*{\fill}\rule{\linewidth}{.5mm}\hspace*{\fill}\vspace{5mm}
                \psfig{figure=#1,#2}
                \caption{#4}
                \label{#3}
               \vspace{5mm}\hspace*{\fill}\rule{\linewidth}{.5mm}\hspace*{\fill}
        \end{figure*}
}

\newcommand{\vfig}[4]{
        \begin{figure}[ltb]
                \centering
               \hspace*{\fill}\rule{\linewidth}{.5mm}\hspace*{\fill}\vspace{1mm}
                \psfig{figure=#1,#2}
                \caption{#4}
                \label{#3}
               \vspace{1mm}\hspace*{\fill}\rule{\linewidth}{.5mm}\hspace*{\fill}
        \end{figure}
}

\newcommand{\vnlfig}[4]{
        \begin{figure}[htb]
                \centering
               \hspace*{\fill}\rule{\linewidth}{0mm}\hspace*{\fill}\vspace{5mm}
                \psfig{figure=#1,#2}
                \caption{#4}
                \label{#3}
               \vspace{0mm}\hspace*{\fill}\rule{\linewidth}{0mm}\hspace*{\fill}
        \end{figure}
}

\newcommand{\dblvfig}[6]{
        \begin{figure}[htb]
                \centering
                \hspace*{\fill}\rule{\linewidth}{0mm}\hspace*{\fill}\vspace{0.5mm}
                \psfig{figure=#1,#2}
	        \hspace{1in}
                \psfig{figure=#3,#4}
                \caption{#6}
                \label{#5}
               \vspace{2mm}\hspace*{\fill}\rule{\linewidth}{0mm}\hspace*{\fill}
        \end{figure}
}
\newc{\myspacing}{
        \let\oldtextheight=\textheight
        \let\oldtextwidth=\textwidth

        \let\oldtopmargin=\topmargin
        \let\oldheadheight=\headheight
        \let\oldfootheight=\footheight
        \let\oldheadsep=\headsep
        \let\oldoddsidemargin=\oddsidemargin


        \textheight 8.5in
        \textwidth 6in

        \topmargin 0in
        \headheight 0in
        \footheight 1.5in
        \headsep 0in
        \oddsidemargin 0in

}

\newc{\oldspacing}{
        \let\textheight=\oldtextheight 
        \let\textwidth=\oldtextwidth

        \let\topmargin=\oldtopmargin 
        \let\headheight=\oldheadheight 
        \let\footheight=\oldfootheight
        \let\headsep=\oldheadsep
        \let\oddsidemargin=\oldoddsidemargin
}
% Local Variables: 
% mode: latex
% TeX-master: t
% End: 


\begin{document}

\newcounter{listcount}
\newcounter{sublistcount}


\handout{H2}{September 22, 2011}{Instructor: Prof. Nick Feamster}
{College of Computing, Georgia Tech}{Final Project Instructions}

As part of this course, you'll perform a final research project.  There
are two due dates:
\begin{itemize}
\itemsep=-1pt
\item {\bf Interim report:} Due October 25, 2011
\item {\bf Final report:} Due December 8, 2011
\end{itemize}
\noindent 
The details are described below.

\paragraph{Team Formation} As soon as possible, you should form a team
of 3--5 students to work on a final project for the course.  In your
project proposal (described below), {\em clearly describe the role that
  each team member will play} in the project.  While you will be
evaluated as a group, it is important that everyone in the group play an
active role.

\paragraph{Project Suggestions}
Selecting a problem is perhaps one of the most important aspects of the
exercise of conducting a research project, so definitely take some time
to think carefully about it.  Feel free to use Piazza or ask the course
staff for feedback about your project ideas.

You should use the final project in this course as an opportunity to
learn something new about an area of networking that other people do not
yet know anything about.  The point of the project should be to perform
a {\em research project} of the appropriate size; that is, the goal is
not to re-create someone else's work, but rather to devise a project to
learn or discover something new.

The T-Square wiki contains a list of project suggestions, which are {\em
  not} complete specifications, and in some cases are intentionally
vague.  A lot of the fun in research is figuring out precisely what to
work on and what questions to ask.  We strongly encourage you to come up
with your own ideas, including by modifying the suggestions in this
document, and run them by. Please send me email if you have any
questions or comments about any of these suggestions. It's fine to do a
project connected to your research or thesis work, or to do a joint
project with another class, {\em provided there is a string networking
  and systems building/empirical evaluation aspect to it}.  

When choosing to work on a particular problem, you might consider
answers to the following questions:
\begin{itemize}
\itemsep=-1pt
\item Has a solution been posed to this problem? What do you think of it?
\item If a solution has been proposed, has it been simulated and evaluated properly?
\item If the system has been simulated and theoretically modeled, has it
  been implemented and its performance studied under realistic
  conditions? 
\item Is the solution scalable? Is it robust? Secure? How well does it
  hold up in the face of failures? Will it work in heterogeneous
  environments? Will it stand up to coming advances in technology? 
\item If several solutions have been proposed, has anyone performed a
comparative study of them? Why do some schemes work better than others?
Can we characterize the conditions under which some schemes work better
than others. 
\end{itemize}

{\bf Note:} It's your job, not mine, to explain why your project has
something new to offer. This means that you should do a diligent and
convincing job of understanding and presenting previous relevant
research.

\paragraph{Interim Report} Your interim report, which is {\bf due
  October 25, 2011}, should be a writeup containing the following:
\begin{itemize}
\itemsep=-1pt
\item A title and 200-word abstract describing your project.
\item A list of team members (``authors''), and the roles that each team
  member will play.
\item A draft introduction and related work section.
\item Expected results and experiments.
\end{itemize}
\noindent
You won't be given a separate grade for the interim report, but you
should take it seriously.  This is your chance to get feedback from the
course staff on your project, as well as to ask the staff for resources
you may need to complete your project.

\paragraph{Final Presentation and Report} 
({\bf Due December 9, 2011.)} You should turn in a 5--10 page writeup in
a conference-style format (\eg double-column, 9--10 point font)
describing your project.  (Course staff will post templates if you need
them.)  You should treat the writeup as kind of a mini-conference paper
submission---while it's unlikely that whatever you complete for your
project will be ``conference ready'' at the end of the term, this should
give you a good start for moving in that direction, should you choose to
do so.  As with any research paper, the writeup should have {\em all} of
the requisite sections: Abstract, Introduction, Related Work,
Data/System, Evaluation, Conclusion.  Prof. Jim Kurose has some thoughts
and suggestions on how to write a paper:
\url{http://gaia.cs.umass.edu/kurose/talks/top_10_tips_for_writing_a_paper.ppt},
as well as some specific tips on writing introductions:
\url{http://www-net.cs.umass.edu/kurose/writing/intro-style.html}.


During the last week (or possibly during the last two weeks) of the
term, you will also give a final presentation to the class of
approximately 15 minutes, plus time for questions.  Your presentation
should be modeled after a short conference talk: introduce and motivate
the problem, describe your approach, show some of your results, and hint
at what else is in the paper.  {\em You should not attempt to summarize
  everything you did in the talk.}  Rather, try to give people a
highlight of why what problem you've chosen to work on is important, and
why your approach is interesting.

\end{document}
