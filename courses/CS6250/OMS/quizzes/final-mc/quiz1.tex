\documentclass[11pt]{article}
\usepackage{graphics,epsfig,amsmath,amssymb}
\usepackage{epsf}
\usepackage{boxedminipage}
\usepackage{fullpage}
\usepackage{fancyheadings}
\usepackage{times}
\usepackage{amsmath}
\usepackage{ifthen}
%\usepackage{pseudocode}
\usepackage{psfrag}
\pagestyle{fancy}

\setlength{\topmargin}{.2in}
\setlength{\parindent}{0in}
\setlength{\parskip}{.15in}
\setlength{\footskip}{0.1in}

\newcounter{pctr}
\stepcounter{pctr}

\newcounter{partctr}

\newcommand{\ie}{{\em i.e.}}
\newcommand{\eg}{{\em e.g.}}

\newcommand{\ch}{\item {\bf True~~/~~False~~}}
\newcommand{\tfnote}{\probnote{Circle True or False for each choice.}}
\newcommand{\allapply}{\probnote{Circle ALL that apply}}
\newcommand{\bestanswer}{\probnote{Circle the BEST answer}}
\newcommand{\ansbelow}{\probnote{Answer legibly in the space below.}}

\renewcommand{\thesection}{{\bf\Roman{section}}}
\renewcommand{\theenumi}{{\bf\Alph{enumi}.}}
\renewcommand{\labelenumi}{{\bf\Alph{enumi}.}}

\newcommand{\setversion}[1]{\def\version{#1}}
%\setversion{answers}
\setversion{quiz}

\ifthenelse{\equal{\version}{answers}}{
    \newcommand{\sols}[1]{#1}
}{
    \newcommand{\sols}[1]{}
}


\begin{document}

\newcounter{answer}
\newenvironment{answer}[1][\relax]{\refstepcounter{answer}\begin{list}%
 {}{\leftmargin 0pt\rightmargin 0pt\labelsep 3pt\parsep 0pt%
 \setlength{\listparindent}{\parindent}}
    \item {\bf Answer \theanswer #1}\
    }{\hspace*{\fill}$\blacksquare$\end{list}} 



% uses these macros to delimit problems
\newcommand\prob[1]%
  {\begin{itemize}\item[]%
   \vspace{.2in}{\bf\thepctr. ~[#1~ points]:}\stepcounter{pctr}}
\newcommand\eprob{\end{itemize}}
\newcommand\probnote[1]%
  {\\\begin{tabular}{cr} \hspace{3in} & {\bf (#1)} \\ \end{tabular}}

% headers/footers
\lhead[\fancyplain{}{\bf Page \thepage ~of \pageref{lastpage}}]%
      {Online CS 6250 Fall 2014, Final Exam}
\lfoot[{\bf Name: }]%
      {{\bf Name: }}
\rhead[Online CS 6250 Fall 2014, Final Exam]%
      {\fancyplain{}{\bf Page \thepage ~of \pageref{lastpage}}}
\cfoot{}
%\setlength{\headrulewidth}{0in}
\setlength{\headsep}{.3in}

 % Compact itemize and enumerate.  Note that they use the same counters and
% symbols as the usual itemize and enumerate environments.
\def\compactify{\itemsep=0pt \topsep=0pt \partopsep=0pt \parsep=0pt}
\let\latexusecounter=\usecounter
\newenvironment{CompactItemize}
  {\def\usecounter{\compactify\latexusecounter}
   \begin{itemize}}
  {\end{itemize}\let\usecounter=\latexusecounter}
\newenvironment{CompactEnumerate}
  {\def\usecounter{\compactify\latexusecounter}
   \begin{enumerate}}
  {\end{enumerate}\let\usecounter=\latexusecounter}


\cfoot{}
\pagestyle{empty}

\begin{center}
\begin{tabular}{lr}
\resizebox{1in}{!}{\includegraphics{GT}}
&
\parbox{4in}{
    {\Large\it College of Computing} \\ \\
    {\LARGE\sf Georgia Institute of Technology} 
}
%
\end{tabular}
\end{center}

\begin{center}
{\Large{\bf Online CS 6250: Computer Networking: Fall 2014} \\
 \vspace{.15in} \Huge{\bf Final Exam}} 
%\vspace{.2in}

% this is the box on the first page with overall quiz information
\begin{boxedminipage}[h]{6in}
There are \underline{15 questions} and \underline{\pageref{lastpage}
  pages} in this quiz booklet (including this page).  Answer each
question according to the instructions given.  You have {\bf 85
  minutes}.

%\vspace{.1in} The last page is an easy question.  {\em Rip this
%page off of your exam for five bonus points.}  Turn it in anonymously if
%you like.


\vspace{.1in} 
If you find a question ambiguous, write down any
assumptions you make.  {\bf Be neat and legible.}  If I can't
understand your answer, I can't give you credit!  You may want to look
through the whole quiz to identify which questions you can complete most
quickly for the most points.

\vspace{.1in} 
Use the empty sides of this booklet if you need scratch space.  You
may also use them for answers, although you shouldn't need to.  {\em If you
do use the blank sides for answers, make sure to clearly say so!}

\vspace{.1in} 
{\bf Note well: Write your name in the space below AND your initials at the bottom of each
page of this booklet.}

\begin{center}{\bf THIS IS AN ``CLOSED BOOK'' QUIZ.\\
YOU ARE PERMITTED ONE DOUBLE-SIDED SHEET OF PAPER FOR NOTES.\\
{\em ABSOLUTELY NO EMAIL OR MESSAGING OF ANY KIND!} \\
MAKE SURE YOU'VE READ ALL THE INSTRUCTIONS ABOVE!}
\end{center}
{\em Initial here to indicate that (1)~you've read the instructions and (2)~
you agree to abide by the Georgia Tech Honor Code: }

\vspace{.1in} The last page has easy bonus questions, which you can
answer outside of the allotted time.  Rip the last page off of your
quiz for five bonus points.  Turn it in anonymously if you like.

\end{boxedminipage}
\end{center}
\vspace*{0.25in}
\begin{center}
{\it Do not write in the boxes below}
\end{center}

\begin{center}
\begin{tabular}{|l|l|l|l|l|l|l|l|l|} \hline \hline
{\bf 1-5 (xx/20)}& {\bf 6-12 (xx/49)}& {\bf 13-15 (xx/16)} & {\bf Bonus (xx/5)} & {\bf Total
  (xx/85)}  \\ \hline 
 & & & & \\ 
 & & & &\\ \hline \hline
\end{tabular}
\end{center}

\vspace{.2in}
{\bf\Large{Name:}}

\newpage
\pagestyle{fancy}


\prob{4} From the Dave Clark paper, {\em Design Principles of the DARPA
  Internet Protocols}, which was the first and foremost fundamental design goal of
the Internet?
\bestanswer

\setcounter{partctr}{0}
\begin{list}{\bf\Alph{partctr}.}{\usecounter{partctr}}
\item Security of end hosts and traffic.
\item Multiplexed utilization of existing interconnected networks.
\item Cost-effectiveness.
\item East of management
\item None of the above.
\end{list}
\eprob

\sols{
\begin{answer}
The answer is: (B).
\end{answer}
}


\prob{4} Which of the following are characteristics of packet switching?
\allapply
\setcounter{partctr}{0}
\begin{list}{\bf\Alph{partctr}.}{\usecounter{partctr}}
%\begin{enumerate}
\item Variable delay.
\item ``Busy signals''
\item Sharing of network resources among multiple recipients.
\item Dedicated resources between each pair of sender and receiver.
\item None of the above.
\end{list}
\eprob

\sols{
\begin{answer}
The answer is: (A), (C).
\end{answer}
}

\prob{4} Which of the following most accurately describes the {\em most common}
uses for eBGP, iBGP, and IGP?
\bestanswer
\setcounter{partctr}{0}
\begin{list}{\bf\Alph{partctr}.}{\usecounter{partctr}}
%\begin{enumerate}
\item eBGP is used
  within an AS for external destinations, iBGP is used between ASes for
  external destinations, and IGP is used within an AS for internal destinations.
\item eBGP is used between ASes for external destinations, iBGP is used
  within an AS for external destinations, and IGP is used within an AS
  for destinations within an AS.
\item eBGP is used between ASes for external destinations, iBGP is used
  within an AS for internal destinations, and IGP is used within an AS
  for external destinations.
\item None of the above
\end{list}
\eprob

\sols{
\begin{answer}
The answer is (B).
\end{answer}
}

\newpage
\prob{4} Which of the following is true about required router buffer
sizing if TCP senders are {\em not} synchronized?
\allapply
\setcounter{partctr}{0}
\begin{list}{\bf\Alph{partctr}.}{\usecounter{partctr}}
%\begin{enumerate}
\item The amount of buffering to sustain complete utilization is more
  than the bandwidth-delay product.
\item The amount of buffering required to sustain complete utilization
  is less than the bandwidth-delay product.
\item Packets from different TCP flows will experience packet drops at
  different times.
\item The total amount of packets in the bottleneck buffer at any time
  will be a normal random variable whose standard deviation is inversely
  proportional to the square root of the number active flows.  
\item None of the above
\end{list}
\eprob

\sols{
\begin{answer}
The answer is: (A), (C), (D).
\end{answer}
}


\prob{4}  Which of the following are characteristics of interdomain
routing policies that are commonly applied?
\allapply
\setcounter{partctr}{0}
\begin{list}{\bf\Alph{partctr}.}{\usecounter{partctr}}
%\begin{enumerate}
\item Given multiple routes to the same IP prefix, an AS will prefer a
  route through a provider over a route through its customer.
\item Given multiple routes to the same IP prefix, an AS will prefer a
  route through a customer over a route through its peer.
\item An AS will not advertise a route that it learns via a provider
  to a peer.
\item An AS will not advertise a route that it learns via a provider
  to another provider.
\item All of the above
\end{list}
\eprob

\sols{
\begin{answer}
The answer is: (B), (C), (D).
\end{answer}
}

\prob{4} What are some of the possible causes of congestion collapse?
\allapply

\setcounter{partctr}{0}
\begin{list}{\bf\Alph{partctr}.}{\usecounter{partctr}}
\item Faulty router software
\item Spurious retransmissions in flight
\item Packets traveling distances that are too far in between routers
\item Undelivered packets
\item None of the above.
\end{list}
\eprob

\sols{
\begin{answer}
The answer is: (B), (D).
\end{answer}
}


\prob{4} Which of the following are true about additive increase
multiplicative decrease (AIMD) and fairness?
\allapply
\setcounter{partctr}{0}
\begin{list}{\bf\Alph{partctr}.}{\usecounter{partctr}}
%\begin{enumerate}
\item Additive increase improves efficiency.
\item Additive increase improves fairness.
\item Multiplicative decrease improves efficiency.
\item Multiplicative increase improves fairness.
\item All of the above.
\end{list}
\eprob

\sols{
\begin{answer}
The answer is: (A), (D).
\end{answer}
}

\prob{4} Which of the following pathologies can streaming audio and
video tolerate by adding more buffering at the receiver?
\allapply
\setcounter{partctr}{0}
\begin{list}{\bf\Alph{partctr}.}{\usecounter{partctr}}
%\begin{enumerate}
\item Packet loss
\item Delay variation or jitter
\item Low throughput
\item Out of order packets
\item All of the above
\end{list}
\eprob

\sols{
\begin{answer}
The answer is (A), (B), (D).
\end{answer}
}

\newpage
\prob{4} 
Which of the following statistics are possible to gather from
information such as flow sampling (\eg, NetFlow)?
\allapply
\setcounter{partctr}{0}
\begin{list}{\bf\Alph{partctr}.}{\usecounter{partctr}}
%\begin{enumerate}
\item The time in between each packet transmission
\item Packet headers
\item The number of bytes that each flow sends
\item The number of packets that each flow sends
\item None of the above
\end{list}
\eprob

\sols{
\begin{answer}
The answer is: (C), (D).
\end{answer}
}


\prob{4}  Which of the following are true about a leaky bucket traffic shaper?
\bestanswer
\setcounter{partctr}{0}
\begin{list}{\bf\Alph{partctr}.}{\usecounter{partctr}}
%\begin{enumerate}
\item A link that is shaped with a leaky bucket traffic shaper will
  never send traffic on the outgoing link at a rate that is higher than
  the average drain rate of the bucket.
\item On a link that is shaped by a leaky bucket, a sender can never
  send traffic faster than the average drain rate of the bucket; the
  link will simply drop any traffic that is sent at a higher rate.
\item Comcast's PowerBoost is likely implemented with a leaky bucket
  traffic shaper.
\item None of the above
\end{list}
\eprob

\sols{
\begin{answer}
The answer is: (A).
\end{answer}
}

\prob{4} What are some advantages of separating the data and control
planes, as in a software defined network (SDN)?
\allapply

\setcounter{partctr}{0}
\begin{list}{\bf\Alph{partctr}.}{\usecounter{partctr}}
\item Independent evolution of data and control plane.
\item Less likelihood of failure.
\item Network-wide view of the state of forwarding elements.
\item Ability to control a network from a single, centralized software program.
\item None of the above.
\end{list}
\eprob

\sols{
\begin{answer}
\end{answer}
}


\prob{4} What is the meaning of ``parallel composition'', in terms of
Pyretic policies?
\allapply
\setcounter{partctr}{0}
\begin{list}{\bf\Alph{partctr}.}{\usecounter{partctr}}
%\begin{enumerate}
\item Apply each policy to the same copy of the packet concurrently.
\item Apply multiple policies to packets in sequence.
\item Make a copy of the original packet, then apply each policy to an
  independent copy of the packet.
\item Apply exactly one of the parallel policies to a copy of the
  packet, depending on which policy matches.
\item All of the above.
\end{list}
\eprob

\sols{
\begin{answer}
\end{answer}
}

\prob{4} What does the Pyretic policy {\tt match(srcip=A) >> fwd(2)} do?
\allapply
\setcounter{partctr}{0}
\begin{list}{\bf\Alph{partctr}.}{\usecounter{partctr}}
%\begin{enumerate}
\item For all packets that will be forwarded via output port 2, rewrite
the source IP address to A.
\item Forward packets matching source IP address A via output port 2.
\item Rewrite packets matching source IP address A so that the virtual packet
  header for {\tt outport} has the value 2.
\item Drop packets whose source IP address is not A.
\item All of the above
\end{list}
\eprob

\sols{
\begin{answer}
\end{answer}
}

%% \newpage
%% \prob{4} 
%% What are some mechanisms to create virtual hosts and links in a virtual network?
%% \allapply
%% \setcounter{partctr}{0}
%% \begin{list}{\bf\Alph{partctr}.}{\usecounter{partctr}}
%% %\begin{enumerate}
%% \item Virtual links with IP-in-IP tunnels
%% \item Virtual links with Ethernet-in-IP (EGRE) encapsulation
%% \item Virtual hosts with virtual machines
%% \item Virtual hosts with virtual environments
%% \item All of the above
%% \end{list}
%% \eprob

%% \sols{
%% \begin{answer}
%% \end{answer}
%% }


\newpage
\prob{4}  Which of the following are true about BGP routing security?
\allapply
\setcounter{partctr}{0}
\begin{list}{\bf\Alph{partctr}.}{\usecounter{partctr}}
%\begin{enumerate}
\item An AS can defend against route hijack attacks by filtering route
  advertisements for IP prefixes that neighbors do not own.
\item An AS can defend against AS path shortening attacks by filtering
  route advertisements for AS paths that it does not own.
\item In Secure BGP (S-BGP), an AS that advertises a route signs a
  version of the AS path that includes the next AS along the path (i.e.,
  the AS to whom it is advertising the route).
\item Attackers can use short-lived BGP routing announcements to make it
  more difficult to trace certain types of attacks (e.g., spam, DoS).
\item None of the above
\end{list}
\eprob

\sols{
\begin{answer}
\end{answer}
}

\prob{4}  What are some mechanisms that can be used to implement censorship?
\allapply
\setcounter{partctr}{0}
\begin{list}{\bf\Alph{partctr}.}{\usecounter{partctr}}
%\begin{enumerate}
\item Blocking DNS requests.
\item Blocking TCP connections.
\item Redirecting URLs to a block page.
\item Withdrawing BGP routes.
\item None of the above.
\end{list}
\eprob

\sols{
\begin{answer}
\end{answer}
}



\label{lastpage}
\end{document}
