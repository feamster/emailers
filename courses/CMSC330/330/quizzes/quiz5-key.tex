\documentclass[11pt,fleqn]{article}

  %
  % Besides saying that the grammars should generate nonempty expressions,
  % the wording should have said that the shortest valid expressions
  % consist of just an operator.
  %

  %
  % Derek said:
  %
  % For #7 on the quiz I had a few students correctly say the grammar was
  % incorrect because it could generate strings like "&& a", but then mention
  % that this was a problem with associativity.  For these cases I wrote a
  % note explaining that this is not what's meant by associativity, but I
  % didn't take off points since they did correctly provide an example of an
  % invalid string that the grammar could generate.
  %

  \usepackage{330-f12}

  \psset{unit=1mm,levelsep=10,treesep=8,nodesep=1}

  \showinfo

\begin{document}

  \header{\course}{Quiz \#5 grading key}{\Term}

  Reminder-- you don't need to write the correct answers every single time
students make mistakes.  It's much quicker to just go give the quiz answers
once in class (and you can answer questions or solve the problems as needed
in the process).  But there \textbf{must} be some indication of what's
wrong.  In other words, you can't just give a score without at least
indicating which parts of the answer were wrong.  Just circling or marking
anything incorrect with an X, and writing a word or two, is often
sufficient.

  Also-- you need to coordinate via email about grading decisions, to ensure
the consistency of grading between TAs and sections.

  \vspace{-.5mm}

  \begin{enumerate}

    \addtolength{\itemsep}{3mm}

    \item The operators' precedence are all reversed.

    \item Correct!

    \item Incorrect-- can't generate expressions with more than one of any
          type of operator in sequence.

    \item Ambiguous for expressions using any operators other than
          \texttt{||}.

    \item This grammar gives the interpretation that the \texttt{\&\&} and
          \texttt{||} operators,, and the \texttt{<}, \texttt{>}, and
          \texttt{!}  operators, have the same precedence.

    \item Incomplete-- parentheses can only surround an entire expression,
          or the leftmost operand (subexpression) of an expression, so
          expressions like \texttt{a < (b || c)} can't be generated.

    \item Incorrect associativity-- generates sequences of \texttt{->} from
          right to left, not left to right.  Also generates the empty
          string, which the problem says is not allowed.

  \end{enumerate}

  \begin{info}{\textbf{\underline{Grading scale:}}}

    \begin{itemize}

      \addtolength{\itemsep}{2mm}

      \item The first two are \pts[circle]{5} each; the remaining five are
            \pts[circle]{6} each.

      \item No credit for answers of ``correct'' for incorrect grammars.

      \item Also no credit for saying that there are any problems with the
            grammar in \#2.

      \item If an answer just says ``incorrect'' without a reason just give
            \pts{1} for that question.

      \item If an answer says ``incorrect'' but gives the wrong reason
            give \pts{2} for tha question.

      \item If an answer says ``incorrect'' and gives the correct reason
            that a gramar is wrong, but doesn't give an example in cases
            that one should have been given, deduct \pts{-2} for that part.
            (If it seems during grading like \pts{-2} is too much of a
            deduction, talk to me about it.)

      \item Note: one of the characteristics of a question like this is that
            it's easy to make unintentional additional errors, in the
            process of trying to create intentional errors.  If several
            people identify something as incorrect in any grammar that the
            answers above don't say, check carefully whether they may
            actually be right.  (If there are any omissions above, let me
            know.)

    \end{itemize}

  \end{info}

\end{document}
