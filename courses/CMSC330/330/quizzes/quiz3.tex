\documentclass[11pt]{article}

  \usepackage{330-f12}

  \usepackage[none,light,outline,timestamp]{draftcopy}

  % for quizzes and exams, due to the nameblock
  \addtolength{\headsep}{-6mm}
  \addtolength{\textheight}{6mm}

\begin{document}

  \hspace{\fill}%
  \nameblock

  \smallskip

  \header{\course}{Quiz\ \#3}{\Term}

  \vspace{-3mm}

    \noindent
    This quiz is 40 points.  \textbf{Do not start} until you're told you can
  begin.  You must turn in your quiz \textbf{immediately} when the end of
  the quiz is announced.

  \vspace{-2mm}

  \begin{enumerate}

    \addtolength{\itemsep}{20mm}

    \item {[10 pts.]} Give an OCaml type definition that will allow the
          following to be valid if typed in at the top level.  (It doesn't
          matter what name you give your type definition.)

          \medskip

          \begin{centering}

            \begin{BVerbatim}
        let wallet = [Card ("Credit Card", "VISA"); Dollars 20.5;
                      Card ("ID", "Driver's License")];;
            \end{BVerbatim}

          \end{centering}

          \medskip\smallskip

          \answerblank{4}{\linewidth}{9mm}

    \item {[12 pts.]} The function \texttt{mapit} below is a variation of
          the (noncurried) \texttt{map} function discussed in class (we named
          it \texttt{mapit}, rather than \texttt{map}, because it's a bit
          different):

          \smallskip

          \begin{centering}

            \begin{BVerbatim}
        let rec mapit (f, l) = match l with
            [] -> []
          | (a::b::t) -> (f (a, b))::(mapit (f, b::t))
          | (a::_) -> [];;
            \end{BVerbatim}

          \end{centering}

          \medskip

          Give the output that would be produced for the following call to
          the \texttt{mapit} function shown below.  The call uses a list
          whose definition is also shown.  (The \texttt{mapit} function, and
          the code below, are both valid.)

          \medskip

          \begin{centering}

            \begin{BVerbatim}
        let numbers = [1; 1; 2; 3; 5; 8];;
        mapit ((fun (a, b) -> a + b), numbers);;
            \end{BVerbatim}

          \end{centering}

          \bigskip

          \answerblank{5}{\linewidth}{9mm}

        %             \item \begin{Verbatim}
        % let game = ["Scissors"; "Paper"; "Rock"; "Lizard"; "Spock"];;
        % mapit ((fun (str1, str2) -> str1 ^ " beats " ^ str2), game);;
        %                   \end{Verbatim}
        % 
        %                   \medskip
        % 
        %                   \answerblank{3}{\linewidth}{9mm}
        % 
        %           \end{enumerate}
        % 
        %           \vspace{-2.5mm}
        % 
        %     \item Give the type of the \texttt{mapit} function:
        % 
        %           \bigskip\medskip
        % 
        %           \begin{centering}
        % 
        %             \ans[4.5in]
        % 
        %           \end{centering}
        %   
        %   \end{enumerate}
        % 
        %   \vspace{-2.5mm}

          \turnover

    \item {[18 pts.]}

          \vspace{-3mm}

          \begin{enumerate}

            \addtolength{\itemsep}{16mm}

            \item Write a recursive OCaml function \texttt{print\_list} that
                  will print a list of integers.  The output produced should
                  consist of each number in the list, with a single blank
                  space separating them.  A blank space should not precede
                  the first number or follow the last one.  The function
                  should just not do anything if called on an empty list.
                  For example, consider the following sample call to your
                  function:

                  \medskip

                  \begin{centering}

                    \begin{BVerbatim}
        let numbers = [3; 3; 0; 2; 0; 1; 2];;
        print_list numbers;
                    \end{BVerbatim}

                  \end{centering}

                  \medskip

                  The output it should produce for this call is
                  (blank spaces are shown as \verb*@ @):
                  \verb*@3 3 0 2 0 1 2@

                  \smallskip

                  You may use the following OCaml library functions:%

                  \medskip

                  \begin{centering}

                    \begin{BVerbatim}
        print_int: int -> unit
        print_string: string -> unit
                    \end{BVerbatim}

                  \end{centering}

                  \bigskip

                  \answerblank{8}{\linewidth}{9mm}

            \item Give the type of your function \texttt{print\_list}:%
                  \hspace{2mm}\ans[3.5in]

          \end{enumerate}

          \vspace{-2.5mm}

  \end{enumerate}

\end{document}
