\documentclass[11pt]{article}

  \usepackage{330-f12}

  \newcommand{\sms}{\ensuremath{\hspace{.1mm}}}
  \newcommand{\sss}{\hspace{.35mm}}
  \newcommand{\mymid}{\mbox{\Large \ensuremath{\sss\mid\sss}}}
  \psset{unit=1mm}
  \psset{arrowsize=2.25 1,arrowinset=0,nodesep=0,offset=0,doublesep=.65}%

  \newcommand{\qthreeanswer}[1]{%
    \begin{tabular}[t]{@{}p{2in}@{\answerblank{2}{4in}{9.5mm}}}%
      \ensuremath{#1}%
    \end{tabular}%
  }

  \usepackage[none,light,outline,timestamp]{draftcopy}

  % for quizzes and exams, due to the nameblock
  \addtolength{\headsep}{-6mm}
  \addtolength{\textheight}{6mm}

\begin{document}

  \hspace{\fill}%
  \nameblock

  \smallskip

  \header{\course}{Quiz\ \#2}{\Term}

  \vspace{-3mm}

    \noindent
    This quiz is 40 points.  \textbf{Do not start} until you're told you can
  begin.  You must turn in your quiz \textbf{immediately} when the end of
  the quiz is announced.

  \vspace{-2mm}

  \begin{enumerate}

    \addtolength{\itemsep}{20mm}

    \item {[4 pts.]} Suppose a DFA has 900 states, and its alphabet $\Sigma$
          is $\{a, b, c\}$.  How many transitions does it have, or is there
          no way to know without seeing it and counting them?  Explain
          \textbf{briefly}.

    \item {[4 pts.]} Consider the following language L:
           \(
             \{ a, b, ba \}
           \).
          What is $\mathrm{L}^2$ for this language?  (Give all the elements
          of the language, rather than trying to describe it in words.)

          \addtolength{\itemsep}{18mm}

    \item {[32 pts.]} All the parts of this question use the following
          language:

          \vspace{-1mm}

          \begin{centering}

            \(
              \left\{
                \:
                \begin{array}[t]{@{}l}
                w
                \end{array}
                \left| \ \:
                \parbox[c]{3.75in}{
                  $w \sms \in \sms \{ \sms a, \, b \sms \}^*$,
                  and \emph{w} contains an even number of
                        \emph{a}s, and every \emph{a} in \textit{w} is
                        immediately followed by at least one \emph{b}
                }
              \:
              \right.
              \right\}
            \)

          \end{centering}

          \smallskip

          (Recall from CMSC 250 that zero is an even number.)  Below are
          some regular expressions, some DFAs, and some NFAs.  Does each
          regular expression, DFA, or NFA recognize or describe the language
          above?

          \vspace{-2.5mm}

          \begin{itemize}

            \addtolength{\itemsep}{0mm}

            \item If so, just write ``correct''.

            \item If \textbf{not}, then do \textbf{one of the following}:

                  \begin{itemize}

                    \addtolength{\itemsep}{.75mm}

                    \item Give a string that \textbf{is} in the language
                          above but is \textbf{not} described by the regular
                          expression, DFA, or NFA, \textbf{and} say that the
                          string is not recognized as it should be,
                          \textbf{\underline{or}}

                    \item Give a string that is \textbf{not} in the language
                          above but \textbf{is} described by the regular
                          expression, DFA, or NFA, \textbf{and} say that the
                          string is recognized but should not be.

                  \end{itemize}

          \end{itemize}

          \vspace{-2.5mm}

          \begin{enumerate}

            \addtolength{\itemsep}{30mm}

            \item {{[8 pts.]}} Do these regular expressions describe the
                  language above?

                  \bigskip

                  \begin{enumerate}

                    \addtolength{\itemsep}{5mm}

                    \item \qthreeanswer{b^* ( \: ab^* ab^* )^*}

                          \bigskip\bigskip

                          \vspace{\fill}

                          \begin{centering}

                            \hspace*{-20mm}(\emph{\LARGE turn over})%
                            \vspace*{-4mm}

                          \end{centering}

                    \item \qthreeanswer{
                            b^* \lpar ( \: abb^* )^* \: ( \: abb^* )^*
                            \rpar^*
                          }
                    \item \qthreeanswer{b^* ( \: abb^* abb^* )^*}

                    \item \qthreeanswer{(abb^*abb^*)^*b^*}

                  \end{enumerate}

                  \vspace{-14mm}

            \item {{[12 pts.]}} Do these DFAs recognize or describe the
                  language above?

                  \vspace{-2.5mm}

                  \begin{enumerate}

                    \addtolength{\itemsep}{8mm}

                    % \item \begin{pspicture}(0,14)(70,34)%
                    %         \Large%
                    %         \state[start,final](10,15){s0}
                    %         \state(40,15){s1}
                    % 
                    %         \large
                    % 
                    %         \transition[offset=-2,labeloffset=-5]%
                    %                    (s0,\emph{a},s1)
                    %         \transition(s0,\emph{b},s0)
                    % 
                    %         \transition[offset=-2,labeloffset=-5]%
                    %                     (s1,\emph{a},s0)
                    %         \transition(s1,\emph{b},s1)
                    % 
                    %       \end{pspicture}%
                    %       \hspace{\fill}\lans\mans

                    \item \begin{pspicture}(0,14)(70,36)%
                            \Large%

                            \state[start](10,15){s0}
                            \state(40,15){s1}
                            \state[final](70,15){s2}

                            \large

                            \transition(s0,\emph{a},s1)
                            \transition(s0,\emph{b},s0)

                            \transition(s1,\emph{a},s2)
                            \transition(s1,\emph{b},s1)

                            \transition[curved,curvature=.9,angle=30]%
                                        (s2,\emph{a},s0)
                            \transition(s2,\emph{b},s2)

                          \end{pspicture}%
                          \hspace{\fill}\parbox[c]{2.9in}{%
                            \vspace*{0mm}\answerblank{2}{2.9in}{9.5mm}
                          }

                    \item \begin{pspicture}(0,39)(70,55)
                            \Large%

                            \state[start,final](10,40){s0}
                            \state(50,40){s1}
                            \state(10,0){s2}
                            \state(50,0){s3}
                            \state(30,20){s4}

                            \large

                            \transition(s0,\emph{b},s0)
                            \transition[labeloffset=-5](s0,\emph{a},s2)

                            \transition(s1,\emph{a},s4)
                            \transition[labeloffset=-6](s1,\emph{b},s0)

                            \transition(s2,\emph{a},s4)
                            \transition[labeloffset=-5](s2,\emph{b},s3)

                            \transition[labeloffset=-5](s3,\emph{a},s1)
                            \transition[loopdirection=down](s3,\emph{b},s3)

                            \transition[loopdirection=nw,labelposition=.75]%
                                        (s4,\emph{a},s4)
                            \transition[loopdirection=se,labelposition=.75]%
                                        (s4,\emph{b},s4)

                          \end{pspicture}%
                          \hspace{\fill}\parbox[c]{3in}{%
                            \vspace*{30mm}\answerblank{2}{3in}{9.5mm}
                          }

                    % \item \begin{pspicture}(0,39)(70,87)
                    %         \Large%
                    %
                    %         \state[start,final](10,40){s0}
                    %         \state(50,40){s1}
                    %         \state(10,0){s2}
                    %         \state(50,0){s3}
                    %         \state(30,20){s4}
                    % 
                    %         \large
                    % 
                    %         \transition(s0,\emph{b},s0)
                    %         \transition[labeloffset=-5](s0,\emph{a},s2)
                    % 
                    %         \transition(s1,\emph{a},s4)
                    %         \transition[labeloffset=-6](s1,\emph{b},s0)
                    % 
                    %         \transition(s2,\emph{a},s4)
                    %         \transition[labeloffset=-5](s2,\emph{b},s3)
                    % 
                    %         \transition[labeloffset=-5](s3,\emph{a},s1)
                    %         \transition[loopdirection=down](s3,\emph{b},s4)
                    % 
                    %         \transition[loopdirection=w,labelposition=.75]%
                    %                     (s4,\emph{a},s4)
                    %         \transition[loopdirection=n,labelposition=.75]%
                    %                     (s4,\emph{b},s4)
                    % 
                    %       \end{pspicture}%
                    %       \hspace{\fill}\parbox[c]{3in}{%
                    %         \vspace*{30mm}\lans\mans
                    %       }
                    % 
                    %       \vspace*{\fill}

                    % \item \begin{pspicture}(0,14)(70,80)%
                    %         \Large%
                    % 
                    %         \state[start,final](10,15){s0}
                    %         \state(35,15){s1}
                    %         \state(60,15){s2}
                    %         \state(85,15){s3}
                    %         \state(60,40){s4}
                    % 
                    %         \large
                    % 
                    %         \transition(s0,\emph{a},s1)
                    %         \transition(s0,\emph{b},s0)
                    % 
                    %         \transition(s1,\emph{a},s4)
                    %         \transition(s1,\emph{b},s2)
                    % 
                    %         \transition(s2,\emph{a},s3)
                    %         \transition(s2,\emph{b},s2)
                    % 
                    %         \transition[labeloffset=-5](s3,\emph{a},s4)
                    %         \transition[curved,curvature=.9,angle=25]%
                    %                     (s3,\emph{b},s0)
                    % 
                    %         \transition[loopdirection=w](s4,\emph{a},s4)
                    %         \transition[loopdirection=e](s4,\emph{b},s4)
                    % 
                    %       \end{pspicture}%
                    %       \hspace{\fill}\parbox[c]{2.35in}{%
                    %         \vspace*{0mm}\answerblank{2}{2.35in}{9.5mm}
                    %       }

                    \item \begin{pspicture}(0,14)(70,80)%
                            \Large%

                            \state[start](10,15){s0}
                            \state(40,15){s1}
                            \state[final](70,15){s2}
                            \state(40,40){s3}

                            \large

                            \transition(s0,\emph{a},s1)
                            \transition[curved,angle=-40,labellocation=below]%
                                        (s0,\emph{b},s2)

                            \transition(s1,\emph{a},s3)
                            \transition[offset=2](s1,\emph{b},s2)

                            \transition[offset=2](s2,\emph{a},s1)
                            \transition(s2,\emph{b},s2)

                            \transition[loopdirection=w](s3,\emph{a},s3)
                            \transition[loopdirection=e](s3,\emph{b},s3)

                          \end{pspicture}%
                          \hspace{\fill}\answerblank{2}{2.35in}{9.5mm}

                  \end{enumerate}

                  \vspace{-2.5mm}

                  \pagebreak

            \item {{[12 pts.]}} Do these NFAs recognize or describe the
                  language above?

                  \vspace{-2.5mm}

                  \begin{enumerate}

                    \addtolength{\itemsep}{10mm}

                    \item \begin{pspicture}(0,14)(70,38)%
                            \Large%

                            \state[start](10,15){s0}
                            \state(40,15){s1}
                            \state[final](70,15){s2}

                            \large

                            \transition(s0,\emph{a},s1)
                            \transition[curved](s0,\emph{b},s2)

                            \transition(s1,\emph{b},s2)

                            \transition[curved,curvature=.9,angle=30]%
                                        (s2,\largeepsilon,s0)
                            \transition(s2,\emph{b},s2)

                          \end{pspicture}%
                          \hspace{\fill}\parbox[c]{2.9in}{%
                            \vspace*{0mm}\answerblank{2}{2.9in}{9.5mm}
                          }

                    \item \begin{pspicture}(0,14)(70,47)%
                            \Large%

                            \state[start,final](10,15){s0}
                            \state(35,15){s1}
                            \state(60,15){s2}
                            \state(85,15){s3}

                            \large

                            \transition(s0,\emph{a},s1)
                            \transition[curved](s0,\largeepsilon,s3)

                            \transition[offset=2](s1,\emph{b},s2)

                            \transition[offset=2](s2,\largeepsilon,s1)
                            \transition(s2,\emph{a},s3)

                            \transition[curved,curvature=.75](s3,\emph{b},s0)

                          \end{pspicture}%
                          \hspace{\fill}\parbox[c]{2.35in}{%
                            \vspace*{0mm}\answerblank{2}{2.35in}{9.5mm}
                          }

                    \item \begin{pspicture}(0,39)(70,73)%
                            \Large%

                            \state[start,final](10,40){s0}
                            \state(40,40){s1}
                            \state(70,40){s2}
                            \state(40,15){s3}

                            \large

                            \transition(s0,\emph{a},s1)
                            \transition[offset=2](s0,\largeepsilon,s3)

                            \transition(s1,\emph{b},s1)
                            \transition(s1,\emph{b},s2)

                            \transition(s2,\emph{a},s3)

                            \transition[offset=2](s3,\emph{b},s0)

                          \end{pspicture}%
                          \hspace{\fill}\answerblank{2}{2.9in}{9.5mm}

                  \end{enumerate}

                  \vspace{-2.5mm}

          \end{enumerate}

          \vspace{-2.5mm}

  \end{enumerate}

\end{document}
