\documentclass[11pt,fleqn]{article}

  \usepackage{330-f12}

  \showinfo

\begin{document}

  \header{\course}{Quiz \#3 grading key}{\Term}

  Reminder-- you don't need to write the correct answers every single time
students make mistakes.  It's much quicker to just go give the quiz answers
once in class (and you can answer questions or solve the problems as needed
in the process).  But there \textbf{must} be some indication of what's
wrong.  In other words, you can't just give a score without at least
indicating which parts of the answer were wrong.  Just circling or marking
anything incorrect with an X, and writing a word or two, is often
sufficient.

  Also-- you need to coordinate via email about grading decisions, to ensure
the consistency of grading between TAs and sections.

  \vspace{-.5mm}

  \begin{enumerate}

    \addtolength{\itemsep}{20mm}

    \item {[10 pts.]}

          \vspace{-2mm}

          \begin{Verbatim}
        type item =
            Dollars of float
          | Card of string * string;;
          \end{Verbatim}

          \vspace{-3mm}

          \begin{info}{\textbf{\underline{Grading scale:}}}

            \begin{itemize}

              \addtolength{\itemsep}{2mm}

              \item Deduct \pts[circle]{-2} for each missing keyword.

              \item Deduct \pts[circle]{-2} for each occurrence of
                    unnecessary or incorrect syntax like:

                    \vspace{0mm}

                    \begin{itemize}

                      \addtolength{\itemsep}{-.5mm}

                      \item Extra syntax like ``\texttt{match item with}''

                      \item \texttt{Card(string, string)}

                      \item \texttt{Card -> string * string}

                    \end{itemize}

                    \vspace{0mm}

              \item Minor errors like missing \texttt{=}, \texttt{|},
                    \texttt{;;} can be ingored (but mark them in grading so
                    the student knows they're wrong; probably we would
                    deduct for some of these in grading an exam).  (Of
                    course the \texttt{;;} would only be needed if the type
                    was entered at the top level anyway; if the type was in
                    a file that was being compiled it would not be
                    necessary).

              \item The name of the type doesn't matter.

            \end{itemize}

          \end{info}

    \item {[12 pts.]} \texttt{[2; 3; 5; 8; 13]}

          \begin{info}{\textbf{\underline{Grading scale:}}}

            \begin{itemize}

              \addtolength{\itemsep}{2mm}

              \item Deduct \pts[circle]{-2} for each incorrect number.

            \end{itemize}

          \end{info}

          \pagebreak

    \item {[18 pts.]}

          \vspace{-1.5mm}

          \begin{enumerate}

            \addtolength{\itemsep}{6mm}

            \item {{[15 pts.]}}

                  \begin{Verbatim}
        let rec print_list a = match a with
            [] -> print_string ""
          | (a::b::c) -> print_int a; print_string " "; print_list (b::c)
          | (a::_) -> print_int a;;
                  \end{Verbatim}

                  \begin{info}{\textbf{\underline{Grading scale:}}}

                    \begin{itemize}

                      \addtolength{\itemsep}{3mm}

                      \item The first line is \pts[circle]{3}:

                            \smallskip

                            Deduct \pts{-1} for missing \texttt{rec}, and
                            deduct \pts{-1} for missing \texttt{match a
                            with}.

                      \item The base case is \pts[circle]{3}:

                            \smallskip

                            \texttt{[] -> ()} is correct, but \texttt{[] ->
                            []} is not (deduct \pts{-1} for this).

                      \item The case \texttt{(a::b::c)} is \pts[circle]{6}:

                            \vspace{0mm}

                            \begin{itemize}

                              \addtolength{\itemsep}{-1mm}

                              \item Deduct \pts{-2} for missing the print
                                    function, or \texttt{(a::b::c)}

                              \item Don't deduct if the answer is something
                                    like \texttt{(a::b) -> print\_int a;
                                      print\_string " "; print\_list b;;},
                                    which will print extra space at the
                                    end-- instead deduct for this in the
                                    next case.

                            \end{itemize}

                            \vspace{0mm}

                      \item The case \texttt{(a::\_)} is \pts[circle]{3}:

                            \smallskip

                            Ignore minor deviations from the given answer.
                            Any major error is \pts{-1}.  Deduct \pts{-3}
                            only if the line is missing.

                      \item If the ordering of cases is incorrect, deduct
                            \pts{-2}

                      \item A missing ';' can be ignored.

                      \item If you think that any answers should get more or
                            less partial credit than what's above be sure to
                            coordinate (via email) with the other TAs
                            grading the quiz, for consistency of grading (so
                            that everyone can deduct the same credit for the
                            same mistake, for all sections).  If you're not
                            sure whether a mistake should get more or less
                            partial credit, email all the TAs grading the
                            quiz.  You can also ask me if you all can't
                            agree.

                    \end{itemize}

                  \end{info}

            \item {{[3 pts.]}} \texttt{int list -> unit}

                   \begin{info}{\textbf{\underline{Grading scale:}}}

                     \begin{itemize}

                       \addtolength{\itemsep}{2mm}

                       \item Deduct \pts[circle]{-1} for any error (give no
                             credit only if there is no answer).

                     \end{itemize}

                   \end{info}

          \end{enumerate}

          \vspace{-2.5mm}

  \end{enumerate}

\end{document}
