\documentclass[11pt,fleqn]{article}

  \usepackage{330-f12}

  \showinfo

\begin{document}

  \header{\course}{Quiz \#2 grading key}{\Term}

  Reminder-- you don't need to write the correct answers every single time
students make mistakes.  It's much quicker to just go give the quiz answers
once in class (and you can answer questions or solve the problems as needed
in the process).  But there \textbf{must} be some indication of what's
wrong.  In other words, you can't just give a score without at least
indicating which parts of the answer were wrong.  Just circling or marking
anything incorrect with an X, and writing a word or two, is often
sufficient.

  Also-- you need to coordinate via email about grading decisions, to ensure
the consistency of grading between TAs and sections.

  \vspace{-.5mm}

  \begin{enumerate}

    \addtolength{\itemsep}{4mm}

    \item {[4 pts.]} We know it'll have exactly $900 \: * \: 3 \ = \ 2700$
          states, because a DFA always has exactly one outgoing transition
          from every state for every alphabet symbol.
          
          \begin{info}{\textbf{\underline{Grading scale:}}}

            \begin{itemize}

              \addtolength{\itemsep}{2mm}

              \item I'm not seeing any circumstances where partial credit
                    might be appropriate, but if you come across any, be
                    sure to coordinate with the other TAs grading the quiz
                    for consistency of grading.  You can also ask me if you
                    all can't agree.

            \end{itemize}

          \end{info}

    \item {[4 pts.]} \( aa, ab, aba, ba, bb, bba, baa, bab, baba \)
          
          \begin{info}{\textbf{\underline{Grading scale:}}}

            \begin{itemize}

              \addtolength{\itemsep}{2mm}

              \item I would deduct \pts{-1} for a small mistake (just a
                    missing string), or half credit for more serious errors
                    (like thinking that $L^2$ was just the strings that are
                    above of length 2 above), that are wrong but still show
                    some understanding of what concatenation of languages
                    is.

              \item If you think that any answers should get more or less
                    partial credit than what's above be sure to coordinate
                    (via email) with the other TAs grading the quiz, for
                    consistency of grading (so that everyone can deduct the
                    same credit for the same mistake, for all sections).  If
                    you're not sure whether a mistake should get more or
                    less partial credit, email all the TAs grading the quiz.
                    You can also ask me if you all can't agree.

            \end{itemize}

          \end{info}

    \item {[32 pts.]}

          \vspace{-1.5mm}

          \begin{enumerate}

            \addtolength{\itemsep}{4mm}

            \item {{[8 pts.]}}

                  \begin{enumerate}

                    \addtolength{\itemsep}{2mm}

                    \item incorrect; describes some invalid strings that are
                          not in the language like \emph{baa} that have
                          \emph{a}s (at the end of the string) that are not
                          followed by \emph{b}s

                    \item incorrect; describes some invalid strings that are
                          not in the language like \emph{bab} that have an
                          odd number of \emph{a}s

                    \item correct

                    \item incorrect; does not describe some valid strings
                          that are in the language like \emph{babab} that
                          begin with \emph{b}s.

                  \end{enumerate}

            \item {{[12 pts.]}}

                  \begin{enumerate}

                    \addtolength{\itemsep}{2mm}

                    \item incorrect; describes some invalid strings that are
                          not in the language like \emph{aa} or
                          \emph{aaaaa}, in which some \emph{a}s are not
                          followed immediately by a \emph{b}

                    \item correct

                    \item incorrect; describes some invalid strings that are
                          not in the language like \emph{bab} that have an
                          odd number of \emph{a}s.

                  \end{enumerate}

            \item {{[12 pts.]}}

                  \begin{enumerate}

                    \addtolength{\itemsep}{2mm}

                    \item incorrect; describes some invalid strings that are
                          not in the language like \emph{abbab} that have an
                          odd number of \emph{a}s

                    \item correct

                    \item correct

                  \end{enumerate}

          \end{enumerate}

          \vspace{-2.5mm}

          \begin{info}{\textbf{\underline{Grading scale (for all parts):}}}

            \begin{itemize}

              \addtolength{\itemsep}{1mm}

              \item In all parts the number of points divides the number of
                    subparts evenly-- \pts{2} for each question in (a), and
                    and \pts{4} for each question in (b) and (c).

              \item If the answer is ``correct'' for an incorrect r.e.\ or
                    automaton I would give no credit.

              \item If the answer is ``incorrect'' for an incorrect r.e.\ or
                    automaton, but the example string is wrong I would give
                    half credit for that question.

              \item If someone read the directions wrong and didn't give any
                    examples-- just wrote ``correct'' or ``incorrect''-- I'm
                    not exactly sure how to handle that.  Hopefully it won't
                    occur.  If it does, maybe talk with each other to see
                    what you think, and run it by me.

              \item If you think that any answers should get more or less
                    partial credit than what's above be sure to coordinate
                    (via email) with the other TAs grading the quiz, for
                    consistency of grading (so that everyone can deduct the
                    same credit for the same mistake, for all sections).  If
                    you're not sure whether a mistake should get more or
                    less partial credit, email all the TAs grading the quiz.
                    You can also ask me if you all can't agree.

            \end{itemize}

          \end{info}

  \end{enumerate}

\end{document}
