\documentclass[11pt]{article}

  \usepackage{330-f12}

  \usepackage[none,light,outline,timestamp]{draftcopy}

  % for quizzes and exams, due to the nameblock
  \addtolength{\headsep}{-6mm}
  \addtolength{\textheight}{6mm}

\begin{document}

  \hspace{\fill}%
  \nameblock

  \smallskip

  \header{\course}{Quiz\ \#4}{\Term}

  \vspace{-3mm}

    \noindent
    This quiz is 40 points.  \textbf{Do not start} until you're told you can
  begin.  You must turn in your quiz \textbf{immediately} when the end of
  the quiz is announced.

  \vspace{-2mm}

  \begin{enumerate}

    \addtolength{\itemsep}{10mm}

    \item {[12 pts.]} Consider the following OCaml code, which uses
          references:

          \vspace{-.5mm}

          \begin{multicols}{2}

            \begin{BVerbatim}
        let jack = ref "Jack";;
        let queen = ref "Queen";;
        let king = ref "King";;
        let ace = ref "Ace";;

        let cards = ref [jack; queen; king; ace];;

        let shuffle = function () ->
          match !cards with
            (a::b::c::d::[]) ->
              let tempB = !b in
                b := !d ;
                d := tempB ;
                cards := (c::b::a::d::[]);;

        shuffle();;
            \end{BVerbatim}

            \columnbreak

            Give the values that the following variables would have after
            execution of the code:

            {

              \renewcommand{\arraystretch}{3}

              \begin{tabular}[t]{l@{\hspace{2mm}\ans[2.5in]}}

                \texttt{jack}:
                  \\

                \texttt{queen}:
                  \\

                \texttt{king}:
                  \\

                \texttt{ace}:
                  \\

              \end{tabular}

            }

          \end{multicols}

          \vspace{-3mm}

    \item {[18 pts.]} Consider the following context--free grammar:
          \begin{grammar}[1.25]

            \production{S}{TU \midspc V}
              \\

            \production{T}{aTb \midspc \largeepsilon}
              \\

            \production{U}{cUd \midspc \largeepsilon}
              \\

            \production{V}{aVd \midspc W}
              \\

            \production{W}{bWc \midspc \largeepsilon}
              \\

          \end{grammar}

          \vspace{-.5mm}

          \begin{enumerate}

            \addtolength{\itemsep}{40mm}

            \item What is the set of strings generated by this grammar?
                  (Hint: express it as a union of two sets of strings.)

            \item Give a string that shows that this grammar is ambiguous:
                  \hspace{1.5mm}\ans[2.5in]

                  \pagebreak

            \item Now prove that the grammar on the previous page is
                  ambiguous, by constructing \textbf{two derivations} for
                  the string that you gave in the previous part, that will
                  demonstrate the ambiguity.

          \end{enumerate}

          \vspace{70mm}

    \item {[10 pts.]} Consider the following context--free grammar:
          \begin{grammar}[1.25]

            \production{E}{T + E \, \midspc \, E - T \, \midspc \, T * E \,
                           \midspc \, T}
              \\

            \production{T}{1 \, \midspc \, 2 \, \midspc \, 3 \, \midspc \, (E)}
              \\

          \end{grammar}

          \medskip

          Construct a parse tree for this grammar for the string $1 \, * \,
          2 \, - \, 3$:

  \end{enumerate}

\end{document}
