\documentclass[11pt,fleqn]{article}

  \usepackage{330-f12}

  \showinfo

\begin{document}

  \header{\course}{Quiz \#1 grading key}{\Term}

  Reminder-- you don't need to write the correct answers every single time
students make mistakes.  It's much quicker to just go give the quiz answers
once in class (and you can answer questions or solve the problems as needed
in the process).  But there \textbf{must} be some indication of what's
wrong.  In other words, you can't just give a score without at least
indicating which parts of the answer were wrong.  Just circling or marking
anything incorrect with an X, and writing a word or two, is often
sufficient.

  Also-- you need to coordinate via email about grading decisions, to ensure
the consistency of grading between TAs and sections.

  \vspace{-0mm}

  \begin{enumerate}

    \addtolength{\itemsep}{8mm}

    \item {[6 pts.]}

          \vspace{-2.5mm}

          \begin{enumerate}

            \addtolength{\itemsep}{2mm}

            \item Because \texttt{rstrip()} is not a mutator-- it returns a
                  new string, but does not modify its current object string.

            \item Two ways would be:

                  \vspace{-2.5mm}

                  \begin{multicols}{3}

                    \begin{Verbatim}
        str= str.rstrip()
        puts(str)
                    \end{Verbatim}

                    \columnbreak

                    \centering or

                    \columnbreak

                    \begin{Verbatim}
        str.rstrip!()
        puts(str)
                    \end{Verbatim}

                  \end{multicols}

          \end{enumerate}

          \vspace{-2.5mm}

          \begin{info}{\textbf{\underline{Grading scale:}}}

            \begin{itemize}

              \addtolength{\itemsep}{2mm}

              \item Each subpart is \pts[circle]{3}

              \item If the answer in either part is partly but not
                    completely right, (deduct \pts{-1} or \pts{-2}).  For
                    example, an answer might say ``\texttt{rstrip()} is
                    wrong, it should be \texttt{chomp()}'', and part (b)
                    gives a version using \texttt{chomp()}.  I would deduct
                    a couple of points in each part, but at least give some
                    credit.

              \item If the answer in either part is missing or completely
                    incorrect then deduct all credit for it.

              \item Note: the answer does not need to repeat the
                    \texttt{puts()}, in fact, it doesn't have to rewrite the
                    code.  Just writing in a \texttt{!} in the right place
                    would be fine.

              \item It's hard to anticipate what types of incorrect answers
                    might be given.  If you think that any answers should
                    get more or less partial credit be sure to coordinate
                    (via email) with the other TAs grading the quiz, for
                    consistency of grading (so that everyone can deduct the
                    same credit for the same mistake, for all sections).  If
                    you're not sure whether a mistake should get more or
                    less partial credit, email all the TAs grading the quiz.
                    You can also ask me if you all can't agree.

            \end{itemize}

          \end{info}

    \item {[20 pts.]}

          \vspace{-1.75mm}

          \begin{Verbatim}
        4 -> 1
        5 -> 3
        6 -> 5
        7 -> 7
        8 -> 9
        9 -> 11

        5 -> 3
        7 -> 7
        9 -> 11
          \end{Verbatim}

          \vspace{-2.5mm}

          \begin{info}{\textbf{\underline{Grading scale:}}}

            \begin{itemize}

              \addtolength{\itemsep}{1mm}

              \item It's hard to think about how to grade this without
                    seeing what types of mistakes students might make.  I
                    would say:

                    \vspace{0mm}

                    \begin{itemize}

                      \addtolength{\itemsep}{-.75mm}

                      % \item If someone gets the results all right it's full
                      %       credit of course.

                      \item If someone gets one or two numbers wrong deduct
                           \pts{-3}

                      \item If someone gets three or four wrong, or has a
                            couple of extra or missing lines, deduct
                            \pts{-6}

                      \item If someone gets various values wrong but it
                            seems like they do at least understand what the
                            keys and values that the hash is storing, deduct
                            \pts{-10}

                      \item If someone is way off but at least wrote some of
                            the numbers that would be printed, deduct \pts{-15}

                      \item Completely bogus or missing answers get no
                            credit.

                    \end{itemize}

                    \vspace{0mm}

              \item If you think that any answers should get more or less
                    partial credit than what's above be sure to coordinate
                    (via email) with the other TAs grading the quiz, for
                    consistency of grading (so that everyone can deduct the
                    same credit for the same mistake, for all sections).  If
                    you're not sure whether a mistake should get more or
                    less partial credit, email all the TAs grading the quiz.
                    You can also ask me if you all can't agree.

            \end{itemize}

          \end{info}

    \item {[10 pts.]} Justification is not necesssary according to the quiz;
          I am just adding explanation in some places anway.

          \vspace{-2.5mm}

          \begin{multicols}{3}

          \begin{enumerate}

            \addtolength{\itemsep}{1.5mm}

            \item yes (note that the string does have \textbf{zero} 0's
                  followed by an 0)

            \item no (because the space between the 30 and the 2 would not
                  be matched by the r.e.)

            \item no (because of the anchors)

            \item yes

            \item yes

          \end{enumerate}

          \end{multicols}

          \vspace{-2.5mm}

          \begin{info}{\textbf{\underline{Grading scale:}}}

            \begin{itemize}

              \addtolength{\itemsep}{2mm}

              \item Each part is \pts[circle]{2}

              \item I'm not seeing any circumstances where partial credit
                    might be appropriate, but if you come across any, be
                    sure to coordinate with the other TAs grading the quiz
                    for consistency of grading.  You can also ask me if you
                    all can't agree.

            \end{itemize}

          \end{info}

          \enlargethispage{10mm}

    \item {[4 pts.]} Because Ruby is interpreted, unlike C and Java, it's
          easy for the interpreter to just treat a string as a piece of code
          and interpret it.

          \vspace{-3.5mm}

          \begin{info}{\textbf{\underline{Grading scale:}}}

            \begin{itemize}

              \addtolength{\itemsep}{2mm}

              \item If an answer is wrong but at least is describing
                    something that is different about Ruby than C and Java
                    (for example ``The reason Ruby has \texttt{eval()}, but
                    not C and Java, is that Ruby has implicit types.'') I
                    would give half credit.

              \item I'm not seeing other circumstances where partial credit
                    might be appropriate, but if you come across any, be
                    sure to coordinate with the other TAs grading the quiz
                    for consistency of grading.  You can also ask me if you
                    all can't agree.

            \end{itemize}

            \smallskip

            By the way, note that compiled languages can actually have an
            \texttt{eval()} mechanism, but it essentially involves embedding
            the compiler, which is large and complex, into an executable
            program.  In contrast, in an interpreted language
            \texttt{eval()} involves adding almost nothing that's not
            already there anyway.

          \end{info}

  \end{enumerate}

\end{document}
