\documentclass[10pt]{article}

  %
  % Should have made the quiz be in 11 pt. font....
  %

  %
  % Some students were confused by the wording in #3 "Which ones [regular
  % expressions] will match the string "330 2012""- they thought that it was
  % asking if the entire string would match the regular expression.  Maybe
  % it should have said something like "Which ones will "330 2012" =~
  % /regular expression/ be true for?
  %

  \usepackage{330-f12}

  \usepackage[none,light,outline,timestamp]{draftcopy}

  % for quizzes and exams, due to the nameblock
  \addtolength{\headsep}{-6mm}
  \addtolength{\textheight}{6mm}

  \newcommand{\qfouranswer}[1]{%
    \begin{tabular}[t]{@{}p{1.25in}@{\lans}}%
      \texttt{/#1/}%
    \end{tabular}%
  }

\begin{document}

  \hspace{\fill}%
  \nameblock

  \smallskip

  \header{\course}{Quiz\ \#1}{\Term}

  \vspace{-3mm}

    \noindent
    This quiz is 40 points.  \textbf{Do not start} until you're told you can
  begin.  You must turn in your quiz \textbf{immediately} when the end of
  the quiz is announced.

  \vspace{-2mm}

  \begin{enumerate}

    \addtolength{\itemsep}{16mm}

    \item {[6 pts.]} Sally Student wrote the following Ruby statements to
          remove any whitespace from the end of the string \texttt{str} and
          print it, but it's not working right for her:

          \vspace{-2.5mm}

          \begin{quote}

            \begin{Verbatim}
        str.rstrip()
        puts(str)
            \end{Verbatim}

          \end{quote}

          \begin{enumerate}

            \addtolength{\itemsep}{16mm}

            \item \textbf{Briefly} say why it doesn't do what she wants.

            \item Change or revise the code so it does do what she wants.

                  \vspace{4mm}

          \end{enumerate}

          \vspace{-2.5mm}

    \item {[20 pts.]} Give the output of the Ruby program below.  If it
          doesn't produce any output say so, or if it has any type of error
          then give any the output it produces before the problem, and
          explain what's wrong.

          \vspace{-1.5mm}

          \begin{multicols}{2}

            \VerbatimInput[gobble=0]{quiz1.rb}

            \columnbreak

            \centering

            \answerblank{13}{3in}{8.4mm}

          \end{multicols}

          \turnover

    \item {[10 pts.]} Consider each of the Ruby regular expressions below.
          Which ones \textbf{will} match the string ``330 2012''?  Just
          write ``yes'' or ``no'' for each one.  (Note the string contains a
          space.)

          \begin{enumerate}

            \addtolength{\itemsep}{6mm}

            \item \qfouranswer{0*0}

            \item \qfouranswer{30*2}

            \item \qfouranswer{\string^0.*0\$}

            \item \qfouranswer{[0123 ]\string{8\string}}

            \item \qfouranswer{3+0 2+012}

          \end{enumerate}

          \vspace{-2.5mm}

    \item {[4 pts.]} Ruby's \texttt{eval} method, illustrated in
          discussion section, allows treating the contents of a string as a
          piece of code, which is executed.  Neither C or Java have a
          similar mechanism.  What is it about Ruby, as opposed to those
          languages, that facilitates the language having an operation like
          this?

  \end{enumerate}

\end{document}
