\documentclass[11pt]{article}

  %
  % Should have explicitly disallowed leading zeros in group numbers;
  % otherwise, are 050 and 50 the same group or not?
  %

  %
  % One of the TAs answered on Piazza that days in dates could have leading
  % 0s, which brings up the same issue- is "Feb 3" the same as "Feb 03", and
  % if so what format should these days be printed in in the third section
  % (if they overlap)?
  %

  %
  % The project doesn't really say specifically what to do if there's more
  % than one error (only one argument, which is an invalid file)....
  %

  %
  % Is it unclear whether days should always be printed with two digits or
  % not???
  %

  %
  % don't test overlapping same hosts or different users in public tests
  % day causes overlap
  % exact boundary times overlap
  % multiple users in third section
  %

  \usepackage[none,light,outline,timestamp]{draftcopy}

  \usepackage{330-f12}

  %
  % This convoluted stuff enables trailing blanks to be shown and not
  % suppressed when showspaces is true.
  %
  \DefineShortVerb{\|}
  \SaveVerb{Verb}| |
  \newcommand{\twospaces}{\UseVerb{Verb}\UseVerb{Verb}}

\begin{document}

  \header{\course}{Project \#2}{\Term}

  \vspace{-3mm}

  \begin{centering}

    Date due: Monday, October 22, 10:00:00 p.m.

  \end{centering}

  \vspace{-6mm}

  % \addtolength{\baselineskip}{1.9mm}

  \section{Introduction}

    A common use of scripting languages is to write programs that read and
  analyze the output of other programs, often in order to perform various
  administrative and maintenance tasks.  In this project you will write a
  Ruby program that will read the hypothetical output of a program producing
  information about users who log in to a machine, and produce a report with
  various results based upon it.

    The project's purpose is to use some of Ruby's data structures, which
  include arrays, hashes, arrays of arrays, hashes of arrays, hashes of
  hashes, etc., and perhaps classes implementing user--defined data
  structures.  There are various ways the information needed could be stored
  using Ruby's data structures; consider different options and choose one
  you think will work well.  The most obvious data structure may not work
  the best.  Although you should evaluate different data structures before
  beginning to code, don't be afraid to change your mind if your approach
  doesn't work well.  Of course, altering the data structures used will be
  easier if you begin writing your program in a flexible, modular, and
  well--structured fashion.  Writing classes and subroutines or methods to
  perform tasks are ways to make a program modular.

    It may be possible to implement some steps in your program very easily
  using methods of Ruby's library classes, particularly in situations where
  it may be difficult or tedious to write your own code.  It would be
  worthwhile to read about those that may be applicable, especially the
  \texttt{Hash} and \texttt{Array} class methods.  Some of the
  \texttt{String} methods could prove useful as well, and in addition the
  \texttt{Time} class could be quite helpful.

    To get the project files log into the Grace systems, execute
  \texttt{cp -r \string~/330public/proj2 \string~/330}, and cd to
  \texttt{\string~/330/proj2} to start working.

    Remember that important questions may be raised and answered
  % , or clarifications made to this project after it has been posted,
  in Piazza.  Be sure to read all messages there often.
  % -- to avoid missing information which could cause you to lose credit on
  % -- your project.
  However, keep in mind that you \textbf{cannot} ask or explain how to
  implement any part of a project (even a small part).  You are expected to
  read all messages before posting questions.  See the project grading policy
  on the class webpage for full expectations and restrictions related to
  Piazza.

    In this project we may \textbf{deduct credit} for students making more
  than fifteen submissions.  This is a 300--level course that is taken after
  students have had a good deal of programming coursework.  Students at this
  level should be able to test programs thoroughly themselves.

  \section{Problem description}

    Suppose a friend of yours has graduated and gotten a job as a system
  administrator at a great company.  The system administrators are
  responsible for keeping the company's systems running properly, securely,
  and reliably for the programmers and developers who work on them, as well
  as the users of the company's services.  As the systems are running they
  generate a large amount of logging data, and for a large system this
  information is quite voluminous.  It would be impossible for the system
  administrators to read through the data manually to look for potential
  problems, so they have written a number of programs (that often run
  automatically at regular periods) to analyze the data collected and inform
  them when there's something that might need further investigation.

    Assume there is a program that automatically keeps track of every time
  someone logs in to one of the company's systems.  The information
  collected includes the IDs of the users who log in, the date and time they
  logged in, the duration of their login session, and where they logged in
  from.

    The hypothetical system administrators are interested in finding out
  several things from this data:

    \enlargethispage{3mm}

    \vspace{-2.5mm}

    \begin{itemize}

      \addtolength{\itemsep}{-1.5mm}

      \item They want to know how many times each user logged into the
            systems during the time the login data covers.

      \item They want to find out how many times each group of users logs
            in.  On a UNIX system users can be members of different groups,
            which allows giving access to files to sets of users.  Every
            user is a member of a default group, which can be used to
            aggregate users by category or type.  If the system
            administrators see that certain groups of users are using the
            machines more heavily it might be desirable to give them higher
            priority, or possibly to consider buying new systems for their
            dedicated use.

      \item Lastly, for security reasons,they want to know whether any user
            logged in at the same time from more than one location.  While
            it might not represent a problem if a user was logged in at the
            same time from more than place (for example, a user might have
            left themselves logged in from home and then also logged into
            the system when they arrived at work), it might represent a
            situation where an intruder was able to crack someone's password
            or take advantage of some vulnerability to break into someone's
            account and gain access to the system some other way, and the
            systems administrators want to be alerted to any such situations
            so they can look into them more closely.

    \end{itemize}

    \vspace{-3.5mm}

  \section{Program input and output}

    \subsection{Input}

      In the discussion below ``whitespace character'' refers to a blank space
    or tab character, and ``whitespace'' refers to a sequence of one or more
    consecutive whitespace characters.

      Your program will read zero or more lines of standard input, each ending
    with a newline, and will also be run with two arguments on its command
    line representing the names of two auxiliary input files.  Validity of
    the input and error conditions are described in a later subsection,
    after processing of valid input is described.

      \subsubsection{Standard input}

        The standard input represents the data produced by the hypothetical
      logging program about the logins that is to be analyzed, where each
      line has the data for one login session, consisting of five groups of
      information.  Note however that one group of information has internal
      whitespace, so there will actually be seven whitespace--separated
      fields on each line (an arbitrary amount of whitespace should separate
      the seven pieces of information on a line).  Whitespace may also
      appear before the first field or following the last field on a line.

        \enlargethispage{3mm}

        \vspace{-2.5mm}

        \begin{itemize}

          \addtolength{\itemsep}{-1mm}

          \item The first field represents the login ID of the user who
                logged in during this session, which should be a sequence of
                one to eight uppercase or lowercase letters or digits that
                begins with a letter.

          \item The second field contains the date when the user logged on,
                which is the day their login session began.  This should
                consist of the common three--letter abbreviation of the day
                of the week the login began, the common three--letter
                abbreviation for the month it began, and a nonnegative
                integer for the day of the month the login began, which will
                obviously be between 1 and 31 (as above, all separated by
                whitespace).  The first letters of the day and month names
                should be capitalized.

          \item The third field contains the starting time of the user's
                login session.  Times are represented in 24--hour format
                using six digits, with colons between the hours and minutes
                and minutes and seconds, for example \texttt{16:07:00},
                \texttt{05:19:10} or \texttt{00:21:20}.

          \item The fourth field represents the duration of the login
                session.  For login sessions less than twenty--four hours in
                length this should be six digits inside parentheses
                consisting of the number of hours, minutes, and seconds
                separated by colons, as in \texttt{(03:20:14)} or
                \texttt{(00:18:27)}, but for login sessions of twenty--four
                hours or longer the number of days, (twenty--four hour
                periods) and a plus sign should precede the number of hours
                and minutes, for example \texttt{(2+03:04:53)}, indicating a
                login session that lasted for two days, three hours, four
                minutes, and fifty--three seconds.  Notice that a user who
                logged in and logged out within the same second (a fast
                typist) would have a login duration that would appear as
                \texttt{(00:00:00)}).

          \item The last field represents the name of the machine the user
                logged in from, which should be a sequence of lowercase
                letters, digit characters, period characters, and dash
                characters beginning with a letter or a digit.

        \end{itemize}

        \vspace{-2.5mm}

        The lines of this file should appear in increasing chronological order
      by login start time.  The input may span one or more month boundaries
      (so obviously it can span multiple days or weeks), but you may assume
      that the dates don't span a year boundary.

      \subsubsection{Password file}

        The first command--line argument should be the name of a file storing
      the login IDs of all of the users on the system, and the number of the
      primary group they belong to (each group on a UNIX system
      has a unique number).  Login IDs are as described above, namely a
      sequence of one to eight uppercase or lowercase letters or digits
      beginning with a letter.  The login ID on each line should be followed
      on the same line by a colon, an asterisk, another colon, the number of
      the user's default group, which should be a nonnegative integer, and a
      newline, so each user's ID and other data will appear on one line in
      this file.  Note that multiple users may have the same primary group,
      and that no whitespace should appear other than the newline ending
      each line.

      \subsubsection{Group file}

        The second command--line argument should be the name of a file storing
      the names and information about the groups on the system.  Group names
      consist of one or more lowercase letters.  Each group name on a line
      should be followed by a colon, an asterisk, another colon, the number
      of the group, which as above should be a nonnegative integer, and a
      newline.  Note that the name of a user's default group can be
      determined by finding their entry in the password file, extracting the
      group number that is the third field of that line, then looking in the
      group file for the line having that group number as its third field.
      As in the password file, terminating newlines are the only whitespace
      that should be present.

        Note that group names and numbers should be unique in that no two
      lines of this file should have the same group name or number, but you
      cannot assume anything about the values of group numbers except that
      they are supposed to be nonnegative integers (for example, the group
      numbers may be in random order in the lines of this file and may not
      represent a consecutive sequence of integers).  Note lastly that some
      groups in this file may be ones that no users have as their primary
      group.

    \subsection{Normal output}

      Your program should write all of its results to its standard output and
    (assuming there are no errors in the input) produce three sections of
    output.

      \subsubsection{Number of times users logged in}

      For the first section of output your program should print a line
      reading ``Section \#1:'', (without the quotes), ending in a newline,
      spelled and punctuated exactly as shown, followed by zero or more
      lines of output each containing the login ID of a user followed by the
      number of times that user logged into the system.  These lines must
      follow the \textbf{exact} format ``\verb*@  mickey: 3 time(s).@'',
      without the quotes, where blank spaces are shown as \verb*@ @, for an
      example user with ID ``mickey'' who logged in three times.  These
      lines are to be printed in increasing sorted order by login ID for the
      users who logged in at least once, where increasing refers to
      lexicographic or dictionary order, the same order that Ruby will
      compare strings using the \texttt{<=>} operator, or sort an array of
      strings using the \texttt{Array.sort()} method (which is done using
      \texttt{<=>}).

        \enlargethispage{3mm}

        Each login session for a user is to be counted once, \textbf{except}
      if a login session spans a day boundary (begins \textbf{before}
      00:00:00 and ends \textbf{at or after} 00:00:00) it is to be counted
      as \textbf{more than one time} that the user logged in.  For example,
      if a user logged in at 23:30 on one day, stayed logged in for an hour,
      and logged out at 00:30 the next day, it would be counted as the user
      being logged in \textbf{twice}.  Furthermore, a login session that
      lasts more than twenty--four hours could be counted as the user being
      logged in three or more times, if it spans the midnight boundary more
      than once.  To determine the number of days a login session lasted, as
      well as to produce the results for the third section of output, you
      will have to add the login duration to the login start time on each
      line of the standard input.

        A completely empty or blank line should separate this section of
      output from the next one.

      \subsubsection{Number of times members of groups logged in}

        In the second section of output your program should aggregate the
      number of times users logged in (for those who logged on at least once)
      by user groups instead of user IDs, using the user group information
      present in the password and group files.  This section should consist
      of a line reading ``Section \#2:'', (without the quotes), ending in a
      newline, spelled and punctuated exactly as shown, followed by a
      sequence of zero or more lines, each containing a group name followed
      by the number of times that users who have that as their primary group
      logged in to the system.  These lines must follow the exact format
      ``\verb*@  users group: 3 time(s).@'', without the quotes, where blank
      spaces are shown as \verb*@ @, for an example group with name ``users''
      whose members logged in three times.  The lines are to be printed in
      increasing order by group name.

        Just as in the previous section, in this section logins crossing a day
      boundary are also to be counted as more than one login.  A completely
      empty line should also separate this section of output from the next
      one.

      \subsubsection{Duplicate logins}

        For the third and last section of output your program should print a
      line reading ``Section \#3:'', (without the quotes), ending in a
      newline, spelled and punctuated exactly as shown, followed by zero or
      more pairs of lines.  If you program detects that the same user logged
      in at any time while they were already logged in from \textbf{another}
      machine, it should print three lines in this section, followed by a
      completely empty line; these should be of the exact format
      illustrated by this example (where spaces are shown as \verb*@ @):

        \medskip

        \begin{centering}

          \begin{BVerbatim}[showspaces,commandchars=\\\{\}]
        {\twospaces}mickey:
        {\twospaces\twospaces}Wed Sep 12 15:00:00 03:20:00 914-45a.umd.edu
        {\twospaces\twospaces}Wed Sep 12 17:00:00 04:01:00 libwkmck1f2.umd.edu
          \end{BVerbatim}

        \end{centering}

        \medskip\smallskip

        In other words, this means that mickey logged in at 17:00:00 on
      Wednesday, September 12, from the host indicated in the second line,
      while already logged in starting at 15:00:00 on September 12 from the
      host indicated in the first line.  If one login time ends at exactly
      the same second that another one begins they are considered to
      overlap.

        The data that should appear on these lines should be evident from the
      example above, except that a duration of twenty--four hours or greater
      should be indicated with a number of days and a preceding plus sign,
      as for example \texttt{1+01:03:04}.  Note that the hours, minutes, and
      seconds should be printed with exactly two digits, with a leading
      zero if less than ten, but not the days.  The first line of the pair
      should be the login time that began earlier, which should be the
      same order these two sessions appeared in the standard input.  If two
      logins began at exactly the same time then the one appearing earlier
      in the input should appear first.  Each such pair should be
      \textbf{followed} by a blank line, so unless there are no overlapping
      logins at all the last such pair printed in this section will be
      followed by a blank line, and blank lines will separate all such
      pairs.

        \enlargethispage{4mm}

        The order that these pairs of lines should be printed in, if there is
      more than one pair, is in increasing chronological order by the
      original login start time, with pairs for multiple duplicate login
      sessions that have the same starting time ordered by the start time of
      the overlapping or later login session.  In other words, your program
      must scan through the login information in chronological order of
      start time (the order of the standard input), and for each login
      session that occurs scan through all the login sessions with later
      start times to see if they began while this one was still going on.
      Notice that these pairs of lines will not be sorted by login ID, but
      chronologically; if multiple users have multiple overlapping login
      sessions then their pairs of output lines may be separated by pairs of
      lines for other users.

        Note that a pair of output lines is to be generated for \textbf{every}
      pair of overlapping logins.  For example, if a user logged in for
      login session A, then logged in later from another host for login
      session B that overlapped chronologically with A, then logged in again
      from another host for login session C, from a host different from the
      two hosts in A and B, at a time that overlaps with both A and B, this
      section would have a pair of lines for sessions A and B, then a pair
      of lines for sessions A and C, then a pair of lines for sessions B and
      C.

        There will be no lines in this section of output (other than the
      heading, terminated by a newline) if no users logged in at the same
      time from different hosts.  Note that a user can be logged in multiple
      times from the \textbf{same} host concurrently, or at
      \textbf{nonoverlapping} times from different hosts, and no output
      should appear in this section as a result, it's only logins at
      overlapping times from different hosts that are a concern.

      \subsubsection{Invalid input and error conditions}

        \vspace{0mm}

        \begin{itemize}

          \addtolength{\itemsep}{-1.25mm}

          \item Unless exactly two command--line arguments are present, or
                if either argument (or both) does not refer to a file that
                is present and can be successfully opened, your program
                should print a single line reading ``Invalid argument.''
                (without the quotes), ending in a newline, spelled and
                punctuated exactly as shown, and quit without producing any
                further output.

          \item Any lines in the password file or group file that are
                invalid (do not match the format described above) should be
                silently ignored (silently ignored means no error message
                should be printed).  If any user ID appears in more than one
                line of the password file, the second and subsequent
                occurrences are just to be ignored; only the first line for
                a user ID is valid.  Similarly, if the same group name or
                number appears in more than one line of the group file only
                the first one is valid and the rest should be silently
                ignored.

                If after ignoring invalid lines from the password file there
                are no valid lines remaining in it, your program cannot
                produce correct results in all cases, so it should just
                print a single line reading ``Invalid file.'', (without the
                quotes), ending in a newline, spelled and punctuated exactly
                as shown, and quit without producing any further output.
                The same applies if there are no valid lines in the group
                file after ignoring invalid ones.  Since some output may
                have already been generated when either of these situations
                is detected, any preceding output doesn't matter (this
                message should just be the last line of output in either of
                these cases).

          \item Your program cannot work correctly if any user's login ID
                does not appear anywhere in the password file or if any
                group number from the password file does not appear anywhere
                in the group file.  It doesn't matter what results are
                produced in these cases.

          \item If any lines from the standard input are ``syntactically''
                invalid in any way (wrong number of fields, or any field is
                not of the format described above) the line is also just to
                be silently ignored.  Lines that are ``semantically''
                invalid should also be ignored, meaning ones that have the
                right number of fields of the proper form, but their values
                are incorrect given the descriptions above.  For example,
                lines with hours not in the range $0..23$, minutes not in
                the range $0..59$, incorrect day numbers for months (such as
                a line with Sep 31 in its start time), etc., should all be
                ignored.  Since the input does not have an indication of the
                year, just assume it is not a leap year.

                If any line in the standard input has a date for its start
                time that is (strictly) earlier than the preceding line it
                is also to be silently ignored.  It's not incorrect if two
                login sessions begin at exactly the same time.

          \item It's not an error if the program's standard input is empty
                (has zero valid lines), either because it has no lines, or
                because it has no valid lines (after ignoring invalid ones).
                In this case the output would just consist of the three
                section headings with two blank lines between them (assuming
                the password and group files were valid).

                \enlargethispage{3mm}

          \item Since the input does not have an indication of the year,
                your program should just assume that the date indicated
                actually does fall on the day of the week given.  For
                example, if a line of the standard input contains Wed Sep 12
                you just have to assume that September 12 was actually a
                Wednesday during whatever year this data was generated.

          \item Other than the fact that dates in the lines of the standard
                input must appear in nondecreasing chronological order, your
                program does not have to enforce any cross--line validity.
                For example, one input line could have ``Mon Sep 10'' in its
                login start date field, and the next line could have ``Wed
                Sep 13'', and although these dates could not possibly occur
                in the same year, you should just consider the dates valid.
                (Note that the day names are not used in producing any of
                the three output sections above, and have no effect other
                than possibly causing some input lines to be ignored if they
                are incorrect).

        \end{itemize}

        \vspace{-3.5mm}

        \enlargethispage{3mm}

  \section{Example}

    The example below assumes that the first line of the Ruby program in the
  file ``\texttt{proj2.rb}'' contains the full pathname of the Ruby
  interpreter preceded by the characters \texttt{\#!}, the UNIX command
  \texttt{chmod} has been used to make the file executable, and the UNIX
  prompt on this system is \texttt{grace2:\string~/330/proj2:}.

    \begin{Verbatim}[gobble=0,formatcom=\small,baselinestretch=.9,
                     commandchars=\\\{\}]
grace2:~/330/proj2: cat login-data
bugs     Wed  Sep  12  15:00:00  (04:00:00)  914-45a.umd.edu
bugs     Wed  Sep  12  17:00:00  (04:01:00)  libwkmck1f2.umd.edu
tweety   Wed  Sep  12  17:22:00  (07:08:00)  c-69-25-171.md.comcast.net
bugs     Wed  Sep  12  18:00:00  (00:01:00)  209.39.175.23
bugs     Wed  Sep  12  20:00:00  (00:02:00)  coredump.umd.edu
tweety   Wed  Sep  12  21:01:00  (03:29:00)  c-69-25-171.md.comcast.net
mickey   Wed  Sep  12  23:58:00  (00:01:00)  c-68-33-204.md.comcast.net
donald   Thu  Sep  13  10:11:00  (00:20:00)  wireless-20-16-67.umd.edu
tweety   Thu  Sep  13  12:20:00  (00:01:00)  ptx-dual-v1.net.umd.edu
donald   Thu  Sep  13  13:08:00  (00:11:00)  ptx-dual-v1.net.umd.edu
jessica  Thu  Sep  13  16:46:00  (00:05:00)  ptx-dual-v1.net.umd.edu
jessica  Fri  Sep  14  03:36:00  (00:00:00)  pool-151.west.verizon.net
goofy    Fri  Sep  14  14:06:00  (00:04:00)  mastercoder.student.umd.edu
mickey   Fri  Sep  14  17:31:00  (00:12:00)  ptx-dual-v1.net.umd.edu

grace2:~/330/proj2: cat passwd-file
bugs:*:3
donald:*:2
goofy:*:1
jessica:*:3
mickey:*:2
tweety:*:2

grace2:~/330/proj2: cat group-file
root:*:0
sysadmin:*:1
users:*:2
devel:*:3
webmaster:*:4

grace2:~/330/proj2: proj2.rb passwd-file group-file < login-data
Section #1:
  bugs: 4 time(s).
  donald: 2 time(s).
  goofy: 1 time(s).
  jessica: 2 time(s).
  mickey: 2 time(s).
  tweety: 5 time(s).

Section #2:
  devel group: 6 time(s).
  sysadmin group: 1 time(s).
  users group: 9 time(s).

Section #3:
  bugs:
    Wed Sep 12 15:00:00 04:00:00 914-45a.umd.edu
    Wed Sep 12 17:00:00 04:01:00 libwkmck1f2.umd.edu

  bugs:
    Wed Sep 12 15:00:00 04:00:00 914-45a.umd.edu
    Wed Sep 12 18:00:00 00:01:00 209.39.175.23

  bugs:
    Wed Sep 12 17:00:00 04:01:00 libwkmck1f2.umd.edu
    Wed Sep 12 18:00:00 00:01:00 209.39.175.23

  bugs:
    Wed Sep 12 17:00:00 04:01:00 libwkmck1f2.umd.edu
    Wed Sep 12 20:00:00 00:02:00 coredump.umd.edu
    \end{Verbatim}

    \vspace{-2mm}

  \section{Development suggestions}

    \begin{itemize}

      \addtolength{\itemsep}{-1mm}

      \item You may want to write your program to handle normal input first,
            then add checks for invalid input.  This is because some checks
            for invalid input may require the data is already stored in some
            manner to be detected, and also because presumably more weight
            will be given to tests of normal cases than to error cases.

      \item You can perform date and time calculations more easily than you
            might think using Ruby's \texttt{Time} class.  In particular,
            \texttt{Time.local()} creates a \texttt{Time} object and the
            \texttt{+} operator adds a number of seconds to a \texttt{Time}.
            Note that \texttt{Time}s can also be compared using the
            \texttt{<=>} comparison operator.  If you print a \texttt{Time}
            object its \texttt{to\_s()} method will be invoked, which (in
            the version of Ruby on the Grace machines) will result in a
            string of the form \texttt{2012-09-12 20:00:00 -0400}.  However,
            the \texttt{Time} class \texttt{strftime()} method can be used
            to format times in a large variety of ways.

      \item As mentioned in discussion section, Ruby has an integrated
            debugger, which can be invoked by running Ruby with the
            \verb@-rdebug@ option before the name of your program (e.g.,
            \texttt{ruby -rdebug proj2.rb}).  The debugger's '\texttt{p}'
            command may be very helpful, as it prints the components of a
            variable or data structure (including arrays and hashes), in a
            formatted fashion, including any nested data structures.  The
            '\texttt{var local}' command prints all of the local variables
            at the current point of execution.  The chapter ``When Trouble
            Strikes'' of The Pragmatic Programmer's Guide linked to on the
            class webpage discusses the debugger in a bit more detail.

    \end{itemize}

    \vspace{-3.5mm}

  \section{Project requirements and submitting your project}

    \begin{enumerate}

      \addtolength{\itemsep}{-.5mm}

      \item Your program should read all of its input from its standard
            input and the two files whose names appear on the command line,
            and write all of its output to its standard output.  Of course
            in UNIX standard input may be redirected from a file, or
            standard output may be redirected to a file.

      \item You may use any Ruby language features in your project that you
            would like, regardless of whether they were covered in class or
            not, so long as your program works successfully using the
            version of Ruby installed on the OIT Grace Cluster and on the
            CMSC project submission server.

      \item Your submitted program file \textbf{must} be in a single file
            named ``\texttt{proj2.rb}'', otherwise the submit server will
            not recognize it, consequently its score will be zero.

      \item To check that your output matches the expected output you can
            use the UNIX \texttt{diff} command, for example:

            \begin{centering}

              \begin{BVerbatim}
        proj2.rb passwd-file group-file < login-data > my-output
        diff -bB public1.output my-output
              \end{BVerbatim}

            \end{centering}

            \smallskip

            If no output at all is produced by \texttt{diff}, your output
            matches the expected output and is correct.  This \texttt{diff}
            command is exactly what the submit server is going to be using
            to check your program's output, so using it you can see yourself
            whether your output is right or not; if you don't get matching
            results when comparing your output this way then the submit
            server is not going say your results are right either.

            Two notes: the diff options \texttt{-bB} ignore differences that
            consist of only blank lines and amount of whitespace, since if
            all the data in your output matches the expected output we'll
            call your results right even if some whitespace doesn't exactly
            agree.  Secondly, you may prefer \texttt{diff}'s unified output
            format, which you can see instead by using the options
            \texttt{-bBu}.  In the unified format surrounding lines of the
            files being compared are shown (not just the lines that differ),
            with lines that should not be present in the first file preceded
            by \texttt{-} and lines that are not missing from in the first
            file preceded by \texttt{+}.  (What is being compared is exactly
            the same whether the unified format is used or not, the results
            may just be easier to follow.)

      % \item Note that the project grading policy on the class syllabus does
      %       not mention anything about losing credit for warnings, so as long
      %       as your output is correct you will not receive any penalty if your
      %       program produces warning messages.

      \item You aren't required but are encouraged to use Ruby's \texttt{-w}
            option as it will assist in finding potential bugs.

      \item Your program \textbf{must have} a comment near the top that
            contains your name, TerpConnect login ID (which is your
            directory ID), your university ID number, and your section
            number.

            The Campus Senate has adopted a policy asking students to
            include the following statement on each major assignment in
            every course: ``I pledge on my honor that I have not given or
            received any unauthorized assistance on this assignment.''
            Consequently you're requested to include this pledge in a
            comment near the top of your program source file.  See the next
            section for important information regarding academic integrity.

      \item Note that although you are not being graded on your source code
            or style (see the separate project grading handout), if the TAs
            cannot read or understand your code they cannot help should you
            have to come to office hours, until you return with a
            well--written and clear program.

      \item As before, to submit your program just type the single command
            \texttt{submit} from the \texttt{proj2} directory that you
            copied above, where your program file \texttt{proj2.rb} should
            be.  As mentioned above you may lose credit for having more than
            fifteen submissions.

    \end{enumerate}

    \vspace{-4.5mm}

  \section{Academic integrity}

    Please \textbf{carefully read} the academic honesty section of the course
  syllabus.  \textbf{Any evidence} of impermissible cooperation on projects,
  use of disallowed materials or resources, or unauthorized use of computer
  accounts, \textbf{will be submitted} to the Student Honor Council, which
  could result in an XF for the course, or suspension or expulsion from the
  University.  Be sure you understand what you are and what you are not
  permitted to do in regards to academic integrity when it comes to project
  assignments.  These policies apply to all students, and the Student Honor
  Council does not consider lack of knowledge of the policies to be a
  defense for violating them.  Full information is found in the course
  syllabus-- please review it at this time.

\end{document}
