\documentclass[11pt]{article}

  %
  % Idea for modifying this project for the future: allow two trains going
  % in opposite directions to be in the same station at the same time.  Or
  % does this really add anything?
  %

  %
  % Should the project have said that you can assume that the same station
  % doesn't appear twice on any train line (there is no circular line)?
  %

  %
  % one test- some lines radiating out of the center, then an
  % almost-circular line around their ends
  %
  % another similar idea- an almost-circular line with a grid of lines
  % inside it
  %
  % another idea- one line goes from A...B (with several stations in
  % between), another from B to C, another from C to D, another from D to E,
  % so a passenter going from A to E has to take them all
  %
  % characters- hobbit, Beatles, Maryland womens' basketball (or other
  % sport) players, famous computer scientists
  %

  %
  % Why didn't I just have metro.rb be named proj5.rb?  It would have made
  % things more uniform, in the Makefile also....
  %

  %
  % Test with no passengers, and a limit of 0?
  %

  %
  % Clarified on Piazza (project should have said):
  %
  % Train line names may consist of any consecutive sequence of one or more
  % upper or lowercase letters.  Train numbers may consist of any
  % consecutive sequence of one or more digit characters, including 0 (there
  % may be more than 9 trains on any line).  Leading zeros will not appear.
  % Station names may consist of any consecutive sequence of one or more
  % characters, including punctuation and whitespace.  Station names consist
  % of everything on a line after the word "entering" or "leaving" and a
  % single blank space, up until the end of the line.
  %

  %
  % In Chau Wen's grader_metro.rb passengers sleep for 0.01 seconds when
  % boarding each train, but the project assignment (his also) only says
  % that trains should sleep at each station, not passengers.
  %

  \usepackage[none,light,outline,timestamp]{draftcopy}

  % For some reason \definecolor is defined on home Linux machine, but not
  % department Linux machine, without this package.
  %
  % \usepackage{color}

  \usepackage{330-f12}

  \usepackage[dvips,
              colorlinks=true,
              urlcolor=red,         % \href{...}{...} external (URL)
              % filecolor=...  seems to have no effect
              % filecolor=green,    % \href{...} local file
              linkcolor=blue,       % \ref{...} and \pageref{...}
              pdftitle={CMSC 330 project},
              pdfauthor={Nick Feamster and Larry Herman},
              pdfsubject={CMSC 330 project},
              pdfkeywords={CMSC, 330},
              pdfproducer={LaTeX},
              pagebackref,
              bookmarks=true,
              pdfstartview=FitH
             ]{hyperref}

\begin{document}

  \header{\course}{Project \#5}{\Term}

  \vspace{-3mm}

  \begin{centering}

    Date due: Monday, December 10, 10:00:00 p.m.

  \end{centering}

  \vspace{-6mm}

  % \addtolength{\baselineskip}{1.25mm}

  \section{Introduction}

    In this project you will develop a multithreaded Ruby program that can be
  used to simulate a subway system like the Washington Metro.  You will also
  write code to process simulation output to display the state of the
  simulation, and (optionally) verify its feasibility.

  \section{End--of--semester TerpConnect account cleanup}

    Before the actual project is discussed, please note this information about
  accessing files and cleaning up your TerpConnect account after classes are
  over.

    At some point after the semester ends you will lose permission to access
  the files in your extra disk space (probably including your projects), as
  well as permission to access the information we provided in the directory
  \texttt{330public} (examples from discussion, secret tests, etc.).  At
  some point after that all these files will be automatically deleted by
  OIT.  If you want to save any of your projects or other coursework, or any
  of the files in \texttt{330public}, it will be necessary to do so after
  finals are over.  Sometimes students are asked for copies of class
  projects during job interviews, as part of scholarship applications, for
  internships, etc.; the instructional staff will not be able to provide
  these in the future, so you'll need to save copies yourself if you might
  ever need them later.  Recall that the \texttt{-r} option to the
  \texttt{cp} command recursively copies a directory and all its contents,
  including subdirectories, so a command like \verb@cp -r ~/330 ~/330.f12@
  would copy everything in your extra course disk space (which the symbolic
  link \texttt{330} points to) to a directory in your TerpConnect account
  home directory named \texttt{330.f12} (or use any other name if you like).
  Of course you have to have enough free disk space in your TerpConnect
  account to store the files; the \texttt{quota} command lets you know how
  much free space you have.  Although you will lose your login permission to
  the Grace systems after the semester, you will still be able to log into
  your TerpConnect account on any TerpConnect machine (just log in to
  terpconnect.umd.edu or glue.umd.edu), since your TerpConnect account will
  remain as long as you're associated with the University.

    You can also download files to your own computer; if you're using
  Windows a WinSCP client that will do this is
    \href{http://winscp.net/eng/index.php}
         {WinSCP}.

    After copying files you want to save be sure to undo the changes that were
  made in the process of setting up your account before working on Project
  \#1, since you may use the Grace systems for other courses in the future
  and the modifications made for this course may conflict with changes
  necessary for those courses:

    \vspace{-2.5mm}

    \begin{enumerate}

      \addtolength{\itemsep}{-1mm}

      \item Remove our directory from your path in your file \texttt{.path},
            by editing it and removing the line beginning
            ``\texttt{setenv PATH }\ldots'' that you added.

      \item Remove the symbolic link named \texttt{330} that you created
            from your home directory to your extra disk space for this
            course, as in \texttt{rm \string~/330}.  You could still
            reach the files in your extra disk space after that if you
            wanted to (until you lose access to them) just by using the full
            pathname, for example you could still use a command like:

            \vspace{-4.5mm}

            \begin{center}

              \texttt{cd /afs/glue/class/\semester\number\year/cmsc/330/%
                      0101/student/\emph{\textrm{loginID}}}

            \end{center}

            \vspace{-3.5mm}

            instead of \texttt{cd \string~/330}.  For convenience though
            you probably want to wait to remove the symlink after you've
            copied any desired files from your extra disk space.

      \item Also remove the symbolic link named \texttt{330public} that you
            created in your home directory, which points to the class
            public directory
            \texttt{/afs/glue/class/\semester\number\year/cmsc/330/%
                    0101/public}
            (\verb@rm ~/330public@).

    \end{enumerate}

    \vspace{-1.5mm}

      At some point after the semester ends the class webpage will become
    inaccessible, so if there's any information you want from it that you
    don't already have, be sure to save it after your finals are over.

      Lastly, after the final exam the class Piazza space will be disabled.
    If there's anything you need to discuss with the instructional staff
    past that time contact us individually.

  \section{Project description}

    \subsection{Metro simulation rules}

      We will begin by describing how the metro simulation works.  In the
    simulation there will be metro trains, each of which runs on a
    particular metro line, and passengers traveling on the metro, each with
    an itinerary.  There are a number of rules governing how trains and
    passengers may move between metro stations.

      \subsubsection{Metro lines and stations}

        The metro consists of a number of metro lines.  A metro line consists
      of a number of metro stations (stops).  The same metro station may be
      on multiple metro lines (a transfer station).

      \subsubsection{Trains}

        At the beginning of the simulation trains enter the initial stop on
      their line (e.g., Greenbelt for the green line).  Trains travel
      between the first and last station on their metro line, in order.
      Trains first move sequentially forward along the stations on their
      metro line, and when they reach the station at the end of the metro
      line that they are on they move sequentially backward along the line,
      until they reach the beginning of the metro line.  The process repeats
      indefinitely, with trains continuing to move back and forth between
      the first and last station on the line, although trains may stop once
      all passengers have reached their destinations.

        If there are no passengers in the entire simulation then each train
      must complete at least one round trip from the first to the last
      station, and back, on its metro line.

        Multiple trains may travel on a metro line.  Each train is assigned a
      number unique to that metro line, beginning with 1 and increasing by 1
      at a time.  Trains are named by the name of the metro line, followed
      by its number, separated by a space. For example, if the metro line
      ``Red'' has three trains, their names would be ``Red 1'', ``Red 2'',
      and ``Red 3''.  Not very creative naming, but it will suffice for our
      purposes.

        At most one train from each metro line may be at a station at a time
      (think of it as metro stations having only one platform per metro
      line).  If there are multiple trains on a metro line, and two trains
      from the same metro line happen to arrive at the same time, one will
      have to wait until the station is clear.  Two trains from the same
      line may not be at the same station even they are traveling in
      opposite directions.  (Of course this is not the case for the real
      Metro, but it must be true in your simulation.)  Trains from different
      metro lines may be at a station simultaneously if that station is a
      transfer station on those different lines.  If there are multiple
      trains waiting to enter a station the order they enter is
      \textbf{unspecified} (again unlike the real Metro).

        There may be multiple trains traveling between metro stations, and
      trains may pass each other between stations (think of it as stations
      being connected by multiple train tracks, for example a regular track
      and some express tracks between stations).  Therefore the order that
      two trains arrive at a station does not affect the order they arrive
      at the next station.

      \subsubsection{Passengers}

        Each passenger has a itinerary, meaning a list of the stations that
      they plan to visit.  Passengers travel on trains from station to
      station on their itinerary until they reach their final destination.
      At the start of the simulation passengers are located at the first
      station on their itinerary.  Passengers wait until a train arrives at
      their station that will also stop at the next station on their
      itinerary.  At that point they leave the station and board the train.
      A passenger may board any train traveling to the next station on their
      itinerary, unless the train is full (see below).  Passengers may board
      trains traveling in either direction, regardless of which direction
      requires fewer stops.  Similarly, if the current and destination
      stations are both on multiple metro lines, then any train on either
      line may be boarded.

        When a train carrying a passenger arrives at the next station on the
      passenger's itinerary, the passenger disembarks (leaves the train and
      enters the station).  Passengers continue traveling until they reach
      the final station on their itinerary.

        It is possible that passengers at a station might miss a train as it
      passes through.  In that case, the passengers remain in the station
      and wait for another opportunity to board a train.  Similarly, if
      passengers on a train miss their stop, they remain on the train and
      wait for another opportunity to exit at the desired station.  It may
      take a while, but the train will eventually return to that station.

        Any line transfers will be explicit in the list of stations on a
      passenger's itinerary.  For example, if there is no direct connection
      from College Park to Vienna you may assume passenger itineraries will
      not try to go directly from College Park to Vienna without using an
      intermediate station as a transfer point.

        All trains in a simulation have a limit as to the maximum number of
      passengers who may be aboard.  If a train is full, passengers may not
      board that train until other passengers leave.  The same passenger
      limit applies to all trains (unlike the real Metro, which has
      four--car trains, six--car trains, and eight--car trains, all trains
      in the simulation are the same size), but since the passenger limit is
      part of your program's input, different limits may apply during
      different executions of your program.

    \subsection{Metro simulation outputs\label{section:outputs}}

      A metro simulation may be described by a number of simulation events,
    and the order they occur.  Four simulation events and their associated
    messages are:

      \vspace{-3.5mm}

      \begin{center}

        \renewcommand{\arraystretch}{1.25}

        \begin{tabular}[t]{l}

          Train \emph{line} \emph{number} entering \emph{station--name}
            \\

          Train \emph{line} \emph{number} leaving \emph{station--name}
            \\

          \emph{passenger--name} boarding train \emph{line} \emph{number} at
          \emph{station--name}
            \\

          \emph{passenger--name} leaving train \emph{line} \emph{number} at
          \emph{station--name}
            \\

        \end{tabular}

      \end{center}

      \vspace{-2.5mm}

    \noindent
    In these messages \emph{line}, \emph{number}, \emph{station--name}, and
    \emph{passenger--name} would be replaced by the line (color), train
    number, station name, and passenger name of the event that occurred.
    The simulator must output these simulation messages in the order they
    occur.  These messages (and their order of occurrence) may then be
    analyzed and used to either display the state of the simulation, or to
    discover whether the simulation results are valid.

    \subsection{Metro simulation parameters and files}

      Each metro simulation is performed for a specific set of simulation
    parameters.  These parameters are read from a simulation file, and
    include the following.  Note that you will not have to write code to
    read these parameters; our code that is being provided to you does this.

      \vspace{-2mm}

      \begin{description}

        \addtolength{\itemsep}{-1.5mm}

        \item[Metro lines:] This is the name of a metro line followed by a
              list of stations on that line.  A simulation may have multiple
              lines.

        \item[Metro trains:] This is the name of a metro line followed by
              the number of trains running on that line.

        \item[Passenger limit:] This is the limit to the number of
              passengers that can be aboard any train at any time, during an
              execution of the program.

        \item[Passengers:] This will be a sequence of passenger names, each
              followed by the list of stations in that passenger's itinerary.

        \item[Simulation output:] A simulation file may contain an example
              output for a simulation performed with this set of simulation
              parameters.

      \end{description}

      \vspace{-2mm}

      If a simulation does not use any of these parameters they will be
    omitted from the simulation parameter file.  Explanatory comments appear
    at the beginning of simulation files, describing what they are testing.
    Here is the first part of an example simulation file:

      \vspace{-2mm}

      \begin{center}

        \begin{BVerbatim}[commandchars=\\\{\}]
        === Lines ===
        Red, Glenmont, Silver Spring, Union Station, Bethesda, Shady Grove
        === Trains ===
        Red=1
        === Passenger Limit ===
        limit=10
        === Passengers ===
        Amy, Silver Spring, Bethesda
        Ann, Glenmont, Bethesda
        Art, Union Station, Silver Spring
        Aaron, Bethesda, Glenmont
        === Output ===
        Train Red 1 entering Glenmont
        Ann boarding train Red 1 at Glenmont
        Train Red 1 leaving Glenmont
        Train Red 1 entering Silver Spring
        Amy boarding train Red 1 at Silver Spring
        Train Red 1 leaving Silver Spring
        \vdots
        \end{BVerbatim}

      \end{center}

      \vspace{-2mm}

    \subsection{Simulation driver}

      In addition to the public test simulation files in the \texttt{proj5}
    directory of the public class directory you will also find the following
    files:

      \vspace{-2.5mm}

      \begin{description}

        \addtolength{\itemsep}{-1.5mm}

          \item[\texttt{simulate.rb}:] This is a driver for the simulator.

          \item[\texttt{metro.rb}:] This is the simulator (write all your
                code in this file).

          \item[\texttt{display-example.rb}:] This is an example showing how
                to use \texttt{display\_state()} (see below).

          \item[\texttt{run-public-tests.rb}:] This is a simple Ruby script
                to run the public tests.

      \end{description}

      \vspace{-2mm}

      The simulation driver in \texttt{simulate.rb} will read in a simulation
    file containing simulation parameters and put the data in a number of
    Ruby hashes for use by your code.  For instance, the simulation
    parameters and simulation output in the simulation file shown above will
    be processed to produce the following:

      \vspace{-1mm}

      \begin{Verbatim}[gobble=6,commandchars=\\\<\>,xleftmargin=22mm]
      lines = {
        "Red" => ["Glenmont", "Silver Spring", "Union Station", "Bethesda",
                  "Shady Grove"]
      }

      numTrains = {
        "Red" => 1
      }

      passengers = {
        "Amy" => ["Silver Spring", "Bethesda"],
        "Ann" => ["Glenmont", "Bethesda"],
        "Art" => ["Union Station", "Silver Spring"],
        "Aaron" => ["Bethesda", "Glenmont"]
      }
      \end{Verbatim}

      \begin{Verbatim}[gobble=6,commandchars=\\\<\>,xleftmargin=22mm]
      sim_output = [
        "Train Red 1 entering Glenmont",
        "Ann boarding train Red 1 at Glenmont",
        "Train Red 1 leaving Glenmont",
        "Train Red 1 entering Silver Spring",
        "Amy boarding train Red 1 at Silver Spring",
        "Train Red 1 leaving Silver Spring",
        \vdots
      ]
      \end{Verbatim}

      \vspace{-2mm}

      The data structures are organized as follows:

      \vspace{-2mm}

      \begin{itemize}

        \addtolength{\itemsep}{-1.5mm}

        \item The hash for metro lines contains the metro line ``Red'', which
              begins at the station Glenmont and ends at the station Shady
              Grove.

        \item The hash for the number of trains contains an integer value for
              each metro line, specifying the number of trains on that line.

        \item The hash for passengers contains a list of stations names
              (forming the itinerary) for each passenger. The first station
              is the passenger starting point, the last station is the final
              passenger destination.

        \item Any simulation output will be put into an array of strings,
              with each string corresponding to one simulation event
              message.

      \end{itemize}

      \vspace{-1.5mm}

      \texttt{simulate.rb} will also take a command--line parameter specifying
    whether the program should perform a simulation or simply display or
    verify the feasibility of the simulation output.  It can be invoked as:

      \vspace{-2.5mm}

      \begin{center}

        \texttt{simulate.rb}
        (\texttt{simulate}|\texttt{display}|\texttt{verify})
        \emph{simulate--filename}

      \end{center}

      \vspace{-1.5mm}

    \noindent
    (I.e., one of the three options \texttt{simulate}, \texttt{display}, or
    \texttt{verify} should appear following the program name and before the
    name of the simulation input file.)  So running the command
    \texttt{simulate.rb simulate public01.input} would execute a simulation
    (calling your simulation code) using the simulation parameters that are
    in the file \texttt{public01.input} (ignoring the simulation output in
    \texttt{public01.input}), while running \texttt{simulate.rb verify
    public01.input} would perform an analysis of the simulation output in
    \texttt{public01.input} to determine whether it is feasible (again
    calling your code).

      Note that \texttt{simulate.rb} outputs simulation parameters before
    simulation messages, so that its output if directed to a file may be
    passed directly to the simulation display/verify routines.

      The simulation files include the simulation parameters and one possible
    output, but some different possible outputs are also provided for the
    tests that perform simulation (see below).

  \section{Project implementation}

    You must implement three major methods: \texttt{display()},
  \texttt{verify()}, and \texttt{simulate()}.  The three parts may be
  implemented independently, though \texttt{display()} and \texttt{verify()}
  are similar.

    \subsection{Part 1: Simulation display}

      A multithreaded simulation can clearly have many different behaviors,
    depending on the thread scheduler.  One way to help determine whether a
    simulation is proceeding correctly (i.e., avoiding race conditions) is
    to model the state of the simulation by processing the simulation
    outputs.  The model can then be used to display the state of the
    simulation, or determine its validity.

      The first part of your project is to implement a model of the simulation
    (by processing simulation event messages) sufficiently detailed to
    display the following:

      \enlargethispage{3mm}

      \vspace{-2.5mm}

      \begin{itemize}

        \addtolength{\itemsep}{-2mm}

        \item the trains at each station,

        \item the passengers at each station, and

        \item the passengers on board each train.

      \end{itemize}

      \vspace{-2.5mm}

      Your code should display the initial state of the simulation.  Then it
    should list each simulation event message in order, followed by a
    display of the state of the simulation after each event.  For instance,
    for the simulation parameters shown above, your code should display the
    initial state of the simulation as follows:

      \vspace{-2mm}

      \begin{Verbatim}[gobble=6,xleftmargin=40mm]
      Red
                         Glenmont       Ann
                    Silver Spring       Amy
                    Union Station       Art
                         Bethesda     Aaron
                      Shady Grove
      \end{Verbatim}

      \vspace{-2mm}

      Then your code should process the simulation event messages in the
    simulation output, displaying the message and the resulting state.  For
    instance, after processing the message \texttt{Amy boarding train Red 1
    at Silver Spring} in the simulation output, your model should contain
    enough information to display the following:

      \vspace{-4mm}

      \begin{Verbatim}[gobble=6,xleftmargin=40mm]
      Amy boarding train Red 1 at Silver Spring
      Red
                         Glenmont
                    Silver Spring            [Red 1 Amy Ann]
                    Union Station       Art
                         Bethesda     Aaron
                      Shady Grove
      \end{Verbatim}

      \vspace{-2mm}

      For the simulation display part of the project you may assume that the
    sample simulation output is valid.  \texttt{simulate.rb} will invoke
    your verifier for this part of the project as:

      \vspace{-2mm}

      \begin{center}

        \texttt{display(lines, passengers, sim\_out)}

      \end{center}

      \vspace{-1.5mm}

    \noindent
    where \texttt{lines}, \texttt{passengers}, and \texttt{sim\_out} are the
    hashes and array produced in \texttt{simulate.rb} from the simulation
    file.  The submit server will be doing the same thing when your code is
    run there.

      A function \texttt{display\_state(lines, stations, trains)} is provided
    for actually printing out the state of the simulation.  If you want to
    use it you will simply need to collect enough information in your model
    to provide two hashes \texttt{stations} and \texttt{trains}, which
    should contain information on trains and passengers at each station, and
    passengers on each train, as follows:

      \vspace{-2.5mm}

      \begin{center}

        \begin{tabular}[t]{|l|l|}

          \hline

          \texttt{stations[station name]}
            & hash for metro line
            \\ \hline

          \texttt{stations[station name][line name]}
            & hash for trains at station (for line)
            \\ \hline

          \texttt{stations[station name][line name][train num]}
            & where train num is "1", "2", etc\ldots
            \\ \hline

          \texttt{stations[station name]["passenger"]}
            & hash for passengers at station
            \\ \hline

          \texttt{stations[station name]["passenger"][passenger name]}
            & exists when passenger is at the station
            \\ \hline

          \texttt{trains[train name]}
            & hash for passengers on train
            \\ \hline

          \texttt{trains[train name][passenger name]}
            & exists when passenger is on the train
            \\ \hline

        \end{tabular}

      \end{center}

      \medskip

      The double--quoted string \texttt{"passenger"} above means that
    literal string is used as a hash key.

      See the example \texttt{display-example.rb} in the project starter files
    for more detailed illustration of how to use \texttt{display\_state()}.

      Notice that this part of the project does not actually use concurrency
    or threads at all.

    \subsection{Part 2: Metro simulation}

      For part 2 (possibly reusing your data structures from part 1) you will
    write a Ruby program that actually performs a multithreaded simulation
    using the simulation parameters supplied. Your simulation should be
    implemented as follows:

      \vspace{-2mm}

      \begin{itemize}

        \addtolength{\itemsep}{-.75mm}

        \item Each train and passenger in the simulation should be
              represented by its own thread, so if you are simulating
              \emph{m} trains and \emph{n} people, there should be $m + n +
              1$ threads in the system (\emph{m} train threads, \emph{n}
              passenger threads, and one thread for the main program.

        \item The initial state of the simulation should be as described in
              the metro simulation rules (i.e., all passengers are at the
              first station in their itinerary, and all trains poised to
              enter the first station on their metro line).

        \item Trains must never carry more passengers at any time than the
              value read from the simulation parameter file.

        \item You \textbf{must} use synchronization (i.e., Ruby monitors) to
              avoid race conditions and ensure your simulation is valid.
              You should use the monitor associated with each metro line to
              prevent all race conditions for that metro line.  So a
              simulation with \emph{n} metro lines could have up to \emph{n}
              train or passenger threads executing concurrently.

        \item You must use condition variables to ensure your simulation
              uses synchronization efficiently.  Use at least two condition
              variables per monitor, one for trains and one for
              passengers.  Trains should wake up passengers when entering
              stations, and wake up other trains when exiting stations.

        \item Use enough locks to permit concurrent execution and avoid race
              conditions, but not so many locks that concurrency is reduced.

        \item Use Ruby's \texttt{wait} condition methods and
              \texttt{broadcast} to avoid busy waiting.

        \item Each train should sleep for 0.01 seconds after entering a
              station before exiting.

        \item Your threads \textbf{may not use busy--waiting} (for example
              trains waiting for passengers or other trains, or passengers
              waiting for trains); you may lose substantial credit on the
              project if your threads do this.  Other than trains sleeping
              for exactly 0.01 seconds after entering a station before
              exiting, your threads also may \textbf{not} just sleep while
              waiting (for passengers or trains).  Instead use the
              \texttt{wait} condition methods as mentioned to wait until the
              desired event occurs, at which time the waiting thread should
              be woken up by a \texttt{broadcast} from another thread.

        \item The TAs will be looking at your submitted code to check that
              you are using synchronization correctly, and have not used
              disallowed operations as described above.

        \item Notice that the list of stations in the itinerary for each
              passenger does not tell you what metro line they want to
              take.  You'll need to compute that based on the current station
              and next destination.  If there is more than one valid metro
              line, the passenger can board a train from any valid line.

        \item The simulation ends when all people have arrived at their
              final stops.  To achieve this, in the main thread you'll
              probably need to do a join on all the threads representing the
              people.  Notice that it is legal if trains continue running
              for a while even after all passengers have arrived.

        \item If there are no passengers, each train must complete at least
              one round trip from the first to the last station (and back)
              on its metro line before the simulation ends.

        \item You should set \texttt{Thread.abort\_on\_exception = true} in
              your code, to avoid hiding errors in threads.

        \item In order to see what's going on during your simulation, your
              program must print out various lines of text as simulation
              events occur.  These lines of text are described in Section
              \ref{section:outputs} above.  For the output to make sense
              you must do the following:

              \vspace{-2.5mm}

              \begin{itemize}

                \addtolength{\itemsep}{0mm}

                \item Only print out a message while you are holding a lock.

                \item Immediately after printing, and before you release the
                      lock, call \texttt{\$stdout.flush()} to flush standard
                      output to the screen.

              \end{itemize}

              \vspace{-2.5mm}

              Following the two rules above should ensure that if you build
              the simulation correctly, your simulation output will be
              valid. Otherwise, you might get strange interleavings of
              output messages that look incorrect even if your simulation
              code is actually correct.

      \end{itemize}

      \vspace{-1.5mm}

      \texttt{simulate.rb} (and the submit server) will invoke your code for
      this part of the project as:

      \vspace{-2.5mm}

      \begin{center}

        \texttt{simulate(lines, num\_trains, passengers, sim\_monitor, limit)}

      \end{center}

      \vspace{-2mm}

      \noindent
      where \texttt{lines}, \texttt{num\_trains}, and \texttt{passengers}
      are the hashes produced in \texttt{simulate.rb} from the simulation
      file.  \texttt{sim\_monitor} is a hash containing a monitor variable
      for each metro line. You must use these monitors (and condition
      variables derived from them) for synchronization in your code.

    \subsection{Optional simulation verifier}

      It should be clear that a multithreaded simulation may have many
    different behaviors, depending on the thread scheduler.  However, there
    are certain restrictions on the simulation output, for example, people
    can board trains only when those trains are at the station where those
    people are.  If you want to you may write a Ruby method that examines
    your simulation outputs and checks whether they are valid (i.e., follow
    all the metro simulation rules in the project description).

      The list of possible errors in simulation output is nearly endless, so
    you can only check some common errors associated with race conditions
    resulting from incorrect synchronization.  Some conditions you can look
    for in the simulation are:

      \vspace{-2mm}

      \begin{itemize}

        \addtolength{\itemsep}{-1.5mm}

        \item Trains start by entering the initial station on their line.

        \item Trains move forward and backward along the stations on their
              line, and visit all stations without skipping any.

        \item Trains enter a station before they leave it.

        \item Two trains from the same metro line are not at the same
              station at the same time.

        \item Each passenger follows their itinerary.

        \item Passengers only board and leave a train while it is at a
              station (i.e., after that train has entered the station, but
              before it has left the station).

        \item All passengers reach their final destination by the end of the
              simulation.

        \item If there are no passengers in the simulation, each train must
              complete at least one round trip from the first to the last
              station (and back) on its line.

        \item Trains never have more than the maximum number of passengers
              on board.

      \end{itemize}

      \vspace{-1mm}

      We are providing some test cases for your verifier if you want to test
    it, but they are not set up on the submit server because they are not
    directly worth any points (of course writing the verifier may help you
    get a higher score on the tests that are worth points, by finding
    problems in your code).  \texttt{simulate.rb} will invoke your verifier
    as:

      \vspace{-4.5mm}

      \begin{center}

        \texttt{result= verify(lines, num\_rains, passengers, sim\_out, limit)}

      \end{center}

      \vspace{-1.5mm}

    \noindent
    where \texttt{lines}, \texttt{numTrains}, \texttt{passengers},
    \texttt{sim\_out} are the hashes and array produced in
    \texttt{simulate.rb} from the simulation file.  The function verify
    should return true if the simulation is valid, and false otherwise.

      Notice that this part of the project does not use concurrency or threads
    either.

    \enlargethispage{3mm}

  \section{Project requirements and submitting your
           project\label{section:requirements}}

    \begin{enumerate}

      \addtolength{\itemsep}{-1mm}

      \item Your methods will be called by \texttt{simulate.rb} so they will
            not read any input.  You may call our method
            \texttt{display\_state()} (or print output directly if you
            prefer to do that for some reason), but all output should be
            written to the program's standard output.

      \item You may use any Ruby language features in your project that you
            would like, regardless of whether they were covered in class or
            not, so long as your program works successfully using the
            version of Ruby installed on the OIT Grace Cluster and on the
            CMSC project submission server, and as long as you use threads,
            and synchronization, as described above.

      \item Your submitted methods \textbf{must} be the file named
            \texttt{metro.rb}, otherwise our \texttt{simulate.rb}
            running on the submit server will not find them, consequently
            your program's score will be zero.  Although Ruby programs can
            be broken up into multiple source files (not shown in class),
            all your code should be in this one source file.  Your program
            would work otherwise, but this is just to make it easy for the
            TAs to check your code.  They will only be looking at the code
            in your \texttt{metro.rb} source file, so if you have code
            that's not in that file you will lose credit because it will not
            be seen.

      \item To check that your output matches the expected output use the
            verify option of \texttt{simulate.rb} as described above.  It
            will print ``\texttt{VALID.}'' if your \texttt{verify()} method
            says that the simulation is valid, and ``\texttt{INVALID.}''
            otherwise.

      \item Your \texttt{metro.rb} source file \textbf{must have} a comment
            near the top that contains your name, TerpConnect login ID
            (which is your directory ID), your university ID number, and
            your section number.

            The Campus Senate has adopted a policy asking students to
            include the following statement on each major assignment in
            every course: ``I pledge on my honor that I have not given or
            received any unauthorized assistance on this assignment.''
            Consequently you're requested to include this pledge in a
            comment near the top of your program source file.  See the next
            section for important information regarding academic integrity.

      \item Note that although you are not being graded on your source code
            or style (see the separate project grading handout), if the TAs
            cannot read or understand your code they cannot help should you
            have to come to office hours, until you return with a
            well--written and clear program.

      \item As before, to submit your program just type the single command
            \texttt{submit} from the \texttt{proj5} directory, where your
            source file should be.
            % You may lose credit for having an excessive number of
            % submissions.

    \end{enumerate}

    \vspace{-4.5mm}

  \section{Academic integrity}

    Please \textbf{carefully read} the academic honesty section of the course
  syllabus.  \textbf{Any evidence} of impermissible cooperation on projects,
  use of disallowed materials or resources, or unauthorized use of computer
  accounts, \textbf{will be submitted} to the Student Honor Council, which
  could result in an XF for the course, or suspension or expulsion from the
  University.  Be sure you understand what you are and what you are not
  permitted to do in regards to academic integrity when it comes to project
  assignments.  These policies apply to all students, and the Student Honor
  Council does not consider lack of knowledge of the policies to be a
  defense for violating them.  Full information is found in the course
  syllabus-- please review it at this time.

\end{document}
