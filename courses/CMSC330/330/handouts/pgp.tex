\documentclass[10pt]{article}

  %
  % Have the students source an aliases file in the public class directory
  % as part of the account setup, so we can create aliases like "ledit
  % ocaml" for "ocaml" for them, since this doesn't work as a shell script.
  %

  %
  % Should have said not only that anonymous messages in Piazza may be
  % deleted, but also that we won't reply to them.
  %

  %
  % In the account setup, should have told students to undo any and all
  % changes made to their files (like .cshrc.mine) in 216.  Leaving "limit
  % maxproc 20" ended up being a problem in the last project (using
  % threads).
  %

  \usepackage{330-f12}

  \usepackage[none,light,outline,timestamp]{draftcopy}

  \newcommand{\firstsection}{0101}
  \newcommand{\class}{330}

\begin{document}

  \header{\course}{Project submission and grading policies}{\Term}

  \vspace{-5mm}

  \section{Grace account setup}

    Programming will be done on the OIT Grace Cluster, which may be reached at
  grace.umd.edu.  You will use your own TerpConnect account to access the
  Grace cluster; your login ID and password are just your University
  directory ID and password.  Below are steps to set up your account for
  this course, including procedures to check that things were done
  correctly.

    Every student gets extra disk space during each semester that they are
  taking class using the Grace systems.  It's \textbf{extremely important}
  to do all your programming work for this course in this extra disk space,
  rather than your home directory, for two reasons:

    \vspace{-1.5mm}

    \begin{itemize}

      \addtolength{\itemsep}{-1mm}

      \item You won't need to worry about running out of disk space in your
            home directory, even if you accidentally create some very large
            files, and

      \item The extra disk space is automatically backed up every night, so
            if you were to accidentally erase an important file, the
            previous day's version can be recovered.

    \end{itemize}

    \vspace{-1.5mm}

    The extra disk space is located elsewhere on the system (not in your home
  directory), but it's easy to create a symbolic link pointing to it, as
  explained below.  (Note that the course extra disk space from CMSC 216
  should \textbf{not} be used, because even if it is currently accessible it
  will become inaccessible during the semester, and you will not be able to
  use any files there after that time.)

   As mentioned in the syllabus, although programming may be done on other
  systems that you have access to, it is recommended that you do all your
  coursework on the Grace cluster.  Note that the Eclipse IDE can also be
  used to do programming, but you are encouraged to use the Grace systems
  (or at least some other UNIX/Linux system).  Experience with more than one
  operating system, not just an IDE, is invaluable for any computer
  scientist or engineer.  Some practice in the UNIX development environment
  can greatly increase your productivity in it.  If you do use any other
  system for programming, keep in mind that projects must work correctly on
  the Grace machines (and the CMSC project submission server), and no
  consideration in grading can be made for errors made in transferring
  files, or submitting the wrong version of a project.

    In a couple of steps below you have to add a line to one of your account
  dotfiles (or account control files); if you add a line at the bottom, be
  sure it ends with a newline, or it won't be recognized by the shell, so
  just press enter at the end of typing the last line of any file.

    \vspace{-1.75mm}

    \begin{enumerate}

      \addtolength{\itemsep}{.75mm}

      \item Log into grace.umd.edu, using your UMCP directory ID and
            password, and use a text editor to edit the file \texttt{.path}
            located in your home directory.  (Emacs was covered in CMSC 216;
            other text editors may be used, but \texttt{pico} is not
            recommended as it is too simple to be used for programming.)
            Add the following line to the bottom:

            \smallskip

            \begin{centering}

              \texttt{setenv PATH "/afs/glue/class/\term/cmsc/\class/%
                      \firstsection/public/bin:\${\textbraceleft}PATH%
                      {\textbraceright}"}

            \end{centering}

            \medskip

            Be sure that \verb@${PATH}@ appears after the other, long
            directory name as shown, and that the curly braces are present.
            Check this using the following commands:

            \begin{centering}

              \begin{BVerbatim}
        source ~/.path
        echo $PATH
              \end{BVerbatim}

            \end{centering}

            \medskip

            You should see \texttt{/afs/glue/class/\term/cmsc/\class/%
            \firstsection/public/bin} as the first component of your path,
            which is where the project submission program is, and other
            programs as needed will be placed.

            Note: if you already have a line like the above at the end of
            your \texttt{.path} file from CMSC 216 just remove it, as that
            directory is no longer accessible, or will become inaccessible
            during the semester.

      \item Project test inputs and outputs, discussion section example
            code, and other files, will be provided in the public class
            directory, which is:

            \begin{centering}

              \texttt{/afs/glue/class/\term/cmsc/\class/\firstsection/public}

            \end{centering}

            \medskip

            However, since using this path is a lot of typing, just create a
            symbolic link to it from your home directory instead.  To do so,
            cd to your home directory, if you're not there already, and use
            the command:

            \smallskip

            \begin{centering}

              \texttt{ln -s /afs/glue/class/\term/cmsc/\class/\firstsection/%
                      public 330public}

            \end{centering}

            \medskip

            This will create a symbolic link named ``\texttt{330public}'' to
            the public class directory.  Check this using the commands:

            \begin{centering}

              \begin{BVerbatim}
        cd 330public/..
        pwd
              \end{BVerbatim}

            \end{centering}

            \smallskip

            This should print the path
            \texttt{/afs/glue/class/\term/cmsc/\class/\firstsection}.

            Note: if you already have a symlink \texttt{216public} from CMSC
            216 in your home directory just remove it, as that directory is
            either no longer accessible, or will become inaccessible during
            the semester.

      \item Similarly, to create a symbolic link to your extra course disk
            space, cd to your home directory if you're not already there,
            and give the command:

            \smallskip

            \begin{centering}

              \texttt{ln -s /afs/glue/class/\term/cmsc/\class/%
                      \emph{\textrm{section}}/student/%
                      \emph{\textrm{loginID}} 330}

            \end{centering}

            \medskip

            where \emph{section} and \emph{loginID} are to be replaced by
            your own discussion section and TerpConnect login ID.  Check
            this using the commands:

            \begin{centering}

              \begin{BVerbatim}
        cd 330/..
        pwd
              \end{BVerbatim}

            \end{centering}

            \medskip

            This should print the path
            \texttt{/afs/glue/class/\term/cmsc/\class/\firstsection/%
            student-cmsc\class-\emph{\textrm{section}}}, where your own
            section number will appear in place of
            \emph{section}.

            From now on, anytime you log in to do any type of coursework for
            this class, be sure to first just ``\texttt{cd 330}'', to change
            to your extra course disk space, so that all your programming is
            done \textbf{in this extra disk space}.

            Note: if you already have a symlink \texttt{216} from CMSC 216
            in your home directory just remove it, as the extra course disk
            space from prior semesters is either no longer accessible, or
            will become inaccessible during the semester.

      \item It's necessary to be able to use Oracle Java, which is not the
            default version on the Grace systems, but this can easily be
            changed.  Add a line that just reads ``\texttt{tap -q java}''
            (without the quotes) to your file \texttt{.cshrc.mine}: Check
            this using the commands:

            \begin{centering}

              \begin{BVerbatim}
        source ~/.cshrc.mine
        rehash
        which java
              \end{BVerbatim}

            \end{centering}

            \medskip

            This should print
            \texttt{/afs/glue.umd.edu/software/java/current/sys/bin/java},
            not \texttt{/usr/bin/java}.

    \end{enumerate}

    \vspace{-1.75mm}

  \section{Project submission, deadlines, and grading policies}

    Projects will be submitted electronically to the CMSC project submission
  server, and directions will be provided with project assignments.
  Projects submitted via any other means (such as an emailed project) cannot
  be considered.  \textbf{Do not} submit projects using the submit server's
  mechanism for uploading a jarfile or zipfile.  Use the procedures to be
  described with assignments.  Projects may fail on the submit server if a
  zipfile or jarfile is submitted, and allowances cannot be made in grading
  for problems with submissions made in this manner.  Only the projects
  electronically submitted according to the procedures provided can be
  graded; it is each student's responsibility to test their program and
  \textbf{verify that it works properly} before submitting, as well as to
  \textbf{log into the submit server} and verify that their submission
  worked correctly there.

    All projects will be due at 10:00:00 p.m.\ on the day indicated on the
  project assignment.  Projects may be submitted up to two days late, with a
  15--point late penalty for each late day.  If you submit more than once,
  you will receive the highest score of your submissions, meaning the
  highest score after the late penalty is taken into account for any late
  submissions.  Submission deadlines are \textbf{\underline{firm}} and
  exceptions cannot be made.  Note there is \underline{\textbf{no grace
  period}} for project submissions-- deadlines are enforced by the
  submit server at exactly 10:00:00 p.m., to the second, the day a project
  is due, and every 24 hours later for one-day--late and two--day--late
  submissions.  Note that project scores as they appear on the project
  submission server will not correctly include late penalties in all cases,
  however we will incorporate these into project grades when they are
  entered electronically.

    Project extensions cannot be given to individual students as a result of
  system problems, network problems, power outages, etc., so do not leave
  finishing and submitting a project until the last minute.  It is strongly
  suggested you finish and submit your program at least one day early, to
  allow time to reread the project assignment and all relevant articles in
  the course Piazza space (see below), to insure you have not missed
  anything which could cause you to lose credit on the project.

    Projects will not be graded on style or documentation, although a
  certain design or structure, or certain language features, may be
  required for some projects.  

    Some public test inputs and outputs will be provided for projects, but
  note that unlike lower--level programming classes, we will not be
  providing extensive automatic testing before projects are due, and
  projects will be graded largely based on test cases \textbf{not provided
  in advance}.  Instead you will be responsible for developing your own
  techniques for testing your own projects, and for checking the correctness of
  your output yourself.  Similarly, projects will not have any release tests.

    Note that any hardcoding in a project assignment will result in a score of
  zero for that project.  Hardcoding refers to attempting to make a program
  appear as if it works correctly and actually calculates and computes
  correct results, when for some reason it actually does not do so.  A few
  examples would include a program which prints the desired output instead
  of computing it, or a program that works only because it takes advantage
  of properties that the provided test inputs happen to have, etc.
  Hardcoding may be considered to be a violation of the University's
  academic integrity policies, as it may be construed to be fabrication or
  falsification of information, so if you have any question about whether a
  particular situation would constitute hardcoding be sure to ask ahead of
  time.

  \section{General project advice}

    \vspace{-.5mm}

    \begin{itemize}

      \addtolength{\itemsep}{-.25mm}

      \item Be sure to carefully read Section 7 of the course syllabus,
            regarding academic integrity.

      \item Be sure to save your work in the text editor often, regardless of
            what computing platform you are using.  A power failure, network
            failure, or dropped connection could otherwise cost many hours of
            lost work.

      \item Be sure to keep backup copies of your project and any other
            important files (like your own test data) in a different
            directory or a subdirectory of the directory where your project
            is located.  It's a good idea to save backup copies every time
            you log in or log out, for example.

            Don't use the submit server to keep backup copies of your
            program, as having a large number of unnecessary submissions
            just slows things down on the submit server for everyone.

      \item Don't just rely on the submit server to check whether your
            program works on any public test inputs, meaning before
            submitting you should run it yourself on any public test inputs,
            and compare its results yourself.

    \end{itemize}

    \vspace{-2.5mm}

  \section{Piazza}

    Everyone registered should have received information via email by now
  about joining the course Piazza space.  This may be very useful for asking
  and answering questions about projects and other coursework.  You should
  check it often, once programming assignments have been made, to see if any
  messages there are relevant to the current project.  Please read and keep
  the following in mind these rules regarding Piazza:

    \vspace{-1mm}

    \begin{itemize}

      \addtolength{\itemsep}{-.25mm}

      \item You may post or reply to questions about project requirements,
            language features, general syntax errors, or course material via
            Piazza.  However, you \underline{\textbf{may not}} ask or
            explain how to implement any part of a project in a public venue
            (or privately, for that matter).  Such questions can be asked in
            person during office hours.  Asking or explaining how to
            implement any part of a project via Piazza would be a violation
            of this course's academic integrity requirements.

      \item Anonymous messages in Piazza may be deleted.  Let your fellow
            students know who is asking and answering questions.

      \item Piazza should be used for communication regarding class
            material.  As it will be read by all the students in the course
            (in addition to the instructional staff) please do not make
            everyone read through unnecessary messages to find information
            they need for their coursework.  In particular, don't post any
            messages which have no relevance to the course material.  There
            are many other venues in which matters unrelated to the course
            (or not of interest to students in the course) may be discussed.

      \item It is not uncommon that questions are asked in a venue like
            Piazza that are actually specifically addressed in a project
            assignment; although we don't want to discourage questions, to
            reduce the number of messages that everyone must read we ask
            that you first check a project assignment before posting a
            question about it Similarly, questions may be asked that have
            already been answered, so first read any pending unread messages
            prior to posting questions in Piazza.

      \item Piazza cannot be used to ask questions about anything particular
            to one student.  If you have any questions or concerns or
            concerns about the course that are not related to the content of
            the course material you are encouraged to discuss them with your
            instructor in person, during office hours, or before or after
            class.

    \end{itemize}

    \vspace{-2.5mm}

\end{document}
