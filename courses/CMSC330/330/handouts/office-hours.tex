\documentclass[11pt]{article}

  %
  % State explicitly next time that students don't need to have a laptop to
  % come to office hours.
  %

  %
  % Yeah, the chart is really ugly Latex, but since it's only used once a
  % semester there's little motivation to spend time trying to rewrite the
  % macros....
  %

  \usepackage{330-f12}

  \usepackage[none,light,outline,timestamp]{draftcopy}

  \usepackage{pstricks}

  \usepackage{multirow}

  % For some reason \definecolor is defined on home Linux machine, but not
  % department Linux machine, without this package.
  %
  \usepackage{color}

  \usepackage{calc}

  \newrgbcolor{lecture}   {0.549 0.549 0.549}   % gray55
  % \newrgbcolor{discussion}{0.000 0.900 1.000} % slightly ligher than
                                                % deep-sky-blue

  % \newrgbcolor{larry}     {0.757 1.000 0.757} % DarkSeaGreen1
  %
  % \newrgbcolor{derek}     {1.000 0.000 0.000} % red
  % \newrgbcolor{christine} {0.706 0.322 0.804} % MediumOrchid3
  % \newrgbcolor{bryan}     {1.000 0.843 0.000} % gold

  \newrgbcolor{larry}     {0.000 0.933 0.000}   % green2
  \newrgbcolor{nick}      {1.000 0.647 0.000}   % orange

  \newrgbcolor{varun}     {0.545 0.353 0.000}   % orange4
  \newrgbcolor{tommy}     {0.706 0.322 0.804}   % MediumOrchid3
  \newrgbcolor{derek}     {1.000 0.000 0.000}   % red
  \newrgbcolor{hao}       {1.000 0.843 0.000}   % gold
  \newrgbcolor{tammy}     {0.000 0.749 1.000}   % deep-sky-blue
  % \newrgbcolor{angjoo}  {1.000 0.647 0.000}   % orange
  % \newrgbcolor{varun}   {1.000 0.961 0.933}   % seashell1
  % \newrgbcolor{hyunjong}{0.000 0.804 0.000}   % green3
  % \newrgbcolor{zhengzheng}{1.000 0.753 0.796} % pink
  % \newrgbcolor{anirudh} {0.933 0.478 0.914}   % orchid2
  % \newrgbcolor{alex}    {0.749 0.937 1.000}   % LightBlue1

  \newcommand{\boxsize}{14mm}
  \newcommand{\breakname}[1]{\parbox[t]{\boxsize}{%
                             \addtolength{\baselineskip}{-.85mm}%
                             \raggedright#1}}
  \newcommand{\class}[2]{\begin{minipage}[t]{\boxsize}#1\\#2\end{minipage}}
  \newcommand{\classcenter}[2]{\begin{minipage}[t]{\boxsize}%
    \centering#1\\%
    \centering#2%
  \end{minipage}}

  %
  % parameters are:
  %  1: x location
  %  2: y location
  %  3: name
  %  4: height (40 for one hour, 60 for 1.5 hours, 80 for 2 hours, etc.)
  %  5: extra x distance for name (negative to move left, positive for right)
  %  6: extra y distance for name (negative to move up, positive for down)
  %  7: name of a command to set the color of the text or line which follows
  %  8: extra amount to add/subtract to width
  %  9: extra amount to add/subtract to height
  \newcounter{onetemporary}
  \newcounter{twotemporary}
  \newcommand{\hour}[9]{%
    \setcounter{onetemporary}{#1+\value{cellwidth}+#8}
    \setcounter{twotemporary}{#2-#4+#9}
    \psframe[fillcolor=#7,fillstyle=solid]
             (#1,#2)(\value{onetemporary},\value{twotemporary})
    \setcounter{onetemporary}{#1+#5+\value{cellheight}+(#8/2)}%
    \setcounter{twotemporary}{#2-#4+#6}%
    {\put(\value{onetemporary},\value{twotemporary})%
      {\makebox(0,#4)[c]{\textbf{#3}}}}%
  }%

\begin{document}

  \header{\course}{Office hours}{\Term}

  \vspace*{-6mm}

  \section{Email contact\label{section:email}}

    Unfortunately we're not able to explain most course material via email,
  due to time constraints and other factors.  Answering such questions is
  more appropriate for class discussion or personal communication; questions
  can sometimes also be asked via Piazza.  Please discuss administrative
  issues as well as course material in person when possible (office hours,
  and before and after class are good times), and use email in case of
  urgent or emergency matters only.  Lastly, it is not practical to provide
  detailed information or assistance regarding programming assignments via
  email, and attempting to do so often results in students receiving
  incomplete or inadequate information.

  \section{Office hours}

    \subsection{Instructors' office hours}

      \begin{center}

        \begin{tabular}[t]{c@{\hspace{8mm}}c}

          \renewcommand{\arraystretch}{1.2}

          \begin{tabular}[t]{|l|l|}

            \multicolumn{2}{c}{Sections 0101, 0102, 0103}
            \\ \hline

            Name:
              & Larry Herman
              \\ \hline

            Office:
              & 1111 A. V. Williams
              \\ \hline

            Phone:
              & (301) 405--2762
              \\ \hline

            Email:
              & larry@cs.umd.edu ($*$)
              \\ \hline

            Office hours:
              & M 5:00--6:00, W \& Th 10:00-11:00
              \\ \hline

          \end{tabular}

          &

          \begin{tabular}[t]{|l|l|}

            \multicolumn{2}{c}{Sections 0201, 0202, 0203, 0204}
            \\ \hline

            Name:
              & Dr.\ Nick Feamster
              \\ \hline

            Office:
              & 4135 A. V. Williams
              \\ \hline

            Phone:
              & (301) 405--8010
              \\ \hline

            Email:
              & feamster@cs.umd.edu ($*$)
              \\ \hline

            Office hours:
              & Tu 4:45--6:00
              \\ \hline

          \end{tabular}

        \end{tabular}

      \end{center}

      \smallskip

      ($*$) See Section \ref{section:email} above regarding email.

    \subsection{Teaching assistants' office hours}

      \begin{center}

        \renewcommand{\arraystretch}{1.25}

        \begin{tabular}{@{}*{4}{|l}|@{}}

          \multicolumn{1}{c}{name}
            & \multicolumn{1}{c}{duties}
            & \multicolumn{1}{c}{email}
            & \multicolumn{1}{c}{office hours}
            \\ \hline

          Tommy Pensyl
            & teaching, 0101 \& 0102
            & tpensyl@cs.umd.edu ($*$)
            & M 2:30--4:30, W 3--5
            \\ \hline

          Tammy Tran
            & teaching, 0103 \& 0201
            & tnt@cs.umd.edu ($*$)
            & Tu 2:30--4:30, Th 12--2
            \\ \hline

          Derek Juba
            & teaching, 0202 \& 0203
            & juba@cs.umd.edu ($*$)
            & Tu 12--2, F 3--5
            \\ \hline

          Varun Nagaraja
            & teaching, 0204, and grading
            & varun@cs.umd.edu ($*$)
            & M 11-1, W 1--3
            \\ \hline

          Hao Li
            & grading
            & haoli@cs.umd.edu ($*$)
            & Th 3:15--5, F 10:45--1
            \\ \hline

        \end{tabular}

      \end{center}

      % All of the TAs' email addresses are ``@cs.umd.edu''.  Also
      ($*$) Please see Section \ref{section:email} above regarding email.

  \section{Regarding office hours}

    All the TAs' office hours will be held in 1112 A.V.\ Williams.

    The TAs' office hours end at the times indicated.  Since the TAs may have
  class or other obligations immediately following their office hours, there
  may be times when they have to leave five minutes before the end of their
  scheduled hours.  The TAs are not required to remain after their hours are
  over.  To receive assistance you must arrive early enough that you can be
  helped, taking this into consideration and keeping in mind that there may
  be others already waiting when you arrive.

    There may be times when the TAs need to swap office hours due to exams
  or other obligations, so at times you may find a different TA in the
  office hours room than the schedule here indicates.

    It's understood that students may occasionally miss class for various
  reasons, but email and office hours are not intended as a replacement for
  class attendance.

    While the TAs can provide assistance with programming assignments during
  office hours, you are responsible for developing and debugging your own
  programs.  You should therefore not rely on the instructional staff for
  getting a project to work.  Lower--level CMSC courses provide extensive
  debugging and development help in office hours, but upper--level CMSC
  courses expect students to complete projects with minimal extra help.
  Therefore in CMSC 330 we will provide less debugging help than some
  students may be used to.  If you come in with a question you should expect
  to be pointed in the right direction, but then it will be up to you to
  finish solving the problem on your own.

  \section{Office hours chart}

    \medskip

    \begin{center}

      \hspace{7mm}%

      \newcounter{ctr}

      % \psset{unit=.2525mm,linewidth=.6pt}
      \psset{unit=.27mm,linewidth=.6pt}

      \begin{pspicture}(625,585)(0,0)

        \thicklines  % write grid in heavy lines

        \newcounter{top}
        \newcounter{horiz}
        \newcounter{vert}
        \newcounter{firsthour}
        \newcounter{numhoursbeforenoon}
        \newcounter{numhoursafternoon}
        \newcounter{totnumhours}
        \newcounter{temp}
        \newcounter{temptwo}
        \newcounter{tempthree}
        \newcounter{cellwidth}
        \newcounter{cellheight}
        \newcounter{halfcellwidth}
        \newcounter{halfcellheight}
        \newcounter{thirdcellwidth}
        \newcounter{thirdcellheight}
        \newcounter{quartercellheight}
        \newcounter{hourcellwidth}
        \newcounter{daynamecellheight}

        \newcounter{mon}
        \newcounter{tue}
        \newcounter{wed}
        \newcounter{thu}
        \newcounter{fri}

        \setcounter{numhoursbeforenoon}{3}  % really, before 1:00
        \setcounter{numhoursafternoon}{5}
        % totnumhours needs to be one more than necessary
        \setcounter{totnumhours}{\value{numhoursbeforenoon} +
                                 \value{numhoursafternoon} + 1}
        \setcounter{cellwidth}{120}
        \setcounter{cellheight}{60}
        \setcounter{hourcellwidth}{\value{cellwidth} / 3}
        \setcounter{daynamecellheight}{(\value{cellheight} * 3) / 4}
        \setcounter{halfcellwidth}{\value{cellwidth} / 2}
        \setcounter{halfcellheight}{\value{cellheight} / 2}
        \setcounter{thirdcellwidth}{\value{cellwidth} / 3}
        \setcounter{thirdcellheight}{\value{cellheight} / 3}
        \setcounter{quartercellheight}{\value{cellheight} / 4}
        \setcounter{horiz}{(5 * \value{cellwidth}) + \value{hourcellwidth}}
        \setcounter{vert}{((\value{totnumhours} - 1) * \value{cellheight}) +
                          \value{daynamecellheight}}
        \setcounter{top}{(\value{totnumhours} + 1) * \value{cellheight}}
        \setcounter{firsthour}{12 - \value{numhoursbeforenoon} + 1}

        \setcounter{mon}{\value{hourcellwidth}}
        \setcounter{tue}{\value{mon} + (1 * \value{cellwidth})}
        \setcounter{wed}{\value{mon} + (2 * \value{cellwidth})}
        \setcounter{thu}{\value{mon} + (3 * \value{cellwidth})}
        \setcounter{fri}{\value{mon} + (4 * \value{cellwidth})}

        \newcounter{eight}
        \newcounter{nine}
        \newcounter{ten}
        \newcounter{eleven}
        \newcounter{twelve}
        \newcounter{one}
        \newcounter{two}
        \newcounter{three}
        \newcounter{four}
        \newcounter{five}
        \newcounter{six}
        \newcounter{seven}

        \setcounter{twelve}{\value{top} - \value{daynamecellheight} -
          ((\value{numhoursbeforenoon} - 1) * \value{cellheight})}
        \setcounter{eleven}{\value{twelve} + \value{cellheight}}
        \setcounter{ten}{\value{eleven} + \value{cellheight}}
        \setcounter{nine}{\value{ten} + \value{cellheight}}
        \setcounter{eight}{\value{nine} + \value{cellheight}}
        \setcounter{one}{\value{twelve} - \value{cellheight}}
        \setcounter{two}{\value{one} - \value{cellheight}}
        \setcounter{three}{\value{two} - \value{cellheight}}
        \setcounter{four}{\value{three} - \value{cellheight}}
        \setcounter{five}{\value{four} - \value{cellheight}}
        \setcounter{six}{\value{five} - \value{cellheight}}
        \setcounter{seven}{\value{six} - \value{cellheight}}

        %
        % We make the part of the picture with the office hours themselves a
        % sub-picture, just so the thickness of the lines applies only to
        % it and not to the remaining part of the picture (the grid which is
        % superimposed on the office hours).
        %

        \thicklines

        % draw horizontal lines
        \setcounter{temp}{\value{top}}
        \addtocounter{temp}{-\value{daynamecellheight}}
        \multiput(0,\value{temp})(0,-\value{cellheight}){
          \value{totnumhours}}{\line(1,0){\value{horiz}}
        }
        % far top horizontal line
        \put(0,\value{top}){\line(1,0){\value{horiz}}}

        % draw vertical lines
        \multiput(\value{hourcellwidth},\value{top})(\value{cellwidth},0){6}
          {\line(0,-1){\value{vert}}}
        % far left vertical line
        \put(0,\value{top}){\line(0,-1){\value{vert}}}

        % extra horizontal line under column headers
        \addtocounter{temp}{3}
        \put(0,\value{temp}){\line(1,0){\value{horiz}}}

        % \thinlines

        % write hour labels

        % Not sure why the -5.  The day labels are horizontally centered, but
        % the hour labels seem to have their left edge centered, so need some
        % horizontal adjustment.  (Note -5 in two places below.)
        %
        \setcounter{ctr}{\value{firsthour}}
        \setcounter{temp}{\value{top} - \value{daynamecellheight} -
                          \value{cellheight}}
        \multiput(-5,\value{temp})(0,-\value{cellheight})
          {\value{numhoursbeforenoon}}{
          {\makebox(\value{hourcellwidth},\value{cellheight})[c]
            {\arabic{ctr}\stepcounter{ctr}}}}

        \setcounter{ctr}{1}
        \setcounter{temp}{\value{top} - \value{daynamecellheight} -
          (\value{cellheight} * (\value{numhoursbeforenoon} + 1))}
        \multiput(-5,\value{temp})(0,-\value{cellheight})
          {\value{numhoursafternoon}}{
          {\makebox(\value{hourcellwidth},\value{cellheight})[c]
            {\arabic{ctr}\stepcounter{ctr}}}}

        % write column headers

        \setcounter{temp}{\value{top}}
        \addtocounter{temp}{-\value{daynamecellheight}}

        \put(\value{mon},\value{temp})
          {\makebox(\value{cellwidth},\value{daynamecellheight})[c]{Monday}}
        \put(\value{tue},\value{temp})
          {\makebox(\value{cellwidth},\value{daynamecellheight})[c]{Tuesday}}
        \put(\value{wed},\value{temp})
          {\makebox(\value{cellwidth},\value{daynamecellheight})[c]{Wednesday}}
        \put(\value{thu},\value{temp})
          {\makebox(\value{cellwidth},\value{daynamecellheight})[c]{Thursday}}
        \put(\value{fri},\value{temp})
          {\makebox(\value{cellwidth},\value{daynamecellheight})[c]{Friday}}

        % Monday

        \hour{\value{mon}}{\value{eleven}}{Varun}
             {120}{0}{0}{varun}{0}{0}

        \setcounter{temp}{\value{two} - \value{halfcellheight}}
        \hour{\value{mon}}{\value{temp}}{Tommy}
             {120}{0}{0}{tommy}{0}{0}

        \hour{\value{mon}}{\value{five}}{Larry}
             {60}{0}{0}{larry}{0}{0}

        % Tuesday

        \hour{\value{tue}}{\value{twelve}}{Derek}
             {120}{0}{0}{derek}{0}{0}

        \hour{\value{tue}}{\value{two}}{%
             \begin{tabular}[t]{@{}c@{}}lecture\\010X\end{tabular}}
             {75}{0}{0}{lecture}{-\value{halfcellwidth}}{0}

        \setcounter{temp}{\value{three} - \value{halfcellheight}}
        \hour{\value{tue}}{\value{temp}}{%
             \begin{tabular}[t]{@{}c@{}}lecture\\020X\end{tabular}}
             {75}{0}{0}{lecture}{-\value{halfcellwidth}}{0}

        \setcounter{temp}{\value{tue} + \value{halfcellwidth}}
        \setcounter{temptwo}{\value{two} - \value{halfcellheight}}
        \hour{\value{temp}}{\value{temptwo}}{Tammy}
             {120}{0}{0}{tammy}{-\value{halfcellwidth}}{0}

        \setcounter{temp}{\value{five} + \value{quartercellheight}}
        \hour{\value{tue}}{\value{temp}}{Nick}
             {75}{0}{0}{nick}{0}{0}

        % Wednesday

        \hour{\value{wed}}{\value{ten}}{Larry}
             {60}{0}{0}{larry}{0}{0}

        \hour{\value{wed}}{\value{one}}{Varun}
             {120}{0}{0}{varun}{0}{0}

        \hour{\value{wed}}{\value{three}}{Tommy}
             {120}{0}{0}{tommy}{0}{0}

        % Thursday

        \hour{\value{thu}}{\value{ten}}{Larry}
             {60}{0}{0}{larry}{0}{0}

        \hour{\value{thu}}{\value{twelve}}{Tammy}
             {120}{0}{0}{tammy}{0}{0}

        \hour{\value{thu}}{\value{two}}{%
             \begin{tabular}[t]{@{}c@{}}lecture\\010X\end{tabular}}
             {75}{0}{0}{lecture}{-\value{halfcellwidth}}{0}

        \setcounter{temp}{\value{three} - \value{halfcellheight}}
        \hour{\value{thu}}{\value{temp}}{%
             \begin{tabular}[t]{@{}c@{}}lecture\\020X\end{tabular}}
             {75}{0}{0}{lecture}{-\value{halfcellwidth}}{0}

        \setcounter{temp}{\value{thu} + \value{halfcellwidth}}
        \setcounter{temptwo}{\value{three} - \value{quartercellheight}}
        \hour{\value{temp}}{\value{temptwo}}{Hao}
             {105}{0}{0}{hao}{-\value{halfcellwidth}}{0}

        % Friday

        \setcounter{temp}{\value{eleven} + \value{quartercellheight}}
        \hour{\value{fri}}{\value{temp}}{Hao}
             {135}{0}{0}{hao}{0}{0}

        \hour{\value{fri}}{\value{three}}{Derek}
             {120}{0}{0}{derek}{0}{0}

      \end{pspicture}

    \end{center}

    \vspace{-18mm}

  %   \noindent
  %   Note: the dark blue entries (0101, 0102, 0401, 0402) indicate the
  % discussion sections, while the light blue ones (010X, 040X) indicate the
  % lectures.

\end{document}
