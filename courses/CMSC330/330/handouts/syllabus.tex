\documentclass[10pt]{article}

  %
  % Maybe should have made the quizzes count for a little more (10%, or at
  % least 7.5% or so)....
  %

  %
  % Should have said something or emphasized more strongly about advance
  % notification being needed for missing an exam.
  %

  \usepackage{330-f12}

  \usepackage[none,light,outline,timestamp]{draftcopy}

  \usepackage[dvips,
              colorlinks=true,
              urlcolor=red,         % \href{...}{...} external (URL)
              % filecolor=...  seems to have no effect
              % filecolor=green,    % \href{...} local file
              linkcolor=blue,       % \ref{...} and \pageref{...}
              pdftitle={CMSC 330 syllabus},
              pdfauthor={Nick Feamster and Larry Herman},
              pdfsubject={CMSC 330 syllabus},
              pdfkeywords={CMSC, 330},
              pdfproducer={LaTeX},
              pagebackref,
              bookmarks=true,
              pdfstartview=FitH
             ]{hyperref}

\begin{document}

  \header{\course}{Organization of Programming Languages}{\Term}

  \vspace{-5mm}

  \thispagestyle{plain}

  \section{Prerequisites and description}

    \vspace{-1mm}

    \hspace{4mm}\begin{tabular}{ll}

      Prerequisites:
        & C$-$ or better in CMSC 216 and in CMSC 250
        \\

      Credits:
        & 3 credits
        \\

    \end{tabular}

    \smallskip

      This course covers the semantics of programming languages and their
    runtime organization.  Different models of languages are discussed,
    including procedural (e.g., C), functional (e.g., ML, LISP), rule-based
    (e.g., Prolog), and object--oriented (e.g., C++, Smalltalk).  Runtime
    structures, including dynamic versus static scope rules, storage for
    strings, arrays, records, and object inheritance are discussed.
    Language features such as scoping and binding of variables,
    higher--order programming, typing and type polymorphism, pointers,
    object inheritance, and exceptions are explored.

  \section{Contact information}

    \subsection{Email contact\label{section:email}}

      Unfortunately we're not able to explain most course material via email,
    due to time constraints and other factors.  Answering such questions is
    more appropriate for class discussion or personal communication.  Please
    discuss administrative issues as well as course material in person when
    possible (office hours, and before and after class are good times), and
    use email in case of urgent or emergency matters only.  Lastly, it is
    not practical to provide detailed information or assistance regarding
    programming assignments via email, and attempting to do so often results
    in students receiving incomplete or inadequate information.

    \subsection{Instructors}

      \vspace{-1mm}

      \begin{center}

        \begin{tabular}[t]{c@{\hspace{8mm}}c}

          \renewcommand{\arraystretch}{1.2}

          \begin{tabular}[t]{|l|l|}

            \multicolumn{2}{c}{Sections 0101, 0102, 0103}
            \\ \hline

            Name:
              & Larry Herman
              \\ \hline

            Office:
              & 1111 A. V. Williams
              \\ \hline

            Phone:
              & (301) 405--2762
              \\ \hline

            Email:
              & larry@cs.umd.edu ($*$)
              \\ \hline

          \end{tabular}

          &

          \begin{tabular}[t]{|l|l|}

            \multicolumn{2}{c}{Sections 0201, 0202, 0203, 0204}
            \\ \hline

            Name:
              & Dr.\ Nick Feamster
              \\ \hline

            Office:
              & 4135 A. V. Williams
              \\ \hline

            Phone:
              & (301) 405--8010
              \\ \hline

            Email:
              & feamster@cs.umd.edu ($*$)
              \\ \hline

          \end{tabular}

        \end{tabular}

      \end{center}

      \smallskip

      ($*$) See Section \ref{section:email} below regarding email.

      Office hours will be provided in a separate handout shortly.

    \subsection{Teaching assistants}

      \vspace{-1mm}

      \begin{center}

        \renewcommand{\arraystretch}{1.1}

        \hspace{5.5mm}%
        \begin{tabular}{@{}|l|l|l|l|@{}}

          \multicolumn{1}{c}{name}
            & \multicolumn{1}{c}{duties}
            & \multicolumn{1}{c}{email}
            \\ \hline

          Tommy Pensyl
            & teaching, 0101 \& 0102
            & tpensyl@cs.umd.edu ($*$)
            \\ \hline

          Tammy Tran
            & teaching, 0103 \& 0201
            & tnt@cs.umd.edu ($*$)
            \\ \hline

          Derek Juba
            & teaching, 0202 \& 0203
            & juba@cs.umd.edu ($*$)
            \\ \hline

          Varun Nagaraja
            & teaching, 0204, and grading
            & varun@cs.umd.edu ($*$)
            \\ \hline

          Hao Li
            & grading
            & haoli@cs.umd.edu ($*$)
            \\ \hline

        \end{tabular}

      \end{center}

      \smallskip

      Office hours will be provided in a separate handout shortly.

      \smallskip

      \enlargethispage{5mm}

      While the TAs can provide assistance with programming assignments during
    office hours, you are responsible for developing and debugging your own
    programs.  You should therefore not rely on the instructional staff for
    getting a project to work.  Lower--level CMSC courses provide extensive
    debugging and development help in office hours, but upper--level CMSC
    courses expect students to complete projects with minimal extra help.
    Therefore in CMSC 330 we will provide less debugging help than some
    students may be used to.  If you come in with a question you should
    expect to be pointed in the right direction, but then it will be up to
    you to finish solving the problem on your own.

  \section{Resources, class webpage, and Piazza}

    There are no required or recommended texts for this course this semester.
  Resources and sources of information for the languages and concepts to be
  taught will be provided during the semester.

    Various course materials will be made available on the class webpage,
  which can be reached by clicking on following link:

    \vspace{-3mm}

    \begin{centering}

    \href{http://www.cs.umd.edu/class/fall2012/cmsc330}
         {\texttt{\underline{www.cs.umd.edu/class/fall2012/cmsc330}}}

    \end{centering}

    \medskip

    Accessing parts of the webpage will require an ID and password to be
  provided in class.

    Course material or concepts may be discussed in Piazza, an online forum,
  linked to from the class webpage.

    Programming will be done on the OIT Grace Cluster, grace.umd.edu.
  Students will use their own TerpConnect accounts to access the Grace
  cluster and do coursework, so students who don't have a TerpConnect
  account should request one online immediately at
   \href{http://www.oit.umd.edu/new}
        {\texttt{\underline{www.oit.umd.edu/new}}}.
  Although programming may be done on other systems that you have access to,
  it is recommended that you do your work on the Grace cluster.  All project
  submissions must work correctly on those machines, and your projects will
  be graded solely based on their results on the cluster and the CMSC
  department submit server.  Because language and library versions may vary
  with the installation, in unfortunate circumstances a program might work
  perfectly on another system but not work at all on the Grace cluster or
  the submit server.  Thus we strongly recommend that if you want to develop
  any project on another system, you should complete it \textbf{several days
  early} to have time to address any compatibility problems.

  %
  %   Projects can be developed on the OIT Grace UNIX Cluster.  You may use any
  % other available system, but all project submissions \textbf{must} work
  % correctly using the language versions on the Grace cluster (details to be
  % provided).  Because different versions of languages may be installed
  % elsewhere, a program may work perfectly on one system, yet not work at all
  % on the Grace cluster.  The program you submit will be graded based on its
  % results on the Grace Cluster, so having a working version on another
  % system at any other time (or even another working version in your Glue
  % account) can not be considered.  No consideration in grading can be made
  % for errors made in transferring files, or submitting the wrong version of
  % your project.  If you want to write any project on another system you are
  % strongly recommended to complete it \textbf{several days early}, to have
  % time to address any problems arising.
  %

  \section{Quiz, exam, and final dates\label{section:examdates}}

    Quizzes will be given in discussion section and will cover discussion and
  lecture material.  They will be announced in a prior class.

    Midterm exams will be held during lecture.  Their dates will be confirmed
  later, and could possibly be changed depending on lecture progress and
  other factors.  The expected dates are below.

    The final exam date and time will be rescheduled \textbf{only} for
  students having another final at exactly the same time, or for students
  with more than three final exams scheduled on the same day.  (The only
  courses that students who are enrolled in CMSC 330 should be able to take
  that have finals at the same time as its final are BIOM 301, EDMS 451, and
  ENMA 300.)  If either of these situations applies to you, you must inform
  your instructor \textbf{at least two weeks in advance} of the final exam
  time for allowances to be made.  Also please inform your instructor
  immediately if you have a conflict with a scheduled midterm date, or any
  other important date as the semester progresses.

    \vspace{-1.5mm}

    \begin{center}

      \renewcommand{\arraystretch}{1.15}

      \begin{tabular}[t]{|l|l|}

        \hline

        Exam \#1:
          & Thursday, October 11
          \\ \hline

        Exam \#2:
          & Tuesday, November 13
          \\ \hline

        Final exam:
          & Thursday, December 13, 4:00--6:00 p.m., location TBA
          \\ \hline

      \end{tabular}

    \end{center}

    \vspace{-2mm}

  \section{Attendance, homework, and general grading policies}

    Students are responsible for the material covered, and announcements,
  deadlines, policies, etc., discussed in lecture and discussion section,
  regardless of whether they were in class or not.  Students are likewise
  responsible for announcements and information appearing in Piazza.
  \textbf{It's understood that students may occasionally have to miss class
  for various reasons, but office hours, email, and Piazza are not
  intended as a replacement for class attendance}.

    Various practice (homework) exercises and solutions will be provided
  during the semester; these will not be collected or graded, but are
  extremely important for testing your knowledge of the material in
  preparation for quizzes and exams.  You are encouraged to work together on
  these problems.  If you have difficulty solving them, feel free to ask
  questions in Piazza, or see the instructional staff in office hours as
  soon as possible.

    \smallskip

    Coursework will count toward the final grade according to the following
  percentages:

    \begin{center}

      \begin{tabular}{@{}|l|l|r|l|@{}}

        \hline

        Midterms:
          & two midterms
          & 32.5\%
          & (equally weighted)
          \\ \hline

        Final:
          & will be comprehensive
          & 30\%
          &
          \\ \hline

        Quizzes:
          & in discussion section
          & 5\%
          & (equally weighted)
          \\ \hline

        Programming projects:
          & five or six coding assignments
          & 32.5\%
          &
          \\ \hline

      \end{tabular}

    \end{center}

    \smallskip

    A request for reconsideration of the grading on any coursework must be
  submitted \textbf{within one week} of when it is returned.  Exam regrading
  requests must be made in writing.
  % Information about resolving any questions about project grading will be
  % provided when the first project is graded.
  Coursework submitted for reconsideration may be regraded in its entirety.
  % which could result in a lower score if warranted.

    Final course grades will be curved as necessary, based on each student's
  total numeric score for all coursework at the end of the semester.

  %   Note: this is a programming course, which teaches languages and concepts
  % essential for later CMSC courses.  As a result, the ability to submit
  % working versions of the projects is necessary.  Therefore, \textbf{no
  % student will be able to pass the course (with a grade of C-- or higher) if
  % at the end of the semester they have a zero grade for any project},
  % regardless of their performance or scores on the other coursework.  Complete
  % project grading policies are below.

    Project submission and grading policies will be provided with the first
  project assignment.

  \section{Absences and accommodations}

    Besides the policies in this syllabus, various University policies may
  apply to students during the semester.  Policies that may be relevant
  appear in the Undergraduate Catalog, at
    \href{http://www.umd.edu/catalog}
         {\texttt{\underline{www.umd.edu/catalog}}}.

    If you experience difficulty during the semester keeping up with the
  academic demands of your courses, you may consider contacting the Learning
  Assistance Service in 2201 Shoemaker Building at (301) 314--7693.  Their
  educational counselors can help with time management issues, reading,
  note--taking, and exam preparation skills.

    \enlargethispage{4mm}

    \subsection{Excused absences}

      \subsubsection{Illness}

        Any student who needs to be excused for an absence from a single
      lecture or discussion section for reasons of medical necessity must,
      according to the University policy:

        \vspace{-2mm}

        \begin{itemize}

          \addtolength{\itemsep}{-1.25mm}

          \item Make a reasonable attempt to inform his or her instructor of
                his or her illness prior to the class.

          \item Upon returning to the class, present his or her instructor
                with a self--signed note attesting to the date of their
                illness.  Each note must contain an acknowledgment by the
                student that the information provided is true and correct.
                Providing false information to University officials is
                prohibited under Part 9(h) of the Code of Student Conduct
                (V--1.00(B) University of Maryland Code of Student Conduct)
                and may result in disciplinary action.

          \item This self--documentation may not be used for the major
                scheduled grading events as defined below and it may only be
                used only once during the semester.

        \end{itemize}

        \vspace{-1.5mm}

          If a student needs to be excused for a prolonged illness (for this
        course this means missing two or more consecutive class meetings) or
        if a major scheduled grading event is missed due to illness, written
        documentation of the illness from the Health Center or from an
        outside health care provider must be provided.  This must include
        the contact information of the provider, verify dates of treatment,
        and indicate the time that the student was incapacitated, in the
        provider's opinion, and unable to meet academic responsibilities.
        Diagnostic information need not be given.  The major scheduled
        grading events for this course are the two midterms and the final
        exam whose dates are given above in Section \ref{section:examdates}.

      \subsubsection{Other excused absences}

        An excused absence will be given for other University--approved reasons,
      such as religious observance, participation in required university
      activities, or family or personal emergency, provided that:

        \vspace{-2.5mm}

        \begin{itemize}

          \addtolength{\itemsep}{-1.75mm}

          \item Students requesting an excused absence furnish documentary
                support of the cause of the absence where feasible.

          \item The maximum possible advance notice is given.

        \end{itemize}

        \vspace{-4mm}

      \subsubsection{Projects\label{section:project-extension}}

        The policies for excused absences don't apply to project assignments.
      These will be assigned with sufficient time to be completed by
      students who have a reasonable understanding of the necessary material
      and begin promptly.  In cases of protracted, serious illness, or
      severe emergency situations, short extensions on projects may be
      considered, depending upon the circumstances.  In such a situation you
      must contact your instructor as soon as possible.

      % Note to myself: my reading of the University policy is that it doesn't
      % require extensions on projects for religious or other reasons, since a
      % project isn't given at a specific time, so a student can't "miss" it.

    \subsection{How excused absences are handled}

      All arrangements for excused absences must be made with the instructor
    (even if the coursework that was missed was in discussion section).

      An excused absence for an exam will be handled by giving a makeup
    exam.

      For an excused absence for a single quiz, rather than a makeup the score
    will be computed at the end of the semester as the average of the
    student's scores for the other quizzes.  Missing more than one quiz will
    be handled by giving a makeup quiz.

    \subsection{Students with disabilities}

      Any student eligible for and requesting reasonable academic
    accommodations due to a disability is requested to provide, to the
    instructor in office hours, a letter of accommodation from the Office of
    Disability Support Services (DSS) within the first two weeks of the
    semester.

      All arrangements for exam accommodations as a result of disability
    \textbf{must} be made and arranged with the instructor \textbf{at least
    three business days prior to the exam date}, or accommodations cannot
    be made.

  \section{Academic integrity}

    The Campus Senate has adopted a policy asking students to include the
  following statement on each examination or major assignment in every
  course: ``I pledge on my honor that I have not given or received any
  unauthorized assistance on this examination (or assignment).''
  Consequently, you will be requested to include this pledge on exams and
  projects.

    Please carefully read the Office of Information Technology's policy
  regarding acceptable use of computer accounts and resources at
    \href{http://www.nethics.umd.edu/aup}
         {\texttt{\underline{www.nethics.umd.edu/aup}}}.

    Unless otherwise noted, all coursework-- including programming
  coursework-- is to be done \textbf{individually}, so cooperation or use of
  unauthorized materials on programming assignments is a violation of the
  University's Code of Academic Integrity.  \textbf{Any evidence} of this,
  or of unacceptable use of computer accounts, use of unauthorized materials
  or cooperation on exams or quizzes, or other possible violations of the
  Honor Code, \textbf{will be submitted} to the Student Honor Council, which
  could result in an XF for the course, suspension, or expulsion.

    \vspace{-2.25mm}

    \begin{itemize}

      \addtolength{\itemsep}{-1.75mm}

      \item For learning the course concepts (including the programming
            languages), students are welcome to study together or to receive
            help from anyone else.  Students may discuss with others the
            project requirements, the features of the programming languages
            used, what was discussed in class and in Piazza, and
            \textbf{general} syntax errors.  Allowable questions are ones
            that convey no information about the contents of a student's
            project solution.

      \item When it comes to actually writing a project assignment, other
            than help from the instructional staff a project must solely and
            entirely be a student's own work.  Working with another student
            or individual, or using anyone else's work \textsc{in any way}
            except as noted in this paragraph, is a violation of the code of
            academic integrity and \textsc{will be reported} to the Honor
            Council.  Students may not discuss design of any part of a
            project with \textbf{anyone} except the instructors or teaching
            assistants.  Examples of questions which students may
            \textbf{not} ask others might be ``How did you implement this
            part of the project?'' or ``Please look at my code and help me
            find the stupid syntax error!''.  Students may not use any
            disallowed source of information in creating either their
            project design or code.  When writing projects students are free
            to use ideas or \textbf{short fragments} of code from
            \textbf{published} textbooks or \textbf{publicly available}
            information, or information or examples provided by the
            instructional staff, provided that the specific source is cited
            in a comment in the relevant section of the program.

    \end{itemize}

    \vspace{-2.25mm}

    \noindent
    \underline{\textsc{Violations of the Code of Academic Integrity
  may include, but are not limited to:}}

    \vspace{-1.5mm}

    \begin{enumerate}

      \addtolength{\itemsep}{-1.25mm}

      \item Failing to do all or any of the work on a project by yourself,
            other than assistance from the instructional staff.

      \item Using any ideas or any part of another person's project, or
            copying any other individual's work in any way.

      \item Giving any parts or ideas from your project, including test
            data, to another student.

      \item Allowing any other students access to your program on any
            computer system.

      \item Transferring any part of a project to or from another student or
            individual by any means, electronic or otherwise.

    \end{enumerate}

    \vspace{-1.5mm}

    If you have any question about a particular situation or source then
  consult with your instructor in advance.  Should you have difficulty with a
  programming assignment you should \textbf{see the teaching assistants in
  office hours}, \textsc{not} solicit help from anyone else in violation of
  these rules.

    \textsc{It is the responsibility, under the honor policy, of anyone who
  suspects an incident of academic dishonesty has occurred to report it to
  their instructor, or directly to the Honor Council.}

    Every semester the department has discovered a number of students
  attempting to cheat on project assignments, in violation of academic
  integrity requirements.  Students' academic careers have been significantly
  affected by a decision to cheat.
  % Think about whether you want to join them before contemplating cheating,
  % or before helping a friend to cheat.

    Students are welcome and encouraged to study and compare or discuss their
  implementations of the programming projects with any others after they are
  graded, \textbf{provided that} the project in question will not be
  extended upon in a later project assignment.

  \section{Course topics {\sc (subject to change)}}

   The following list of lecture topics may vary according to the pace of
  lecture:

    \vspace{-2mm}

    \begin{multicols}{2}

      \begin{itemize}

        \addtolength{\itemsep}{0mm}

        \item Administrative and course introduction

        \item Scripting languages (Ruby)

              \vspace{-2.5mm}

              \begin{itemize}

                \addtolength{\itemsep}{-.5mm}

                \item Implicit vs.\ explicit declarations

                \item Dynamic vs.\ static typing

                \item Text processing and string manipulation

                \item Data structures in Ruby

              \end{itemize}

              \vspace{-2mm}

        \item Regular expressions

              \vspace{-2.5mm}

              \begin{itemize}

                \addtolength{\itemsep}{-.5mm}

                \item Regular expressions in Ruby

                \item Formal definitions

              \end{itemize}

              \vspace{-2mm}

        \item Finite automata

              \vspace{-2.5mm}

              \begin{itemize}

                \addtolength{\itemsep}{-.5mm}

                \item DFAs

                \item NFAs

                \item Uses and applications

              \end{itemize}

              \vspace{-2mm}

        \item Environments, scoping, and binding

              \vspace{-2.5mm}

              \begin{itemize}

                \addtolength{\itemsep}{-.5mm}

                \item Functions and procedures

                \item Parameter passing mechanisms

                \item Dynamic vs.\ static scoping

                \item Runtime implementations

              \end{itemize}

              \vspace{-2mm}

        \item Functional programming (OCaml)

              \vspace{-2.5mm}

              \begin{itemize}

                \addtolength{\itemsep}{-.5mm}

                \item Lists and recursion

                \item Higher--order programming

                \item Types and polymorphism

                \item Data types and pattern matching

                \item Modules

                \item Closures

              \end{itemize}

              \vspace{-2mm}

        \item Context--free grammars

        \item Concurrency and multithreading

        \item Parameters and scope

        \item Polymorphism and generics

        \item Functional programming in object--oriented languages

        \item Programming language theory

              \vspace{-2mm}

              \begin{itemize}

                \addtolength{\itemsep}{-.5mm}

                \item Lambda calculus

                \item Operational semantics

              \end{itemize}

              \vspace{-2.5mm}

        \item Pointers and garbage collection

        % \item Exceptions

        \item Historical overview of programming languages

      \end{itemize}

      \vspace{-2mm}

    \end{multicols}

  \section{Right to change information}

    Although every effort has been made to be complete and accurate,
  unforeseen circumstances arising during the semester could require the
  adjustment of any material given here.  Consequently, given due notice to
  students, the instructor reserves the right to change any information on
  this syllabus or in other course materials.

  \section{Copyright}

    All course materials are copyright Larry Herman and Nick Feamster (and
  other CMSC faculty) \copyright\ 2005--2012.  All rights reserved.
  Students are permitted to use course materials for their own personal use
  only.  Course materials may not be distributed publicly or provided to
  others (excepting other students in the course), in any way or format.

\end{document}
