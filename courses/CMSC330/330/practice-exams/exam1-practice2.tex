\documentclass[11pt]{article}

  \usepackage{330-f12}

  \usepackage{graphicx}

  \usepackage{color}

\begin{document}

  \header{\course}{Exam \#1 practice questions \#2}{\Term}

  \noindent
  {\large\textbf{\underline{Do not open this exam until you are told.}  Read
  these instructions:}}

  \vspace{-2mm}

  \thispagestyle{myheadings}

  \begin{enumerate}

    \addtolength{\itemsep}{-2mm}  % reduced space between lines

    \item This is a closed book exam.  \textbf{No
          % calculators,
          notes %,
          or other aids are allowed.}

    \item {\bf You must turn in your exam \underline{immediately} when time is
          called at the end.}

    \item This exam contains % \pageref{end}
          6 pages, including this one.  \textbf{Make sure you have all the
          pages.}  Each question's point value is next to its number.
          \textbf{Write your name on the top of all pages before starting
          the exam.}

    \item In order to be eligible for as much partial credit as possible,
          show all of your work for each problem, and \textbf{clearly
          indicate} your answers.  Credit \textbf{cannot} be given for
          illegible answers.

    \item If you finish at least 15 minutes early, bring your exam to the
          front when you are finished; otherwise, wait until the end of the
          exam to turn it in. Please be as quiet as possible.

    \item If you have a question, raise your hand.  If you feel an exam
          question assumes something that is not written, write it down on
          your exam sheet.  Barring some unforeseen error on the exam,
          however, you shouldn't need to do this at all, so be careful when
          making assumptions.

    \item If you need scratch paper during the exam, please raise your hand.
          Scratch paper must be turned in with your exam, with your name and
          ID number written on it.  Scratch paper \textbf{will not} be
          graded.

    \item Small syntax errors will be ignored in any code you have to write
          on this exam, as long as the concepts are correct.

    \item The Campus Senate has adopted a policy asking students to include
          the following handwritten statement on each examination and
          assignment in every course: ``\textit{I pledge on my honor that I
          have not given or received any unauthorized assistance on this
          examination\/}.''  Therefore, \textbf{just before turning in your
          exam}, you are requested to write this pledge \textbf{in full
          \textmd{and} sign it} below:

          \medskip

          \begin{minipage}[t]{6.6in}

            \addtolength{\baselineskip}{4mm}

            \underline{\hspace{6.6in}}

            \underline{\hspace{6.6in}}

          \end{minipage}

    \medskip

  \end{enumerate}

  \vspace{-1.5mm}

  \noindent
  Good luck!

  % \enlargethispage{2mm}
  % 
  % \vspace{\fill}
  % 
  % % 3 exam questions, .6mm scale factor
  % \hspace{\fill}\parbox[t]{3.15in}{\scoreblank{3}{.6mm}}

  \pagebreak

    %%%%%%%%%%%%%%%%%%%%%%%%%%%%%%%%%%%%%%%%%%%%%%%%%%%%%%%%%%%%%%%%%%%%%%%%%%

  \pagestyle{headings}

  \markright{Name: \hspace{.2mm} \lans\ans}

  \begin{enumerate}

    %%%%%%%%%%%%%%%%%%%%%%%%%%%%%%%%%%%%%%%%%%%%%%%%%%%%%%%%%%%%%%%%%%%%%%%%%%

    \item {[20 pts.]} \textbf{Short Answer.}

          \vspace{-1.5mm}

          \begin{enumerate}

            \item { [5 pts.]} \textbf{Briefly} explain the difference
                  between a \textit{formal parameter} and an \textit{actual
                  parameter}.

                  \vspace{1.65in}

            \item { [5 pts.]} \textbf{Briefly} explain what a
                  \textit{control statement} is and give examples of two
                  control statements in Ruby.

                  \vspace{1.65in}

            \item { [10 pts.]} Write down the type of each of the following
                  OCaml expression, or ``error'' if the expression has no
                  valid type.

                  \begin{enumerate}

                    \item \texttt{(1, [3; 4; 5], 'a')}

                          \vspace{.65in}

                    \item \texttt{[(1, 2, 3); (4, 5); (6, 7, 8)]}

                          \vspace{.65in}

                    \item \texttt{if true then (1, 2) else [3; 4]}

                          \vspace{.65in}

                    \item \texttt{((1, 2), [3; 4])}

                          \vspace{.65in}

                    \item \texttt{[(fun x -> x + 1); (fun y -> -y);
                                   (fun z -> z * z)]}

                          \vspace{.25in}

                  \end{enumerate}

          \end{enumerate}

          \pagebreak

    %%%%%%%%%%%%%%%%%%%%%%%%%%%%%%%%%%%%%%%%%%%%%%%%%%%%%%%%%%%%%%%%%%%%%%%%%%

    \item {[50 pts.]} \textbf{Regular languages from Planet Zorg.}  Aliens
          from the planet Zorg have taken control of Friendly Computer
          Corporation and are requiring that all software be written in the
          programming language ``Zorg'' (the aliens are not very original).
          Zorg has a number of peculiarities.

          \vspace{-2mm}

          \begin{enumerate}

            \item {{[10 pts.]}} Keywords in Zorg are all the strings
                  recognized by the following NFA.  Write down the set of
                  Zorg keywords.  (Your answer should just be a list of
                  strings or words.)

                  \begin{pspicture}(0,0)(145,67)

                    \Large

                    \state(10,60){s0}         % top row
                    \state[final](35,60){s1}
                    \state[start](10,35){s2}  % middle row
                    \state(35,35){s3}
                    \state[final](85,35){s4}
                    \state(10,10){s5}         % bottom row
                    \state(35,10){s6}
                    \state(60,10){s7}
                    \state[final](85,10){s8}

                    \large

                    \transition(s0,\emph{a},s3)
                    \transition(s0,\emph{z},s1)

                    \transition(s1,\emph{b},s3)

                    \transition(s2,\emph{b},s3)
                    \transition(s2,\emph{z},s0)

                    \transition(s3,\emph{a},s4)
                    \transition[offset=2,labelposition=.6](s3,\emph{a},s7)
                    \transition[offset=-2,labelposition=.6,labeloffset=-5.5]%
                                (s3,\emph{b},s7)

                    \transition[loopdirection=right](s4,$\epsilon$,s4)

                    \transition(s5,\emph{a},s3)
                    \transition(s5,\emph{b},s2)

                    \transition[loopdirection=up,curvature=4,labeloffset=2.5,
                                labelposition=.85](s6,\emph{a},s6)
                    \transition[loopdirection=left,curvature=4](s6,\emph{b},s6)
                    \transition[loopdirection=down,curvature=4](s6,\emph{z},s6)

                    \transition(s7,\emph{z},s6)
                    \transition(s7,\emph{z},s8)

                    \transition[labeloffset=-5](s8,$\epsilon$,s4)

                  \end{pspicture}

                  \vspace{1.4in}

            \item {{[10 pts.]}} Identifiers in Zorg are made up of a sequence
                  of $a$'s, $b$'s, and $z$'s, must be an even length, and
                  must contain an occurrence of the string $az$.  Write a
                  regular expression for Zorg identifiers.  You should use
                  \textbf{only} the formal regular expression
                  operations concatenation, alternation, and Kleene closure,
                  which were defined in class.  These operations can be
                  nested, and terminal symbols, parentheses, and $\epsilon$
                  may be used as well, but do \textbf{not} use any other
                  regular expression operations, and do \textbf{not} write a
                  Ruby regular expression.

                  \vspace{2in}

                  \pagebreak

            \item {{[10 pts.]}} Class names in Zorg (it's an object--oriented
                  language) are strings of $z$'s such that the length of the
                  string is \textit{not} divisible by three.  Write a
                  regular expression for Zorg class names, again using only
                  the notation allowed in part (b).  \textit{Hint: Start by
                  writing down the strings whose length is divisible by
                  three and work from there.}

                  \vspace{2.25in}

            \item {{[20 pts.]}} In Zorg, strings begin with $<$ and end with
                  $>$, and may contain occurrences of $<$, $z$, and $>$.
                  Inside of a string, $>$ may appear but only if it is
                  preceded by $z$ (i.e., $z$ is the ``escape'' character).
                  $z$ may appear without a following $>$.  Moreover, there
                  must be an even number of characters between $<$ and $>$,
                  where the pair $z\!\!>$ is counted as a single character.
                  Note that a string cannot end in $z\!\!>$, because the
                  pair $z\!\!>$ counts as a single character, so the string
                  would lack a terminating $>$.  Construct a DFA that
                  accepts valid Zorg strings.  Be sure to create a DFA,
                  \textbf{not} an NFA.  Note: \textbf{do not} use the
                  notational shortcuts for DFAs that were given in lecture.

                  \vspace{3in}

          \end{enumerate}

          \pagebreak

    %%%%%%%%%%%%%%%%%%%%%%%%%%%%%%%%%%%%%%%%%%%%%%%%%%%%%%%%%%%%%%%%%%%%%%%%%%

    \item {[30 pts.]} \textbf{Linked lists in Ruby.} Write code for a
          complete Ruby class \texttt{List} that implements singly--linked
          lists.  Below is a class \texttt{Empty} that is a subclass of
          your \texttt{List} class.  Instances of \texttt{Empty}
          represent the empty list.

          \vspace{-3mm}

          \begin{verbatim}
class Empty < List

  def length
    return 0
  end

  # appending any list l to the empty list results in the list l
  def append(l)
    return l
  end
end\end{verbatim}

          \vspace{-1.25mm}
          
          Your class \texttt{List} must be a complete Ruby class, and
          must include definitions of the following methods:

          \vspace{-3.25mm}

          \begin{center}

            \begin{tabular}{lp{5.85in}@{}}

              \texttt{length}
                & Return the length of this list.  Your method \textbf{must}
                  be recursive. \\

              \texttt{append(l)}
                & Return a new list containing this list followed by list
                  \texttt{l}, \textbf{without} changing self.  Your method
                  \textbf{must} be recursive.  Your class should allow
                  appending a \texttt{List} to an \texttt{Empty List}

            \end{tabular}

          \end{center}

          \vspace{-7.5mm}

          \textit{Hint: Recall that \emph{\texttt{self}} refers to the
          current instance within a method.  The methods we've given
          you in \emph{\texttt{Empty}} represent the base cases of
          your recursive functions.}  \textbf{Note:} your
          implementation must create a linked structure, and not use any
          library classes (such as \texttt{Array}).  You can use the space
          below, plus the next page if necessary.

          \vspace{5in}

          \pagebreak

          ~

          \pagebreak

    %%%%%%%%%%%%%%%%%%%%%%%%%%%%%%%%%%%%%%%%%%%%%%%%%%%%%%%%%%%%%%%%%%%%%%%%%%

  \end{enumerate}

  \label{end}

\end{document}
