\documentclass[11pt]{article}

  \usepackage{330-f12}

  \usepackage{graphicx}

\begin{document}

  \showinfo

  \header{\course}{Exam \#1 practice questions \#2-- solutions}{\Term}

  \begin{enumerate}

    \addtolength{\itemsep}{10mm}

    \item {[20 pts.]} Short answer.

          \vspace{-1.5mm}

          \begin{enumerate}

            \addtolength{\itemsep}{2mm}

            \item {{[5 pts.]}} A formal parameter is the name for a parameter
                  as used in the body of a method or function.  An actual
                  parameter is the argument that is passed in at a method or
                  function call.

            \item {{[5 pts.]}} A control statement is one that changes what
                  the next statement to be executed is.  Examples are
                  \texttt{if}, \texttt{unless}, \texttt{while}, and
                  \texttt{until}.  Method invocation and return can also be
                  considered control statements.

            \item {{[10 pts.]}}

                  \begin{enumerate}

                    \addtolength{\itemsep}{2mm}

                    \item \texttt{int * int list * char}

                    \item error (can't mix tuples of different sizes)

                    \item error (both branches of an \texttt{if} must have
                          the same type)

                    \item \texttt{(int * int) * int list}

                    \item \texttt{(int -> int) list}

                  \end{enumerate}

          \end{enumerate}

    \item {[50 pts.]} Regular languages from Planet Zorg.

          \vspace{-2mm}

          \begin{enumerate}

            \addtolength{\itemsep}{2.5mm}

            \item {{[10 pts.]}}
                  \(
                    zz, \, zzba, \, zaa, \, ba, \, zzbaz, \, zzbbz, \, zaaz,
                    \, zabz, \, baz, \, bbz
                  \)

            \item {{[10 pts.]}} $((a|b|z)(a|b|z))^*$ is the set of strings
                  of even length, and $(a|b|z)((a|b|z)(a|b|z))^*$ is the set
                  of strings of odd length.  Thus one correct answer is:

                  \begin{displaymath}
                    \lpar \:
                      ((a|b|z)(a|b|z))^*az((a|b|z)(a|b|z))^* \ \mid \
                      (a|b|z)((a|b|z)(a|b|z))^*az(a|b|z)((a|b|z)(a|b|z))^* \:
                    \rpar
                  \end{displaymath}

            \item {{[10 pts.]}} The set of strings of $z$'s whose length is
                  divisible by three is $(zzz)^*$.  To get strings whose
                  length is not divisible by three we can add one or two
                  more $z$'s to it.  Thus two correct answers are:

                  \begin{displaymath}
                    (z \! \mid \! zz) \, (zzz)^*
                  \end{displaymath}

                  \vspace{-5mm}

                  \begin{displaymath}
                    \lpar \: z \: (zzz)^* \: \mid \: zz \: (zzz)^* \: \rpar
                  \end{displaymath}

            \item {{[20 pts.]}} Just for ease of drawing the DFA the
                  notational shortcut is used of assuming that all omitted
                  states go to a dead state that is not shown.  (You were
                  not supposed to use this shortcut, but it's easier to draw
                  DFAs by hand than to typeset them\ldots)

                  \enlargethispage{5mm}

                  \bigskip

                  \hspace{17mm}%
                  \begin{pspicture}(0,0)(145,77)

                    \Large

                    \state[final](10,70){s0}  % top row
                    \state(55,70){s1}
                    \state[start](10,40){s2}  % middle row
                    \state(40,40){s3}
                    \state(70,40){s4}
                    \state(55,10){s5}         % bottom row

                    \large

                    \transition[offset=-2,labeloffset=-6](s1,$<$,s4)
                    \transition[labeloffset=-6](s1,$>$,s3)
                    \transition[curved,curvature=1.15,angle=65](s1,\emph{z},s5)

                    \transition(s2,$<$,s3)

                    \transition[offset=2](s3,$<$,s4)
                    \transition[labeloffset=-6.5](s3,$>$,s0)
                    \transition[offset=-2,labeloffset=-5](s3,\emph{z},s5)

                    \transition[offset=2](s4,$<$,s3)
                    \transition[offset=-2,labeloffset=-5.5](s4,\emph{z},s1)

                    \transition[offset=-2,labeloffset=-5.5](s5,$<$,s3)
                    \transition[labeloffset=-6.5](s5,$>$,s4)
                    \transition[curved,curvature=1.30,angle=-85,%
                                labeloffset=-5](s5,\emph{z},s1)

                  \end{pspicture}

          \end{enumerate}

    \item {[30 pts.]} Linked lists in Ruby.

          Here's the \texttt{Empty} class again:

          \vspace{-3mm}

          \begin{verbatim}
class Empty < List

  def length
    return 0
  end

  # appending any list l to the empty list results in the list l
  def append(l)
    return l
  end
end\end{verbatim}

          A correct \texttt{List} class would be:

          \begin{verbatim}
class List

  def initialize(h, t)
    @head = h
    @tail = t
  end

  def length
    return 1 + (@tail.length)
  end

  def append(l)
    return List.new(@head, @tail.append(l))
  end

end\end{verbatim}

  \end{enumerate}

\end{document}
