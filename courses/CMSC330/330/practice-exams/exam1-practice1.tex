\documentclass[11pt]{article}

  \usepackage{330-f12}

  \newcommand{\sms}{\ensuremath{\hspace{.1mm}}}

  \newcommand{\terminal}[1]{\texttt{#1}}

  \newcommand{\midspcsmall}{\mbox{\LARGE \ensuremath{\ \mid \ }}}

  \newcommand{\formula}[1]{{\Large\ensuremath{#1}}}

\begin{document}

  % We have to use \header{}{}{}, since \maketitle generates a new page
  % beforehand.
  %
  \header{\course}{Exam \#1 practice questions \#1}{\Term}

  \noindent
  {\large
    \textbf{\underline{Do not open this exam until you are told.}
            Read these instructions:
    }
  }

   \vspace{-2mm}

  \thispagestyle{myheadings}

  \begin{enumerate}

    \addtolength{\itemsep}{-2mm}  % reduced space between lines

    \item This is a closed book exam.  {\bf No
          % calculators,
          notes %,
          or other aids are allowed.}  If you have a question during the
          exam, please raise your hand.  Each question's point value is next
          to its number.

    \item {\bf You must turn in your exam \underline{immediately} when time is
          called at the end.}

    % \item 6 pages, 5 problems, 100 points, 50 minutes.

    \item In order to be eligible for as much partial credit as possible,
          show all of your work for each problem, and \textbf{clearly
          indicate} your answers.  Credit \textbf{cannot} be given for
          illegible answers.

    \item You will \textbf{lose credit} if \emph{\textbf{any}} information
           above is incorrect or missing, or your name is missing from any
           side of any page.

    \item Parts of this page and of two other pages are
    % \item Part of this page, and part of the next page, are
          for scratch work.  If you need extra scratch paper
    % \item Parts of this page and of another page are
    % \item Part of this page, and the last page, are
    %       for scratch work.  If you need extra scratch paper
    % \item Part of this page is
    %       for scratch work.  If you need extra scratch paper
    % \item Part of this page, and parts of two other pages, are
    %       for scratch work.  If you need extra scratch paper
          after you have filled
          % this page
          % these pages
          these areas
          % this area
          up, please raise your hand.
          Scratch paper must be turned in with your exam, with your name and
          ID number written on it.
          % \textbf{Nothing written on scratch paper will be graded.}
          Scratch paper \textbf{will not} be graded.
          % Figure out your answer on scratch paper if necessary, then write
          % it {\bf neatly} in the answer space provided.

    \item To avoid distracting others, {\bf no one} may leave until the exam
          is over.

    \item The Campus Senate has adopted a policy asking students to include
          the following handwritten statement on each examination and
          assignment in every course: ``\textit{I pledge on my honor that I
          have not given or received any unauthorized assistance on this
          examination\/}.''  Therefore, \textbf{just before turning in your
          exam}, you are requested to write this pledge \textbf{in full
          \textmd{and} sign it} below:

          \medskip

          \begin{minipage}[t]{6.6in}

            \addtolength{\baselineskip}{4mm}

            \underline{\hspace{6.6in}}

            \underline{\hspace{6.6in}}

          \end{minipage}

    \medskip

  \end{enumerate}

  \vspace{-2mm}

  \noindent
  Good luck!

  % \enlargethispage{6mm}
  % 
  % \vspace{\fill}
  % 
  % \noindent
  % \begin{tabular}[t]{@{}p{2.75in}@{\hspace{.5in}}p{3.75in}@{}}
  % 
  %   \vspace{8mm}
  % 
  %   \scratchpaper[2.75in]
  % 
  % &
  % 
  %   \vspace{0mm}
  % 
  %   % 4 exam questions, .5mm scale factor
  %   \hspace{\fill}\parbox[t]{3.325in}{\scoreblank{5}{.45mm}}
  %   % \scoreblank{4}{.5mm}
  % 
  %   %\vspace{-.75in}
  % 
  % \end{tabular}

  \pagebreak

    %%%%%%%%%%%%%%%%%%%%%%%%%%%%%%%%%%%%%%%%%%%%%%%%%%%%%%%%%%%%%%%%%%%%%%%%%%

  \pagestyle{headings}

  \markright{Name: \hspace{.2mm} \lans\ans}

  \begin{enumerate}

    %%%%%%%%%%%%%%%%%%%%%%%%%%%%%%%%%%%%%%%%%%%%%%%%%%%%%%%%%%%%%%%%%%%%%%%%%%

    \item {[16 pts.]} Short--answer questions based on material discussed in
          class-- answer them quickly and \textbf{very \underline{briefly}},
          each in \textbf{two sentences} or less.

          \vspace{-2.5mm}

          \begin{enumerate}

            \addtolength{\itemsep}{28mm}

            \item Give a short example (just a few lines of code) that
                  illustrate that Ruby has dynamic typing.

            \item \textbf{Briefly} describe any \emph{two} differences
                  between NFAs and DFAs.

            \item \textbf{Briefly} explain what it means to say that a
                  function has \emph{polymorphic} type.

            \item \textbf{Briefly} explain what it means to say that a
                  language has \emph{higher--order functions}.

          \end{enumerate}

          \vspace*{26mm}

          \underline{\hspace{6.6225in}}

          \medskip

          \begin{centering}

            \scratchpaper[6.8in]

          \end{centering}

          \pagebreak

    %%%%%%%%%%%%%%%%%%%%%%%%%%%%%%%%%%%%%%%%%%%%%%%%%%%%%%%%%%%%%%%%%%%%%%%%%%

    \item {[24 pts.]} Draw a valid, complete, deterministic finite automaton
          (DFA) that recognizes or accepts the language:

          \vspace{-2mm}

          \begin{centering}

            \(
              \left\{
                \ \:
                \begin{array}[c]{@{}l}
                w
                \end{array}
                \left| \ \ \:
                w \sms \in \sms \{ \sms a, \, b \sms \}^*, \:
                \textrm{and the \textbf{second--to--last} symbol of
                        \emph{w} is an \emph{a}}
              \ \:
              \right.
              \right\}
            \)

            \vspace{1.25mm}

          \end{centering}

          Strings in this language can be of any length and can have any
          number of \emph{a}'s and \emph{b}'s, as long as they satisfy the
          property above.  In particular, all the symbols of a string other
          than the second--to--last one can be anything at all.  Note that
          any string that doesn't have a second--to--last character can
          never have this property.  Some examples are:

          \vspace{-2.75mm}

          \begin{itemize}

            \addtolength{\itemsep}{-.5mm}

            \item The strings \emph{aaaaa}, \emph{bbbab}, and \emph{babab}
                  are all in this language, since the second--to--last
                  symbol of each of them is an \emph{a}.

            \item The strings \emph{bbbbb}, \emph{aaaba}, and \emph{ababa}
                  are all \textbf{not} in this language, since none of them
                  have a second--to--last symbol that is an \emph{a}.

          \end{itemize}

          \vspace{-2.5mm}

          Note these examples do \textbf{\underline{not}} illustrate all
          possible valid or invalid strings.

          Be sure to give a \textbf{complete} DFA, not an NFA, and
          \textbf{do not use any notational shortcuts}.  \emph{\textbf{Write
          neatly}} for your answer to be graded.

          % -- figure out your DFA in the scratch area first, then write it
          % here.

          \pagebreak

    %%%%%%%%%%%%%%%%%%%%%%%%%%%%%%%%%%%%%%%%%%%%%%%%%%%%%%%%%%%%%%%%%%%%%%%%%%

    \item {[18 pts.]} Consider the following programming problem.  We need
          to read the transitions of an NFA from a text file and count how
          many outgoing transitions the state that is named ``\texttt{S0}''
          has.  A transition is a line of the form (\emph{from\_state},
          \emph{letter}, \emph{to\_state}), which indicates a transition
          from \emph{from\_state} on the symbol \emph{letter} to the state
          \emph{to\_state}.  State names may contain uppercase and lowercase
          letter and digit characters and may be of length 1 or more, while
          \emph{letter} is a single lowercase letter.  No other characters
          can appear on an input transition line, and no whitespace may
          appear other than the newline terminating each line.  If any input
          lines are not of this form they represent invalid transitions and
          are to be ignored.

          Write a complete Ruby program that reads a sequence of such lines
          from its standard input, and prints the number of
          \textbf{outgoing} transitions from the state named \texttt{S0};
          there may be zero or more.  Don't try to ignore duplicate
          transitions-- every transition that is outgoing from \texttt{S0}
          should be counted, regardless of what letter the transition is for
          or which state it goes to.  You may use any Ruby standard library
          functions.  Minor mistakes in syntax will be ignored if you have
          the correct concept.

          \pagebreak

    %%%%%%%%%%%%%%%%%%%%%%%%%%%%%%%%%%%%%%%%%%%%%%%%%%%%%%%%%%%%%%%%%%%%%%%%%%

    \item {[26 pts.]} \label{re-problem} Write a regular expression that
          describes or recognizes the language:

          \begin{centering}

            \vspace{.25mm}

            \(
              \left\{
                \ \:
                \begin{array}[c]{@{}l}
                w
                \end{array}
                \left| \ \ \:
                \parbox[c]{3.5in}{
                  $w \sms \in \sms \{ \sms a, \, b , \, c , \, d \sms \}^*$,
                  and \emph{w} does not contain any \emph{a} that is adjacent
                  to a \emph{d} (or vice versa of course)
                }
              \ \:
              \right.
              \right\}
            \)

            \vspace{1.5mm}

          \end{centering}

          Strings in this language can be of any length, and can contain any
          number of zero or more \emph{a}'s, \emph{b}'s, \emph{c}'s, and
          \emph{d}'s, as long as they satisfy the property above.  Note that
          there are no restrictions on the \emph{b}'s and \emph{c}'s in a
          string.  Some examples are:

          \vspace{-2.75mm}

          \begin{itemize}

            \addtolength{\itemsep}{-.5mm}

            \item The string \emph{abcddcba} is in this language, since none
                  of its \emph{a}'s and \emph{d}'s are adjacent.

            \item The string \emph{aabbccaabbcc} is also trivially in this
                  language, as it doesn't even have any \texttt{d}'s.

            \item \emph{bbbbb} also trivially has the property above, so
                  it's in the language as well.

            \item The string \emph{abcdabcd} is \textbf{not} in this
                  language, because its first \texttt{d} is adjacent to its
                  second \emph{a}.

          \end{itemize}

          \vspace{-2.5mm}

          Note these examples do \textbf{\underline{not}} illustrate all
          possible valid or invalid strings.

          Your regular expression may \textbf{only} use the three formal
          regular expression operations concatenation, alternation, and
          Kleene closure, which were defined in class.  These operations can
          be nested, and terminal symbols, parentheses, and $\epsilon$ may
          be used as well, but do \textbf{not} use any other regular
          expression operations, and do \textbf{not} write a Ruby regular
          expression.

         \pagebreak

    %%%%%%%%%%%%%%%%%%%%%%%%%%%%%%%%%%%%%%%%%%%%%%%%%%%%%%%%%%%%%%%%%%%%%%%%%%

    \item {[16 pts.]} Writing an OCaml function.

          \vspace{-2.75mm}

          \begin{enumerate}

            \addtolength{\itemsep}{3in}

            \item Write an OCaml function named \texttt{nth} that has a
                  tuple of an \texttt{int} named \texttt{n} and a list as a
                  parameter and that returns the \texttt{n}'th element of
                  the list.  The list's first element is considered to be
                  element number 1.  For example:

                  \vspace{-.75mm}

                  \begin{itemize}

                    \addtolength{\itemsep}{-.25mm}

                    \item \texttt{nth (1, [2; 4; 6; 8; 10])} would return
                          \texttt{2},

                    \item \texttt{nth (2, [2; 4; 6; 8; 10])} would return
                          \texttt{4}, and

                    \item \texttt{nth (3, ["hi"; "ciao"; "bye"])} would
                          return \texttt{"bye"}.

                  \end{itemize}

                  \vspace{-.75mm}

                  You can assume the list will always have an \texttt{n}'th
                  element to be returned, and you can also assume that
                  $\mathtt{n} > 0$.  It doesn't matter if your function
                  would generate any incomplete match warnings.

            \item Give the exact type of the function \texttt{nth} that you
                  wrote above.

          \end{enumerate}

          \vspace*{8mm}

          \underline{\hspace{6.6225in}}

          \medskip

          \begin{centering}

            \scratchpaper[6.8in]

          \end{centering}

          \pagebreak

    %%%%%%%%%%%%%%%%%%%%%%%%%%%%%%%%%%%%%%%%%%%%%%%%%%%%%%%%%%%%%%%%%%%%%%%%%%

  \end{enumerate}

  % end of numbered questions

  % \markright{}

  % \begin{center}
  % 
  %   \scratchpaper
  % 
  %   % \bigskip
  %   % 
  %   % \Large
  %   % 
  %   % You can separate this page if you like, to use in solving earlier
  %   % \linebreak
  %   % questions, as long as you \textbf{\underline{write your name} on the
  %   % other side.}
  % 
  % \end{center}

   %%%%%%%%%%%%%%%%%%%%%%%%%%%%%%%%%%%%%%%%%%%%%%%%%%%%%%%%%%%%%%%%%%%%%%%%%%

\end{document}
