\documentclass[11pt]{article}

  \usepackage{330-f12}

  \usepackage{multicol}

  \newcommand{\quest}[3]{%
    \begin{tabular}[t]{@{}p{#1}@{\hspace{3mm}}%
                         @{\underline{\hspace{#2}}}@{}}%
                       \sloppy#3\end{tabular}%
  }

\begin{document}

  % We have to use \header{}{}{}, since \maketitle generates a new page
  % beforehand.
  %
  \header{\course}{Exam \#2 practice questions \#1}{\Term}

  \noindent
  {\large
    \textbf{\underline{Do not open this exam until you are told.}
            Read these instructions:
    }
  }

   \vspace{-2mm}

  \thispagestyle{myheadings}

  \begin{enumerate}

    \addtolength{\itemsep}{-2mm}  % reduced space between lines

    \item This is a closed book exam.  {\bf No
          % calculators,
          notes %,
          or other aids are allowed.}  If you have a question during the
          exam, please raise your hand.  Each question's point value is next
          to its number.

    \item {\bf You must turn in your exam \underline{immediately} when time is
          called at the end.}

    % \item 6 pages, 5 problems, 100 points, 50 minutes.

    \item In order to be eligible for as much partial credit as possible,
          show all of your work for each problem, and \textbf{clearly
          indicate} your answers.  Credit \textbf{cannot} be given for
          illegible answers.

    \item You will \textbf{lose credit} if \emph{\textbf{any}} information
           above is incorrect or missing, or your name is missing from any
           side of any page.

    \item Parts of this page and of two other pages are
    % \item Part of this page, and part of the next page, are
          for scratch work.  If you need extra scratch paper
    % \item Parts of this page and of another page are
    % \item Part of this page, and the last page, are
    %       for scratch work.  If you need extra scratch paper
    % \item Part of this page is
    %       for scratch work.  If you need extra scratch paper
    % \item Part of this page, and parts of two other pages, are
    %       for scratch work.  If you need extra scratch paper
          after you have filled
          % this page
          % these pages
          these areas
          % this area
          up, please raise your hand.
          Scratch paper must be turned in with your exam, with your name and
          ID number written on it.
          % \textbf{Nothing written on scratch paper will be graded.}
          Scratch paper \textbf{will not} be graded.
          % Figure out your answer on scratch paper if necessary, then write
          % it {\bf neatly} in the answer space provided.

    \item To avoid distracting others, {\bf no one} may leave until the exam
          is over.

    \item The Campus Senate has adopted a policy asking students to include
          the following handwritten statement on each examination and
          assignment in every course: ``\textit{I pledge on my honor that I
          have not given or received any unauthorized assistance on this
          examination\/}.''  Therefore, \textbf{just before turning in your
          exam}, you are requested to write this pledge \textbf{in full
          \textmd{and} sign it} below:

          \medskip

          \begin{minipage}[t]{6.6in}

            \addtolength{\baselineskip}{4mm}

            \underline{\hspace{6.6in}}

            \underline{\hspace{6.6in}}

          \end{minipage}

    \medskip

  \end{enumerate}

  \vspace{-2mm}

  \noindent
  Good luck!

  % \enlargethispage{6mm}
  % 
  % \vspace{\fill}
  % 
  % \noindent
  % \begin{tabular}[t]{@{}p{2.75in}@{\hspace{.5in}}p{3.75in}@{}}
  % 
  %   \vspace{8mm}
  % 
  %   \scratchpaper[2.75in]
  % 
  % &
  % 
  %   \vspace{0mm}
  % 
  %   % 4 exam questions, .5mm scale factor
  %   \hspace{\fill}\parbox[t]{3.325in}{\scoreblank{5}{.45mm}}
  %   % \scoreblank{4}{.5mm}
  % 
  %   %\vspace{-.75in}
  % 
  % \end{tabular}

  \pagebreak

    %%%%%%%%%%%%%%%%%%%%%%%%%%%%%%%%%%%%%%%%%%%%%%%%%%%%%%%%%%%%%%%%%%%%%%%%%%

  \pagestyle{headings}

  \markright{Name: \hspace{.2mm} \lans\ans}

  \begin{enumerate}

    %%%%%%%%%%%%%%%%%%%%%%%%%%%%%%%%%%%%%%%%%%%%%%%%%%%%%%%%%%%%%%%%%%%%%%%%%%

    \item Suppose you dropped and broke your cell phone, so you don't have
          your contact list or phone list any more.  However, you do have
          the OCaml interpreter downloaded onto your computer.  After you
          have finished weeping over your broken phone, write an OCaml
          function that can be used to create a contact list, for later
          lookup:

          \vspace{-2.5mm}

          \begin{itemize}

            \addtolength{\itemsep}{1mm}

            \item Your function should be called
                  \texttt{create\_contact\_list} and it should have one
                  parameter, a list of strings.  The list consists of a
                  sequence of strings, alternating between your friends'
                  first names (in no particular order) and their phone
                  numbers.

            \item Your function \texttt{create\_contact\_list} should
                  \textbf{create and return a function} that can then be
                  used to look up your friends' phone numbers, given their
                  first names.  For instance:

                  \smallskip

                  \begin{Verbatim}
        let lookup_friends = create_contact_list ["Tommy"; "(301) 123-4567";
                                                  "Tammy"; "(301) 111-2222";
                                                  "Derek"; "(301) 987-6543";
                                                  "Varun"; "(301) 222-3333"];;

        lookup_friends "Derek";;
        - : string = "(301) 987-6543"
        lookup_friends "Tommy";;
        - : string = "(301) 123-4567"
                  \end{Verbatim}

            \item You may assume the list passed into
                  \texttt{create\_contact\_list} will be valid and consist
                  of alternating names and phone numbers as described.  You
                  may assume the function it returns will only be called
                  with names that are in the list, and that your friends all
                  have unique first names.  (You can always get rid of
                  friends who have the same first names as others.)

          \end{itemize}

          \vspace{-2.5mm}

          It doesn't matter if your function would cause incomplete match
          warnings.  To receive full credit, your solution \textbf{should
          not use any references}.

          \pagebreak

    %%%%%%%%%%%%%%%%%%%%%%%%%%%%%%%%%%%%%%%%%%%%%%%%%%%%%%%%%%%%%%%%%%%%%%%%%%

    \item Write a context--free grammar that generates the following
          language:

          \begin{centering}

            \(
              \left\{
                \:
                {a^\mathrm{m}}{b^\mathrm{n}}
                \midspc
                \mathrm{m \, + \, n \ \, \textrm{is \textbf{not} a multiple
                        of 3}
                }
                \:
              \right\}
            \)

          \end{centering}

          For full credit, your grammar should be \textbf{unambiguous}.  In
          order that your answer can be graded as accurately and quickly as
          possible, please use consecutive nonterminals beginning with S (S,
          T, U, etc.) when writing your grammar.

          \vspace{4in}

          \underline{\hspace{6.8225in}}

          \medskip

          \begin{centering}

            \scratchpaper[6.8225in]

          \end{centering}

          \pagebreak

    %%%%%%%%%%%%%%%%%%%%%%%%%%%%%%%%%%%%%%%%%%%%%%%%%%%%%%%%%%%%%%%%%%%%%%%%%%

    \item Consider the following program, written in C synax:

          \vspace{-3.25mm}

          \begin{Verbatim}
        #include <stdio.h>

        int x= 1;

        int f(int a, int b) {
          int c= a + x;
          x += 1;
          x += b + 2;
          return a + c;
        }

        int main() {
          int y= 7;
          y= f(x + 1, y + x);
          printf("%d %d\n", x, y);
          return 0;
        }
          \end{Verbatim}

          \vspace{-1.5mm}

          \quest{4in}{2.7in}{What would the program print if C used
                             call--by--name as its default parameter
                             transmission method?}

          \vspace{8mm}

          \underline{\hspace{6.8225in}}

          \medskip

          \begin{centering}

            \scratchpaper[6.8225in]

          \end{centering}

          \pagebreak

    %%%%%%%%%%%%%%%%%%%%%%%%%%%%%%%%%%%%%%%%%%%%%%%%%%%%%%%%%%%%%%%%%%%%%%%%%%

    \item Java, subtyping, and generics.

          \vspace{-2mm}

          \begin{enumerate}

            \addtolength{\itemsep}{12mm}

            \item Consider the following Java program:

                  \vspace{-2.75mm}

                  \begin{tabular}[t]{@{}p{4in}@{\hspace{.6in}}p{1.8in}@{}}

                    \begin{Verbatim}
        class A {
          void g() {
            System.out.println("g1");
          }
        }
        
        class B extends A {
          void g() {
            System.out.println("g2");
          }
        }
        
        class C {
          static void f(A a) {
            System.out.println("f1");
          }
        
          static void f(B b) {
            System.out.println("f2");
          }
        
          public static void main(String[] args) {
            A x= new A();
            A y= new B();
            B z= new B();
        
            f(x);
            f(y);
            f(z);
        
            x.g();
            y.g();
            z.g();
          }
        }
                    \end{Verbatim}

                    & Give the program's complete output below:

                      \bigskip

                      \answerblank{8}{1.8in}{9mm}

                  \end{tabular}

                  \vspace{-6mm}

            \item Suppose the following method is added to class \texttt{C}:

                  \smallskip

                  \begin{Verbatim}
        static <T extends B> void h(T t) {
          System.out.println("h");
        }
                  \end{Verbatim}

                  \medskip

                  Which, if any, of \texttt{x}, \texttt{y}, and \texttt{z}
                  could be passed into \texttt{h}?  ~ \lans

            \item What kind of polymorphism are Java's generics an example
                  of?

                  \bigskip\medskip

                  \lans\ans

          \end{enumerate}

          \pagebreak

    %%%%%%%%%%%%%%%%%%%%%%%%%%%%%%%%%%%%%%%%%%%%%%%%%%%%%%%%%%%%%%%%%%%%%%%%%%

  \end{enumerate}

  % end of numbered questions

  % \markright{}

  % \begin{center}
  %
  %   \scratchpaper
  %
  %   % \bigskip
  %   %
  %   % \Large
  %   %
  %   % You can separate this page if you like, to use in solving earlier
  %   % \linebreak
  %   % questions, as long as you \textbf{\underline{write your name} on the
  %   % other side.}
  %
  % \end{center}

   %%%%%%%%%%%%%%%%%%%%%%%%%%%%%%%%%%%%%%%%%%%%%%%%%%%%%%%%%%%%%%%%%%%%%%%%%%

\end{document}
