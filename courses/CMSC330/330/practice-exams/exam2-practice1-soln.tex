\documentclass[11pt,fleqn]{article}

  \usepackage{330-f12}

  \usepackage{pst-tree}

  \hideinfo

\begin{document}

  \header{\course}{Exam \#2 practice questions \#1-- solutions}{\Term}

  \begin{enumerate}

    \addtolength{\itemsep}{15mm}

    %%%%%%%%%%%%%%%%%%%%%%%%%%%%%%%%%%%%%%%%%%%%%%%%%%%%%%%%%%%%%%%%%%%%%%%%%%

    % \item \begin{enumerate}
    % 
    %         \vspace{-2.25mm}
    % 
    %         \addtolength{\itemsep}{6mm}
    % 
    %         \item The result would be 8.  The rule is saying that in an
    %               application of a closure to an argument, variable bindings
    %               in the closure'e environment take precedence over the
    %               binding of the value of the argument to the closure's
    %               parameter.  In other words, it's like an inverse static
    %               scoping, in which outermost variable bindings take
    %               precedence over inner ones.  (Of course this behavior
    %               would make no sense for a real language, it's just to test
    %               understanding of operational semantics rules.)
    % 
    %               Note that the original rule (which describes what real
    %               Scheme would use) the function application would return 6,
    %               as the outer x and its value 4 are shadowed by the inner x
    %               and its value 3.
    % 
    %         \item \begin{enumerate}
    % 
    %                 \vspace{-2.75mm}
    % 
    %                 \addtolength{\itemsep}{4mm}
    % 
    %                 \item $\mathrm{\lambda y.z \: y}$
    % 
    %                 \item $\mathrm{\lambda z.y \: z}$ (where z can be any
    %                       variable other than y)
    % 
    %                 \item $\mathrm{\lambda y.y}$
    % 
    %               \end{enumerate}
    % 
    %               \vspace{-1mm}
    % 
    %         \item In translation via erasure Java replaces type parameters
    %               with \texttt{Object}, since any type which the type
    %               parameter is instantiated with will be a subclass of
    %               \texttt{Object}.  C++ does not have a class like
    %               \texttt{Object} which is the root of the entire
    %               inheritance hierarchy, without which this approach won't
    %               work.
    % 
    %       \end{enumerate}
    % 
    %       \begin{info}{\textbf{\underline{Grading scale:}}}{6.5in}
    % 
    %         \begin{itemize}
    % 
    %           \addtolength{\itemsep}{1mm}
    % 
    %           \item Part (a) is \pts{8}, part (b) is \pts{10}, and part (c) is
    %                 \pts{8}
    % 
    %           \item For part (a):
    % 
    %                 \begin{itemize}
    % 
    %                   \addtolength{\itemsep}{1mm}
    % 
    %                   \item Give full credit if the answer is 8 and a
    %                         reasonable explanation is given.
    % 
    %                   \item Give \pts{2} for giving the correct numeric answer
    %                         8, if the explanation is missing or incorrect.
    % 
    %                   \item If the numeric answer is wrong (e.g., 6), but
    %                         somehow the explanation is correct (maybe the
    %                         student just made a calculation mistake, or got
    %                         their x's mixed up), then give \pts{4}.
    % 
    %                 \end{itemize}
    % 
    %           \item For part (b):
    % 
    %                 \begin{itemize}
    % 
    %                   \addtolength{\itemsep}{1mm}
    % 
    %                   \item Part (i) is \pts{2}, part (ii) is \pts{3}, and
    %                         part (iii) is \pts{5}.
    % 
    %                   \item If the answer is wrong but is partly right, in
    %                         that the student did some reductions correctly
    %                         (e.g., in (ii) the answer could be
    %                         $\mathrm{\lambda y.y \: y}$, or in (iii) some
    %                         reductions might have been carried out correctly
    %                         but not all, or the result wasn't completely
    %                         reduced), then give half credit for that part
    %                         (rounding upward for parts with odd point values).
    % 
    %                 \end{itemize}
    % 
    %           \item For part (c), if you find a case where partial credit
    %                 seems appropriate then give it, using uniform and
    %                 consistent criteria (and keeping track of the criteria).
    % 
    %         \end{itemize}
    % 
    %       \end{info}

    %%%%%%%%%%%%%%%%%%%%%%%%%%%%%%%%%%%%%%%%%%%%%%%%%%%%%%%%%%%%%%%%%%%%%%%%%%

    \item Here are three slightly different solutions:

          \vspace{-2.75mm}

          \begin{Verbatim}
        let create_contact_list names =
          let rec search list name =
            match list with first::phone::rest ->
              if name = first
                then phone
                else search rest name
          in search names

        let rec create_contact_list names =
          let rec lookup name =
            match names with first::phone::rest ->
              if first = name
                then phone
                else ((create_contact_list rest) name)
          in lookup

        let rec create_contact_list names =
          fun name -> match names with
            first::phone::rest ->
              if first = name
                then phone
                else ((create_contact_list rest) name)
          \end{Verbatim}

          \vspace{-1mm}

          \begin{info}{\textbf{\underline{Grading scale:}}}{6.5in}

            \begin{itemize}

              \addtolength{\itemsep}{1mm}

              \item Just defining a function with the correct name (the first
                    line) is \pts{2}

              \item Defining a (recursive) function inside the function, with
                    the proper parameters, using any correct syntax (anonymous
                    or named) is \pts{5}.

              \item Correctly using \texttt{match} in the inner function on
                    the list of names, to be able to extract the first name
                    and phone number, and the rest of the list, is \pts{4}

              \item The base case (returning the phone number corresponding to
                    the first name in the list if it's equal to the inner
                    function's parameter, including testing that the first
                    name in the list is equal to the parameter) is \pts{4}

              \item The recursive case (the inner function calling itself on
                    the rest of the list if the first name in the list isn't
                    equal to the inner function's parameter) is \pts{4}

              \item Returning the inner function is \pts{5}

            \end{itemize}

          \end{info}

          \begin{info}{\textbf{\underline{Grading notes:}}}{6.5in}

            \begin{itemize}

              \addtolength{\itemsep}{1mm}

              \item Ignore minor mistakes in syntax.

              \item Give partial credit for the above tasks as appropriate.
                    For instance, if a recursive function is defined inside
                    the outer function but its parameters aren't correct,
                    give part of the second \pts{5} for that tasks, using
                    some consistent criteria (and keep track of decisions in
                    such cases).

            \end{itemize}

          \end{info}

    %%%%%%%%%%%%%%%%%%%%%%%%%%%%%%%%%%%%%%%%%%%%%%%%%%%%%%%%%%%%%%%%%%%%%%%%%%

    \item One note before giving the solution: recall from
          CMSC 250 that zero is a multiple of 3 (the multiples of 3 are $\{
          \ldots, \, -9, \, -6, \, -3, \, 0, \, 3, \, 6, \, 9, \,
          \ldots\}$).  Therefore $\largeepsilon$ is not a valid string in this
          language.

          % This is a language which is regular, but it's atypical
          % in that for this language it's actually probably easier to just
          % write the grammar directly, rather than writing the DFA and
          % converting it to a regular grammar, which is usually simpler
          % (although that could definitely work though, as shown below).

          Your first thought might be to write a grammar that looked
          something like the following, which would generate strings where
          $\mathrm{m \, + \, n}$ \textbf{is} a multiple of 3 and then
          generate either one or two extra \emph{a}'s or \emph{b}'s, thereby
          forcing $\mathrm{m \, + \, n}$ to \textbf{not} be a multiple of 3:

          \vspace{-2mm}

          \raggedcolumns

          \begin{multicols}{2}

            \begin{center}

              \begin{grammar}[1.5]
                \production{S}{\mathit{aaa}S \midspc \mathit{aa}S\mathit{b}
                               \midspc \mathit{a}S\mathit{bb} \midspc
                               S\mathit{bbb} \midspc T}
                  \\
                \production{T}{\mathit{a} \midspc \mathit{aa} \midspc
                               \mathit{b} \midspc \mathit{bb} \midspc
                               \mathit{ab}}
                  \\
              \end{grammar}

            \end{center}

            \begin{center}

              \begin{grammar}[1.5]
                \production{S}{\mathit{a}T \midspc \mathit{aa}T \midspc
                               T\mathit{b} \midspc T\mathit{bb} \midspc
                               \mathit{a}T\mathit{b}}
                  \\
                \production{T}{\mathit{aaa}T \midspc \mathit{aa}T\mathit{b}
                               \midspc \mathit{a}T\mathit{bb} \midspc
                               T\mathit{bbb} \midspc \largeepsilon}
                  \\
              \end{grammar}

            \end{center}

          \end{multicols}

          \vspace{-2mm}

          The problem with these grammars is that they're ambiguous.  For
          example, in the grammar on the left:

          \vspace{-2.75mm}

          {

            \psset{unit=1mm,levelsep=12,treesep=10,nodesep=1.5}

            \raggedcolumns

            \begin{multicols}{2}

                \pstree{\TR{S}}{
                  \pstree{\TR{\emph{a}}}{
                  }
                  \pstree{\TR{\emph{a}}}{
                  }
                  \pstree{\TR{\emph{a}}}{
                  }
                  \pstree{\TR{S}}{
                    \pstree{\TR{\emph{a}}}{
                    }
                    \pstree{\TR{S}}{
                      \pstree{\TR{\emph{a}}}{
                      }
                    }
                    \pstree{\TR{\emph{b}}}{
                    }
                    \pstree{\TR{\emph{b}}}{
                    }
                  }
                }

            \columnbreak

              \begin{center}

                \pstree{\TR{S}}{
                  \pstree{\TR{\emph{a}}}{
                  }
                  \pstree{\TR{S}}{
                    \pstree{\TR{\emph{a}}}{
                    }
                    \pstree{\TR{\emph{a}}}{
                    }
                    \pstree{\TR{\emph{a}}}{
                    }
                    \pstree{\TR{S}}{
                      \pstree{\TR{\emph{a}}}{
                      }
                    }
                  }
                  \pstree{\TR{\emph{b}}}{
                  }
                  \pstree{\TR{\emph{b}}}{
                  }
                }

              \end{center}

            \end{multicols}

          }

          \vspace{-7.75mm}

          Here are several unambiguous versions:

          \vspace{-2mm}

          \raggedcolumns

          {

          \setlength{\columnsep}{30mm}

          \begin{multicols}{3}

            \begin{center}

              \begin{grammar}[1.5]
                \production{S}{\mathit{aaa}S \midspc T}
                  \\
                \production{T}{\mathit{aa}T\mathit{b} \midspc U}
                  \\
                \production{U}{\mathit{a}U\mathit{bb} \midspc V}
                  \\
                \production{V}{V\mathit{bbb} \midspc W}
                  \\
                \production{W}{\mathit{a} \midspc \mathit{aa} \midspc
                               \mathit{b} \midspc \mathit{bb} \midspc
                               \mathit{ab}}
                  \\
              \end{grammar}

            \end{center}

          \columnbreak

            \begin{center}

              \begin{grammar}[1.5]
                \production{S}{S\mathit{bbb} \midspc T}
                  \\
                \production{T}{\mathit{a}T\mathit{bb} \midspc U}
                  \\
                \production{U}{\mathit{aa}U\mathit{b} \midspc V}
                  \\
                \production{V}{\mathit{aaa}V \midspc W}
                  \\
                \production{W}{\mathit{a} \midspc \mathit{aa} \midspc
                               \mathit{b} \midspc \mathit{bb} \midspc
                               \mathit{ab}}
                  \\
              \end{grammar}

            \end{center}

          \end{multicols}

          }

          \bigskip\bigskip

              \begin{grammar}[1.5]
                \production{S}{\mathit{a}T \midspc \mathit{aa}T \midspc
                               T\mathit{b} \midspc T\mathit{bb} \midspc
                               \mathit{a}T\mathit{b}}
                  \\
                \production{T}{\mathit{aaa}T \midspc U}
                  \\
                \production{U}{\mathit{aa}U\mathit{b} \midspc V}
                  \\
                \production{V}{\mathit{a}V\mathit{bb} \midspc W}
                  \\
                \production{W}{W\mathit{bbb} \midspc \largeepsilon}
                  \\
              \end{grammar}

              \bigskip\bigskip

          % \vspace{-3mm}
          % 
          % Below is a DFA which recognizes the language, and a grammar which
          % is generated from it, using the construction shown in discussion
          % section.  Note that the DFA keeps track of the number of
          % \emph{a}'s generated and whether they're a multiple of 3, and once
          % a \emph{b} is seen it's able to keep track of whether the
          % remainder of the string consists of only \emph{b}'s and would
          % preserve the total string length as being not a multiple of 3:
          % 
          % \pagebreak
          % 
          % \begin{multicols}{2}
          % 
          %   \begin{automaton}(0,0)(80,91)
          % 
          %     \state[start,label=S](5,75){S}
          %     \state[final,label=T](35,75){T}
          %     \state[final,label=U](65,75){U}
          %     \state[label=V](5,45){V}
          %     \state[final,label=W](35,45){W}
          %     \state[final,label=X](65,45){X}
          %     \state[label=Y](35,25){Y}
          % 
          %     \transition(S,\emph{a},T)
          %     \transition(S,\emph{b},W)
          % 
          %     \transition(T,\emph{a},U)
          %     \transition(T,\emph{b},X)
          % 
          %     \transition[labellocation=below,angle=-35,curvature=.67]%
          %                 (U,\emph{a},S)
          %     \transition[labellocation=below](U,\emph{b},V)
          % 
          %     \transition[labellocation=below](V,\emph{a},Y)
          %     \transition(V,\emph{b},W)
          % 
          %     \transition(W,\emph{a},Y)
          %     \transition(W,\emph{b},X)
          % 
          %     \transition(X,\emph{a},Y)
          %     \transition[curved,curvature=1.65,angle=70,labellocation=below]%
          %                 (X,\emph{b},V)
          % 
          %     \transition[loopdirection=sw,labelposition=.8,labeloffset=2]%
          %                 (Y,\emph{a},Y)
          %     \transition[loopdirection=se,labelposition=.8](Y,\emph{b},Y)
          % 
          %   \end{automaton}
          % 
          % \columnbreak
          % 
          %   \begin{center}
          % 
          %     \begin{grammar}[1.5]
          %       \production{S}{\mathit{a}T \midspc \mathit{a} \midspc
          %                      \mathit{b}W \midspc \mathit{b}}
          %         \\
          %       \production{T}{\mathit{a}U \midspc \mathit{a} \midspc
          %                      \mathit{b}X \midspc \mathit{b}}
          %         \\
          %       \production{U}{\mathit{a}S \midspc \mathit{b}V}
          %         \\
          %       \production{V}{\mathit{a}Y \midspc \mathit{b}W \midspc
          %                      \mathit{b}}
          %         \\
          %       \production{W}{\mathit{a}Y \midspc \mathit{b}X \midspc
          %                      \mathit{b}}
          %         \\
          %       \production{X}{\mathit{a}Y \midspc \mathit{b}V}
          %         \\
          %       \production{Y}{\mathit{a}Y \midspc \mathit{b}Y}
          %         \\
          %     \end{grammar}
          % 
          %   \end{center}
          % 
          %   Note: the productions for Y would be optional (the grammar would
          %   generate the same language if they weren't shown).
          % 
          % \end{multicols}

          Any completely--correct answer should have the following
          properties:

          \vspace{-2mm}

          \begin{enumerate}

            \item[p1:] (completeness) It generates every valid
                       string.

            \item[p2:] (correctness) It generates only valid
                       strings (it does not generate any invalid strings).

            \item[p3:] (ambiguity) It's unambiguous.

          \end{enumerate}

          \vspace{-2mm}

          \begin{info}{\textbf{\underline{Grading scale:}}}{6.5in}

            \begin{itemize}

              \addtolength{\itemsep}{1mm}

              \item \begin{itemize}

                      \addtolength{\itemsep}{1mm}

                      \item Each property above which the grammar fully
                            enforces is full--credit for that property.

                      \item Each property above which the grammar partly
                            enforces or attempts to enforce, but there is one
                            mistake in the way that property is handled in the
                            grammar, is \pts{-2} for that property.

                      \item Each property above which the grammar partly
                            enforces or attempts to enforce, but there are
                            two mistakes in the way that property is handled
                            in the grammar, is \pts{-5} for that property.

                      \item Each property above which the grammar has more
                            than two mistakes trying to enforce, or which
                            the grammar makes no attempt to enforce, is no
                            credit for that property.

                 \end{itemize}

            \item Deduct \pts{1} for minor mistakes such as using angle
                  braces around terminals (\verb@<a>@, \verb@<b>@).

                  % Add any new properties to the list as necessary while
                  % grading, and let me know their names/numbers.

            \end{itemize}

          \end{info}

          \vspace{-1.5mm}

          \begin{info}{\textbf{\underline{Grading notes:}}}{6.5in}

            \begin{itemize}

              \addtolength{\itemsep}{1mm}

              \item For ambiguity look for whether some productions
                    incorrectly contain nonterminals of several other
                    productions as right--side alternatives, for example
                    something like

                    \production{S}{\mathit{aaa}S \midspc T \midspc U}

                    where T also goes to U.  Also be careful whenever any
                    productions go to nonterminals of earlier productions,
                    such as in
                    \production{T}{\mathit{aa}T\mathit{b} \midspc U \midspc S}

                    Nothing prevents derivations containing repeated
                    cycles of
                    \(
                      \mathrm{S} \: \Longrightarrow \:
                      \mathrm{T} \: \Longrightarrow \:
                      \mathrm{S} \: \Longrightarrow \:
                      \mathrm{S} \: \Longrightarrow \:
                      \mathrm{T} \: \Longrightarrow \ldots
                    \)

                    Needless to say, any production in which the left--side
                    nonterminal is also a right--side alternative, as in
                    \production{T}{\mathit{aa}T\mathit{b} \midspc U \midspc T}
                    is about as ambiguous as you can get.

              \item There's no penalty for extra productions.  In other words,
                    the grammar need not be minimal and there is no penalty
                    for useless or unnecessary nonterminals.

              \item Consider each mistake in productions in isolation, and
                    pretend that mistake isn't present when considering other
                    mistakes.  Otherwise one mistake in one production could
                    cause the grammar not to generate any strings, although
                    other than that one problem the grammar might be perfect.

                    Another way to look at this would be to consider each
                    mistake in a production (or each thing omitted in a
                    production) to be a single mistake, rather than
                    considering each form of string generated or not generated
                    to be a single mistake.

              \item If the answer uses regular expressions in productions
                    deduct \pts{-5} from property p1.

            \end{itemize}

          \end{info}

    %%%%%%%%%%%%%%%%%%%%%%%%%%%%%%%%%%%%%%%%%%%%%%%%%%%%%%%%%%%%%%%%%%%%%%%%%%

    \item \texttt{13 17}

          \begin{info}{\textbf{\underline{Grading scale:}}}{6.5in}

            \begin{itemize}

              \addtolength{\itemsep}{1mm}

              \item Each correct value is \pts{5}

              \item If you can tell that the student knew what
                    call--by--name is and how it works, and set things up
                    correctly, but just made an arithmetic error, then
                    deduct \pts{-2} for each value incorrect as a result of
                    an arithmetic error.

            \end{itemize}

          \end{info}

    %%%%%%%%%%%%%%%%%%%%%%%%%%%%%%%%%%%%%%%%%%%%%%%%%%%%%%%%%%%%%%%%%%%%%%%%%%

    \item  \begin{enumerate}

             \addtolength{\itemsep}{8mm}

             \item \begin{Verbatim}
        f1
        f1
        f2
        g1
        g2
        g2
             \end{Verbatim}

             \item Only \texttt{z}.

             \item Parametric polymorphism.

           \end{enumerate}

           \begin{info}{\textbf{\underline{Grading scale:}}}{3.6in}

             \begin{itemize}

               \addtolength{\itemsep}{2mm}

               \item Part (a) is \pts{10}; the points for each output value
                     are as follows:

                     \begin{tabular}[t]{ll}

                       \texttt{f1}
                         & \pts{1}
                         \\

                       \texttt{f1}
                         & \pts{3}
                         \\

                       \texttt{f2}
                         & \pts{1}
                         \\

                       \texttt{g1}
                         & \pts{1}
                         \\

                       \texttt{g2}
                         & \pts{3}
                         \\

                       \texttt{g2}
                         & \pts{1}
                         \\

                     \end{tabular}

                     \smallskip

               \item Part (b) is \pts{3}; if the answer \texttt{z} is given
                     but extra incorrect answers are also given, deduct
                     \pts{-1} for each additional incorrect answer.

               \item Part (c) is \pts{2};I'm not sure what kind of situation
                     would deserver partial credit for part (c).

             \end{itemize}

           \end{info}

  \end{enumerate}

\end{document}
