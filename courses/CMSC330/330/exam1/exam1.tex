\documentclass[11pt]{article}

  %
  % 3:30 lecture n = 110
  %

  %
  % Can take draft exam:
  %
  % Tommy Tu 9:45-11, 12:30-1:45, 4:30+; Wed 1-2:15, 5+
  % Tammy anytime Wed
  % Derek Tuesday 9-12, 5+; Wed 4+
  % Varun 12-3 Tue
  % Hao
  %
  % Varun 12 Tuesday
  % Tommy 5 Tuesday
  % Tammy 1:45 Wednesday
  % Derek 4 Wednesday
  % Hao 4:30 Wednesday
  %

  %
  % The "good style and formatting" part could have been dropped (dropped in
  % the second exam).
  %

  %
  % Next time just drop the instruction about minor syntax errors being
  % ignored; students don't need to know it, and all it does is encourage
  % regrades.
  %

  %
  % #3 was actually isomorphic to the practice problem question
  % (practice04-automata2, #8).  This question could or should have been a
  % little different.  (Also, the originally-postged practice problem
  % solution was actually incorrect....)
  %

  \usepackage{330-f12}

  \usepackage[none,light,outline,timestamp]{draftcopy}

  \newcommand{\sms}{\ensuremath{\hspace{.1mm}}}

  \addtolength{\headsep}{-6mm}

  \addtolength{\textheight}{6mm}

\begin{document}

  \hspace{\fill}%
  \nameblock

  % We have to use \header{}{}{}, since \maketitle generates a new page
  % beforehand.
  %
  \header{\course}{Exam \#1}{\Term}

  \vspace{-5mm}

  \noindent
  {\large
    \textbf{\underline{Do not open this exam until you are told.}
            Read these instructions:
    }
  }

  \vspace{-2mm}

  \thispagestyle{myheadings}

  \begin{enumerate}

    \addtolength{\itemsep}{-2mm}  % reduced space between lines

    \item This is a closed book exam.  {\bf No
          % calculators,
          notes %,
          or other aids are allowed.}  If you have a question during the
          exam, please raise your hand.  Each question's point value is next
          to its number.

    \item {\bf You must turn in your exam \underline{immediately} when time is
          called at the end.}

    \item 8 pages, 5 problems, 100 points, 75 minutes.

    \item In order to be eligible for as much partial credit as possible,
          show all of your work for each problem, and \textbf{clearly
          indicate} your answers.  Credit \textbf{cannot} be given for
          illegible answers.

    \item You will \textbf{lose credit} if \emph{\textbf{any}} information
          above is incorrect or missing, or your name is missing from any
          side of any page.

    \item Parts of several pages, and the last page, are for scratch work.
          If you need extra scratch paper after you have filled these areas
          up, please raise your hand.  Scratch paper must be turned in with
          your exam, with your name and ID number written on it.  Scratch
          paper \textbf{will not} be graded.
          % Figure out your answer on scratch paper if necessary, then write
          % it {\bf neatly} in the answer space provided.

    \item To avoid distracting others, \textbf{no one may leave until the
          exam is over}.

    % \item To avoid distracting others, \textbf{no one may leave in the
    %       last 10 minutes}.

    \item The Campus Senate has adopted a policy asking students to include
          the following handwritten statement on each examination and
          assignment in every course: ``\textit{I pledge on my honor that I
          have not given or received any unauthorized assistance on this
          examination\/}.''  Therefore, \textbf{just before turning in your
          exam}, you are requested to write this pledge \textbf{in full
          \textmd{and} sign it} below:

          \medskip

          \begin{minipage}[t]{\linewidth}

            \addtolength{\baselineskip}{4mm}

            \ans[\linewidth]

            \ans[\linewidth]

          \end{minipage}

    \medskip

  \end{enumerate}

  \vspace{-2mm}

  \noindent
  Good luck!

  \enlargethispage{14mm}

  \vspace{\fill}

  \noindent
  \begin{tabular}[t]{@{}p{2.75in}@{\hspace{.25in}}p{4in}@{}}

    \vspace{8mm}

    \scratchpaper[2.75in]

  &

    \vspace{0mm}

    % 4 exam questions, .5mm scale factor
    \hspace{\fill}\parbox[t]{3.5in}{\scoreblank{5}{.475mm}}
    % \scoreblank{4}{.5mm}

    %\vspace{-.75in}

  \end{tabular}

  \pagebreak

  \addtolength{\headsep}{4mm}

  \addtolength{\textheight}{2mm}

    %%%%%%%%%%%%%%%%%%%%%%%%%%%%%%%%%%%%%%%%%%%%%%%%%%%%%%%%%%%%%%%%%%%%%%%%%%

  \pagestyle{headings}

  \markright{Name: \hspace{.2mm} \lans\ans}

  \begin{enumerate}

    %%%%%%%%%%%%%%%%%%%%%%%%%%%%%%%%%%%%%%%%%%%%%%%%%%%%%%%%%%%%%%%%%%%%%%%%%%

    \item {[20 pts.]} \label{re-question} Write a regular expression that
          describes or recognizes the language:

          \begin{centering}

            \vspace{-.5mm}

            \(
              \left\{
                \
                w \
                \left| \ \:
                w \sms \in \sms \{ \sms a, \, b \sms \}^*, \
                \textrm{and \emph{w} does not contain any individual
                        \emph{a}'s or any repeated \emph{b}'s}
                \
              \right.
              \right\}
            \)

          \end{centering}

          An individual \emph{a} is an \emph{a} that is not adjacent to any
          other \emph{a}, while repeated \emph{b}'s refer to sequences of
          more than one \emph{b} that are adjacent.

          Strings in this language may be of any length, and may contain any
          number of zero or more \emph{a}'s and \emph{b}'s, as long as they
          satisfy the property above.  Some examples are:

          \vspace{-2.5mm}

          \begin{itemize}

            \addtolength{\itemsep}{-.25mm}

            \item The string \emph{aabaaabaa} is in this language, since
                  every \emph{a} it contains is adjacent to another
                  \emph{a}, while there is no \emph{b} that is adjacent to
                  another \emph{b}.

            \item The string \emph{aaaaa} is also in the language.

            \item The string \emph{ab} is \textbf{not} in this language,
                  because it has an \emph{a} that is not adjacent to another
                  \emph{a}.

            \item The string \emph{aabb} is also \textbf{not} in this
                  language, as it contains two \emph{b}'s that are adjacent
                  to each other.

          \end{itemize}

          \vspace{-2.25mm}

          Note these examples do \textbf{\underline{\underline{not}}}
          illustrate all possible valid or invalid strings.  Base your
          regular expression on the definition of the language, not only on
          the examples.

          Your regular expression may \textbf{\underline{\emph{only}}} use
          the three formal regular expression operations concatenation,
          alternation, and Kleene closure, which were defined in class.
          These operations can be nested, and alphabet symbols, parentheses,
          and $\epsilon$ may be used as well, but \textbf{\underline{do
          not}} use any other regular expression operations, and
          \textbf{\underline{\underline{do not}}} write a Ruby regular
          expression.

          \vspace{2.5in}

          \ans[\linewidth]

          \medskip

          \begin{centering}

            \scratchpaper[\linewidth]

          \end{centering}

          \pagebreak

    %%%%%%%%%%%%%%%%%%%%%%%%%%%%%%%%%%%%%%%%%%%%%%%%%%%%%%%%%%%%%%%%%%%%%%%%%%

    \item {[26 pts.]} Give a valid, complete, deterministic finite automaton
          (DFA) that recognizes or accepts the language from Problem
          \#\ref{re-question}.  (Show the DFA by drawing its transition
          diagram, rather than listing out all its components.)  It is not
          necessary to ensure that your DFA is minimal or to minimize it;
          it just needs to be correct.

          Be sure to give a \textbf{complete} DFA, not an NFA, and
          \textbf{\underline{do not} use any notational shortcuts}.
          \emph{\textbf{\underline{\underline{Write}}
          \underline{\underline{neatly}}}} for your DFA to be graded.
          Clearly indicate what your final DFA answer is (as opposed to any
          scratch work).

          \pagebreak

    %%%%%%%%%%%%%%%%%%%%%%%%%%%%%%%%%%%%%%%%%%%%%%%%%%%%%%%%%%%%%%%%%%%%%%%%%%

    % new

    \item {[12 pts.]} Apply the construction given in class to produce an
          NFA from the regular expression
          \(
            (a{\mid}c)^*b
          \).
          (Show the NFA by drawing its transition diagram, rather than
          listing out all its components.)  \textbf{Note}: do \textbf{not}
          simplify the resulting NFA at all, and do \textbf{not} create a
          DFA.  The purpose of this problem is not to see whether you can
          create an NFA or DFA ``by hand'' for this language, but to apply
          the specific construction that was given to transform a regular
          expression to an NFA.

          \pagebreak

    %%%%%%%%%%%%%%%%%%%%%%%%%%%%%%%%%%%%%%%%%%%%%%%%%%%%%%%%%%%%%%%%%%%%%%%%%%

    % s07

    \item {[12 pts.]} OCaml.

          \vspace{-2.75mm}

          \begin{enumerate}

            \addtolength{\itemsep}{10mm}

            \item Give the value returned by each of the following OCaml
                  expressions.  If any of the expressions wouldn't have a
                  value because of any kind of error, just write
                  ``incorrect'' for them.  Don't worry about any incomplete
                  match warnings that might occur.  (\textbf{Note:} be careful
                  not to miss any braces.)

                  \vspace{.5mm}

                  \begin{enumerate}

                    \addtolength{\itemsep}{12mm}

                    \renewcommand{\labelenumiii}{\arabic{enumiii})}%

                    \item \begin{multicols}{2}

                            \begin{Verbatim}
        match [ [1;2]::[[3;4]] ] with a::b -> b
                            \end{Verbatim}

                            \flushright\vspace*{1mm}\ans[2.42in]

                          \end{multicols}

                    \item \begin{multicols}{2}

        %                     \begin{Verbatim}
        % match [[1;2]; [3;4]] with a::[b::c] -> [c;a]
        %                     \end{Verbatim}

                            \begin{Verbatim}
        match [9;8;7;6;5] with _::a::b -> b
                            \end{Verbatim}

                            \flushright\vspace*{1mm}\ans[2.42in]

                          \end{multicols}

                    \item \begin{multicols}{2}

                            \begin{Verbatim}
        match [ [1::[2]]; [3::[4]]; [5::[]] ] with
          | a::b::c::d::e -> 1
          | a::b::c -> 2
          | a::b -> 3
          | a -> 4
                            \end{Verbatim}

                            \flushright\vspace*{1mm}\ans[2.42in]

                          \end{multicols}

                  \end{enumerate}

                  \vspace{-2.5mm}

            \item For each of the following OCaml types, either write a
                  \textbf{short} OCaml expression that has that type, or
                  just state that the type is invalid:

                  \newcommand{\ocamlquestionii}[1]{%
                    \begin{tabular}[t]{@{}p{2in}
                                       @{\underline{\hspace{4.17in}}}@{}}
                      \texttt{#1}
                    \end{tabular}
                  }

                  \vspace{2.5mm}

                  \begin{enumerate}

                    \addtolength{\itemsep}{7mm}

                    \renewcommand{\labelenumiii}{\arabic{enumiii})}

                    \item \ocamlquestionii{int list list}

                    \item \ocamlquestionii{(int * int) list}

                    \item \ocamlquestionii{int * (int list)}

                  \end{enumerate}

                  \vspace{-2.5mm}

          \end{enumerate}

          \medskip

          \ans[\linewidth]

          \medskip

          \begin{centering}

            \scratchpaper[\linewidth]

          \end{centering}

          \pagebreak

    %%%%%%%%%%%%%%%%%%%%%%%%%%%%%%%%%%%%%%%%%%%%%%%%%%%%%%%%%%%%%%%%%%%%%%%%%%

    % new

    \item {[30 pts.]} On the next page, write a complete Ruby program that
          reads an input line, which should be in the form of a valid,
          nonempty OCaml list of integers, and prints the element that
          appeared the most times in the list.

          \vspace{-2.5mm}

          \begin{itemize}

            \addtolength{\itemsep}{0mm}

            \item A nonempty OCaml list consists of a pair of square braces
                  (\texttt{[]}) surrounding one or more elements.  Your
                  program only has to work for nonempty lists whose elements
                  are \textbf{integers} (meaning sequences of one or more
                  digit characters), in which no whitespace appears.  If a
                  list has more than one element they should be separated by
                  semicolons.

            \item You may assume that one line of input (terminated by a
                  newline) will be entered when your program is run.  If the
                  input line read does not match the form of a valid
                  nonempty OCaml list of integers as described above, your
                  program should just print a single line reading ``Invalid
                  list.''\ and quit without printing anything else.

            \item If the input line \textbf{was} in the form of a valid
                  nonempty OCaml list of integers, your program should print
                  the element of the list that appeared the \textbf{most}
                  number of times.  For example, if the input list was
                  \texttt{[5;3;5;5;12;3;5]} the output should be the single
                  number 5, because there is only one 12 in the input list,
                  there are two 3's, and there are four 5's (the most
                  occurrences).

            \item If more than one element of the list has the same maximal
                  number of occurrences it doesn't matter which one your
                  program prints.  For example, for the list
                  \texttt{[4;6;6;4;3]} either 4 or 6 could be printed.

            \item Your program should work as described no matter how many
                  elements a valid nonempty list has (one or more).

          \end{itemize}

          \vspace{-2.5mm}

          You may use any Ruby standard library classes and their methods.
          Minor syntax errors will be ignored if you have the correct
          concept.  To get a perfect score, your program should be
          reasonably efficient.  If you're not sure how to get your program
          to do everything mentioned, do the best you can to get the most
          credit.

          Comments are optional, but your program must be written
          \textbf{\underline{\underline{\emph{\large neatly}}}}, with good
          style and formatting.  Use the scratch area below to plan your
          program, then write it \textbf{\underline{\underline{\emph{\large
          neatly}}}} on the next page.

          \vspace{-1mm}

          \underline{\hspace{6.8225in}}

          \medskip

          \begin{centering}

            \scratchpaper[\linewidth]

          \end{centering}

          \pagebreak

          \vspace*{0mm}

          \enlargethispage{8mm}

          \answerblank{27}{\linewidth}{9.5mm}

          % \begin{center}
          % 
          %   \LARGE
          % 
          %   \textsl{Don't miss the last problem on the next page!}
          % 
          % \end{center}

          \pagebreak

    %%%%%%%%%%%%%%%%%%%%%%%%%%%%%%%%%%%%%%%%%%%%%%%%%%%%%%%%%%%%%%%%%%%%%%%%%%

  \end{enumerate}

  % end of numbered questions

  \markright{}

  \begin{center}

    \scratchpaper

    \bigskip

    \Large

    You can separate this page if you like, to use in solving any
    \linebreak
    questions, as long as you \textbf{\underline{write your name} on the
    other side.}

  \end{center}

   %%%%%%%%%%%%%%%%%%%%%%%%%%%%%%%%%%%%%%%%%%%%%%%%%%%%%%%%%%%%%%%%%%%%%%%%%%

\end{document}
