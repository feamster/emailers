\documentclass[11pt,fleqn]{article}

  %
  % After the grading has been going on a while doublecheck the TAs'
  % grading.
  %

  %
  % Should have added something to the grading key for most questions that
  % grading decisions should be written down on the grading sheet.
  %

  %
  % The final exam grading key has more detailed procedural instructions
  % after each question.
  %

  \usepackage{330-f12}

  \addtolength{\headsep}{-7mm}

  \addtolength{\textheight}{7mm}

  \showinfo

\begin{document}

  \header{\course}{Exam \#1 grading key}{\Term}

  {

    \setlength{\leftmargini}{5mm}

    \vspace*{-6mm}

    \begin{info}{\textbf{\underline{\Large Common grading criteria for all
                 parts}}}

      \vspace*{-2mm}

      \enlargethispage{6mm}

      \setlength{\leftmarginii}{5mm}

      \begin{enumerate}

        \addtolength{\itemsep}{.5mm}

        \item Be sure to first check that our answers are correct, that the
              grading criteria make sense, and that the point values add up
              right.

        \item Circle or mark \textbf{anything} that is incorrect.  You don't
              need to write the correct answers but students
              \textbf{\underline{must}} be able to tell what they are losing
              credit for (what was wrong or missing in their answer), and not
              have any points subtracted without being clear what the reason
              is.  Students will ask us about the grading, and we need to be
              able to tell why they lost points.

              If something is wrong circle or mark it with an 'X'.  If
              something that was supposed to be present is missing you can
              just circle the part of the question describing what was missed,
              rather than writing out a description.  If you need to write a
              description of an error just write a \textbf{few short} words.

        \item Mark negative deductions next to any mistakes, but write the
              total (positive) points for each question \textbf{next to its
              number} (\underline{next to where the question says ``[XX
              pts\.]}'').

              Also write the total score for that question in the space
              \textbf{\underline{on the front of the exam}} (so the total
              score for each question is written twice).

        \item For any problems with subparts where the subparts are worth
              different numbers of points, if the points for each subpart
              are not already marked on the exam then mark the points for each
              subpart as a fraction $\left(\frac{n}{m}\right)$ next to that
              subpart.

        \item Don't deduct fractional points; grade using whole number points
              only.

        \item Unless noted below, give partial credit according to the
              grading key if something is partially, but not completely,
              incorrect.

              Unless otherwise indicated, the points for tasks should be
              prorated if the task is partially but not completely right
              (i.e., partial credit should be given).

              In general try to avoid penalizing twice for the same problem,
              or don't penalize for a problem and then for each of its
              effects.

              % For example, suppose the grading key for a question says that
              % making a proper function call is \pts{3}-- you can deduct
              % (using uniform criteria) \pts{-1} if a call was given but has
              % some small mistakes, \pts{-2} if a call was given but has
              % larger mistakes, and \pts{-3} if the call is completely
              % missing or totally wrong.

        \item Anything correct gets full credit, even if it's not the same as
              the solution on the answer key (unless it contradicts what the
              question was asking for in some way, such as using some language
              feature that the problem said could not be used).

        \item Put your initials at the bottom of the page for each question you
              grade.

        \item Carefully add the point deductions for each problem to avoid
              mistakes.

        \item Write the numbers (the score for each question next to the
              question number, and on the front page)
              \textbf{\underline{\underline{neatly}}}.

        \item If you find common mistakes students made, or that several
              students have given the same incorrect answer, let us know.  It
              may be a common conceptual mistake that we didn't anticipate
              when writing the answer key.  In that case we may want to adjust
              the grading key, or make up a special deduction to handle that
              case.

        \item Write down all decisions about grading that are made on the
              grading key (anything that's not already on the grading key).
              Write your name on the pages of the key on the problems you
              graded.

        \item If you're grading a question together with anyone else the
              most important things are correctness of grading and
              \underline{consistency} of grading.  If you see any situations
              not addressed by the grading key, and we make any decisions
              about how to handle them, everyone grading that question must
              commnunicate that between yourselves, and you need to ask when
              you come across different situations to make certain you're
              deducting the same credit as everyone is for the same mistake.
              Anytime you see something you aren't sure about, check with the
              other graders grading that question to ensure any special
              situations are handled uniformly.

        \item If you have questions about anything while grading, be sure to
              ask.  We want to avoid having to go back and correct mistakes
              after things have been graded!

        \item After the exams are graded and returned, don't tell the students
              what the grading criteria were (number of points off for
              different mistakes).

      \end{enumerate}

      \vspace{-2.5mm}

      \pagebreak

    \end{info}

  }

  \begin{enumerate}

    \addtolength{\itemsep}{12mm}

    %%%%%%%%%%%%%%%%%%%%%%%%%%%%%%%%%%%%%%%%%%%%%%%%%%%%%%%%%%%%%%%%%%%%%%%%%%

    \item {[20 pts.]} Any correct answer should have the following
          properties:

          \vspace{-1.75mm}

          \begin{description}

            \item[completeness:] The regular expression should describe all
                  valid strings.

            \item[correctness:] The regular expression should not describe
                  any invalid strings.

          \end{description}

          \vspace{-1.75mm}

          Some correct regular expressions are:

          \vspace{-1.75mm}

          \begin{centering}

            \addtolength{\baselineskip}{2.5mm}  % increased space between lines

            \(
              \left(
                ( baaa^{*} )^{*}
                \midspc
                ( aaa^{*}b )^{*}
                \midspc
                b( aaa^{*}b )^{*}
                \midspc
                ( aaa^{*}b )^{*} aaa^{*}
              \right)
            \)

            \smallskip

            \(
              \left(
                ( b \mid \epsilon)
                \left(
                  aaa^{*} \mid aabaa
                \right)^{*}
                ( b \mid \epsilon)
              \midspc
                b
              \right)
            \)

            \smallskip

            \(
              ( b \mid \epsilon)
              ( aaa^{*}b )^{*}
              (\epsilon \mid aaa^{*})
            \)

            \smallskip

            \(
              ( aaa^{*} \mid \epsilon )
              ( baaa^{*} )^{*}
              ( b \mid \epsilon)
            \)

            \smallskip

            \(
              \left(
                ( b \mid \epsilon) \, aaa^{*}
              \right)^{*}
              (b \mid \epsilon)
            \)

          \end{centering}

          \medskip

          Of course other answers could be correct also.

          \medskip

          \begin{info}{\textbf{\underline{Grading key:}}}

            \smallskip

            \begin{itemize}

              \addtolength{\itemsep}{1mm}

              \item Correctness and completeness are \pts[circle]{10} each:

                    \smallskip

                    \begin{itemize}

                      \addtolength{\itemsep}{.5mm}

                      \item Give full credit if the property is completely
                            satisfied.

                      \item Deduct \pts{-2} per property for one small
                            problem in that property.

                      \item Deduct \pts{-5} per property for a larger
                            problem in that property.

                      \item Deduct \pts{-7} per property if there are major
                            problems in that property, but at least some
                            attempt in the solution (successful or not) to
                            enforce it.

                      \item No credit for that property otherwise.

                    \end{itemize}

                    \smallskip

              \item Deduct up to \pts{-2} for omitting parentheses around
                    regular expressions that use alternation when the
                    effect could be unclear, or using incorrect parentheses.

              \item Deduct up to \pts{-5} for using disallowed (Ruby)
                    regular expression operations, such as $+$ or ? or
                    numeric superscripts or something like $a\{1,2\}$ or
                    character classes, none of which should have been used
                    (up to \pts{-5} depending upon how many disallowed Ruby
                    regular expression operations were used).

            \end{itemize}

          \end{info}

          \begin{info}{\textbf{\underline{Grading notes:}}}

            \smallskip

            \begin{itemize}

              \addtolength{\itemsep}{1mm}

              \item Since different valid answers exist, check the
                    correctness of the answer by ensuring it describes valid
                    strings of various forms, and does not describe invalid
                    strings of various forms.  Note special cases, for
                    example the answer should allow strings that don't have
                    any \emph{a}'s or any \emph{b}'s, ones that can begin
                    and with each letter, etc.

              \item If parentheses were omitted around regular expressions
                    where the meaning is clear and unambiguous, then don't
                    deduct any credit, but show where the parentheses should
                    have been so the student knows they should have been
                    used.

              \item No answer should receive so many deductions it has a
                    negative score (in other words, stop deducting points if
                    you ever get to zero).

              \item Don't penalize for using extra unnecessary parentheses
                    in regular expressions.

              \item Although it's incorrect, don't penalize for omitting
                    $\epsilon$ in a regular expression and using an empty
                    alternative instead, such as in $(a \mid \: )$.  (Write
                    it in, but don't deduct.)

            \end{itemize}

          \end{info}

          \pagebreak

    %%%%%%%%%%%%%%%%%%%%%%%%%%%%%%%%%%%%%%%%%%%%%%%%%%%%%%%%%%%%%%%%%%%%%%%%%%

    \item {[26 pts.]} As in the previous problem, a correct answer must
          accept all valid strings and only valid strings.  One correct DFA
          is:

          \begin{automaton}(-25,0)(120,68)

            \psset{unit=.9mm}

            \state[start,final](5,65){s0}
            \state(35,65){s1}
            \state[final](65,65){s2}
            \state[final](5,35){s3}
            \state(5,5){s4}

            \transition(s0,\emph{a},s1)
            \transition[labellocation=below](s0,\emph{b},s3)

            \transition(s1,\emph{a},s2)
            \transition(s1,\emph{b},s4)

            \transition[loopdirection=right](s2,\emph{a},s2)
            \transition(s2,\emph{b},s3)

            \transition(s3,\emph{a},s1)
            \transition(s3,\emph{b},s4)

            \transition[loopdirection=left](s4,\emph{a},s4)
            \transition[loopdirection=right](s4,\emph{b},s4)

          \end{automaton}

          \enlargethispage{3mm}

          \bigskip

          As the problem emphasizes, no notational shortcuts should be used.
          A DFA always has exactly one outgoing transition for every symbol
          of its alphabet (otherwise it is not a valid DFA).  Omitting a
          dead state and transitions to it, or transitions from the dead
          state to itself, are notational shortcuts that the problem said
          were not allowed.

          \begin{info}{\textbf{\underline{Grading key:}}}

            \smallskip

            \begin{itemize}

              \addtolength{\itemsep}{2mm}

              \item The grading procedure for this question involves finding
                    the closest isomorphism between the student's answer and
                    the correct DFA.  In doing this, ignore things that are
                    extra in the student's DFA that don't make it
                    incorrect, for example, an answer may have more states
                    than necessary, but there's no penalty as long as it
                    recognizes the correct language.  However, if the answer
                    is missing something required (states, transitions,
                    etc.), causing it not to correctly recognize the
                    language in some way, then deduct as described below.

              \item For each missing state that the DFA must have to
                    recognize the correct language but that is not present,
                    deduct \pts{-4}  This deduction is for both the missing
                    state and for all of its outgoing (but not incoming)
                    transitions.

              \item For each missing or incorrect transition deduct \pts{-3}
                    Incorrect transitions are transitions that aren't shown
                    at all, or transitions that should go to a state other
                    than the state they actually do go to (including
                    transitions that go to some existing state but that
                    should really go to some state that does not appear in
                    the DFA).

                    (Note the transitions from the dead state are handled as
                    a special case below).

                    When a needed state is missing from the DFA, the answer
                    should lose credit for both the missing state, and for
                    all of the incorrect transitions that should have gone
                    to that state but that obviously don't go there
                    (they're missing or they go somewhere else).

              \item For each state in the DFA that is final when it should
                    be nonfinal, or that is nonfinal when it should be
                    final, deduct \pts{-2}

              \item Deduct \pts{-2} if the start state is unlabelled or
                    wrong.

              \item If notational shortcuts were used then deduct anywhere
                    from \pts{-2} to \pts{-6}( for the whole problem),
                    depending on how many notational shortcuts were used.
                    Disallowed notational shortcuts would include omitting
                    the dead state, omitting the transitions to it, or
                    having transitions labelled with multiple symbols.

              \item If the answer is an NFA, not a DFA, talk to me about how
                    to grade it.

              % \item Strings in this language must have two basic properties
              %       (the first two symbols aren't the same, and the last two
              %       symbols the same and in the same order as the first
              %       two).  If one of the two properties is basically
              %       enforced by the DFA, but there are problems with the
              %       other property, then the minimum score the answer should
              %       receive should be \pts{13} In other words, in this case
              %       indicate all mistakes in the solution, but stop
              %       deducting at \pts{-13}
              %
              % \item If the DFA is a poor solution that doesn't really
              %       enforce either property but at least makes some partial
              %       attempt at enforcing one property, then the minimum
              %       score the answer should receive should be \pts{8} In
              %       other words, in this case indicate all mistakes in the
              %       solution, but stop deducting points at \pts{-20} An
              %       answer should receive less than \pts{8} only if it's
              %       completely missing or does not show even a minimal
              %       understanding or attempt.
              %
              % \item If the same mistakes were made in both halves of the DFA
              %       then we may not want to deduct twice for of them.  If in
              %       the process of grading it seems like we should deduct
              %       only once for such errors, let me know and we can
              %       discuss it.

              \item If the DFA is a poor solution that doesn't really
                    enforce the language property but at least makes some
                    partial attempt at enforcing it, then the
                    minimum score the answer should receive should be
                    \pts{6} In other words, in this case indicate all
                    mistakes in the solution, but stop deducting points at
                    \pts{-14} An answer should receive less than \pts{6}
                    only if it's completely missing or does not show even a
                    minimal understanding of DFAs or a reasonable attempt to
                    solve the problem.

            \end{itemize}

          \end{info}

          \begin{info}{\textbf{\underline{Grading notes:}}}

            \smallskip

            \begin{itemize}

              \addtolength{\itemsep}{1mm}

              \item Make sure the answer accepts $\epsilon$.

              \item Note as above that there is \textbf{no deduction} for
                    having extra states, as long as the DFA is correct (many
                    answers are correct but the DFA just isn't minimal).
                    But if the DFA has extra states and any extra states
                    have incorrect transitions, then deduct for each
                    incorrect transition as described above.

              \item Carefully check that each state has two outgoing
                    transitions (one on each symbol), and only two outgoing
                    transitions.

                    Carefully check that each transition goes to the correct
                    state in order to recognize or accept all and only valid
                    strings.

              % \item Note that entirely omitting the dead state and all of
              %       the transitions to it counts as a missing state
              %       (\pts{-3}).  But including it and omitting only the
              %       transitions from itself to itself is \pts{-2}

            \end{itemize}

          \end{info}

          \vspace{.25in}

    %%%%%%%%%%%%%%%%%%%%%%%%%%%%%%%%%%%%%%%%%%%%%%%%%%%%%%%%%%%%%%%%%%%%%%%%%%

    \item {[12 pts.]} Notice that the problem says that not any NFA is
          desired (even if it is correct), but the specific NFA that would
          result from applying the construction given in class.  Although
          there are many different NFAs that would recognize or accept this
          langauge (an infinite number in fact), there is only one NFA that
          would result from following the construction given in class.

          \begin{automaton}(-5,0)(130,44)

            \psset{unit=.8mm}

            \state[start](10,20){s0}
            \state(35,20){s1}
            \state(60,35){s2}
            \state(60,5){s3}
            \state(85,35){s4}
            \state(85,5){s5}
            \state(110,20){s6}
            \state(135,20){s7}
            \state(160,20){s8}
            \state[final](185,20){s9}

            \transition(s0,$\epsilon$,s1)
            \transition[curved,angle=48](s0,$\epsilon$,s7)

            \transition(s1,$\epsilon$,s2)
            \transition[labeloffset=-7](s1,$\epsilon$,s3)

            \transition(s2,\emph{a},s4)

            \transition[labeloffset=-7](s3,\emph{c},s5)

            \transition(s4,$\epsilon$,s6)

            \transition[labeloffset=-7](s5,$\epsilon$,s6)

            \transition(s6,$\epsilon$,s7)

            \transition(s7,$\epsilon$,s8)
            \transition[curved,angle=48](s7,$\epsilon$,s0)

            \transition(s8,\emph{b},s9)

          \end{automaton}

          \bigskip

          \begin{info}{\textbf{\underline{Grading key:}}}

            \smallskip

            \begin{itemize}

              \addtolength{\itemsep}{2mm}

              \item Each missing or extra state is \pts{-3} This
                    \textbf{includes} the missing state, \textbf{and} all of
                    its outgoing transitions.

              \item Each missing or incorrect or extra transition is
                    \pts{-2}

              \item If the start state is missing or incorrect, deduct
                    \pts{-2}

              \item If the final state is missing or incorrect, deduct
                    \pts{-2}

              \item All of the above is for NFAs that are basically similar
                    to the above (that were created using the procedure
                    given in class).  If anyone wrote a small two--state or
                    three--state NFA (not using the procedure given in
                    class) that is correct then deduct \pts{-7}.  If anyone
                    wrote a small NFA (not using the procedure given in
                    class) but it's not even correct then deduct \pts{-9}

            \end{itemize}

          \end{info}

    %%%%%%%%%%%%%%%%%%%%%%%%%%%%%%%%%%%%%%%%%%%%%%%%%%%%%%%%%%%%%%%%%%%%%%%%%%

    \item {[12 pts.]}

          \vspace{-2.5mm}

          \begin{enumerate}

            \addtolength{\itemsep}{4mm}

            \item \begin{enumerate}

                    \addtolength{\itemsep}{2mm}

                    \renewcommand{\labelenumiii}{\arabic{enumiii})}

                    \item \texttt{[]}

                    \item \texttt{[7; 6; 5]}

                    \item \texttt{2}

                  \end{enumerate}

            \item \begin{enumerate}

                    \addtolength{\itemsep}{2mm}

                    \renewcommand{\labelenumiii}{\arabic{enumiii})}

                    \item Three examples are \texttt{[[1]]} and
                          \texttt{[[1; 2]]} and \texttt{[[1; 2]; [3; 4]]}

                    \item Two examples are \texttt{[(1, 2)]} and
                          \texttt{[(1, 2); (3, 4)]}

                    \item Two examples are \texttt{(1, [2])} and
                          \texttt{(1, [2; 3; 4])}

                  \end{enumerate}

          \end{enumerate}

          \begin{info}{\textbf{\underline{Grading key:}}}

            \smallskip

            \begin{itemize}

              \addtolength{\itemsep}{2mm}

              \item Each subpart is \pts[circle]{2}

              \item Give partial credit as you feel appropriate on
                    subparts (although I'm not really seeing right now where
                    partial credit might be warranted for part (a)), using
                    uniform and consistent criteria.

            \end{itemize}

          \end{info}

    %%%%%%%%%%%%%%%%%%%%%%%%%%%%%%%%%%%%%%%%%%%%%%%%%%%%%%%%%%%%%%%%%%%%%%%%%%

    \item {[30 pts.]} Different solutions are obviously possible.  The first
          solution below builds a hash (\texttt{element\_counts}) mapping values
          to their number of occurrences in the input list.  Then we want to
          get the largest value in the \texttt{element\_counts} hash, and
          determine the key corresponding to it.  The easiest way is to
          invert the hash by creating a new hash with the keys and values of
          \texttt{element\_counts} reversed.  There's actually a method in
          the \texttt{Hash} class that does that (\texttt{invert()}), which
          those who read about \texttt{Hash} may know about, but it's not
          required.  (The solution below just does the hash inversion
          explicitly, without using the library method.)

          The second solution is a variation that just iterates over the
          hash to find the key that has the largest corresponding value:

          Another approach that could work would be to use nested loops, but
          that would be $\mathcal{O}(n^2)$, and the problem says that
          efficiency matters.

          The easiest way to extract the body of the list is probably a
          backreference, as used in the solution below.

          Notice that the problem says that the program should only read
          \textbf{one} input line.  Also, integers are sequences of one or
          more digit characters, so a regular expression like
          \texttt{\string\d*(;\string\d*)*} would recognize sequences of
          adjacent semicolons with zero--length integers between them.

          \vspace{-.5mm}

          \VerbatimInput[gobble=0,baselinestretch=.925]{most-occurrences.rb}

          \medskip

          \enlargethispage{8mm}

          \VerbatimInput[gobble=0,baselinestretch=.925]{most-occurrences2.rb}

          \vspace{-2.5mm}

          \begin{info}{\textbf{\underline{Grading key:}}}

            \begin{itemize}

              \addtolength{\itemsep}{2mm}

              \item A correct solution should perform the following
                    conceptual tasks, which are listed with their point
                    values:

                    \smallskip

                    \begin{itemize}

                      \addtolength{\itemsep}{0mm}

                      \renewcommand{\labelitemii}{$\ast$}

                      \item Read one line from standard input.
                            (\pts[circle]{1})

                      \item Test the line read for being of the correct
                            syntactic form. (\pts[circle]{5})

                      \item Print the specified message and don't produce
                            any other output if the line is not of the right
                            form. (\pts[circle]{2})

                      \item Extract the body of the list from the input
                            line. (\pts[circle]{4})

                      \item Iterate over all the integers in the list (using
                            \texttt{split()} or \texttt{scan()} would be
                            easiest, but repeatedly matching a regular
                            expression against the beginning line, then
                            removing the part that matched, would also
                            work). (\pts[circle]{5})

                      \item Keep track of the count of occurrences of each
                            value in the input list. (\pts[circle]{4})

                      \item Find the number of occurrences that was largest
                            (or a maximal one). (\pts[circle]{5})

                      \item Print the element that had that largest number
                            of occurrences.  (This mostly refers to
                            identifying the element that had the most
                            occurrences; just printing it is probably just
                            worth a point.)  (\pts[circle]{4})

                    \end{itemize}

                    Partial credit should be given for tasks that are
                    attempted and partly right, but that have some
                    mistakes, using consistent and uniform criteria.

                    \smallskip

              \item Ignore minor mistakes in syntax if the intention is
                    clear.

            \end{itemize}

          \end{info}

          \begin{info}{\textbf{\underline{Grading notes:}}}

            \smallskip

            \begin{itemize}

              \addtolength{\itemsep}{1mm}

              \item The first line (\texttt{!\#}) isn't required (mark, but
                    don't deduct for mistakes in it if it is present but
                    incorrect).

              \item If anyone's r.e.\ is accepting signed integers I would
                    circle the part of the problem that says that integers
                    are just ``sequences of one or more digit characters'',
                    but I wouldn't deduct anything.

              \item If the solution is not efficient (uses nested loops
                    rather than a hash to find the element with the most
                    occurrences) I'm inclined to deduct \pts{-5}, but if
                    anyone actually does this talk to me about it and we can
                    decide for sure.

            \end{itemize}

          \end{info}

    %%%%%%%%%%%%%%%%%%%%%%%%%%%%%%%%%%%%%%%%%%%%%%%%%%%%%%%%%%%%%%%%%%%%%%%%%%

  \end{enumerate}

\end{document}
