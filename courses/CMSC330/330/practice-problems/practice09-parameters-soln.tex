\documentclass[11pt]{article}

  \usepackage{330-f12}

  \pagestyle{empty}

\begin{document}

  \header{\course}{Parameters-- solutions}{\Term}

  \begin{enumerate}

    \addtolength{\itemsep}{8mm}

    \item {

            \renewcommand{\arraystretch}{1.5}

            \begin{tabular}[t]{l|rrrrrrr}

              parameter transmission mechanism
                & \multicolumn{6}{c}{output}
                \\

              \hline

              call--by--value &
                \texttt{10} &
                \texttt{13} &
                \texttt{0} &
                \texttt{200} &
                \texttt{150} &
                \texttt{175} &
                \texttt{190} \\

              call--by--reference &
                \texttt{5} &
                \texttt{8} &
                \texttt{0} &
                \texttt{200} &
                \texttt{20} &
                \texttt{175} &
                \texttt{190} \\

              call--by--name &
                \texttt{5} &
                \texttt{8} &
                \texttt{0} &
                \texttt{200} &
                \texttt{150} &
                \texttt{175} &
                \texttt{190} \\

            \end{tabular}

          }

    \item \begin{enumerate}

            \addtolength{\itemsep}{2mm}

            \item The output would be \texttt{7}.

            \item The output would be \texttt{14}.

            \item The output would be \texttt{24}.

          \end{enumerate}

    \item Using call--by--value, when the argument \texttt{func(1, 1)} is
          evaluated the function would recurse indefinitely, since the
          actual parameter \texttt{func(1, 1)} corresponding to the formal
          parameter \texttt{a} would first be evaluated.  For the call
          \texttt{func(1, 1)} the parameter \texttt{b} is nonzero, and the
          function will call itself recursively with the exact same
          parameter values it was itself called with.  This is not a good
          idea if you have anything productive you want to do after that.

          If call--by--need were used instead, \texttt{a} would never be
          evaluated for the call \texttt{func(1, 1)} since it's not used, so
          \texttt{func(1, 1)} would never be called.  Evaluating \texttt{b}
          is safe, as \texttt{func(0, 0)} would cause a recursive call in
          which \texttt{b}'s value was 0, causing the function to simply
          return 0 instead of calling itself recursively again.

    \item Any variable whose address can only be resolved at runtime may
          cause this procedure to not function correctly; specifically when
          the address of one parameter depends on the other's value.  For
          example, the call \texttt{swap(i, c[i])} won't swap the values of
          \texttt{i} and \texttt{c[i]}.

    \item The least restrictive condition: do not allow a variable whose
          address can only be resolved at runtime to be an actual parameter.

    \item The output that would be produced if dynamic scoping were used
          would be:

          \begin{Verbatim}
        e:  9  1  8  2
        b:  8  2  6  3
        a:  9  3  8  7
          \end{Verbatim}

          The output that would be produced if static scoping were used
          would be:

          \begin{Verbatim}
        e:  9  1  8  2
        b:  9  6  8  8
        a:  9  6  8  8
          \end{Verbatim}

  \end{enumerate}

\end{document}
