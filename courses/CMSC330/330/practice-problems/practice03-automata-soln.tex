\documentclass[11pt]{article}

  \usepackage{330-f12}

\begin{document}

  \header{\course}{Finite automata-- solutions}{\Term}

  \begin{enumerate}

    \item Note that different solutions are possible.

          \vspace{-3mm}

          \begin{enumerate}

          \addtolength{\itemsep}{20mm}

          \item \begin{automaton}(0,24)(100,35)

                  \state[start](10,25){s0}
                  \state(35,25){s1}
                  \state(60,25){s2}
                  \state(85,25){s3}
                  \state[final](110,25){s4}
                  \state[label={\renewcommand{\arraystretch}{.75}%
                                \normalsize%
                                \begin{tabular}[t]{c}%
                                  dead%
                                    \\%
                                  state%
                                \end{tabular}}%
                         ](35,0){s5}%

                  \transition(s0,\emph{a},s1)
                  \transition(s0,\emph{b},s5)

                  \transition[labellocation=below](s1,\emph{a},s5)
                  \transition(s1,\emph{b},s2)

                  \transition(s2,\emph{a},s3)
                  \transition[labellocation=below](s2,\emph{b},s5)

                  \transition[offsetto=-1mm](s3,\emph{a},s5)
                  \transition(s3,\emph{b},s4)

                  \transition(s4,\emph{a},s4)
                  \transition[loopdirection=down](s4,\emph{b},s4)

                  \transition[loopdirection=left](s5,\emph{a},s5)
                  \transition[loopdirection=right](s5,\emph{b},s5)

                \end{automaton}

                \vspace{-1mm}

          \item \begin{automaton}(0,19)(100,58)

                  \state[start](10,20){s0}
                  \state(35,20){s1}
                  \state(60,20){s2}
                  \state(85,20){s3}
                  \state[final](110,20){s4}

                  \transition(s0,\emph{a},s1)
                  \transition[labelposition=.85](s0,\emph{b},s0)

                  \transition(s1,\emph{a},s1)
                  \transition(s1,\emph{b},s2)

                  \transition(s2,\emph{a},s3)
                  \transition[angle=-45,curvature=1.2,labelposition=0.8,
                              labellocation=below](s2,\emph{b},s0)

                  \transition[angle=36](s3,\emph{a},s1)
                  \transition[offset=2](s3,\emph{b},s4)

                  \transition[offset=2](s4,\emph{a},s3)
                  \transition[curved,labelposition=.85,angle=40]%
                              (s4,\emph{b},s0)

                \end{automaton}

                \vspace{5mm}

          \setlength{\columnsep}{15mm}

          \raggedcolumns

          \begin{multicols}{2}

            \item \begin{automaton}(0,24)(70,60)

                    \state[start](10,25){s0}
                    \state(30,25){s1}
                    \state(50,25){s2}
                    \state[final](70,25){s3}
                    \state[label={\renewcommand{\arraystretch}{.75}%
                                  \normalsize%
                                  \begin{tabular}[t]{c}%
                                    dead%
                                      \\%
                                    state%
                                  \end{tabular}}%
                           ](20,5){s4}
                    \state(60,5){s5}

                    \transition(s0,\emph{a},s1)
                    \transition[labelposition=.35,labellocation=below]%
                                (s0,\emph{b},s4)

                    \transition[labelposition=.35](s1,\emph{a},s4)
                    \transition(s1,\emph{b},s2)

                    \transition[offset=2](s2,\emph{a},s3)
                    \transition[labelposition=.85](s2,\emph{b},s2)

                    \transition(s3,\emph{a},s5)
                    \transition[offset=2](s3,\emph{b},s2)

                    \transition[loopdirection=left](s4,\emph{a},s4)
                    \transition[loopdirection=right](s4,\emph{b},s4)

                    \transition[loopdirection=down](s5,\emph{a},s5)
                    \transition(s5,\emph{b},s2)

                  \end{automaton}

          \columnbreak

            \item \begin{automaton}(0,24)(70,60)

                    \state[start](10,25){s0}
                    \state(30,25){s1}
                    \state[final](50,15){s2}
                    \state(10,5){s3}
                    \state(30,5){s4}

                    \transition(s0,\emph{a},s1)
                    \transition[labelposition=.85](s0,\emph{b},s0)

                    \transition(s1,\emph{a},s2)
                    \transition(s1,\emph{b},s1)

                    \transition(s2,\emph{a},s4)
                    \transition[loopdirection=right](s2,\emph{b},s2)

                    \transition(s3,\emph{a},s0)
                    \transition[loopdirection=down,labelposition=.85]%
                                (s3,\emph{b},s3)

                    \transition(s4,\emph{a},s3)
                    \transition[loopdirection=down](s4,\emph{b},s4)

                  \end{automaton}

          \end{multicols}

          \vspace{16mm}

          \item \begin{tabular}[t]{@{}p{1.75in}@{\hspace{.25in}}p{4.25in}@{}}

                  \begin{automaton}(0,29)(42,40)

                    \state[start,final](10,30){s0}
                    \state[final](35,30){s1}
                    \state[final](10,5){s2}
                    \state(35,5){s3}

                    \transition[offset=2,labeloffset=1,labelposition=.15]%
                                (s0,\emph{a},s3)
                    \transition[labellocation=below,offset=-2](s0,\emph{b},s2)

                    \transition[offset=2,labeloffset=1,labelposition=.85]%
                                (s1,\emph{a},s2)
                    \transition[offset=-2,labeloffset=.5,labelposition=.5]%
                               (s1,\emph{b},s3)

                    \transition[offset=2,labeloffset=1,labelposition=.85]%
                                (s2,\emph{a},s1)
                    \transition[offset=-2,labeloffset=.25,labelposition=.5]%
                                (s2,\emph{b},s0)

                    \transition[offset=2,labeloffset=1,labelposition=.15]%
                                (s3,\emph{a},s0)
                    \transition[labellocation=below,offset=-2](s3,\emph{b},s1)

                  \end{automaton}

              &

                Note that when a language is expressed as a conjunction or
                disjunction of properties, as this one is, it is often
                convenient to construct a DFA by identifying all of the
                possibilities for each property and then creating a separate
                state corresponding to each combination of the properties.
                Then for each state you need to determine for each symbol of
                the input alphabet which other state a transition should
                lead to.  Lastly you have to identify which of the states
                represent final states and which is the start state.

                \end{tabular}

                \vspace{-4mm}

                In this DFA, the states represent the following combinations of
                properties:

                \begin{itemize}

                  \addtolength{\itemsep}{.5mm}

                  \item The upper--left state represents strings with an even
                        number of $a$'s and that have even lengths.

                  \item The lower--left state represents strings with an even
                        number of $a$'s and that have odd lengths.

                  \item The upper--right state represents strings with an odd
                        number of $a$'s and that have even lengths.

                  \item The upper--left state represents strings with an odd
                        number of $a$'s and that have odd lengths.

                \end{itemize}

                \vspace{0mm}

          \begin{multicols}{2}

            \item \begin{automaton}(0,14)(70,32)

                    \state[start](10,15){s0}
                    \state(30,15){s1}
                    \state(50,15){s2}
                    \state[final](70,15){s3}

                    \transition[offset=2](s0,\emph{a},s1)
                    \transition[labelposition=.85](s0,\emph{b},s0)

                    \transition(s1,\emph{a},s2)
                    \transition[offset=2](s1,\emph{b},s0)

                    \transition(s2,\emph{a},s3)
                    \transition[angle=55,labelposition=.15](s2,\emph{b},s0)

                    \transition(s3,\emph{a},s3)
                    \transition[loopdirection=down](s3,\emph{b},s3)

                  \end{automaton}

          \columnbreak

            \item \begin{automaton}(0,14)(70,32)

                    \state[start,final](10,15){s0}
                    \state[final](30,15){s1}
                    \state[final](50,15){s2}
                    \state[label={\renewcommand{\arraystretch}{.75}%
                                  \normalsize%
                                  \begin{tabular}[t]{c}%
                                    dead%
                                      \\%
                                    state%
                                  \end{tabular}}%
                           ](70,15){s3}

                    \transition[offset=2](s0,\emph{a},s1)
                    \transition[labelposition=.85](s0,\emph{b},s0)

                    \transition(s1,\emph{a},s2)
                    \transition[offset=2](s1,\emph{b},s0)

                    \transition(s2,\emph{a},s3)
                    \transition[angle=55,labelposition=.15](s2,\emph{b},s0)

                    \transition(s3,\emph{a},s3)
                    \transition[loopdirection=down](s3,\emph{b},s3)

                  \end{automaton}

          \end{multicols}

          \item \begin{automaton}(0,28)(100,47)

                  \state[start](10,30){s0}
                  \state(35,30){s1}
                  \state(60,30){s2}
                  \state[final](85,30){s3}
                  \state[final](110,30){s4}
                  \state[final](135,30){s5}
                  \state[label={\renewcommand{\arraystretch}{.75}%
                                \normalsize%
                                \begin{tabular}[t]{c}%
                                  dead%
                                    \\%
                                  state%
                                \end{tabular}}%
                        ](85,5){s6}
                  \state[final](135,5){s7}

                  \transition[offset=2](s0,\emph{a},s1)
                  \transition[labelposition=.85](s0,\emph{b},s0)

                  \transition(s1,\emph{a},s2)
                  \transition[offset=2](s1,\emph{b},s0)

                  \transition(s2,\emph{a},s3)
                  \transition[angle=45,curvature=.75](s2,\emph{b},s0)

                  \transition(s3,\emph{a},s6)
                  \transition(s3,\emph{b},s4)

                  \transition[offset=2](s4,\emph{a},s5)
                  \transition(s4,\emph{b},s4)

                  \transition(s5,\emph{a},s7)
                  \transition[offset=2](s5,\emph{b},s4)

                  \transition[loopdirection=left](s6,\emph{a},s6)
                  \transition[loopdirection=down](s6,\emph{b},s6)

                  \transition(s7,\emph{a},s6)
                  \transition(s7,\emph{b},s4)

                \end{automaton}

                \vspace{26mm}

          \item \begin{tabular}[t]{@{}p{3.25in}@{\hspace{.25in}}p{2.95in}@{}}

                  \begin{automaton}(0,33)(70,52)

                    \state[start,final](10,35){s0}
                    \state[final](30,35){s1}
                    \state[final](50,35){s2}
                    \state[final](70,35){s3}
                    \state[final](10,10){s4}
                    \state(40,10){s5}
                    \state(70,10){s6}

                    \transition[labellocation=below](s0,\emph{a},s5)
                    \transition(s0,\emph{b},s1)

                    \transition(s1,\emph{a},s5)
                    \transition(s1,\emph{b},s2)

                    \transition(s2,\emph{a},s5)
                    \transition(s2,\emph{b},s3)

                    \transition(s3,\emph{a},s3)
                    \transition[loopdirection=down](s3,\emph{b},s3)

                    \transition[labellocation=below](s4,\emph{a},s5)
                    \transition[loopdirection=down,labelposition=.85]%
                                (s4,\emph{b},s4)

                    \transition[labellocation=below](s5,\emph{a},s6)
                    \transition[loopdirection=down](s5,\emph{b},s5)

                    \transition[angle=45,curvature=1,labelposition=.85]%
                                (s6,\emph{a},s4)
                    \transition[loopdirection=down](s6,\emph{b},s6)

                  \end{automaton}

                &

                  Below is a regular expression based directly on the DFA.

                  Note that every subsidiary clause in the regular expression
                  corresponds to some path in the DFA from the start state to
                  a final state, with Kleene closure corresponding to a cycle
                  in the DFA.

                  This regular expression isn't as short as it could be.  I
                  wrote this version this way as one example illustration of
                  the correspondence between DFAs and regular expressions.

                \end{tabular}

                \vspace{-4mm}

                \(
                  \lpar \
                    \largeepsilon \ | \
                    b \ | \
                    bb \ | \
                    bbb(a|b)^* \ | \
                    ab^*ab^*ab^*(ab^*ab^*ab^*)^* \ | \
                    bab^*ab^*ab^*(ab^*ab^*ab^*)^* \ | \
                    bbab^*ab^*ab^*(ab^*ab^*ab^*)^*
                    \
                  \rpar
                \)

                \medskip

                Here's a shorter version that also recognizes or accepts
                the same langauge as the DFA does:

                \smallskip

                \(
                  \lpar \
                    b \ | \
                    bb \ | \
                    bbb(a|b)^* \ | \
                    (b^*ab^*ab^*ab^*)^*
                    \
                  \rpar
                \)

          \item \begin{tabular}[t]{@{}p{2.8in}@{\hspace{.25in}}p{3.4in}@{}}

                  \begin{automaton}(0,28)(70,45)

                    \state[start](10,30){s0}
                    \state(35,30){s1}
                    \state(60,30){s2}
                    \state(10,5){s3}
                    \state[final](35,5){s4}
                    \state(60,5){s5}

                    \transition(s0,\emph{a},s1)
                    \transition[offset=-2,labellocation=below](s0,\emph{b},s3)

                    \transition(s1,\emph{a},s2)
                    \transition[offset=-2,labellocation=below](s1,\emph{b},s4)

                    \transition[angle=-35,curvature=.75,labelposition=.85,
                                labellocation=below](s2,\emph{a},s0)
                    \transition[offset=-2,labellocation=below](s2,\emph{b},s5)

                    \transition(s3,\emph{a},s4)
                    \transition[offset=-2,labellocation=below](s3,\emph{b},s0)

                    \transition(s4,\emph{a},s5)
                    \transition[offset=-2,labellocation=below](s4,\emph{b},s1)

                    \transition[angle=35,curvature=.75,labelposition=.85]%
                                (s5,\emph{a},s3)
                    \transition[offset=-2,labellocation=below](s5,\emph{b},s2)

                  \end{automaton}

                &

                  This DFA can again be constructed by creating a state
                  corresponding to each possible combination of the two
                  properties given in the definition of the language.  The
                  top row of states represents strings that have an even
                  number of $b$'s and the bottom row represents strings
                  that have an odd number of $b$'s.  The left column of
                  states represents strings $w$ for which $\#a(w) \, \equiv
                  0 \pmod{3}$, while the middle column of states represents
                  strings $w$ for which $\#a(w) \, \equiv 1 \pmod{3}$, and
                  the rightmost column of states represents strings $w$ for
                  which $\#a(w) \, \equiv 2 \pmod{3}$.

                \end{tabular}

                \vspace{-4mm}

                After determining each such state, the transitions between
                them should be relatively easy to identify.

                \vspace{-10mm}

          \item \begin{automaton}(0,28)(100,37)

                  \state[start,final](10,30){s0}
                  \state[final](35,30){s1}
                  \state[final](10,5){s2}
                  \state[label={\renewcommand{\arraystretch}{.75}%
                                \normalsize%
                                \begin{tabular}[t]{c}%
                                  dead%
                                    \\%
                                  state%
                                \end{tabular}}%
                        ](35,5){s3}

                  \transition(s0,\emph{a},s1)
                  \transition[labellocation=below](s0,\emph{b},s2)

                  \transition(s1,\emph{a},s3)
                  \transition[offset=-2,labellocation=below](s1,\emph{b},s2)

                  \transition[offset=-2,labellocation=below](s2,\emph{a},s1)
                  \transition[labellocation=below](s2,\emph{b},s3)

                  \transition[loopdirection=right](s3,\emph{a},s3)
                  \transition[loopdirection=down](s3,\emph{b},s3)

                \end{automaton}

                \vspace{26mm}

          \item \begin{tabular}[t]{@{}p{3.8in}@{\hspace{.25in}}p{2.4in}@{}}

                  \begin{automaton}(0,18)(90,45)

                    \state[start,final](10,20){s0}
                    \state(35,35){s1}
                    \state(60,35){s2}
                    \state[final](85,35){s3}
                    \state[final](35,5){s4}
                    \state(60,5){s5}
                    \state(85,5){s6}

                    \transition(s0,\emph{b},s1)
                    \transition[labellocation=below](s0,\largeepsilon,s4)

                    \transition(s1,\emph{b},s2)

                    \transition(s2,\emph{b},s3)

                    \transition(s3,\emph{a},s3)
                    \transition[loopdirection=down](s3,\emph{b},s3)

                    \transition[labellocation=below](s4,\emph{a},s5)
                    \transition[loopdirection=down](s4,\emph{b},s4)

                    \transition[labellocation=below](s5,\emph{a},s6)
                    \transition[loopdirection=down](s5,\emph{b},s5)

                    \transition[angle=-35,curvature=.75,labellocation=below]%
                                (s6,\emph{a},s4)
                    \transition[loopdirection=down](s6,\emph{b},s6)

                  \end{automaton}

                  &

                  Note that not only does this NFA have an
                  \largeepsilon--transition, but some of its states don't have
                  transitions for every alphabet symbol.  There are some paths
                  for some strings in the language (such as \emph{baaa}) that
                  don't lead to a final state, but there is a path that does
                  lead to a final state for each string in the language.

                \end{tabular}

          \end{enumerate}

          \pagebreak

    \item Similar to the previous practice problems, you may have noticed in
          a few cases in this problem that even making a small change in a
          DFA can have a big effect on the language it accepts.

          \vspace{-1.5mm}

          \begin{enumerate}

            \addtolength{\itemsep}{3mm}

            \item This DFA is wrong.  Strings like 01011 for example are in
                  the language, so the DFA should recognize or accept them,
                  but it does not.  Furthermore, some strings like 10 and 01
                  for example are not in the language, so the DFA should not
                  recognize or accept them, but it does.

            \item This DFA is wrong.  Strings like 010, 100, and 0011 for
                  example are in the language, so the DFA should recognize
                  or accept them, but it does not.

            \item This DFA is wrong.  Strings like 1010 for example are in
                  the language, so the DFA should recognize or accept them,
                  but it does not.  Furthermore, a string like 01 for
                  example is not in the language, so the DFA should not
                  recognize or accept it, but it does.

            \item This DFA is wrong.  Strings like \largeepsilon, 100, and 010,
                  and 0101 for example are in the language, so the DFA
                  should recognize or accept them, but it does not.

            \item This DFA is wrong.  A string like 10011 for example is in
                  the language, so the DFA should recognize or accept it,
                  but it does not.

            \item This DFA is wrong.  A string like 0101010 for example is
                  in the language, so the DFA should recognize or accept it,
                  but it does not.

            \item This DFA is wrong.  Some strings like 110 and 0010 for
                  example are not in the language, so the DFA should not
                  recognize or accept them, but it does.

            \item This DFA is correct.

          \end{enumerate}

  \end{enumerate}

\end{document}
