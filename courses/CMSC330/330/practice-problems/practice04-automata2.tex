\documentclass[11pt]{article}

  \usepackage{330-f12}

  \psset{unit=1mm}
  \psset{arrowsize=2.25 1,arrowinset=0,nodesep=0,offset=0,doublesep=.65}%

\begin{document}

  \header{\course}{Automata transformations}{\Term}

  \begin{enumerate}

    \addtolength{\itemsep}{4mm}

    \item When does an NFA accept a string?

    \item How long could it take to reduce a NFA with \emph{n} states and
          \emph{t} transitions to a DFA?

    \item What language (or set of strings) is accepted by the following
          NFA?

          \begin{pspicture}(-20,57)(55,5)%

            \Large%

            \state[start,label=S0](10,30){s0}
            \state[label=S1](40,50){s1}
            \state[label=S2](70,50){s2}
            \state[final,label=S3](100,30){s3}
            \state[label=S4](40,10){s4}
            \state[label=S5](70,10){s5}

            \large

            \transition(s0,0,s1)
            \transition[labelposition=.15](s0,\largeepsilon,s3)

            \transition(s1,1,s2)
            \transition[labelposition=.15,labeloffset=-6](s1,\largeepsilon,s5)

            \transition(s2,0,s3)
            \transition[labelposition=.15](s2,0,s4)

            \transition(s4,\largeepsilon,s0)
            \transition[labeloffset=-5](s4,0,s5)

            \transition[labeloffset=-5](s5,\largeepsilon,s3)

          \end{pspicture}

    \item Construct an NFA that accepts or recognizes binary numbers that
          contain the digit sequence 101.

    \item Construct an NFA that accepts or recognizes binary numbers that
          include either the digit sequence 00 or the digit sequence 11.

    \item Construct an NFA that accepts or recognizes binary numbers that
          include the digit sequence 00 \textbf{and} the digit sequence 11.

    \item Compute the \largeepsilon--closure of the start state for the
          the NFAs in the four subparts above.

    \item Apply the construction given in lecture to produce an NFA that
          accepts or recognizes the same language as the regular expression
          \(
            (a{\mid}b)^* \, b
          \).

   \item For the regular expression $1^*$:

         \vspace{-2.5mm}

         \begin{enumerate}

           \addtolength{\itemsep}{1mm}

           \item Apply the construction given in lecture to produce an NFA that
                 accepts or recognizes the same language.

           \item Reduce the resulting NFA to a DFA using the subset
                 algorithm.

           \item Minimize the resulting DFA using Hopcroft reduction

         \end{enumerate}

         \vspace{-2.5mm}

   \item For the regular expression $(0|01)^*0$:

         \vspace{-2.5mm}

         \begin{enumerate}

           \addtolength{\itemsep}{1mm}

           \item Apply the construction given in lecture to produce an NFA that
                 accepts or recognizes the same language.

           \item Reduce the resulting NFA to a DFA using the subset
                 algorithm.

           \item Minimize the resulting DFA using Hopcroft reduction

         \end{enumerate}

         \vspace{-2.5mm}

  \end{enumerate}

\end{document}
