\documentclass[11pt]{article}

  \usepackage{330-f12}

  \pagestyle{empty}

\begin{document}

  \header{\course}{Context--free grammars}{\Term}

    To make your answers easier to compare with the solutions, use
  consecutive nonterminals beginning with S when writing your grammars (S,
  T, U, etc.).

  \vspace{-1.5mm}

  \begin{enumerate}

    \addtolength{\itemsep}{8mm}

    \item In Pascal both of the following syntax forms can be used for an
          array type

          \vspace{-4.5mm}

          \begin{center}

            \texttt{array[$\textit{\textrm{lower--bound}}_1\texttt{..}
                           \textit{\textrm{upper--bound}}_1$,
                          \ldots ,
                          $\textit{\textrm{lower--bound}}_n\texttt{..}
                           \textit{\textrm{upper--bound}}_n$]
                    of \textit{\textrm{element--type}};}

            \texttt{array[$\textit{\textrm{lower--bound}}_1\texttt{..}
                           \textit{\textrm{upper--bound}}_1$]%
                          \ldots%
                          [$\textit{\textrm{lower--bound}}_n\texttt{..}
                           \textit{\textrm{upper--bound}}_n$]
                    of \textit{\textrm{element--type}};}

          \end{center}

          \vspace{-4mm}

          What an array's bounds and an array's element type can be are
          described below.

          Note that an array's lower subscript is not necessarily 0 as in C,
          Ruby, and Java.  For example, here are a few array types in Pascal
          (all of which have integer bounds):

          \vspace{-2mm}

          \begin{multicols}{2}

            \texttt{array[1..10] of integer;}

            \texttt{array[5..50] of char;}

            \texttt{array[1..10,10..20] of integer;}

            \texttt{array[1..10][10..20] of integer;}

          \end{multicols}

          \vspace{-2.75mm}

          The second one has subscripts ranging between 5 and 50
          (inclusive), while the last two are two--dimensional arrays.

          Write an unambiguous grammar generating array types similar to
          Pascal's as specifically described below.  (Array types in this
          problem are based upon, but not completely the same as, Pascal's
          arrays.)

          \vspace{-1.75mm}

          \begin{itemize}

            \addtolength{\itemsep}{.25mm}

            \item For purposes of this problem, an array type begins with
                  the word \texttt{array}, which is followed by the array's
                  dimensions, then the word \texttt{of}, then an identifier
                  and a semicolon.  Don't worry about indicating where
                  whitespace may appear.

            \item An array's dimensions are either a sequence of one or more
                  ranges each inside its own set of square braces, or a
                  single comma--separated list of one or more ranges all
                  together inside one set of square braces.

            \item For purposes of this problem a range consists of two
                  constant values of the same type separated by \texttt{..}
                  (two adjacent period characters).  A range can contain two
                  boolean constants, two identifiers, two character
                  constants, or two integer constants.

            \item The only two boolean constants are \texttt{true} and
                  \texttt{false}.  Assume that identifers consist of
                  sequences of one or more occurrences of of the letters
                  \texttt{a}, \texttt{b}, and \texttt{c}, and the characters
                  that can appear as character constants are \texttt{x},
                  \texttt{y}, and \texttt{z}.  A character constant is
                  surrounded by single quotes (\texttt{''}).  An integer
                  constant consists of a sequence or one or more decimal
                  digits (\texttt{0} through \texttt{9}), optionally
                  preceded by a single plus (\texttt{+}) or minus
                  (\texttt{-}) sign.

          \end{itemize}

          \vspace{-1.75mm}

          Note that your grammar must generate all the tokens (integer
          constants, character constants, etc.) in array types; the only
          thing you don't have to worry about is where whitespace may
          appear.

    \item Write unambiguous grammars for the following languages:

          \vspace{-1mm}

          \begin{enumerate}

            \addtolength{\itemsep}{4mm}

            \item \(
                    \{ \,
                      {a^\mathrm{n}}{b^\mathrm{n}} \midspc
                       \mathrm{n} \: \geq \: 0 \ \:
                       \mathrm{and} \ \mathrm{n} \ \textrm{is even}
                     \, \}
                  \)

            \item \(
                    \{ \,
                      {a^\mathrm{i}}{b^\mathrm{j}}
                       {c^{2\mathrm{i}+1}}{d^\mathrm{k}} \midspc
                        \mathrm{i,j,k} \: \geq \: 0
                    \, \}
                  \)

            \item \(
                    \{ \,
                      {\gamma_1}{\gamma_2} \ldots {\gamma_\mathrm{n}}
                      {\gamma_\mathrm{n}} \ldots {\gamma_2}{\gamma_1}
                      \midspc \: \gamma_i \: \in \: \{a,b\},
                      \ 1 \: \leq \: i \: \leq \mathrm{n}
                    \, \}
                  \)

                  \medskip

                  Note this language is describing all strings that are
                  palindromes, that is, they read the same forward as
                  backwards, over the alphabet $\Sigma \: = \: \{a,b\}$.

            \item \(
                    \{ \,
                      {a^\mathrm{m}}{b^\mathrm{n}}{a^{\mathrm{m+n}}} \midspc
                       \mathrm{m} \: \geq \: 0 \ \mathrm{and} \ 
                       \mathrm{n} \: \geq \: 1
                    \, \}
                  \)

            \item \(
                    \{ \,
                      {a^\mathrm{n}}{b^\mathrm{m}}{a^{\mathrm{n-m}}}
                      \midspc \: \mathrm{n} \: \geq \: \mathrm{m} \: \geq \:
                      0
                    \, \}
                  \)

            \item All possible sequences of balanced parentheses.  Each
                  string in the language is composed of zero or more of the
                  symbols \texttt{(} and \texttt{)}.  Each valid string must
                  have same number of left parentheses as right parentheses,
                  and they must be properly balanced, with every opening
                  left parenthesis paired with a later closing right
                  parentheses.  Valid strings may contain nested
                  parentheses.  For example, the strings (()) and (()(()()))
                  contain matching balanced parentheses, while (()(()()) and
                  (()))( do not.

            \item \(
                    \{ \,
                      w \midspc w \: \in \: \{a,b\}^* \ \mathrm{and} \ w \
                      \textrm{has an equal number of} \
                      a\textrm{'s} \ \mathrm{and} \ b\textrm{'s}
                      \, \}
                  \)

            \item \(
                    \{ \,
                      {a^\mathrm{m}}{b^\mathrm{n}} \midspc
                       \mathrm{m \, \neq \, n} \ \mathrm{and} \ 
                       \mathrm{m,\, n} \: \geq \: 0
                    \, \}
                  \)

            \item \(
                    \{ \,
                      {a^\mathrm{m}}{b^\mathrm{n}}{c^\mathrm{p}}{d^\mathrm{q}}
                       \midspc \mathrm{m \: + \: n \: = \: p \: + \: q}
                    \, \}
                  \)

          \end{enumerate}

    \item Write a grammar for the language
          \(
            \{ \,
              {a^n}{b^n}{a^m}{b^m} \midspc \: m,n \: \geq \: 1
            \, \}
            \ \cup \
            \{ \,
              {a^n}{b^m}{a^m}{b^n} \midspc \: m,n \: \geq \: 1
            \, \}
          \).
          If your grammar is ambiguous, prove that it is.

  \end{enumerate}

\end{document}
