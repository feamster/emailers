\documentclass[11pt]{article}

  %
  % Should have omitted the (unnecessary) \_ in the solution to the last
  % problem, since it confused some students (omit next time).
  %

  \usepackage{330-f12}

\begin{document}

  \header{\course}{Ruby and Ruby regular expressions -- solutions}{\Term}

  \begin{enumerate}

    \addtolength{\itemsep}{6mm}

    \item \begin{enumerate}

            \addtolength{\itemsep}{3mm}

            \item Shifted from efficiency to ease-of-programming

            \item \begin{itemize}

                    \addtolength{\itemsep}{2mm}

                    \item Naturalness of application -­ text processing is
                          easier in Ruby

                    \item Cost of use ­ Small Ruby programs are
                          simpler/quicker to write

                  \end{itemize}

            \item Interpretation and compilation.  Ruby is interpreted.

          \end{enumerate}

    \item \begin{enumerate}

            \addtolength{\itemsep}{3mm}

            \item \begin{Verbatim}
        def printit(x)
          print(x)
        end

        printit(2)
                  \end{Verbatim}

                   \texttt{x} is the formal paramameter, \texttt{2} is the
                   actual parameter

             \item \begin{multicols}{3}

                     \begin{Verbatim}
        1.upto(10) {|i|
          puts(i)
        }
                     \end{Verbatim}

                     \bigskip

                     \begin{Verbatim}
        (1..10).each() {|i|
          puts(i)
        }
                     \end{Verbatim}

                     \bigskip

                     \begin{Verbatim}
        for i in (1..10)
          puts(i)
        end
                     \end{Verbatim}

                     \bigskip

                     \begin{Verbatim}
        i = 1
        while (i <= 10)
          puts(i)
          i += 1
        end
                     \end{Verbatim}

                     \bigskip

                     \begin{Verbatim}
        i = 0
        until ((i += 1) > 10)
          puts(i)
        end
                     \end{Verbatim}

                   \end{multicols}

                  \smallskip

                  Of course other versions are possible as well.

            \item \begin{itemize}

                    \addtolength{\itemsep}{2mm}

                    \item Explicit-­ declarations must specify the type of
                          each variable used

                    \item Implicit-­ the first use of a variable declares it
                          and determines its type

                  \end{itemize}

            \item Helps prevent subtle errors, catches more type errors at
                  compile time

            \item \begin{Verbatim}
        class Teacher

           @@totalStudents = 0

           def initialize
             @students = 0
             @@totalStudents += @students
          end

        end
                  \end{Verbatim}

            \item \texttt{x = "a" ; y = x}

            \item \texttt{x == y}

          \end{enumerate}

    \item \begin{enumerate}

            \addtolength{\itemsep}{3mm}

            \item Strings that contain two \texttt{3}s.

            \item Strings that contain an uppercase letter character.

            \item Strings that contain zero or more uppercase letter
                  character, followed by a digit character.

            \item Strings that contain a \texttt{0} as their last character.

            \item Strings containing a period character.

            \item \begin{Verbatim}
        4
        6
        nil
        nil
        7
        1

                  \end{Verbatim}

            \item \begin{Verbatim}
        c
        b
        a
        a
        b
        c

                  \end{Verbatim}

            \item \begin{Verbatim}
        6
        nil
        7
        6

                  \end{Verbatim}

            \item \begin{Verbatim}
        f
                  \end{Verbatim}

            \item An array of strings where each string is a line from the
                  file \texttt{filename}.

            \item A string for the first command--line argument

            \item \begin{Verbatim}
        def m()
          2.times() { yield }
        end

        m() { puts("Running")}  # prints "Running Running"
                  \end{Verbatim}

          \end{enumerate}

    \item \begin{Verbatim}
        # version that reads entire file into an array
        file = File.new(ARGV[0], "r")
        lines = file.readlines()
        lines.each{ |line|
          # need to get rid of the newline, otherwise it would be a character
          # that doesn't match the r.e. below
          line.chomp!()
          if (line !~ /[^A-Za-z0-9\_]+/) then
            puts(line)
          end
        }
                  \end{Verbatim}

  \end{enumerate}

\end{document}
