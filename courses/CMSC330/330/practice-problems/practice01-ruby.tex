\documentclass[11pt]{article}

  \usepackage{330-f12}

\begin{document}

  \header{\course}{Ruby and Ruby regular expressions}{\Term}

  \begin{enumerate}

    \addtolength{\itemsep}{6mm}

    \item Programming languages

          \vspace{-1.5mm}

          \begin{enumerate}

            \addtolength{\itemsep}{3mm}

            \item Explain how goals for programming languages have changed
                  since the 1960’s.

            \item List two desirable attributes for a programming language
                  where Ruby is better than C.  Explain why.

            \item Give two methods for executing a program.  Which method is
                  used by Ruby?

          \end{enumerate}

    \item Ruby basics

          \vspace{-1.5mm}

          \begin{enumerate}

            \addtolength{\itemsep}{3mm}

            \item Write a Ruby method \texttt{printit} that takes an integer
                  as a parameter, and prints it.  Write a call to
                  \texttt{printit} with 2 as its argument.  Circle and label
                  the formal and actual parameters in your code.

            \item Using different Ruby control statements, write four
                  different code fragments that each iterate and print the
                  values between 1 and 10.

            \item Explain the difference between explicit and implicit
                  variable declarations.

            \item Give two advantages of static types.

            \item \sloppy Write a Ruby class \texttt{Teacher} that contains
                  an integer field \texttt{students} and an integer field
                  named \texttt{totalStudents} that is shared across all
                  objects of the class.  You don't have to write any methods
                  for the class. \fussy

            \item Give an example of shallow (reference) copy in Ruby.

            \item Give an example of testing for structural equality in
                  Ruby.

          \end{enumerate}

    \item Ruby advanced features

          \vspace{-1.5mm}

          \begin{enumerate}

            \addtolength{\itemsep}{3mm}

            \item Describe the strings accepted by the Ruby regular
                  expression \texttt{/3\string{2\string}/}.

            \item Describe the strings accepted by the Ruby regular
                  expression \texttt{/[A-Z]/}.

            \item Describe the strings accepted by the Ruby regular
                  expression \texttt{/[A-Z]*[0-9]/}.

            \item Describe the strings accepted by the Ruby regular
                  expression \texttt{/0\$/}.

            \item Describe the strings accepted by the Ruby regular
                  expression \texttt{/\string\./}.

            \item What is the output of the following Ruby program?

                  \begin{Verbatim}
        a = [4, 5, 6]
        a[5] = 7
        a.delete_at(1)
        puts a
        a.push(2)
        a.push(1)
        puts(a.pop)
                  \end{Verbatim}

            \item What is the output of the following Ruby program?

                  \begin{Verbatim}
        a = ["c", "b", "a"]
        puts(a)
        b = a
        a.sort!()
        puts(b)
                  \end{Verbatim}

            \item What is the output of the following Ruby program?

                  \begin{Verbatim}
        a = {4 => 6, 5 => 7}
        puts(a[4])
        puts(a[6])
        puts(a.values())
                  \end{Verbatim}


        %
        % Didn't cover....
        %
        %     \item What is the output of the following Ruby program?
        % 
        %           \begin{Verbatim}
        % h = Hash.new(0)
        % h["a"] = h["b"]
        % h["b"] = 7
        % h["c"] += 2
        % puts("#{h["a"]} #{h["b"]} #{h["c"]}")
        %           \end{Verbatim}

            \item What is the output of the following Ruby program?

                  \begin{Verbatim}
        if ("CMSC 330" =~ /1/) then
          puts("t")
        elsif ("CMSC 330" !~ /1/) then
          puts("f")
        else
          puts("n")
        end
                  \end{Verbatim}

        %
        % Didn't cover yet.
        %
        %     \item What is the output of the following Ruby program?
        % 
        %           \begin{Verbatim}
        % "CMSC 330" =~ /([0-9]+)/
        % puts $1
        % puts $2
        %           \end{Verbatim}

        %
        % Didn't cover yet.
        %
        %     \item What is the output of the following Ruby program?
        % 
        %           \begin{Verbatim}
        % a = CMSC 330 CMSC 351"
        % b = a.scan(/[A-Z]+/)
        % puts(b)
        % a.scan(/[0-9]+ [A-Z]+/) { |x| puts x }
        %           \end{Verbatim}

            \item What is returned by \texttt{file = File.new("filename",
                  "r"); lines = file.readlines();}?

            \item What is returned by \texttt{x = ARGV[0]}?

            \item Write a Ruby method named \texttt{m} that takes a code
                  block (one that does not have any parameters) and executes
                  it twice.

          \end{enumerate}

    \item Write a Ruby program that reads the name of a text file from the
          command line, opens the file, reads every line of text in the
          file, and prints only the lines that contain exclusively the
          following characters: uppercase and lowercase letters, digits, and
          underscore.  For example, lines that contain any spaces or
          punctuation should not be printed.

    %
    % Didn't cover yet.
    %
    % \item Ruby programming
    % 
    %       \vspace{-1.5mm}
    % 
    %       \begin{enumerate}
    % 
    %         \addtolength{\itemsep}{3mm}
    % 
    %         \item Write a Ruby program that reads in lines of input from its
    %               standard input and remembers all integers (consecutive
    %               digits) encountered.  Each line of input may contain zero
    %               or more integers or non--integers.  The program should
    %               stop and print the list of integers in sorted order (from
    %               smallest to largest) when the word ``Done!'' is
    %               encountered.
    % 
    %       \end{enumerate}
    % 
    %       \vspace{-2.5mm}

  \end{enumerate}

\end{document}
