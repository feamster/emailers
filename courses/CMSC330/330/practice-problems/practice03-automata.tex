\documentclass[11pt]{article}

  \usepackage{330-f12}

  \psset{unit=1mm}
  \psset{arrowsize=2.25 1,arrowinset=0,nodesep=0,offset=0,doublesep=.65}%

\begin{document}

  \header{\course}{Finite automata}{\Term}

  \begin{enumerate}

    \addtolength{\itemsep}{30mm}

    \item For each of the following problems construct a deterministic
          finite automaton that describes or recognizes the language given.
          Write all DFAs in the form of a transition diagram.  The
          underlying alphabet is $\Sigma = \{a, \: b\}$.

          Be sure to give a DFA for each part, and not an NFA unless the
          problem asks for one.  Do \textbf{\underline{not}} use any of the
          notational conveniences or shortcuts shown in lecture.

          The notation $\#a(w)$ is used below to refer to the number of
          $a$'s occurring in the string $w$.  For example, $\#a(bbaba) = 2$.

          Note: the first seven parts are the languages from the second
          question in the prior practice problem set.  It would be
          instructive to compare the DFA and the regular expression for
          these languages.

          \vspace{-1.5mm}

          \begin{enumerate}

            \addtolength{\itemsep}{3mm}

            \item $ \{ \: w\ | \ w \ \textrm{begins with} \ abab\: \} $

            \item $ \{ \: w\ | \ w \ \textrm{ends with} \ abab\: \} $

            \item $ \{ \: w\ | \ w \ \textrm{begins with} \ ab \
                    \textrm{and ends with} \ ba\: \} $

                  Note: The string \textit{aba\/} is in this language.

            \item $ \{ \: w\ | \ \#a(w) \: \equiv \: 2 \ \pmod{5} \: \}$

                  Recall that $i \: \equiv \: j \pmod{k}$ if and only if
                  $i - j$ is divisible by $k$.

            \item $ \{ \: w\ | \ \#a(w) \ \textrm{is even or} \
                    |w| \ \textrm{is even} \: \} $

            \item $ \{ \: w\ | \ aaa \ \textrm{is a substring of} \ w\: \} $

            \item $ \{ \: w\ | \ aaa \ \mathrm{is} \ {\bf not}
                    \ \textrm{a substring of} \ w \: \} $

            \item $ \{ \: w \ | \ w \
                             \textrm{contains exactly one occurrence of the
                             substring}
                             \ aaa \: \} $

                  Note: the string $aaaa$ has two occurrences of $aaa$.

            \item \label{problem}

                  $ \{ \: w \ | \ \textrm{either} \ \#a(w)
                                  \ \textrm{is divisible by 3 or} \ w
                                  \ \textrm{begins with} \ bbb \: \} $

                  Also write a regular expression that describes or
                  recognizes this language.

            \item $ \{ \: w \ | \ \#a(w) \: \equiv \: 1 \pmod{3}
                             \ \mathrm{and} \ \#b(w) \ \textrm{is odd} \: \} $

            \item $ \{ \: w \ | \ \mathrm{neither} \ aa \ \mathrm{nor} \ bb \
                             \textrm{is a substring of} \ w \: \} $

            \item Write an NFA for the language in part \#\ref{problem}.

          \end{enumerate}

          \vspace{-6mm}

    \item Consider the following language:

          $ \{ \: w \ | \ w \: \in \: \{ \, 0, \, 1 \}^* \
                        \textrm{and \textit{w} contains an even number of
                                0s, and \textit{w} does not contain three
                                consecutive 1s} \: \} $

          Determine whether each of the following DFAs correctly describes
          or recognizes this language or not.  Identify why each incorrect
          DFA is wrong-- give a string that the DFA doesn't give the right
          results for, and identify what result the DFA should give for that
          string, and what result it actually gives.

          \pagebreak

          \begin{enumerate}

            \addtolength{\itemsep}{46mm}

            \item \begin{pspicture}(0,14)(70,30)%
                    \Large%
                    \state[start,final](10,15){s0}
                    \state(40,15){s1}
                    \state[final](70,15){s2}
                    \state[label={\renewcommand{\arraystretch}{.75}%
                                  \normalsize%
                                  \begin{tabular}[t]{c}%
                                    dead%
                                      \\%
                                    state%
                                  \end{tabular}}%
                          ](100,15){s3}

                    \large

                    \transition[offset=-2,labeloffset=-5](s0,0,s1)
                    \transition[offset=2](s0,1,s1)

                    \transition[offset=-2,labeloffset=-5](s1,0,s2)
                    \transition[offset=2](s1,1,s2)

                    \transition[curved,angle=80,curvature=.9,angle=60]%
                                (s2,0,s1)
                    \transition(s2,1,s3)

                    \transition(s3,0,s3)
                    \transition[loopdirection=down](s3,1,s3)

                  \end{pspicture}

                  \vspace{-10mm}

            \item \begin{pspicture}(0,29)(100,37)

                    \state[start,final](10,30){s0}
                    \state(40,30){s1}
                    \state(70,30){s2}
                    \state[label={\renewcommand{\arraystretch}{.75}%
                                  \normalsize%
                                  \begin{tabular}[t]{c}%
                                    dead%
                                      \\%
                                    state%
                                  \end{tabular}}%
                          ](100,30){s3}
                    \state(10,0){s4}

                    \large

                    \transition[offset=-2,labeloffset=-4](s0,0,s4)
                    \transition(s0,1,s1)

                    \transition[offset=-2,labeloffset=-5](s1,0,s4)
                    \transition(s1,1,s2)

                    \transition[offsetto=3](s2,0,s4)
                    \transition(s2,1,s3)

                    \transition(s3,0,s3)
                    \transition[loopdirection=down](s3,1,s3)

                    \transition[offset=-2,labeloffset=-4](s4,0,s0)
                    \transition[offset=-2,labeloffset=-5](s4,1,s1)

                  \end{pspicture}

            \item \begin{pspicture}(0,28)(100,40)

                    \state[start,final](10,30){s0}
                    \state[final](40,30){s1}
                    \state[final](70,30){s2}
                    \state[label={\renewcommand{\arraystretch}{.75}%
                                  \normalsize%
                                  \begin{tabular}[t]{c}%
                                    dead%
                                      \\%
                                    state%
                                  \end{tabular}}%
                          ](100,30){s3}
                    \state(10,0){s4}
                    \state(40,0){s5}
                    \state(70,0){s6}

                    \large

                    \transition[offset=-2,labeloffset=-4](s0,0,s4)
                    \transition(s0,1,s1)

                    \transition[offset=-2,labeloffset=-4](s1,0,s5)
                    \transition(s1,1,s2)

                    \transition[offset=-2,labeloffset=-4](s2,0,s6)
                    \transition(s2,1,s3)

                    \transition(s3,0,s3)
                    \transition[loopdirection=down](s3,1,s3)

                    \transition[offset=-2,labeloffset=-4](s4,0,s0)
                    \transition[labeloffset=-5,labellocation=below](s4,1,s1)

                    \transition[offset=-2,labeloffset=-4](s5,0,s1)
                    \transition(s5,1,s2)

                    \transition[offset=-2,labeloffset=-4](s6,0,s2)
                    \transition(s6,1,s3)

                  \end{pspicture}

            \item \begin{pspicture}(0,29)(100,45)

                    \state[start](10,30){s0}
                    \state(40,30){s1}
                    \state(70,30){s2}
                    \state[label={\renewcommand{\arraystretch}{.75}%
                                  \normalsize%
                                  \begin{tabular}[t]{c}%
                                    dead%
                                      \\%
                                    state%
                                  \end{tabular}}%
                          ](100,30){s3}
                    \state(10,0){s4}
                    \state[final](40,0){s5}

                    \large

                    \transition[labeloffset=-4](s0,0,s4)
                    \transition[offset=2,labeloffset=2](s0,1,s1)

                    \transition[offset=2,labeloffset=2](s1,0,s0)
                    \transition(s1,1,s2)

                    \transition[curved,angle=-40,labeloffset=-5](s2,0,s0)
                    \transition(s2,1,s3)

                    \transition(s3,0,s3)
                    \transition[loopdirection=down](s3,1,s3)

                    \transition[offset=2,labeloffset=2](s4,0,s5)
                    \transition[labeloffset=-5,labellocation=below](s4,1,s1)

                    \transition[offset=2,labeloffset=2](s5,0,s4)
                    \transition[labellocation=below](s5,1,s1)

                  \end{pspicture}

                  \pagebreak

            \item \begin{pspicture}(0,29)(100,40)

                    \state[start,final](10,30){s0}
                    \state[final](40,30){s1}
                    \state[final](70,30){s2}
                    \state[label={\renewcommand{\arraystretch}{.75}%
                                  \normalsize%
                                  \begin{tabular}[t]{c}%
                                    dead%
                                      \\%
                                    state%
                                  \end{tabular}}%
                          ](100,30){s3}
                    \state(10,0){s4}
                    \state(40,0){s5}
                    \state(70,0){s6}

                    \large

                    \transition[offset=-2,labeloffset=-4](s0,0,s4)
                    \transition(s0,1,s1)

                    \transition[offset=-2,labeloffset=-3.5](s1,0,s5)
                    \transition(s1,1,s2)

                    \transition[offset=-2,labeloffset=-4](s2,0,s6)
                    \transition(s2,1,s3)

                    \transition(s3,0,s3)
                    \transition[loopdirection=down](s3,1,s3)

                    \transition[offset=-2,labeloffset=-4](s4,0,s0)
                    \transition[offset=1.5,labeloffset=-5,labellocation=below]%
                                (s4,1,s1)

                    \transition[offset=-2,labeloffset=-4](s5,0,s1)
                    \transition[curved,curvature=.8](s5,1,s1)

                    \transition[offset=-2,labeloffset=-4](s6,0,s2)
                    \transition(s6,1,s1)

                  \end{pspicture}

            \item \begin{pspicture}(0,29)(100,40)

                    \state[start,final](10,30){s0}
                    \state[final](40,30){s1}
                    \state[final](70,30){s2}
                    \state[label={\renewcommand{\arraystretch}{.75}%
                                  \normalsize%
                                  \begin{tabular}[t]{c}%
                                    dead%
                                      \\%
                                    state%
                                  \end{tabular}}%
                          ](100,15){s3}
                    \state(10,0){s4}
                    \state(40,0){s5}
                    \state(70,0){s6}

                    \large

                    \transition[offset=-2,labeloffset=-4](s0,0,s4)
                    \transition(s0,1,s1)

                    \transition[offset=-2,labeloffset=-4](s1,0,s5)
                    \transition(s1,1,s2)

                    \transition[offset=-2,labeloffset=-4](s2,0,s6)
                    \transition(s2,1,s3)

                    \transition(s3,0,s3)
                    \transition[loopdirection=down](s3,1,s3)

                    \transition[offset=-2,labeloffset=-4](s4,0,s0)
                    \transition[labeloffset=-5](s4,1,s5)

                    \transition[offset=-2,labeloffset=-4](s5,0,s1)
                    \transition[labeloffset=-5](s5,1,s6)

                    \transition[offset=-2,labeloffset=-4](s6,0,s2)
                    \transition[labeloffset=-5.5](s6,1,s3)

                  \end{pspicture}

            \item \begin{pspicture}(0,29)(100,45)

                    \state[start,final](10,30){s0}
                    \state[final](40,30){s1}
                    \state[final](70,30){s2}
                    \state[label={\renewcommand{\arraystretch}{.75}%
                                  \normalsize%
                                  \begin{tabular}[t]{c}%
                                    dead%
                                      \\%
                                    state%
                                  \end{tabular}}%
                          ](100,15){s3}
                    \state(10,0){s4}
                    \state(40,0){s5}
                    \state(70,0){s6}

                    \large

                    \transition[offset=-2,labeloffset=-4](s0,0,s4)
                    \transition[offset=2,labeloffset=2](s0,1,s1)

                    \transition[offset=2,labeloffset=2,labelposition=.25]%
                                (s1,0,s0)
                    \transition(s1,1,s2)

                    \transition[curved,angle=-40,labellocation=below](s2,0,s0)
                    \transition(s2,1,s3)

                    \transition(s3,0,s3)
                    \transition[loopdirection=down](s3,1,s3)

                    \transition[offset=-2,labeloffset=-4](s4,0,s0)
                    \transition[labeloffset=-5](s4,1,s5)

                    \transition(s5,0,s0)
                    \transition[labeloffset=-5](s5,1,s6)

                    \transition[labellocation=below,labelposition=.375]%
                                (s6,0,s0)
                    \transition[labeloffset=-5.5](s6,1,s3)

                  \end{pspicture}

            \item \begin{pspicture}(0,29)(100,40)

                    \state[start,final](10,30){s0}
                    \state[final](40,30){s1}
                    \state[final](70,30){s2}
                    \state[label={\renewcommand{\arraystretch}{.75}%
                                  \normalsize%
                                  \begin{tabular}[t]{c}%
                                    dead%
                                      \\%
                                    state%
                                  \end{tabular}}%
                          ](100,15){s3}
                    \state(10,0){s4}
                    \state(40,0){s5}
                    \state(70,0){s6}

                    \large

                    \transition[offset=-2,labeloffset=-4](s0,0,s4)
                    \transition(s0,1,s1)

                    \transition[labeloffset=-5,labelposition=.175](s1,0,s4)
                    \transition(s1,1,s2)

                    \transition[labeloffset=-5,labelposition=.225](s2,0,s4)
                    \transition(s2,1,s3)

                    \transition(s3,0,s3)
                    \transition[loopdirection=down](s3,1,s3)

                    \transition[offset=-2,labeloffset=-4](s4,0,s0)
                    \transition[labeloffset=-5](s4,1,s5)

                    \transition[labeloffset=-5,labelposition=.175,
                                labellocation=below](s5,0,s0)
                    \transition[labeloffset=-5](s5,1,s6)

                    \transition[labeloffset=-5,labelposition=.225,
                                labellocation=below]](s6,0,s0)
                    \transition[labeloffset=-5.5](s6,1,s3)

                  \end{pspicture}

          \end{enumerate}

          \vspace{1in}

    % \item Apply the construction given in lecture to produce an NFA that
    %       accepts the same language as the regular expression
    %       \(
    %         (a \mid b)^* \, b
    %       \).          

  \end{enumerate}

\end{document}
