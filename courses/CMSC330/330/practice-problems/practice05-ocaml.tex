\documentclass[11pt]{article}

  %
  % Turned out there were many small OCaml problems in Chau Wen's spring 12
  % /www/tests/prac4-fall09.pdf, that I hadn't come across at the time that
  % I was scavenging these problems from other things in his files.
  %

  \usepackage{330-f12}

  \pagestyle{empty}

\begin{document}

  \header{\course}{Ocaml}{\Term}

  \enlargethispage{3mm}

  \vspace{-3mm}

  \begin{enumerate}

    \addtolength{\itemsep}{5mm}

    \item Give the type of the following OCaml expressions.  If any has an
          error, describe it.

          \vspace{-2.5mm}

          \begin{enumerate}

            \addtolength{\itemsep}{0mm}

            \item \texttt{[("1", 2); ("3", 4)]}

            \item \texttt{fun f a -> [a; a + 1]}

            \item \texttt{fun x y -> [x ; y]}

          \end{enumerate}

          \vspace{-2.5mm}

    \item Give the value of the following OCaml expressions.  If any has an
          error, describe it.

          \vspace{-2.5mm}

          \begin{enumerate}

            \addtolength{\itemsep}{0mm}

            \item \texttt{[1; 2]::[3]}

            \item \texttt{let x y = y 3 in x (fun z -> z - 1)}

            \item \texttt{map ((fun x y -> x+y) 1) [2; 3; 4]}

          \end{enumerate}

          \vspace{-2.5mm}

    \item Write an OCaml expression with the type of each of the following:

          \vspace{-2.5mm}

          \begin{enumerate}

            \addtolength{\itemsep}{0mm}

            \item \texttt{int * int list}

            \item \texttt{int list -> (int -> int)}

            \item \texttt{(int -> 'a) -> 'a}

          \end{enumerate}

          \vspace{-2.5mm}

    \item Solve the following OCaml programming problems.  You are allowed
          to use \texttt{List.rev} (an OCaml library function that reverses
          a list) and the following curried versions of the function map and
          fold, but no other OCaml library functions. Your solution must run
          in $\mathcal{O}(n)$ time for input lists of length \emph{n}.

          \vspace{-1.5mm}

          \begin{multicols}{2}

            \begin{Verbatim}
        let rec map f l = match l with
            [] -> []
          | (h::t) -> (f h)::(map f t);;
            \end{Verbatim}

            \columnbreak

            \begin{Verbatim}
        let rec fold f a l = match l with
            [] -> a
          | (h::t) -> fold f (f a h) t;;
            \end{Verbatim}

          \end{multicols}

          \vspace{-4mm}

          \begin{enumerate}

            \addtolength{\itemsep}{4mm}

            \item Write a function \texttt{makeLists} that when applied to a
                  list \texttt{lst}, creates a new list for every element of
                  \texttt{lst}, returning the results in a single list.  You
                  may use map or fold if you wish, but it is not required.
                  Example: \texttt{makeLists [1; 2; 4]} should produce
                  \texttt{[[1]; [2]; [4]]}.

            \item Using either map or fold and an anonymous function, write
                  a function \texttt{over20} which when applied to a list of
                  ints \texttt{lst}, returns a list of all elements of lst
                  that are 21 or over (preserving their relative order in
                  \texttt{lst}).  Example: \texttt{over20 [33; 18; 21; 19]}
                  should produce \texttt{[33; 21]}.

            \item Write a function \texttt{growNum} which when given a
                  number \texttt{k}, returns a list of the first \texttt{k}
                  integers, starting from \texttt{k}.  Examples:

                  \begin{tabular}[t]{l}

                    \texttt{growNum 0} should return \texttt{[]}
                      \\

                    \texttt{growNum 1} should return \texttt{[1]}
                      \\

                    \texttt{growNum 4} should return \texttt{[1; 2; 3; 4]}
                      \\

                  \end{tabular}

            \item Using either map or fold and an anonymous function, write
                  a curried function \texttt{countNum} which when given a
                  number \texttt{n} and a list of ints \texttt{lst}, returns
                  the number of times \texttt{n} occurs in \texttt{lst}.

                  Examples:

                  \begin{tabular}[t]{l}

                    \texttt{countNum 5 [1; 2; 3; 4]} should return
                    \texttt{0}
                      \\

                    \texttt{countNum 5 [1; 2; 3; 4; 5; 6]} should return
                    \texttt{1}
                      \\

                    \texttt{countNum 5 [5; 5; 6; 5]} should return \texttt{3}
                      \\

                  \end{tabular}

          \end{enumerate}

          \vspace{-2.5mm}

  \end{enumerate}

\end{document}
