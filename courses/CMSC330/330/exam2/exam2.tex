\documentclass[11pt]{article}

  %
  % all on Wednesday;
  % Varun 1
  % Tommy 1
  % Tammy after 3
  % Derek 4
  % Hao   after 4
  %

  %
  % Not sure why I numbered the subparts of #2 (a), then (c), then (b), then
  % (d), going by columns.  Oh well, doesn't make any difference....
  %

  %
  % Next time just drop the instruction about minor syntax errors being
  % ignored; students don't need to know it, and all it does is encourage
  % regrades.
  %

  \usepackage{330-f12}

  \usepackage[none,light,outline,timestamp]{draftcopy}

  \newcommand{\sms}{\ensuremath{\hspace{.1mm}}}

  \newcommand{\mathstr}[1]{\ensuremath{\mathit{#1}}}

  \addtolength{\headsep}{-6mm}

  \addtolength{\textheight}{6mm}

\begin{document}

  \hspace{\fill}%
  \nameblock

  % We have to use \header{}{}{}, since \maketitle generates a new page
  % beforehand.
  %
  \header{\course}{Exam \#2}{\Term}

  \vspace{-2mm}

  \noindent
  {\large
    \textbf{\underline{Do not open this exam until you are told.}
            Read these instructions:
    }
  }

  \vspace{-2mm}

  \thispagestyle{myheadings}

  \begin{enumerate}

    \addtolength{\itemsep}{-2mm}  % reduced space between lines

    \item This is a closed book exam.  {\bf No
          % calculators,
          notes %,
          or other aids are allowed.}  If you have a question during the
          exam, please raise your hand.  Each question's point value is next
          to its number.

    \item {\bf You must turn in your exam \underline{immediately} when time is
          called at the end.}

    \item 6 pages, 4 problems, 100 points, 75 minutes.

    \item In order to be eligible for as much partial credit as possible,
          show all of your work for each problem, and \textbf{clearly
          indicate} your answers.  Credit \textbf{cannot} be given for
          illegible answers.

    \item You will \textbf{lose credit} if \emph{\textbf{any}} information
          above is incorrect or missing, or your name is missing from any
          side of any page.

    \item Parts of this page and the next page are for scratch work.  If
          you need extra scratch paper after you have filled these areas up,
          please raise your hand.  Scratch paper must be turned in with your
          exam, with your name and ID number written on it.  Scratch paper
          \textbf{will not} be graded.
          % Figure out your answer on scratch paper if necessary, then write
          % it {\bf neatly} in the answer space provided.

    % \item To avoid distracting others, \textbf{no one may leave until the
    %      exam is over}.

    \item To avoid distracting others, \textbf{no one may leave in the
          last 10 minutes}.

    \item The Campus Senate has adopted a policy asking students to include
          the following handwritten statement on each examination and
          assignment in every course: ``\textit{I pledge on my honor that I
          have not given or received any unauthorized assistance on this
          examination\/}.''  Therefore, \textbf{just before turning in your
          exam}, you are requested to write this pledge \textbf{in full
          \textmd{and} sign it} below:

          \medskip

          \begin{minipage}[t]{\linewidth}

            \addtolength{\baselineskip}{4mm}

            \ans[\linewidth]

            \ans[\linewidth]

          \end{minipage}

    \medskip

  \end{enumerate}

  \vspace{-2mm}

  \noindent
  Good luck!

  \enlargethispage{13mm}

  \vspace{\fill}

  \noindent
  % \begin{tabular}[t]{@{}p{2.5in}@{\hspace{.5in}}p{4.15in}@{}}
  \begin{tabular}[t]{@{}p{3.1in}@{\hspace{.5in}}p{3.4in}@{}}

    \vspace{14mm}
    \scratchpaper[2.5in]

  &

    \vspace{0mm}

    % \hspace*{\fill}\parbox[t]{3.6in}{\scoreblank{5}{.5mm}}
    \hspace*{\fill}\scoreblank{4}{.5mm}

  \end{tabular}

  \pagebreak

  \addtolength{\headsep}{4mm}

  \addtolength{\textheight}{2mm}

  \pagestyle{headings}

  \markright{Name: \hspace{.2mm} \lans\ans}

    %%%%%%%%%%%%%%%%%%%%%%%%%%%%%%%%%%%%%%%%%%%%%%%%%%%%%%%%%%%%%%%%%%%%%%%%%%

  \begin{enumerate}

    % \item { [16 pts.]} Short--answer.  Answer each one \textbf{briefly}.
    %
    %       \vspace{-3mm}
    %
    %       \begin{enumerate}
    %
    %         \addtolength{\itemsep}{40mm}
    %
    %         \item What do we mean by saying that OCaml has \emph{type
    %               inference}?
    %
    %         \item \textbf{Briefly} describe what \emph{dynamic scoping} is.
    %
    %         \item \textbf{Briefly} describe what a \emph{side effect} is,
    %               \textbf{and} give an example of a code fragment that has a
    %               side effect in C.  (Clearly describe what the side effect
    %               is in your code fragment.)
    %
    %               \vspace{12mm}
    %
    %         \item Why do we say that a recursive descent parser is a
    %               \emph{predictive} parser?
    %
    %         \item What are \emph{downward funargs}, and why are they more
    %               efficient than upward funargs in OCaml?
    %
    %       \end{enumerate}
    %
    %       \vspace{-2.5mm}
    %
    %       \pagebreak

    %%%%%%%%%%%%%%%%%%%%%%%%%%%%%%%%%%%%%%%%%%%%%%%%%%%%%%%%%%%%%%%%%%%%%%%%%%

    \item {[16 pts.]} The following grammar cannot be used for a recursive
          descent parser, because the first sets of its productions are not
          unique.  (The language the grammar generates isn't anything in
          particular.)  Left factor the grammar so it generates the same
          language but does not have this problem.  Clearly indicate your
          answer.

          \vspace{-6.5mm}

          \begin{center}

            \begin{grammar}

              \production{S}{\mathit{ca}S \midspc \mathit{co}S \midspc
                             \mathit{db}T \midspc \mathit{bd}T \midspc
                             T}
                \\

              \production{T}{\mathit{aa}T \midspc \mathit{ab}T \midspc
                             \mathit{a} \midspc \mathit{b}}
                \\

            \end{grammar}

          \end{center}

          \vspace{-3mm}

          \textbf{Note}: many grammars can be written that generate the same
          language.  We are \textbf{not} asking you to write another grammar
          from scratch.  We are asking you to apply the \textbf{specific}
          construction or algorithm that was given in class to left factor a
          grammar, in order to receive full credit.

          \vspace{4in}

          \underline{\hspace{\linewidth}}

          \medskip

          \begin{centering}

            \scratchpaper[6.8in]

          \end{centering}

          \pagebreak

    %%%%%%%%%%%%%%%%%%%%%%%%%%%%%%%%%%%%%%%%%%%%%%%%%%%%%%%%%%%%%%%%%%%%%%%%%%

    \item {[20 pts.]} Some students were given the job of writing an
          \textbf{unambiguous, context--free} grammar that generates the
          following language:

          \vspace{-3.75mm}

          \begin{center}

            \(
              \left\{
                \:
                {a^\mathrm{p}}{b^\mathrm{q}}{c^\mathrm{r}}{d^\mathrm{s}}
                \midspc
                \mathrm{p, \, q, \, r \: \ge \: 0, \ p \: < \: s,
                        \: and \ q \: - \: r \ \textrm{is a multiple of 3}
                }
                \:
              \right\}
            \)

          \end{center}

          \vspace{-3.5mm}

          Note that $\mathrm{q \: - \: r}$ may be negative, and recall that
          0 is a multiple of 3.

          The grammars the students came up with are below.  However, since
          they took CMSC 330 last semester they are not as smart as you, so
          they may have made some mistakes.  Determine whether each grammar
          is correct.  If so, just write ``Correct!''.  If a grammar is
          incorrect, say why \textbf{\underline{and}} give an example
          demonstrating the problem.  For instance, if there are any strings
          in the language above that a grammar can't generate, say so and
          give one.  If a grammar generates some invalid strings not in the
          language above, say so and give one.  If a grammar is ambiguous
          for some strings, say so and give one.  If a grammar has any other
          problems, explain and give an example if possible.  If a grammar
          has multiple errors you only need to identify one problem.

          \vspace{-2mm}

          \begin{multicols}{2}

            \begin{enumerate}

              \addtolength{\itemsep}{0mm}

              \item \begin{grammar}[1.35]

                      \production{S}{TU}
                        \\

                      \production{T}{\mathstr{a}T\mathstr{d} \midspc V}
                        \\

                      \production{U}{\mathstr{d}U \midspc \mathstr{d}}
                        \\

                      \production{V}{\mathstr{b}V\mathstr{c} \midspc
                                     \mathstr{b}X\mathstr{c}}
                        \\

                      \production{X}{\mathstr{b}X\mathstr{c} \midspc
                                     \largeepsilon}
                        \\

                    \end{grammar}

                    \bigskip

                    \begin{center}

                      \answerblank{5}{2.75in}{9.5mm}

                    \end{center}

                    \columnbreak

              \item \begin{grammar}[1.35]

                      \production{S}{\mathstr{a}S\mathstr{d} \midspc T}
                        \\

                      \production{T}{T\mathstr{d} \midspc \mathstr{d}U}
                        \\

                      \production{U}{\mathstr{b}U\mathstr{c} \midspc V
                                     \midspc W}
                        \\

                      \production{V}{\mathstr{bbb}V \midspc \mathstr{bbb}}
                        \\

                      \production{W}{W\mathstr{ccc} \midspc \mathstr{ccc}}
                        \\

                    \end{grammar}

                    \medskip

                    \begin{center}

                      \answerblank{5}{2.75in}{9.5mm}

                    \end{center}

                    \end{enumerate}

            \vspace{-2.5mm}

          \end{multicols}

          \vspace{14mm}

          \begin{multicols}{2}

            \vspace{-2.5mm}

            \begin{enumerate}

              \addtolength{\itemsep}{0mm}

              \setcounter{enumii}{2}

              \item \begin{grammar}[1.35]

                      \production{S}{\mathstr{a}S\mathstr{d} \midspc
                                     S\mathstr{d} \midspc T\mathstr{d}}
                        \\

                      \production{T}{\mathstr{b}T\mathstr{c} \midspc
                                     \mathstr{bbb}T \midspc T\mathstr{ccc}
                                     \midspc \largeepsilon}
                        \\

                    \end{grammar}

                    \bigskip

                    \begin{center}

                      \answerblank{5}{2.75in}{9.5mm}

                    \end{center}

                    \columnbreak

              \item \begin{grammar}[1.35]

                      \production{S}{S\mathstr{d} \midspc T\mathstr{d}}
                        \\

                      \production{T}{\mathstr{a}T\mathstr{d} \midspc U}
                        \\

                      \production{U}{\mathstr{b}U\mathstr{c} \midspc V}
                        \\

                      \production{V}{W \midspc X \midspc \largeepsilon}
                        \\

                      \production{W}{\mathstr{bbb}W \midspc \mathstr{bbb}}
                        \\

                      \production{X}{X\mathstr{ccc} \midspc \mathstr{ccc}}
                        \\

                    \end{grammar}

                    \medskip\smallskip

                    \begin{center}

                      \answerblank{5}{2.75in}{9.5mm}

                    \end{center}

            \end{enumerate}

            \vspace{-2.5mm}

          \end{multicols}

          \pagebreak

    %%%%%%%%%%%%%%%%%%%%%%%%%%%%%%%%%%%%%%%%%%%%%%%%%%%%%%%%%%%%%%%%%%%%%%%%%%

    \item {[32 pts.]}  Write an \textbf{unambiguous, context--free} grammar
          on the next page that describes a small subset of UNIX commands
          according to the following rules:

          \vspace{-2.5mm}

          \begin{itemize}

            \addtolength{\itemsep}{.25mm}

            \item The only three program names that may be used in commands
                  are \texttt{a}, \texttt{b}, and \texttt{c}.  The shortest
                  valid commands consist simply of one of the three program
                  names.

            \item Programs may also be grouped together in a pipeline
                  separated by the pipe symbol \verb@|&@.  (This is the only
                  pipe symbol that can be used).  If multiple programs
                  appear in a pipeline, your grammar should express the
                  interpretation that they are performed or grouped in
                  \textbf{left--to--right} order.

            \item Either a single program, or the first program in a
                  pipeline, may optionally use input redirection, indicated
                  with the symbol \texttt{<}\hspace{.35mm} followed by a
                  filename.  The only three filenames are \texttt{f},
                  \texttt{g}, and \texttt{h}.

            \item Either a single program, or the last program in a
                  pipeline, may optionally use output redirection, indicated
                  with the symbol \texttt{>}\hspace{.35mm} followed by a
                  filename.  Programs in the middle of a pipeline (any
                  programs besides the first or last ones) cannot use input
                  or output redirection.  If a single program uses both
                  input and output redirection, the input redirection must
                  appear first.

            \item Input redirection has priority over pipelining, but
                  pipelining has priority over output redirection (see the
                  examples below).  If a single program uses both input and
                  output redirection, the input redirection should have
                  priority.

            \item No other program names, filenames, or other symbols or
                  operations may be used in commands.  (Don't worry about
                  spaces in commands.)

          \end{itemize}

          \vspace{-2mm}

          Some valid commands your grammar should generate are:

          \smallskip

          \hspace{.35in}%
          \begin{tabular}[t]
                {c@{\hspace{1.05in}}c@{\hspace{1.05in}}c@{\hspace{1.05in}}c@{}}

            \verb@a@
              & \verb@a |& b |& c@
              & \verb@a < f |& b@
              & \verb@b |& c > g@
              \\

          \end{tabular}

          \smallskip

          Since input redirection has priority over pipelining, in the third
          command above the program \texttt{a} is run with its input
          redirected from the file \texttt{f}, then \texttt{a}'s output is
          used as the input for the program \texttt{b}.  (This means that
          your grammar should give the interpretation \verb@(a < f) |& b@ to
          this command.)  But since pipelining has priority over output
          redirection, in the fourth command above the program \texttt{b} is
          run, its output is used as the input for the program \texttt{c},
          and \texttt{c}'s results are sent to the file \texttt{g}.

          Some invalid commands, and reasons why they're incorrect, are:

          {

            \renewcommand{\arraystretch}{1.25}

            \hspace{.35in}%
            \begin{tabular}[t]{p{1in}@{\hspace{.5in}}p{4.65in}@{}}

              \verb@a b@
                & no pipeline or redirection symbol separates \texttt{a} and
                  \texttt{b}
                \\

              \verb@a |& |& b@
                & is missing a program name, or has an extra pipe symbol
                \\

              \verb@a < f < g@
                & input redirection can only be used once in a command
                \\

            \end{tabular}

          }

          \medskip

          In order that your answer can be graded as accurately and quickly
          as possible, your grammar should use nonterminals beginning with S
          (S, T, U, etc.).  Clearly indicate your answer, and please write
          the vertical bar symbol used to separate productions in a grammar
          much larger than the pipeline symbol, so they can be clearly
          distinguished.  Use the area below to figure out your grammar,
          then write it \textbf{\underline{neatly}} on the next page.
          \textbf{Note:} if you have trouble writing the full grammar, try
          to give a version that's correct except it doesn't use input
          redirection.

          \pagebreak

          \hspace*{0mm}

          \enlargethispage{6mm}

          \vspace{\fill}

          \begin{center}

            \huge

            \textsl{\textcolor{black}{Don't miss the last problem on the next
            page!}}

          \end{center}

          \pagebreak

    %%%%%%%%%%%%%%%%%%%%%%%%%%%%%%%%%%%%%%%%%%%%%%%%%%%%%%%%%%%%%%%%%%%%%%%%%%

    \item {[32 pts.]} Write a function \texttt{quicksort} in OCaml, that
          will sort a list of integers using the quicksort algorithm.  Its
          only argument is the list to be sorted, and it should return the
          sorted list.

          Here's a review of quicksort: it picks a pivot element (the first
          element of the list may be used), and arranges the list elements
          such that all the elements less than the pivot are before it, and
          all those greater than or equal to the pivot (note that the list
          elements are not necessarily unique) are after it.  This is called
          partitioning.  Note that the elements before and after the pivot
          are not yet sorted, but quicksort then calls itself recursively
          on the sublist of elements before the pivot, and on the sublist of
          elements after the pivot.  Combining the sorted sublists, with the
          pivot between them, results in the sorted list.

          You may use anything in the \texttt{Pervasives} module, but
          \textbf{no other library modules may be used}. The infix
          \texttt{@} operator from the \texttt{Pervasives} module, which
          appends two lists, may be useful.  You may write helper functions
          but you must show them in full, and they should use currying if
          they have multiple arguments.  No loops or references may be used.
          Comments are optional, but your function must be written
          \textbf{\underline{\underline{\emph{\large neatly}}}}.

          \medskip

          \enlargethispage{8mm}

          \answerblank{21}{\linewidth}{9.15mm}

          \pagebreak

    %%%%%%%%%%%%%%%%%%%%%%%%%%%%%%%%%%%%%%%%%%%%%%%%%%%%%%%%%%%%%%%%%%%%%%%%%%

  \end{enumerate}

  % % end of numbered questions
  %
  % \markright{}
  %
  % \begin{center}
  %
  %   \scratchpaper
  %
  %   \bigskip
  %
  %   \Large
  %
  %   You can separate this page if you like, to use in solving any
  %   \linebreak
  %   questions, as long as you \textbf{\underline{write your name} on the
  %   other side.}
  %
  % \end{center}
  %
  % \pagebreak
  %
  % \markright{}
  %
  % \underline{\hspace{\linewidth}}
  %
  % \begin{center}
  %
  %   \scratchpaper
  %
  %   \bigskip
  %
  %   \Large
  %
  %   You can separate this page if you like, to use in solving earlier
  %   \linebreak
  %   questions, as long as you \textbf{\underline{write your name} on the
  %   other side.}
  %
  % \end{center}

   %%%%%%%%%%%%%%%%%%%%%%%%%%%%%%%%%%%%%%%%%%%%%%%%%%%%%%%%%%%%%%%%%%%%%%%%%%

\end{document}
