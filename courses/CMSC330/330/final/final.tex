\documentclass[11pt]{article}

  %
  % finals schedule:
  %
  % Varun no finals; free Dec. 13 and Dec. 15-16.  Plans starting 3:00 Dec. 18
  % Tommy Dec. 13 8-10, Dec. 18 8-10
  % Tammy final project due by the end of the Dec. 19
  % Derek no finals
  % Hao   Dec. 10 take-home final image understanding, Dec. 12 final
  %       project flight Dec. 19
  %

  %
  % As given, the sentence
  %
  % Any of the four operands can appear as functions in a function
  % application, and they can also appear as data (meaning functions can be
  % applied to them, lists can be constructed with them, etc.).
  %
  % in #7 said:
  %
  % The identifiers can appear as functions in a function application, and
  % they can also appear as data (meaning functions can be applied to them,
  % lists can be constructed with them, etc.).
  %
  % and the expression [] f g was not an example (it was added later, before
  % the second DSS student took the exam).
  %

  %
  % A few students- and some TAs- were confused by where the wording on #5
  % (the DFA question) said w \in {a, b}*- they thought that the * meant
  % that the empty string was in the language.
  %

  \usepackage{330-f12}

  \usepackage[none,light,outline,timestamp]{draftcopy}

  \newcommand{\quest}[3]{%
    \begin{tabular}[t]{@{}p{#1}@{\hspace{3mm}}%
                         @{\underline{\hspace{#2}}}@{}}%
                       \sloppy#3\end{tabular}%
  }

  \newcommand{\sms}{\ensuremath{\hspace{.1mm}}}

  \newcommand{\mathstr}[1]{\ensuremath{\mathit{#1}}}

  \setlength{\columnsep}{16mm}

  \addtolength{\headsep}{-6mm}

  \addtolength{\textheight}{6mm}

\begin{document}

  \hspace{\fill}%
  \nameblock

  % We have to use \header{}{}{}, since \maketitle generates a new page
  % beforehand.
  %
  \header{\course}{\textsc{\huge Final exam}}{\Term}

  \vspace{-2mm}

  \noindent
  {\large
    \textbf{\underline{Do not open this exam until you are told.}
            Read these instructions:
    }
  }

  \vspace{-2mm}

  \thispagestyle{myheadings}

  \begin{enumerate}

    \addtolength{\itemsep}{-2mm}  % reduced space between lines

    \item This is a closed book exam.  \textbf{No calculators, notes, or
          other aids are allowed.}  If you have a question during the exam,
          please raise your hand.  Each question's point value is next to
          its number.

    \item To avoid distracting others, \textbf{no one may leave in the
          last 15 minutes}.

    \item \textbf{You must turn in your exam \underline{immediately} when
          time is called at the end.}

    \item 12 pages, 10 problems, 200 points, 120 minutes.

    \item Show all of your work for each problem, \textbf{write legibly},
          and \textbf{clearly indicate} your answers.

    \item You will \textbf{lose credit} if \emph{\textbf{any}} information
          above is incorrect or missing, or your name is missing from any
          side of any page.

    \item \label{code-style} Comments are unnecessary, but any code you have
          to write must be written \textbf{\underline{neatly}}, with proper
          indentation.  Credit \textbf{cannot} be given for illegible
          answers or code.

    \item Parts of this page and another page, and the last page, are for
          scratch work.  If you need extra scratch paper \textbf{after} you
          have filled all of these areas up, please raise your hand.
          Scratch paper must be turned in with your exam, with your name and
          ID number written on it.  Scratch paper \textbf{will not} be
          graded.

    % \item When you are finished, you may turn in your exam at the front and
    %       leave \textbf{\underline{quietly}}.
    %       % However, to avoid distracting others, \textbf{no one may leave
    %       % during the last fifteen minutes}.

    % \item To avoid distracting others, \textbf{no one may leave until the
    %      exam is over}.

    \item The Campus Senate has adopted a policy asking students to include
          the following handwritten statement on each examination and
          assignment in every course: \emph{I pledge on my honor that I
          have not given or received any unauthorized assistance on this
          examination}.  Therefore, \textbf{just before turning in your
          exam}, you are requested to write this pledge
          \textbf{\underline{in full} \emph{\textmd{and}} \underline{sign
          it}} below:

          \medskip

          \begin{minipage}[t]{6.7in}

            \addtolength{\baselineskip}{4mm}

            \ans[\linewidth]

            \ans[\linewidth]

          \end{minipage}

          \medskip

  \end{enumerate}

  \vspace{-2mm}

  \noindent
  Good luck!

  \enlargethispage{10mm}

  \vspace{\fill}

  \noindent
  % \begin{tabular}[t]{@{}p{2.5in}@{\hspace{.5in}}p{4.15in}@{}}
  \begin{tabular}[t]{@{}p{2in}@{\hspace{.35in}}p{3.4in}@{}}

    \vspace{12mm}
    \hspace*{-7mm}\scratchpaper[2.5in]

  &

    \vspace{0mm}

    % \hspace*{\fill}\parbox[t]{3.6in}{\scoreblank{5}{.5mm}}
    \hspace*{\fill}\scoreblank{10}{.365mm}

  \end{tabular}

  \pagebreak

  \addtolength{\headsep}{4mm}

  \addtolength{\textheight}{2mm}

  \pagestyle{headings}

  \markright{Name: \hspace{.2mm} \lans\ans}

  \begin{enumerate}

    %%%%%%%%%%%%%%%%%%%%%%%%%%%%%%%%%%%%%%%%%%%%%%%%%%%%%%%%%%%%%%%%%%%%%%%%%%

    \item {[20 pts.]} Short--answer.  Answer each one \textbf{briefly}.

          \vspace{-3.5mm}

          \begin{enumerate}

            \addtolength{\itemsep}{39.5mm}

            \item \textbf{\underline{Briefly}} describe what polymorphism
                  refers to, \textbf{and} describe an example where
                  polymorphism would occur in OCaml.  (A code example is not
                  necessary, just describe in words something in OCaml that
                  would be an example of polymorphism.)

            \item \textbf{\underline{Briefly}} describe what closures and
                  objects have in common (besides both being mentioned
                  in CMSC 330).

            \item We briefly discussed operational semantics.  What is the
                  purpose of operational semantics?  (Meaning what does it
                  do or what is it used for, other than trying to confuse
                  CMSC 330 students?)

            \item \textbf{\underline{Briefly}} describe the main
                  characteristic of markup languages, and their general
                  purpose.

            \item \textbf{\underline{Briefly}} describe
                  \textbf{\underline{one}} advantage of stop and copy
                  garbage collection.
                  % (It has more than one advantage; you only need to describe
                  % one.)

          \end{enumerate}

          \vspace{-2.5mm}

          \pagebreak

    %%%%%%%%%%%%%%%%%%%%%%%%%%%%%%%%%%%%%%%%%%%%%%%%%%%%%%%%%%%%%%%%%%%%%%%%%%

    \item {[14 pts.]} \label{filter-problem} In class we discussed two
          higher--order functions: \texttt{map}, which applies a function to
          a list and returns the list of results, and \texttt{fold}, which
          returns the result of applying a function to an accumulator and
          each element of a list.

          Another useful higher--order function is \texttt{filter p l},
          which takes a function \texttt{p} and a list \texttt{l} as
          arguments.  \texttt{p} is a predicate, meaning it returns true or
          false, and \texttt{filter} returns the elements of \texttt{l} for
          which \texttt{p} returns true, in the same order they appear in
          \texttt{l}.

          \vspace{-3mm}

          \begin{tabular}[t]{@{}p{3.75in}@{\hspace{.515in}}p{2.55in}@{}}

            \vspace{0mm}

            For example, on the right are OCaml definitions of a predicate
            \texttt{even}, which returns true if its \texttt{int} argument
            is even, and an \texttt{int} list \texttt{l}.

            & \vspace{0mm}

              \begin{BVerbatim}
        let even n = (n mod 2 = 0)
        let l = [1; 2; -1; -6; 2; 4; -5]
              \end{BVerbatim}

          \end{tabular}

          \vspace{-4.5mm}

          Given the above, \texttt{filter even l} should return \texttt{[2;
          -6; 2; 4]}, since the elements in this list are those in
          \texttt{l} for which \texttt{even} returned true.

          \vspace{-3.5mm}

          \begin{tabular}[t]{@{}p{3.75in}@{\hspace{.515in}}p{2.55in}@{}}

            \vspace{0mm}

            Ruby has its own version of \texttt{filter}, named
            \texttt{select()} (part of the \texttt{Enumerable} module).
            Supposing that \texttt{select()} didn't exist, you are to
            implement it, by writing a Ruby method \texttt{select()} that
            has the functionality of \texttt{filter} described above.  It
            should take an array and a code block as arguments, and return
            an array with only the elements of its argument array for which
            the code block returns true (there may be zero or more such
            elements).  For example, consider the code on the right.

            \vspace{1.75mm}

            & \vspace{0mm}

              \addtolength{\baselineskip}{-.275mm}

              \begin{BVerbatim}
        def even(x)
          return x % 2 == 0
        end

        def nonnegative(x)
          return x > 0
        end

        arr= [1, 2, -1, -6, 2, 4, -5]
              \end{BVerbatim}

          \end{tabular}

          \vspace{-4.25mm}

          The call \verb@select(arr) { |e| even(e) }@ to your method
          should return the array \texttt{[2, -6, 2, 4]}, and the call
          \verb@select(arr) { |e| nonnegative(e) }@ should return \texttt{[1,
          2, 2, 4]}.  Recall that \texttt{yield} calls a code block in
          Ruby.  \textbf{Read \#\ref{code-style} on the front page.}

          Since you are supposed to be implementing \texttt{select} as
          if it did not exist, your method \textbf{may not} use it.

          \smallskip

          \answerblank{14}{\linewidth}{9.15mm}

          \pagebreak

    %%%%%%%%%%%%%%%%%%%%%%%%%%%%%%%%%%%%%%%%%%%%%%%%%%%%%%%%%%%%%%%%%%%%%%%%%%

    \item {[20 pts.]} Write a regular expression that describes or
          recognizes the following language:

          \vspace{-.5mm}

          \begin{centering}

            \(
              \left\{
                \:
                {a^\mathrm{m}}{b^\mathrm{n}}
                \midspc
                \mathrm{m \, + \, n \ \, \textrm{is \textbf{not} a multiple
                        of 3}
                }
                \:
              \right\}
            \)

          \end{centering}

          Strings in this language can be of any length, and can contain any
          number of \emph{a}'s and \emph{b}'s, as long as they satisfy the
          property above.  Recall from CMSC 250 that 0 is a multiple of 3.

          Your regular expression may use \textbf{\underline{only}} the
          three formal regular expression operations concatenation,
          alternation, and Kleene closure, which were defined in class.
          These operations may be nested, and terminal symbols, parentheses,
          and $\epsilon$ may be used as well, but do \textbf{\underline{do
          not use}} \textbf{any} other regular expression operations,
          and do \textbf{\underline{not}} write a Ruby regular expression.

          \vspace{4in}

    %%%%%%%%%%%%%%%%%%%%%%%%%%%%%%%%%%%%%%%%%%%%%%%%%%%%%%%%%%%%%%%%%%%%%%%%%%

    \item {[12 pts.]} Simplify each of the following lambda calculus
          expressions as much as possible (but no further):

          \bigskip\bigskip

          \begin{enumerate}

            \addtolength{\itemsep}{16mm}

            \item \quest{3.7in}{2.375in}
                        {($\mathrm{\lambda x.\lambda y.x) \ w \ z}$}

            \item \quest{3.7in}{2.375in}
                        {($\mathrm{\lambda x.\lambda y.y \ x) \ y}$}

            \item \quest{3.7in}{2.375in}
                        {$\mathrm{
                          ( \lambda x.x \ (\lambda y.\lambda z.z) ) \
                          \lambda x.( x \ w \ v )
                          }$}

          \end{enumerate}

          \pagebreak

    %%%%%%%%%%%%%%%%%%%%%%%%%%%%%%%%%%%%%%%%%%%%%%%%%%%%%%%%%%%%%%%%%%%%%%%%%%

    \item {[24 pts.]} Draw a valid, complete, deterministic finite automaton
          (DFA) that recognizes or accepts the language:

          \medskip

          \begin{centering}

            \(
              \left\{
                \
                \begin{array}[c]{@{}l@{}}
                w \:
                \end{array}
                \left| \ \
                \parbox[c]{5.8in}{
                  $w \sms \in \sms \{ \sms a, \, b \sms \}^*$,
                  and \emph{w}'s first two symbols are different, its last
                  two symbols are the same as its first two symbols, and
                  its last two symbols are in the same order as the first two
                }
              \ \:
              \right.
              \right\}
            \)

          \end{centering}

          \medskip

          Strings in this language can be of any length and can have any
          number of \emph{a}'s and \emph{b}'s, as long as they have the
          property above.  Note: a string's first two symbols and last two
          symbols may overlap if the string is less than 4 characters; in
          this case the string might or might not be valid.  Three examples
          are:

          \vspace{-2.5mm}

          \begin{itemize}

            \addtolength{\itemsep}{-.25mm}

            \item The string \emph{abbbabab} is in this language, since it
                  begins with \emph{ab} and also ends with \emph{ab}.

            \item The string \emph{aabaa} is \textbf{not} in this language
                  because its first two symbols aren't different.

            \item The string \emph{abaaba} is also \textbf{not} in this
                  language, because although its last two symbols are the
                  same as its first two symbols they don't appear in the
                  same order as the first two symbols.

          \end{itemize}

          \vspace{-2.5mm}

          Note these examples \textbf{do \underline{not}} illustrate all
          possible valid or invalid strings.

          Be sure to give a DFA, not an NFA, and \textbf{\underline{do not
          use}} any of the notational shortcuts for DFAs that were
          given in class.  \textbf{Write neatly}-- try figuring out your DFA
          in the scratch area first.

          \pagebreak

    %%%%%%%%%%%%%%%%%%%%%%%%%%%%%%%%%%%%%%%%%%%%%%%%%%%%%%%%%%%%%%%%%%%%%%%%%%

    \item {[26 pts.]} Write a function \texttt{product\_of\_lasts} in OCaml,
          which has a list of \texttt{int} lists as its single parameter,
          and which returns an \texttt{int} (meaning the function's type is
          \texttt{int list list -> int}).  It should return the product of
          the \textbf{last} or final elements of each of the inner lists in
          its list parameter.  For example:

          \vspace{-4.5mm}

          \begin{center}

            \begin{tabular}[t]{|l|c|}

              \multicolumn{1}{c}{the call}
                & \multicolumn{1}{c}{should return}
                \\ \hline

              \texttt{product\_of\_lasts [[1; 2]; [3; 4]; [5; 6]]}
                & 48 (which is $2 \, * \, 4 \, * \, 6$)
                \\ \hline

              \texttt{product\_of\_lasts [[2; 3]; [4; 5; 6]; [7]]}
                & 126 (which is $3 \, * \, 6 \, * \, 7$)
                \\ \hline

              \texttt{product\_of\_lasts [[330]]}
                & 330
                \\ \hline

            \end{tabular}

          \end{center}

          \vspace{-1.5mm}

          Note that the inner lists may have different lengths.  Your
          function may assume, without checking, that there will be at least
          one inner list, and each inner list will have at least one
          element.

          Your function may use anything in the \texttt{Pervasives} module,
          \textbf{and} anything in the \texttt{List} module.  Other than
          that, no other library modules may be used.  For \textbf{full
          credit}, your function should use or employ
          \textbf{\underline{at least one of} the functions}
          \texttt{List.map}, \texttt{List.fold\_left} (the same as the
          \texttt{fold} function discussed in class), or
          \texttt{List.filter} (which are all described in Problem
          \#\ref{filter-problem}, as you saw).  Of course if you use any
          helper functions you must show them in full.  No loops or
          references may be used.  \textbf{Read \#\ref{code-style} on the
          front page.}

          \smallskip

          \enlargethispage{6mm}

          \answerblank{20}{\linewidth}{9mm}

          \pagebreak

    %%%%%%%%%%%%%%%%%%%%%%%%%%%%%%%%%%%%%%%%%%%%%%%%%%%%%%%%%%%%%%%%%%%%%%%%%%

    \item {[28 pts.]} This problem involves writing a context--free grammar
          that generates the small set of \textbf{nonempty} expressions
          similar (but not identical) to those in OCaml described below.

          \vspace{-2.75mm}

          \begin{itemize}

            \addtolength{\itemsep}{2mm}

            \item Expressions may contain only the binary operators below,
                  listed in order of decreasing precedence:

                  \vspace{-2.5mm}

                  \begin{center}

                    \begin{tabular}[t]{@{}|@{\hspace{3mm}}c@{\hspace{3mm}}|
                                       @{\hspace{3mm}}p{4.45in}
                                       @{\hspace{2.25mm}}|@{}}

                          \hline

                      \multirow{4}{*}{function application}
                        & Function application is \textbf{left associative},
                          and has the highest precedence of all the operators.
                          Function application \textbf{does not use} an
                          operator symbol-- for example, the expression
                          \texttt{f g} indicates that the function
                          \texttt{f} is being called on the operand
                          \texttt{g}.
                          \\ \hline

                      \texttt{::}
                        & This operator prepends an element to a list and
                          is \textbf{right associative}.
                          \\ \hline

                      \multirow{2}{*}{\texttt{;}}
                         & The sequencing operator is \textbf{left
                           associative} and has the lowest precedence of all
                           the operators.
                           \\ \hline

                    \end{tabular}

                  \end{center}

                  \smallskip

            \item The operands that may appear in expressions are the
                  identifiers \texttt{f}, \texttt{g}, and \texttt{h}, as
                  well as the empty list \verb@[]@.  Any of the four
                  operands can appear as functions in a function
                  application, and they can also appear as data (meaning
                  functions can be applied to them, lists can be constructed
                  with them, etc.).

            \item Valid expressions and subexpressions may also be
                  surrounded by parentheses, which can be thought of as
                  operators having the highest precedence.

            \item Some valid expressions are:

                  \begin{tabular}[t]{@{}p{2.1in}p{2.9in}p{1.15in}@{}}

                    \verb@f g h@
                      & \verb@f::([])::[]@
                      & \verb@f ; [] ; h@
                      \\

                    \verb@(f::g)::(h::g)@
                      & \verb@f::g ; (f (g h)) ; (g [])@
                      & \verb@[] f g@
                      \\

                  \end{tabular}

                  \vspace{1.5mm}

                  % Note that \verb@([])@ is valid, since the empty list is
                  % not used as a function application (an argument does not
                  % appear following it).  The expression \verb@[]@ is
                  % likewise valid.

            \item Some invalid expressions are:\hspace{1.5mm}
                  \begin{tabular}[t]{@{}p{.5in}@{\hspace{.3in}}
                                     p{2.925in}@{}}

                    \verb@f::g::@
                      & missing operand
                      \\

                    \verb@f:: ;@
                      & also missing operand
                      \\

                    % \verb@[] f g@
                    %   & empty list used as function application
                    %   \\
                    %
                    % \verb@f ([] g) h@
                    %   & empty list also used as function application
                    %   \\

                    \verb@f ()@
                      & empty parenthesized expression
                      \\

                    \verb@[f; g]@
                      & nonempty list
                      \\

                  \end{tabular}

          \end{itemize}

          \vspace{-1mm}

          The examples above are for illustration but do
          \textbf{\underline{not}} exemplify all possible valid or invalid
          expressions.

          Write an \textbf{unambiguous} context--free grammar below that
          captures the properties described above.  In order that your
          answer can be graded as accurately and quickly as possible, use
          consecutive nonterminals beginning with S (S, T, U, etc.).

          % \vspace{-1.5mm}
          % 
          % \begin{enumerate}
          % 
          %   \item 
          % 
          %         % Note that in this part there are \textbf{no restrictions}
          %         % on where in expressions the empty list \verb@[]@ may
          %         % appear, so an expression like \verb@[] f g@ is valid.  For
          %         % full credit, your grammar should be \textbf{unambiguous}.
          % 
          %         \pagebreak
          % 
          %   \item Now modify your grammar-- it will not be a large
          %         modification-- to include the following restriction on the
          %         empty list.  Since the empty list is not a function, it
          %         \textbf{cannot appear first in a function application.}
          %         Either a function name, or any parenthesized
          %         subexpression, can appear first in a function application.
          %         The empty list may appear anywhere else.  The modified
          %         grammar should still enforce the same associativity and
          %         precedence of the operators and be unambiguous.
          % 
          %         Given this restriction, two invalid expressions that use
          %         the empty list as a function application are \verb@[] f g@
          %         and \verb@f ([] g) h@.  Note that \verb@([])@ is valid,
          %         since the empty list is not used as a function application
          %         (an argument does not appear following it).  The
          %         expression \verb@[]@ is likewise valid.  Lastly, an
          %         expression like \verb@f []@ is valid, as \texttt{f} is
          %         being applied to the empty list, so the empty list is not
          %         being used as a function.  These examples are for
          %         illustration but do \textbf{\underline{not}} exemplify all
          %         possible invalid expressions given the restriction.
          % 
          % \end{enumerate}

          \pagebreak

    %%%%%%%%%%%%%%%%%%%%%%%%%%%%%%%%%%%%%%%%%%%%%%%%%%%%%%%%%%%%%%%%%%%%%%%%%%

    \item {[24 pts.]} Consider the multithreaded Ruby code below.  Notice
          there are line numbers in comments to the right of some of the
          statements.  The program is supposed to implement a buffer for
          threads to exchange data.

          \vspace{-.5mm}

          \setlength{\columnseprule}{.15mm}

          \begin{multicols}{2}

            \enlargethispage{10mm}

            \begin{Verbatim}[baselinestretch=.9]
        class Buffer
        
          def initialize()
            @buffer= nil
            @empty= true
            @monitor= Monitor.new()
            @cond= @monitor.new_cond()
          end
        
          def produce(obj)
            @monitor.synchronize() {
              @cond.wait_while { !@empty }  # 1
              @empty= false                 # 2
              @cond.broadcast()             # 3
              @buffer= obj                  # 4
            }
          end
        
          def consume()
            @monitor.synchronize() {
              @cond.wait_while { @empty }  # 5
              @empty= true                 # 6
              @cond.broadcast()            # 7
              return @buffer               # 8
            }
          end
        
        end
        
        buffer= Buffer.new()
        
        p1= Thread.new() {
          buffer.produce(1)
        }
        
        p2= Thread.new() {
          buffer.produce(2)
        }
        
        p3= Thread.new() {
          buffer.produce(3)
        }
        
        c1= Thread.new() {
          j= buffer.consume()
        }
        
        c2= Thread.new() {
          k= buffer.consume()
        }
        
        p1.join()
        p2.join()
        p3.join()
        c1.join()
        c2.join()
            \end{Verbatim}

            \columnbreak

            Thread schedules are given below as a list of thread name/line
            number (or range) pairs, e.g., (p1, 1), (c1, 5-8), would mean
            thread p1 executed line 1, followed by thread c1 executing lines
            5--8.

            For each of the schedules shown, determine whether the schedule
            is possible.  If it is, say what values are assigned to
            \texttt{j} and \texttt{k} in threads c1 and c2.  If the schedule
            is \textbf{not} possible, \textbf{briefly} explain why.

            \vspace{-1mm}

            \begin{enumerate}

              \addtolength{\itemsep}{32mm}

              \item (p1, 1--4), (p2, 1--4), (c1, 5--8), (c2, 5--8)

                    \bigskip

                    \answerblank{6}{\linewidth}{9mm}

              \item (p3, 1--4), (c2, 5--8), (p2, 1--4), (c1, 5--8), (p1,
                    1--4)

                    \bigskip

                    \answerblank{6}{\linewidth}{9mm}

                    \columnbreak

              \item (p2, 1--4), (p3, 1), (p1, 1), (c2, 5-8), (c1, 5), (p1,
                    2--4), (c1, 6-8), (p3, 2--4)

                    \bigskip

                    \answerblank{6}{\linewidth}{9mm}

              \item (c1, 5), (c2, 5), (p1, 1--4), (p3, 1), (p2, 1), (c2,
                    6-8), (c1, 6--8), (p3, 2--4)

                    \bigskip

                    \answerblank{6}{\linewidth}{9mm}

              \item (c1, 5), (p1, 1-3), (c1, 6--8), (p1,4), (p3, 1--4), (c2,
                    5-8), (p2, 1--4)

                    \bigskip

                    \answerblank{6}{\linewidth}{9mm}

            \end{enumerate}

            \vspace{-2.5mm}

            \columnbreak

            \begin{centering}

              \scratchpaper[\linewidth]

            \end{centering}

          \end{multicols}

          \setlength{\columnseprule}{0mm}

    %%%%%%%%%%%%%%%%%%%%%%%%%%%%%%%%%%%%%%%%%%%%%%%%%%%%%%%%%%%%%%%%%%%%%%%%%%

    \begin{multicols}{2}

      \item {[16 pts.]} Some languages allow different parameters to a
            function to be passed using different transmission mechanisms.
            For example, parameters are passed by value by default in
            Pascal, but putting the keyword \texttt{var} before any formal
            parameter makes it a reference parameter instead.  Suppose that
            C allowed different parameters to be passed in different ways,
            and the function \texttt{f} in the following program used call
            by \textbf{reference} for its first two parameters \texttt{r1}
            and \texttt{r2}, and call by \textbf{name} for its second two
            parameters \texttt{n1} and \texttt{n2}.  Give the complete
            output that the program would produce under these circumstances.
            (All other features of the program work the same way they would
            in real C.)

            \VerbatimInput[gobble=0]{parameters.c}

            \answerblank{4}{\linewidth}{9mm}

            \columnbreak

    %%%%%%%%%%%%%%%%%%%%%%%%%%%%%%%%%%%%%%%%%%%%%%%%%%%%%%%%%%%%%%%%%%%%%%%%%%

      \item {[16 pts.]} Give the complete output that the program below
            would produce if C used \textbf{dynamic scoping} (instead of
            static scoping, as it really does).  The program would be valid
            and have no errors if C used dynamic scoping.

            \VerbatimInput[gobble=0]{scope.c}

            \medskip

            \answerblank{7}{\linewidth}{9mm}

    \end{multicols}

    \pagebreak

  \end{enumerate}

  \pagebreak

  \begin{center}

    \scratchpaper

    \medskip\smallskip

    \Large

    You can separate this page to use in solving any questions
    \linebreak
    as long as you \textbf{\underline{write your name} above.}

  \end{center}

  \pagebreak

  \begin{center}

    \scratchpaper

    \medskip\smallskip

    \Large

    You can separate this page to use in solving any questions
    \linebreak
    as long as you \textbf{\underline{write your name} above.}

  \end{center}

  \hspace*{1mm}

    %%%%%%%%%%%%%%%%%%%%%%%%%%%%%%%%%%%%%%%%%%%%%%%%%%%%%%%%%%%%%%%%%%%%%%%%%%

  % % end of numbered questions
  %
  % \markright{}
  %
  % \begin{center}
  %
  %   \scratchpaper
  %
  %   \bigskip
  %
  %   \Large
  %
  %   You can separate this page if you like, to use in solving any
  %   \linebreak
  %   questions, as long as you \textbf{\underline{write your name} on the
  %   other side.}
  %
  % \end{center}
  %
  % \pagebreak
  %
  % \markright{}
  %
  % \underline{\hspace{\linewidth}}
  %
  % \begin{center}
  %
  %   \scratchpaper
  %
  %   \bigskip
  %
  %   \Large
  %
  %   You can separate this page if you like, to use in solving earlier
  %   \linebreak
  %   questions, as long as you \textbf{\underline{write your name} on the
  %   other side.}
  %
  % \end{center}

   %%%%%%%%%%%%%%%%%%%%%%%%%%%%%%%%%%%%%%%%%%%%%%%%%%%%%%%%%%%%%%%%%%%%%%%%%%

\end{document}
