\documentclass[11pt,fleqn]{article}

  %
  % Emphasize to graders writing the score for subparts.
  %

  %
  % After the grading has been going on a while doublecheck the TAs'
  % grading.
  %

  %
  % Emphasize before grading starts:
  %
  % There has to be an explanation of what's wrong with an answer in any
  % grading.
  %
  % Be sure they put their initials.
  %
  % Write neatly!
  %
  % Positive points next to question number, not negative deductions.
  %
  % The TAs should be giving partial credit for things that are attempted
  % and partly right, but not completely right.
  %

  \usepackage{330-f12}

  \addtolength{\headsep}{-7mm}

  \addtolength{\textheight}{7mm}

  \newcommand{\terminal}[1]{\texttt{#1}}

  \showinfo

\begin{document}

  \header{\course}{Final exam grading key}{\Term}

  {

    \setlength{\leftmargini}{5mm}

    \vspace*{-6mm}

    \begin{info}{\textbf{\underline{\Large Common grading criteria for all
                 parts}}}

      \vspace*{-2mm}

      \enlargethispage{6mm}

      \setlength{\leftmarginii}{5mm}

      \begin{enumerate}

        \addtolength{\itemsep}{.5mm}

        \item Be sure to first check that our answers are correct, that the
              grading criteria make sense, and that the point values add up
              right.

        \item Circle or mark \textbf{anything} that is incorrect.  You don't
              need to write the correct answers but students
              \textbf{\underline{must}} be able to tell what they are losing
              credit for (what was wrong or missing in their answer), and not
              have any points subtracted without being clear what the reason
              is.  Students will ask us about the grading, and we need to be
              able to tell why they lost points.

              If something is wrong circle or mark it with an 'X'.  If
              something that was supposed to be present is missing you can
              just circle the part of the question describing what was missed,
              rather than writing out a description.  If you need to write a
              description of an error just write a \textbf{few short} words.

        \item Mark negative deductions next to any mistakes, but write the
              total (positive) points for each question \textbf{next to its
              number} (\underline{next to where the question says ``[XX
              pts\.]}'').

              Also write the total score for that question in the space
              \textbf{\underline{on the front of the exam}} (so the total
              score for each question is written twice).

        \item For any problems with subparts where the subparts are worth
              different numbers of points, if the points for each subpart
              are not already marked on the exam then mark the points for each
              subpart as a fraction $\left(\frac{n}{m}\right)$ next to that
              subpart.

        \item Don't deduct fractional points; grade using whole number points
              only.

        \item Unless noted below, give partial credit according to the
              grading key if something is partially, but not completely,
              incorrect.

              Unless otherwise indicated, the points for tasks should be
              prorated if the task is partially but not completely right
              (i.e., partial credit should be given).

              In general try to avoid penalizing twice for the same problem,
              or don't penalize for a problem and then for each of its
              effects.

              % For example, suppose the grading key for a question says that
              % making a proper function call is \pts{3}-- you can deduct
              % (using uniform criteria) \pts{-1} if a call was given but has
              % some small mistakes, \pts{-2} if a call was given but has
              % larger mistakes, and \pts{-3} if the call is completely
              % missing or totally wrong.

        \item Anything correct gets full credit, even if it's not the same as
              the solution on the answer key (unless it contradicts what the
              question was asking for in some way, such as using some language
              feature that the problem said could not be used).

        \item Put your initials at the bottom of the page for each question you
              grade.

        \item Carefully add the point deductions for each problem to avoid
              mistakes.

        \item Write the numbers (the score for each question next to the
              question number, and on the front page)
              \textbf{\underline{\underline{neatly}}}.

        \item If you find common mistakes students made, or that several
              students have given the same incorrect answer, let us know.  It
              may be a common conceptual mistake that we didn't anticipate
              when writing the answer key.  In that case we may want to adjust
              the grading key, or make up a special deduction to handle that
              case.

        \item Write down all decisions about grading that are made on the
              grading key (anything that's not already on the grading key).
              Write your name on the pages of the key on the problems you
              graded.

        \item If you're grading a question together with anyone else the
              most important things are correctness of grading and
              \underline{consistency} of grading.  If you see any situations
              not addressed by the grading key, and we make any decisions
              about how to handle them, everyone grading that question must
              commnunicate that between yourselves, and you need to ask when
              you come across different situations to make certain you're
              deducting the same credit as everyone is for the same mistake.
              Anytime you see something you aren't sure about, check with the
              other graders grading that question to ensure any special
              situations are handled uniformly.

        \item If you have questions about anything while grading, be sure to
              ask.  We want to avoid having to go back and correct mistakes
              after things have been graded!

        \item After the exams are graded and returned, don't tell the students
              what the grading criteria were (number of points off for
              different mistakes).

      \end{enumerate}

      \vspace{-2.5mm}

      \pagebreak

    \end{info}

  }

  \begin{enumerate}

    \addtolength{\itemsep}{15mm}

    %%%%%%%%%%%%%%%%%%%%%%%%%%%%%%%%%%%%%%%%%%%%%%%%%%%%%%%%%%%%%%%%%%%%%%%%%%

    \item {[20 pts.]}

          \vspace{-2.5mm}

          \begin{enumerate}

            \addtolength{\itemsep}{4mm}

            \item Polymorphism refers to code that can be applied to
                  different types (or data structures that can store
                  different types of objects); examples in OCaml would be
                  functions that return the length of a list or the first
                  element of a list, that can operate on lists of any type
                  (nothing in them requires the list to have any particular
                  element type).  (A type like \texttt{'a option} can be
                  used to store any type of data as well.)

            \item Both closures and objects encapsulate both code and data.

            \item Operational semantics is used to formally or precisely
                  describe the semantics of programming languages.

            \item Markup languages use annotations or tags that are
                  distinguishable from the text of a document to specify the
                  formatting, appearance or structure of documents.

            \item Two advantages of stop and copy garbage collection are
                  that it only touches live objects, and it eliminates
                  fragmentation.  (Only one advantage was asked for; we just
                  give two here.)

          \end{enumerate}

          \vspace{-2.5mm}

          \begin{info}{\textbf{\underline{Grading key:}}}

            \begin{itemize}

              \addtolength{\itemsep}{1mm}

              \item Each part is \pts[circle]{4}

              \item For the parts that are asking for two things ((a) and
                    (d)), each thing is \pts{2}

              \item Be sure to coordinate with anyone else also grading this
                    question for consistency of grading; ask me if you're
                    not sure how much credit to give.  \textbf{Write down on
                    this grading key} how many points you decide to give
                    for common situations not described here.

              \item Put your initials at the bottom of the page, write
                    negative deductions for mistakes or omissions in the
                    answers (clearly indicate what any deductions are for),
                    write the score for each subpart as a positive number
                    \textbf{next to its letter} (not just a checkmark, but a
                    number), and write the total score next to the question
                    number and on the front page.

            \end{itemize}

          \end{info}

          \pagebreak

    %%%%%%%%%%%%%%%%%%%%%%%%%%%%%%%%%%%%%%%%%%%%%%%%%%%%%%%%%%%%%%%%%%%%%%%%%%

    \item {[14 pts.]} Here's one solution:

          \vspace{-2.75mm}

          \begin{Verbatim}
        def select(list)
          result= Array.new
          list.each { |elt|
            if (yield(elt)) then
              result.push(elt)
            end
          }
          return result
        end
          \end{Verbatim}

          \vspace{-1.75mm}

          The method could also have been added to the \texttt{Array} class
          (although we didn't show this in class) by defining it as
          \texttt{def Array.select(list)}.

          \begin{info}{\textbf{\underline{Grading key:}}}

            \begin{itemize}

              \addtolength{\itemsep}{1mm}

              \item The five conceptual tasks that have to be performed, and
                    their point values, are:

                    \begin{itemize}

                      \addtolength{\itemsep}{1mm}

                      \item \pts[circle]{2} for declaring and initializing
                            a new array.

                      \item \pts[circle]{3} for iterating across the array
                            parameter.

                      \item \pts[circle]{4} for calling \texttt{yield} to test
                            each element of the array parameter.

                      \item \pts[circle]{3} for properly adding each element
                            for which the code block returned true to the
                            result array (in the correct order).

                      \item \pts[circle]{2} for returning the result array.

                    \end{itemize}

                    \smallskip

              \item Ignore minor mistakes in syntax.

              \item If the student didn't remember \texttt{yield}, but wrote
                    code that attempts to call and something, give half
                    credit (\pts{2}) for that task.

              \item Be sure to coordinate with anyone else also grading this
                    question for consistency of grading; ask me if you're
                    not sure how much credit to give.  \textbf{Write down on
                    this grading key} how many points you decide to give
                    for common situations not described here.

              \item Put your initials at the bottom of the page, write
                    negative deductions for mistakes or omissions in the
                    problem (clearly indicate what any deductions are for),
                    write the total score as a positive number next to the
                    question number, and on the front page.

            \end{itemize}

          \end{info}

          \pagebreak

    %%%%%%%%%%%%%%%%%%%%%%%%%%%%%%%%%%%%%%%%%%%%%%%%%%%%%%%%%%%%%%%%%%%%%%%%%%

    \item {[20 pts.]} The multiples of 3 are $\{ \ldots, \, -9, \, -6, \,
          -3, \, 0, \, 3, \, 6, \, 9, \, \ldots\}$ (all numbers \emph{x}
          such that $x \equiv 0\pmod 3$).  Therefore $\largeepsilon$ is not
          a valid string in this language.  Two correct regular expressions
          would be:

          \enlargethispage{3mm}

          \vspace{-4.5mm}

          \begin{centering}

            \addtolength{\baselineskip}{2mm}  % increased space between lines

            \(
              ( aaa )^{*}
              \,
              ( \, a \mid aa \mid b  \mid bb \mid ab \, )
              \,
              ( bbb )^{*}
            \)

            \(
              \lpar
                ( aaa )^{*}( bbb )^{*}b
                \midspc
                ( aaa )^{*}( bbb )^{*}bb
                \midspc
                ( aaa )^{*}a( bbb )^{*}
                \midspc
                ( aaa )^{*}a( bbb )^{*}b
                \midspc
                ( aaa )^{*}aa( bbb )^{*}
                \midspc
                ( aaa )^{*}aa( bbb )^{*}bb
              \rpar
            \)

          \end{centering}

          \medskip

          Any correct answer should have the following properties:

          \vspace{-1.5mm}

          \begin{itemize}

            \item Recognizes strings consisting of zero or more \emph{a}'s,
                  followed by zero or more \emph{b}'s (i.e., intermixed
                  \emph{a}'s and \emph{b}'s are not allowed).

            \item Recognizes only strings with length not a multiple of 3.

          \end{itemize}

          \vspace{-1.75mm}

          \begin{info}{\textbf{Grading key:}}

            \vspace{-2.25mm}

            \begin{itemize}

              \addtolength{\itemsep}{1mm}

              \item The first property is \pts[circle]{8}; the second
                    property is \pts[circle]{12}

              \item Deduct \pts{-2} for a small or minor error in either
                    property (\pts{-4} if there are small errors in both).
                    For any non--minor error in a property, \textbf{deduct
                    all credit} for that property.

              \item Deduct \pts{-3} if the empty string is recognized.

              \item Deduct up to \pts{-2} for omitting parentheses around
                    regular expressions that use alternation when the
                    effect could be unclear, or using incorrect parentheses
                    (such as square braces).

              \item Deduct up to \pts{-5} for using disallowed (Ruby)
                    regular expression operations, such as $+$ or ? or
                    numeric superscripts or character classes (all of which
                    should not be used) depending upon how often they were
                    used.

            \end{itemize}

          \end{info}

          \begin{info}{\textbf{\underline{Grading notes:}}}

            \begin{itemize}

              \addtolength{\itemsep}{2mm}

              \item Since different valid answers exist, check the
                    correctness of the answer by ensuring it describes valid
                    strings of various forms, and does not describe invalid
                    strings of various forms.  In particular, the answer
                    should allow valid strings that begin with an \emph{a}
                    and end with an \emph{a}, those that begin with an
                    \emph{a} and end with a \emph{b}, those that begin with
                    a \emph{b} and end with an \emph{a}, and those that
                    begin with a \emph{b} and end with a \emph{b}.

              \item If parentheses were omitted around regular expressions
                    where the meaning is clear and unambiguous, then don't
                    deduct any credit, but show where the parentheses should
                    have been so the student knows they should have been
                    used.

              \item No answer should receive so many deductions it has a
                    negative score (in other words, stop deducting points if
                    you ever get to zero).

              \item There is no grading penalty for using extra unnecessary
                    parentheses in regular expressions-- but please make a
                    note that they're unnecessary.

              \item There is no grading penalty for omitting $\largeepsilon$
                    in a regular expression and using an empty alternative
                    instead, such as in $(a \mid \: )$.

              \item Be sure to coordinate with anyone else also grading this
                    question for consistency of grading; ask me if you're
                    not sure how much credit to give.  \textbf{Write down on
                    this grading key} how many points you decide to give
                    for common situations not described here.

              \item Put your initials at the bottom of the page, write
                    negative deductions for mistakes or omissions in the
                    problem (clearly indicate what any deductions are for),
                    write the total score as a positive number next to the
                    question number, and on the front page.

            \end{itemize}

          \end{info}

          \pagebreak

    %%%%%%%%%%%%%%%%%%%%%%%%%%%%%%%%%%%%%%%%%%%%%%%%%%%%%%%%%%%%%%%%%%%%%%%%%%

    \item {[12 pts.]}

          \begin{enumerate}

            \vspace{-2.75mm}

            \addtolength{\itemsep}{4mm}

            \item y

            \item $\mathrm{\lambda z.z \: y}$ (where z can be any variable
                  other than y)

            \item v

          \end{enumerate}

          \begin{info}{\textbf{\underline{Grading key:}}}

            \begin{itemize}

              \addtolength{\itemsep}{1mm}

              \item Part (a) is \pts[circle]{3}, part (b) is
                    \pts[circle]{4}, and part (c) is \pts[circle]{5}

              \item If the answer is wrong but is partly right, in that the
                    student did some reductions correctly (e.g., in (ii) the
                    answer could be $\mathrm{\lambda y.y \: y}$, or in (iii)
                    some reductions might have been carried out correctly
                    but not all, or the result wasn't completely reduced),
                    then give half credit for that part (rounding upward
                    to a whole number).

              \item If you find other cases where partial credit seems
                    appropriate then give it, using uniform and consistent
                    criteria (and keeping track of the criteria).

              \item Be sure to coordinate with anyone else also grading this
                    question for consistency of grading; ask me if you're
                    not sure how much credit to give.  \textbf{Write down on
                    this grading key} how many points you decide to give
                    for common situations not described here.

              \item Put your initials at the bottom of the page, write
                    negative deductions for mistakes or omissions in the
                    problem (clearly indicate what any deductions are for),
                    write the score \textbf{for each subpart, as a positive
                    fraction, next to its letter} (not just a checkmark,
                    but a fraction), and the total score as a positive
                    number (not a fraction) next to the question number, and
                    on the front page.

            \end{itemize}

          \end{info}

          \pagebreak

    %%%%%%%%%%%%%%%%%%%%%%%%%%%%%%%%%%%%%%%%%%%%%%%%%%%%%%%%%%%%%%%%%%%%%%%%%%

    \item {[24 pts.]} One correct DFA is below; other correct variants exist
          as well.

          \medskip

          \begin{automaton}(0,0)(135,70)

            \state[start](10,35){s0}
            \state(40,55){s1}
            \state[final](70,55){s2}
            \state(100,55){s3}
            \state(130,55){s4}
            \state(40,15){s5}
            \state[final](70,15){s6}
            \state(100,15){s7}
            \state(130,15){s8}
            \state[label={\renewcommand{\arraystretch}{.75}%
                          \normalsize%
                          \begin{tabular}[t]{c}%
                            dead%
                              \\%
                            state%
                          \end{tabular}}%
                  ](40,35){s9}

            \transition(s0,\emph{a},s1)
            \transition[labellocation=below](s0,\emph{b},s5)

            \transition[labellocation=below,labelposition=.4]%
                        (s1,\emph{a},s9)
            \transition[offset=2](s1,\emph{b},s2)

            \transition[offset=2](s2,\emph{a},s3)
            \transition[curved](s2,\emph{b},s4)

            \transition[loopdirection=down](s3,\emph{a},s3)
            \transition[offset=2](s3,\emph{b},s2)

            \transition%[labelposition=.6,offsetfrom=3]
                       (s4,\emph{a},s3)
            \transition[loopdirection=down](s4,\emph{b},s4)

            \transition(s5,\emph{a},s6)
            \transition[labelposition=.4](s5,\emph{b},s9)

            \transition[curved,angle=-35,labellocation=below](s6,\emph{a},s8)
            \transition[offset=2](s6,\emph{b},s7)

            \transition[offset=2](s7,\emph{a},s6)
            \transition[loopdirection=up](s7,\emph{b},s7)

            \transition[loopdirection=up](s8,\emph{a},s8)
            \transition%[labelposition=.6,offsetfrom=-3,labellocation=below]%
                        (s8,\emph{b},s7)

            \transition[loopdirection=left,labelposition=.5](s9,\emph{a},s9)
            \transition[loopdirection=right,labelposition=.5](s9,\emph{b},s9)

          \end{automaton}

          \vspace{-2mm}

          \begin{info}{\textbf{\underline{Grading key:}}}

            \begin{itemize}

              \addtolength{\itemsep}{1mm}

              \item The grading procedure for this question involves finding
                    the closest isomorphism between the student's answer and
                    the correct DFA.  In doing this, ignore things that are
                    extra in the student's DFA that don't make it
                    incorrect, for example, an answer may have more states
                    than necessary, but there's no penalty as long as it
                    recognizes the correct language.  However, if the answer
                    is missing something required (states, transitions,
                    etc.), causing it not to correctly recognize the
                    language in some way, then deduct as described below.

              \item For each missing state that the DFA must have to
                    recognize the correct language but that is not present,
                    deduct \pts[circle]{-4} This deduction is for both the
                    missing state and for all of its outgoing (but not
                    incoming) transitions.

         %    \end{itemize}
         %
         %  \end{info}
         %
         %  \begin{info}{\textbf{\underline{Grading key, con't:}}}
         %
         %    \begin{itemize}
         %
         %      \addtolength{\itemsep}{.5mm}

              \item For each missing or incorrect transition deduct
                    \pts[circle]{-2}  Incorrect transitions are transitions
                    that aren't shown at all, or transitions that should
                    go to a state other than the state they actually do go
                    to (including transitions that go to some existing
                    state but that should really go to some state that
                    does not appear in the DFA).

                    (Note the transitions from the dead state are handled as
                    a special case below).

                    When a needed state is missing from the DFA, the answer
                    should lose credit for both the missing state, and for
                    all of the incorrect transitions that should have gone
                    to that state but that obviously don't go there
                    (they're missing or they go somewhere else).

              \item For each state in the DFA that is final when it should
                    be nonfinal, or that is nonfinal when it should be
                    final, deduct \pts[circle]{-2}

              \item Deduct \pts[circle]{-2} if the start state is unlabelled
                    or wrong.

              \item If notational shortcuts were used then deduct a flat
                    \pts[circle]{-3} for the whole problem (no matter how
                    many times they were used).  Notational shortcuts are:

                    \medskip

                    \begin{itemize}

                      \item Any transition labelled with more than one
                            symbol.

                      \item Omitting transition(s) from the dead state to
                            itself%
                            %(that shouldn't occur in this DFA anyway)%
                            .

                    \end{itemize}

                    \vspace{-.75mm}

              \item Strings in this language must have two basic properties
                    (the first two symbols aren't the same, and the last two
                    symbols the same and in the same order as the first
                    two).  If one of the two properties is basically
                    enforced by the DFA, but there are problems with the
                    other property, then the minimum score the answer should
                    receive should be \pts{14}  In other words, in this case
                    indicate all mistakes in the solution, but stop
                    deducting at \pts{-14}

              \item If the DFA is a poor solution that doesn't really
                    enforce either property but at least makes some partial
                    attempt at enforcing one property, then the minimum
                    score the answer should receive should be \pts{7}  In
                    other words, in this case indicate all mistakes in the
                    solution, but stop deducting points at \pts{-17} An
                    answer should receive less than \pts{7} only if it's
                    completely missing or does not show even a minimal
                    understanding or attempt.

              \item If the same mistakes were made in both halves of the DFA
                    then we may not want to deduct twice for of them.  If in
                    the process of grading it seems like we should deduct
                    only once for such errors, let me know and we can
                    discuss it.

            \end{itemize}

          \end{info}

          \enlargethispage{4mm}

          \begin{info}{\textbf{\underline{Grading notes:}}}

            \begin{itemize}

              \addtolength{\itemsep}{1mm}

              \item Make sure the answer can accept the two valid strings of
                    length 2 \emph{ab} and \emph{ba}, can't generate any
                    strings of length three, can generate the two valid
                    strings of length 4 \emph{abab} and \emph{baba}, and of
                    course valid strings longer than length 4 as well.

              \item Note as above that there is \textbf{no deduction} for
                    having extra states, as long as the DFA is correct (many
                    answers are correct but the DFA just isn't minimal).
                    But if the DFA has extra states and any extra states
                    have incorrect transitions, then deduct \pts{2} for each
                    incorrect transition as described above.

              \item Carefully check that each state has two outgoing
                    transitions (one on each symbol), and only two outgoing
                    transitions.

                    Carefully check that each transition goes to the correct
                    state in order to recognize or accept all and only valid
                    strings.

              \item Note that entirely omitting the dead state and all of
                    the transitions to it counts as a missing state
                    (\pts{-3}).  But including it and omitting only the
                    transitions from itself to itself is \pts{-2}

              \item Be sure to coordinate with anyone else also grading this
                    question for consistency of grading; ask me if you're
                    not sure how much credit to give.  \textbf{Write down on
                    this grading key} how many points you decide to give
                    for common situations not described here.

              \item Put your initials at the bottom of the page, write
                    negative deductions for mistakes or omissions in the
                    problem (clearly indicate what any deductions are for),
                    write the total score as a positive number next to the
                    question number, and on the front page.

            \end{itemize}

          \end{info}

          \pagebreak

    %%%%%%%%%%%%%%%%%%%%%%%%%%%%%%%%%%%%%%%%%%%%%%%%%%%%%%%%%%%%%%%%%%%%%%%%%%

    \item {[26 pts.]} Here is a solution using two of the higher--order
          functions from the \texttt{List} module:

          \vspace{-2.5mm}

          \begin{Verbatim}[xleftmargin=10mm]
        let rec last = function
          [x] -> x
        | h::t -> last t;;

        let product_of_lasts l = List.fold_left ( * ) 1 (List.map last l);;
          \end{Verbatim}

          \vspace{-1mm}

          (Note the spaces around the multiplication operator \texttt{*}, to
          avoid it looking like the beginning of a comment, as in
          \texttt{(*)}.)

          Here is a version that does not use any of the higher order
          functions from the \texttt{List} module:

          \vspace{-2.5mm}

          \begin{Verbatim}[xleftmargin=10mm]
        let rec product_of_lasts l =
          match l with
              [x] -> last x
            | h::t -> (last h) * (product_of_lasts t);;
          \end{Verbatim}

          \vspace{-1mm}

          Here are two versions that each use one higher--order function.
          The first uses just \texttt{List.map}:

          \vspace{-2.5mm}

          \begin{Verbatim}[xleftmargin=10mm]
        let rec multiply = function
            [] -> 1
          | h::t -> h * (multiply t);;

        let product_of_lasts l = multiply (List.map last l);;
          \end{Verbatim}

          \vspace{-1mm}

          This one uses just \texttt{List.fold\_left}:

          \vspace{-2.5mm}

          \begin{Verbatim}[xleftmargin=10mm]
        let rec get_last_elements = function
            [] -> []  (* note the base case can't be [x] -> (last x) *)
          | h::t -> (last h)::(get_last_elements t);;

        let product_of_lasts l = List.fold_left ( * ) 1 (get_last_elements l);;
          \end{Verbatim}

          \vspace{-4mm}

          \begin{info}{\textbf{\underline{Grading key:}}}

            \begin{itemize}

              \addtolength{\itemsep}{1mm}

              \item A correct solution that satisfies the criteria in the
                    problem should perform the following conceptual tasks:

                    \smallskip

                    \begin{itemize}

                      \addtolength{\itemsep}{0mm}

                      \renewcommand{\labelitemii}{$\ast$}

                      \item Having some code to get the last element in a
                            list (the \texttt{last} function above)
                            (\pts[circle]{6})

                      \item Extracting or isolating the last element in each
                            inner list (applying the \texttt{last} function
                            to each inner list, either using
                            \texttt{List.map} or recursively)
                            (\pts[circle]{6})

                      \item Multiplying all of the last elements in each
                            inner list produced by the process above (either
                            using \texttt{List.fold\_left} or recursively)
                            (\pts[circle]{6})

                      \item Using at least one higher--order function
                            (\pts[circle]{6})

                    \end{itemize}

                    \medskip

                    Partial credit should be given for tasks that are
                    attempted and partly right, but that have some
                    mistakes, using consistent and uniform criteria.

                    \smallskip

                    \enlargethispage{6mm}

              \item Ignore minor mistakes in syntax if the intention is
                    clear.  Deduct \pts{-2} for each more significant minor
                    error.

              \item No penalty for having the order of the arguments to
                    \texttt{List.fold\_left} wrong.

              \item Don't penalize for inefficiency.

              \item Be sure to coordinate with anyone else also grading this
                    question for consistency of grading; ask me if you're
                    not sure how much credit to give.  \textbf{Write down on
                    this grading key} how many points you decide to give
                    for common situations not described here.

              \item Put your initials at the bottom of the page, write
                    negative deductions for mistakes or omissions in the
                    problem (clearly indicate what any deductions are for),
                    write the total score as a positive number next to the
                    question number, and on the front page.

            \end{itemize}

            \vspace{-2.5mm}

          \end{info}

          \pagebreak

    %%%%%%%%%%%%%%%%%%%%%%%%%%%%%%%%%%%%%%%%%%%%%%%%%%%%%%%%%%%%%%%%%%%%%%%%%%

    \item {[28 pts.]}

          \vspace{-6mm}

          \begin{center}

            \begin{grammar}[1.5]

              \production{S}{S \ \terminal{;} \ T \midspc T}
                \\

              \production{T}{U\terminal{::}T \midspc U}
                \\

              \production{U}{U \ \: V \midspc V}
                \\

              \production{V}{\terminal{f} \midspc \terminal{g} \midspc
                             \terminal{h} \midspc \terminal{[]} \midspc
                             \terminal{(}S\terminal{)}}
                \\

            \end{grammar}

          \end{center}

          Any completely--correct answer should have the following
          properties:

          \vspace{-2mm}

          \begin{description}

            \addtolength{\itemsep}{2mm}

            \item[completeness:] It generates every valid
                  expression.  \pts[circle]{5}

            \item[correctness:] It generates only valid expressions (and
                  does not generate any invalid expressions).
                  \pts[circle]{6}

            \item[associativity:] It correctly enforces the described
                  associativity of the operators.  \pts[circle]{6}

            \item[precedence:] It correctly enforces the described
                  precedence of the operators.  \pts[circle]{6}

            \item[ambiguity:] It's unambiguous.  \pts[circle]{5}

          \end{description}

          \vspace{-3.75mm}

          \begin{info}{\textbf{\underline{Grading scale:}}}

            \begin{itemize}

              \addtolength{\itemsep}{2mm}

              \item Each property above that the grammar fully enforces is
                    full--credit for that property.  Deduct \pts{-3} if the
                    grammar contains one source of error in a property, and
                    deduct all points for a property if there's more than
                    one source of source of error in it.

                    Examples: reversing the operators' precedence would
                    count as one mistake in enforcing precedence.  Or
                    writing right--recursive productions for all of the
                    left--associative operators would be one mistake in
                    enforcing associativity.

              \item Deduct \pts{-1} for minor mistakes such as using
                    angle braces around terminals (for example,
                    \verb@<f>@, \verb@<;>@, etc.

              \item If regular expressions are used in productions
                    then deduct \pts{-6}

              \item If the grammar generates empty expressions it
                    would be considerd a mistake in correctness.

            \end{itemize}

          \end{info}

          \vspace{-1.5mm}

          \begin{info}{\textbf{\underline{Grading notes:}}}

            \begin{itemize}

              \addtolength{\itemsep}{2mm}

              \item If all the operators have the same precedence
                    then no attempt was made to enforce precedence
                    at all, so that should be counted as more than
                    one mistake in precedence (no credit for p4).
                    Or writing productions like \production{S}{S \
                    \terminal{;} \ S \midspc T} is not only
                    ambiguous but doesn't make any effort at all to
                    enforce associativity, so that should count as
                    more than one mistake in associativity (no
                    credit for associativity).

              \item For ambiguity look for whether some productions
                    incorrectly contain nonterminals of several
                    other productions as right side alternatives,
                    for example something like

                    \smallskip

                    \begin{centering}

                      \production{S}{S \ \terminal{;} \ T \midspc
                                     T \midspc U}

                    \end{centering}

                    \smallskip

                    where T also goes to U.  Also be careful
                    whenever any productions go to nonterminals of
                    earlier productions, such as in

                    \smallskip

                    \begin{centering}

                      \production{T}{U\terminal{::}T \midspc U
                                     \midspc S}

                    \end{centering}

                    \smallskip

                    Nothing prevents derivations containing repeated
                    cycles of

                    \smallskip

                    \begin{centering}

                      \(
                        \mathrm{S} \: \Longrightarrow \:
                        \mathrm{T} \: \Longrightarrow \:
                        \mathrm{S} \: \Longrightarrow \:
                        \mathrm{T} \: \Longrightarrow \:
                        \mathrm{S} \: \Longrightarrow \:
                        \mathrm{T} \: \Longrightarrow \ldots
                      \)

                    \end{centering}

                    \smallskip

                    Needless to say, any production in which the
                    left side nonterminal is also a right side
                    alternative, as in
                    \production{T}{U\terminal{::}T \midspc U \midspc
                    T} is about as ambiguous as you can get.

              \item There's no penalty for extra productions.  In
                    other words, the grammar need not be minimal and
                    there is no penalty for useless nonterminals.
                    For example, no penalty for a grammar with
                    productions like the following:

                    \smallskip

                    \begin{centering}

                      \begin{grammar}[1.25]
                        \production{V}{Lparen \ \: S \ \: Rparen}
                          \\
                        \production{Lparen}{\terminal{(}}
                          \\
                        \production{Rparen}{\terminal{)}}
                          \\
                      \end{grammar}

                    \end{centering}

                    \smallskip

              \item Consider each mistake in productions in
                    isolation, and pretend that mistake isn't
                    present when considering other mistakes.
                    Otherwise one mistake in one production could
                    cause the grammar not to generate any strings,
                    although other than that one problem the grammar
                    might be perfect.

                    Another way to look at this would be to consider
                    each mistake in a production (or each thing
                    omitted in a production) to be a single mistake,
                    rather than considering each form of string
                    generated or not generated to be a single
                    mistake.

              \end{itemize}

          \end{info}

          \pagebreak

    %%%%%%%%%%%%%%%%%%%%%%%%%%%%%%%%%%%%%%%%%%%%%%%%%%%%%%%%%%%%%%%%%%%%%%%%%%

    \item {[24 pts.]}

          \vspace{-2.5mm}

          \begin{enumerate}

            \addtolength{\itemsep}{4mm}

            \item Not possible-- \texttt{@empty} is false after p1 executes
                  so p2 cannot execute until one of the consumer threads
                  executes (two producers cannot execute in a row, or one
                  would try to store something into a full buffer).

            \item $\mathtt{j} = 2$, $\mathtt{k} = 3$

            \item $\mathtt{j} = 1$, $\mathtt{k} = 2$

            \item Not possible-- \texttt{@empty} is true after c2 finishes
                  executing so c1 cannot execute until one of the producer
                  threads executes (two consumers cannot execute in a row,
                  or one would consume from an empty buffer).

            \item Not possible-- p1 cannot get interrupted after line 3 (it.
                  has the lock on the monitor until line 4).

          \end{enumerate}

          \vspace{-2.5mm}

          \begin{info}{\textbf{\underline{Grading key:}}}

            \begin{itemize}

              \addtolength{\itemsep}{1mm}

              \item Each part is \pts[circle]{6}

              \item No credit if an answer for parts (a), (d), or (e) says
                    that the schedule is valid, or if an answer for parts
                    (b) or (c) says that the schedule is invalid.

              \item If the answer says that part (b) or (c) is valid, but
                    has the wrong values for \texttt{j} or \texttt{k}, give
                    half credit for that part.

              \item It doesn't matter what order the answer lists the values
                    of \texttt{j} and \texttt{k} in parts (b) and (c) (as
                    long as it's clear which one the answer is writing
                    first).

              \item Be sure to coordinate with anyone else also grading this
                    question for consistency of grading; ask me if you're
                    not sure how much credit to give.  \textbf{Write down on
                    this grading key} how many points you decide to give
                    for common situations not described here.

              \item Put your initials at the bottom of the page, write
                    negative deductions for mistakes or omissions in the
                    problem (clearly indicate what any deductions are for),
                    write the total score as a positive number next to the
                    question number, and on the front page.

            \end{itemize}

          \end{info}

          \pagebreak

    \begin{multicols}{2}

    %%%%%%%%%%%%%%%%%%%%%%%%%%%%%%%%%%%%%%%%%%%%%%%%%%%%%%%%%%%%%%%%%%%%%%%%%%

    \item {[16 pts.]} The output would be \texttt{1 31 21 -9}.

          \begin{info}{\textbf{\underline{Grading key:}}}

            \begin{itemize}

              \addtolength{\itemsep}{2mm}

              \item Each correct value is \pts[circle]{4}.

              \item If there are extra values besides the four values that
                    are supposed to be printed, deduct \pts{-2} each.

              \item No deduction if the answer printed the three array
                    elements first, then the value of x.

              \item If you find other cases where partial credit seems
                    appropriate then give it, using uniform and consistent
                    criteria (and keeping track of the criteria).  If you
                    see any common incorrect answers talk to me about
                    whether they should be graded specially.

              \item Be sure to coordinate with anyone else also grading this
                    question for consistency of grading; ask me if you're
                    not sure how much credit to give.  \textbf{Write down on
                    this grading key} how many points you decide to give
                    for common situations not described here.

              \item Put your initials at the bottom of the page, write
                    negative deductions for mistakes or omissions in the
                    problem (clearly indicate what any deductions are for),
                    write the total score as a positive number next to the
                    question number, and on the front page.

            \end{itemize}

          \end{info}

          \columnbreak

    %%%%%%%%%%%%%%%%%%%%%%%%%%%%%%%%%%%%%%%%%%%%%%%%%%%%%%%%%%%%%%%%%%%%%%%%%%

    \item {[16 pts.]} The output would be:

          \begin{Verbatim}
        f: 3 8
        g: 3 8
        h: 2 8
        main: 2 1
          \end{Verbatim}

          \begin{info}{\textbf{\underline{Grading key:}}}

            \begin{itemize}

              \addtolength{\itemsep}{2mm}

              \item Each correct value is \pts[circle]{2}.

              \item If there are extra values besides the four values that
                    are supposed to be printed, deduct \pts{-1} each.

              \item It doesn't matter if the labels (\texttt{f:},
                     \texttt{g:}, etc.) are missing.

              \item If the lines are in the wrong order, talk to me about
                    how much to deduct.

              \item If you find other cases where partial credit seems
                    appropriate then give it, using uniform and consistent
                    criteria (and keeping track of the criteria).  If you
                    see any common incorrect answers talk to me about
                    whether they should be graded specially.

              \item Be sure to coordinate with anyone else also grading this
                    question for consistency of grading; ask me if you're
                    not sure how much credit to give.  \textbf{Write down on
                    this grading key} how many points you decide to give
                    for common situations not described here.

              \item Put your initials at the bottom of the page, write
                    negative deductions for mistakes or omissions in the
                    problem (clearly indicate what any deductions are for),
                    write the total score as a positive number next to the
                    question number, and on the front page.

            \end{itemize}

          \end{info}

    \end{multicols}

  \end{enumerate}

\end{document}
