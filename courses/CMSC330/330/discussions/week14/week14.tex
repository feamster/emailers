\documentclass[12pt]{article}

  %
  % Scanning the exams and returning them are after the \end{document},
  % because they were returned in lecture this time, and they were already
  % scanned before these topics were given to the TAs.
  %

  \usepackage{330-f12}

  \pagestyle{empty}

\begin{document}

  \header{\course}{Discussion section topics -- Week \#12}{\Term}

  \section{Discussion \#14, Friday, November 30}

    \subsection{Ruby concurrency examples}

      There are several Ruby concurrency examples in my directory to show and
    explain:

      \vspace{-.5mm}

      \begin{description}

        \addtolength{\itemsep}{0mm}

        \item[produce--consume.rb:] The producer/consumer problem (with a
              single--element buffer being used to store elements that are
              produced) was explained in lecture, and a solution outline was
              shown in Java (see the posted lecture slides).  This is a
              solution in Java, with a few threads created to illustrate
              things happening.  Ideally you can review what the problem is
              and show a version of the program with the synchronization
              constructs removed, and ask the students what Ruby constructs
              should be added and where they should be added to make things
              work right.

              % final-spr10.pdf:

        \item[multithreading1.rb:] Now consider a producer/consumer pair
              where two threads are using a \texttt{Container} object to
              store produced values before they are consumed, where the
              container can again store one object.  Now we want to be able
              to create multiple \texttt{Container}s, so instead of the
              monitor and lock being global variables we want them to be
              fields of a \texttt{Container}, and we want the
              \texttt{produce()} and \texttt{consume()} methods to just
              throw an exception if the buffer is full or empty
              respectively.  Show and explain the \texttt{produce()} and
              \texttt{consume()} methods that do this.

              Then suppose we want to add two methods to the
              \texttt{Container} class that implement variants of
              \texttt{produce()} and \texttt{consume()} that either produce
              or consume if possible, or wait until the shared buffer is
              empty or full if not, without throwing exceptions.  This is
              similar to the first example except in this case we want on
              the value of the buffer, not a boolean variable.  Show these
              methods and illustrate the difference.

              % prac7-fall09.pdf:

        \item[multithreading2.rb:] Using Ruby monitors we want to implement
              a synchronization construct called \texttt{MyBarrier}.  A
              \texttt{MyBarrier} object is created with a certain value
              \texttt{n}.  When a thread calls the method \texttt{enter()} ,
              it enters the barrier and blocks until a total of \texttt{n}
              threads have entered the barrier.  When the \texttt{n}'th
              threads enters the barrier, all the threads waiting at the
              barrier wake up and unblock, and the \texttt{n}th thread
              continues without blocking.  When a thread calls the method
              \texttt{reset()}, the barrier is reset so that it starts fresh
              in counting up to \texttt{n} (i.e., \texttt{n} \texttt{more}
              threads must enter the \texttt{MyBarrier}).

              Show and explain the code.

        \item[multithreading3.rb:] This is a version of a bounded buffer, for
              the producer/consumer problem, that can store multiple items.
              Show and explain the code.

      \end{description}

      \vspace{-1.5mm}

    \subsection{Quiz \#5}

      Please review the quiz information from the earlier discussion topics
    so everything is clear.

      Stop the above so the students have 15 minutes at the end of the section
    to do the quiz (or earlier or later if we decide that any particular
    quiz should have more time).

      When handing out the quiz, emphasize that students \textbf{must} put
    their complete identifying information on the front (name, University ID
    number, TA's name, and discussion section time).

      Give the quiz at the end of the discussion.  Count the number of
    students taking the quiz during each section, and write it down and save
    it.  Count the number of quizzes turned in and make sure it agrees.

\end{document}

    \subsection{Scanning the exams}

      As mentioned earlier, for reasons given then, after the exams are graded
    (before they're returned to students) you need to scan the exams for your
    sections and give the PDFs of the scans to me.

      \textbf{Do not wait} to scan the exams.  Sometimes it gets done wrong and
    you would ave to rescan them, or you may have questions about how to do
    it.

      The copier/scanner will email the scans to you.  \textbf{\underline{Before
    scanning the entire stack of exams}}, take a small stack of just two
    or three exams to the copier, scan them, go back to your office, and
    \textbf{make sure the emailed PDF is correct}.  The PDF pages should be
    \textbf{face-up,} all the pages of every student's exam should be included
    (make sure that \textbf{both sides of each exam were scanned}, not just
    the odd-numbered pages), and the exams in the PDFs should be in the
    \textbf{same alphabetic order} as the stack of exams.  The PDF should be
    exactly the same as if you were looking at the stack of exams from the top
    to the bottom.

      If everything is OK, bring the rest of the exams and scan them.
    \textbf{Then doublecheck the full PDFs and make sure they are also
    correct-- check the same things as above-- before giving them to me.}

      Scanning instructions are:

      \vspace{-2.5mm}

      \begin{enumerate}

        \addtolength{\itemsep}{-1mm}

        \item Unstaple the exams, make sure none of the pages are
              stuck together, squish the corner where the exams were
              stapled as flat as possible, and put the entire stack of
              exams, \textbf{face-up}, in the photocopier feeder.

        \item Press ``Scan and Send''.

        \item Press ``New Destination''.

        \item Press ``E-Mail''.

        \item Type in your CS login ID (only the ID is necessary, not
              \texttt{@cs.umd.edu}), and press ``OK'' twice.

        \item Press ``PDF (Compact)'' and make sure PDF and Compact
              are both selected (highlighted in yellow), and press
              ``OK''.

        \item Press ``2-Sided Original'' and then ``Book Type'', and
              press ``OK'' (you may have to press ``Book Type'' even
              if it's highlighted).

        \item Just press ``Start'' and the scanning will be done and
              the resulting PDFs will be emailed to you.

      \end{enumerate}

      \vspace{-2.5mm}

      Then just restaple the exams exactly like they originally were.

      \textbf{Be sure-- as described above-- to check a small stack first,
    and to check the full PDFs when they're done-- to ensure that
    everything is correct.}  Once or twice I have seen a corrupt PDF get
    sent, and that part of the stack of exams had to be rescanned.

      The scanner may split the full PDFs up into two or three parts, when
    you scan the whole section, since it may be too large for it to create
    a single PDF; this is fine.

      Emailing the exam scans may not work due to their size-- some email
    programs choke on really large files so \textbf{do not email them to
    me}.  Instead, just copy them to my directory instead.  I will
    create a directory there where you can copy them.  Be sure to set the
    file permissions (\texttt{g+rw}) and group correctly if needed.  Name
    them with the number of your section, and a suffix if there are
    multiple parts (e.g., 0101a.pdf, 0101b.pdf, 0101c.pdf, etc.).  Let me
    know when you've put them there so I can grab them.

      Two more notes: if you may want to use the scanner at other times in
    the future can add your email address to the copier's addressbook, so
    you don't have to type it in every time.  And when you enter the
    settings for one scan, you don't have to re-enter them all to do a
    second scan right afterwards-- you can just use the history option to
    reuse the settings that were used for the previous scan.

      It is not necessary to say anything to students about the exams being
    scanned, meaning please don't mention to the students that we have
    copies of the exams (I will mention why at the meeting).  (If anyone
    happens to ask you if we have copies of their exams I would suggest
    just smiling and not replying.)

    \subsection{Return Exam \#2}

      Return the exams near the end of the discussion section, otherwise, if
    you were to return them at the beginning, some students would just be
    reading their exam and wouldn't pay attention to anything else after
    that.  I would suggest returning them after the quiz.

      Grades are considered very confidential-- hand the exams out
    individually, folded over or upside--down, so no one sees anyone else's
    grade.

      Don't give the students any statistics (e.g., average) about the exams;
    I'll give them this information (average, standard deviation, probably
    some type of distribution) in lecture.

      I'll discuss some of the points below in lecture as well, but please
    emphasize them also.

      \vspace{-2.5mm}

      \begin{itemize}

        \addtolength{\itemsep}{-1mm}

        \item The exam statistics will be given in lecture, so students
              should be sure to attend.

        \item \textbf{Emphasize that students should \underline{not} write
              anything anywhere on their exam when it is returned}, in the
              (unlikely) occurrence that there is a mistake in grading and
              they have to get something regraded; obviously changing the
              exam in any way beforehand is not permitted.

        \item The solutions will be posted shortly.  However, please
              emphasize that students will learn more about the material by
              fixing any problems in their answers themselves, possibly with
              assistance in office hours.  In other words, if a student had
              trouble with some questions they will improve much more by
              going back and fixing mistakes indicated in their answers
              (possibly with assistance) instead of just looking at the
              correct answers in the posted solutions.

              For this reason, please don't go over the solutions to any of
              the exam problems in class today.  Even if students ask you
              after class, I would encourage them to try to fix their
              answers themselves before asking what the right answers are.

        \item We'll discuss in class (and I'll let you know at the meeting)
              what students should do if they feel that a mistake was made
              in the grading of their exam.  (The summary is: first they
              should compare their answer with the posted solutions, and if
              they still feel their answer is correct and the grading is
              wrong, they should attach a sheet of paper and write two or
              three sentences explaining the mistake.  They can give it to
              me or to you.  If they give it to you, just bring it to me.)

              \textbf{Do not accept any exam regrades until the exam
              solutions are posted and the student has compared their
              answer with them.}

        \item After looking over their exam, anyone who's concerned about
              how they did is encouraged to meet with their instructor in
              person during office hours.

      \end{itemize}

      \vspace{-2.5mm}

      \textbf{Note}: if you have any quizzes that haven't been picked up yet,
    \textbf{staple them to the students' exams before returning the exams.}
    Since most students usually end up picking up their exams, this will be
    a good opportunity to get these quizzes picked up at the same time.

      \textbf{Immediately on your way back after your discussions}, bring any
    any exams that are not picked up back to me, so I can return them if
    students ask.  I want to have these in time for my next office hours in
    case any students come and ask for their exams, so please don't forget.
    If I'm not in, just shove the envelope with the exams \textbf{way under
    the door}.

