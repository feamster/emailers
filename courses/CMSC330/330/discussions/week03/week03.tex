\documentclass[12pt]{article}

  %
  % Add that the TAs should count the students during the quiz, and write
  % the count down somewhere:
  %
  %      \item \textbf{Important}: count the number of students taking the
  %            quiz during each section, and write it down and save it.
  %            Count the number of quizzes turned in and make sure it agrees.
  %

  \usepackage{330-f12}

  \usepackage{wasysym}

\begin{document}

  \header{\course}{Discussion section topics -- Week~\#3}{\Term}

  \section{Discussion \#2, Friday, September 14}

    \subsection{More Ruby}
  
      \subsubsection{Small topics}
  
        Here are a few small topics to either mention (that were not explained
      in class), or to reinforce.  Illustrating them in irb would be
      helpful.
  
        \vspace{-1.5mm}
  
        \begin{enumerate}
  
          \addtolength{\itemsep}{.5mm}
  
          \item The subscript $-1$ always refers to the last element of an
                array.
  
          \item If you refer to an element of a hash whose key has never
                been assigned to (i.e., a nonexistent element), you just get
                \texttt{nil}.  That is the default behavior.  However, you
                can use an argument in the \texttt{Hash} constructor (e.g.,
                \texttt{Hash.new(10)}, and all elements of that hash that
                are referred to without first being assigned to will start
                off with that value.
  
                (There's a third form of the constructor that takes a code
                block as a parameter, which is called any time elements of
                that hash that are referred to without first being assigned
                to, and its value becomes the starting value-- note that it
                can produce different results at different calls, use global
                varibles, etc.  If you have extra time you could mention
                this, but I think it's not as important.)
  
          \item Take a few moments to illustrate the difference between
                \texttt{Array.delete()} and \texttt{Array.delete\_at()}, by
                giving some example calls on an array.
  
                (Yes, the quiz uses \texttt{Array.delete\_at()}-- I wonder
                  if we may have covered it too quickly in lecture, and
                 students won't remember it for the quiz.  So taking a few
                 moments to mention it again today will ensure that students
                 remember it.  Of course, don't tell them that it's asked
                 about on the quiz. \smiley)
                
        \end{enumerate}
  
        \vspace{-2.5mm}
  
      \subsubsection{Larger example (two versions)}
  
        In my directory is a larger example of a Ruby program to solve the
      problem below As before, the idea is not to just show the code, but
      ideally to explain what we want to do, and where possible to develop
      the programs one part at a time.  Ask the students how to do the first
      step, and show it, or type it into a new file.  Then ask them what the
      next step should be and type it in, etc.  If they tell you the wrong
      thing you can ask them if it'll work, or why not.  If there are
      different ways to do something you can discuss them.
  
        Here's the problem:
  
        \begin{quote}
  
          Write a method that will take a filename as an argument and read
          that file, counting the number of times that each group of three
          letters appears, so these numbers can be accessed from a hash.
          Assume that number of characters per line is a multiple of 3.  It
          should print the results in alphabetical order according to the
          amino acid names.
  
        \end{quote}
  
        You might first ask the students to think about how they would code
      this up in Java.
  
        The solution uses I/O, hashes, and regular expressions with
      \texttt{scan()} (which will be covered in lecture on Thursday).
      Explain what it's doing as needed.  Although this is a matter of
      opinion of course, many people might say that the Ruby solution is
      shorter, easier to understand (for someone who knows Ruby as well as
      they know Java), and could be written more quickly.
  
        In the second version, suppose we want to print the output in a more
      complex format, namely we want to list the amino acids in decreasing
      order by occurrence, with the ones that have the same number of
      occurrences printed in alphabetical order according to the amino acid
      names.
  
        This could be done in different ways, but my solution creates an
      auxililary array in which each element is a two--element array (a
      pair) with a count and the corresponding amino acid name, for example
  
        \begin{centering}
  
          \begin{BVerbatim}[commandchars=\\\<\>]
          [[3, gct], [2, gca], [5, gcg], [2, tat], \ldots]
          \end{BVerbatim}
  
        \end{centering}
  
        \medskip\smallskip
  
        Then it calls \texttt{sort()} on this array, using the form of
      \texttt{sort()} that takes a code block as a parameter.  The code
      block has two parameters, and will be called every time the sorting
      algorithm compares two elements of the array in the proces of sorting.
      In this case the code block first compares the amino acid occurrence
      counts (the first elements of the two two--element arrays), and if
      they're not identical it returns 1 if the comparison result was $-1$,
      and $-1$ if the comparison result was 1 (this is to sort in decreasing
      order).  If the comparison result was 0, the result of comparing the
      second elements of both pairs (the amino acid names) are used to break
      the tie.  Note that these comparisons use Ruby's \texttt{<=>}
      operator, which is also new in these examples.  Then another code
      block prints the results.
  
    \subsection{If you have extra time}
  
      If you have any extra time, instead of letting the students go early,
    see if they have any questions about Ruby, and try to answer them if so.
    Or make up some regular expression problems if needed.  You could give
    some regular expressions (using only the features covered so far, of
    comparable difficulty to the ones on the quiz) and ask the students to
    describe what strings would be matched by them.  Or describe a set of
    strings with some property, and ask them to tell you a regular
    expression that would match them.
  
    \subsection{Quiz \#1}
  
      Stop the above so the students have 15 minutes at the end of the section
    to do the quiz (or earlier or later if we decide that any particular
    quiz should have more time).
  
      Regarding quizzes:
  
      \vspace{-1.5mm}
  
      \begin{itemize}
  
        \addtolength{\itemsep}{-1mm}
  
        \item Make the quiz photocopies double--sided if the quiz is more
              than one page.
  
        \item Do not give the quiz early.  If the material before the quiz
              doesn't take all the time, prepare enough extra material (see
              the prior subsection) to use up the time until the quiz should
              begin.  (If the students are able to leave early it increases
              the chance they'll tell others out in the hallway who are in
              the next section what the quiz questions are.)
  
        \item If possible have students spread out before the quiz as much
              as possible, although if the room isn't that large, it might
              not be possible to spread out very much.
              % (It's easier for you if we give quizzes that are quick to
              % grade, meaning all the answers are just a few numbers, but
              % these types of questions make it much easier for one student
              % to glance at their neighbor's paper and copy their answer.)
  
        \item Quizzes are closed--book, closed notes, so have students put
              away all their books and materials (including all electronic
              devices) before passing out the quizzes.
  
        \item When you pass out the quizzes, tell students not to start
              until you say so, so be sure everyone has the quiz before saying
              they can start.
  
              Also as you hand out the quizzes, mention that the quiz is
              two--sided, so students should not miss the second page.
  
        \item When you pass out the quizzes, make sure to hand out exactly
              the same number as students in the room.  (Since students in
              later sections will be taking the same quiz we don't want
              anyone to be able to walk out with an extra copy of the quiz
              and give it to a friend in a later section.)
  
        \item When handing out the quiz, announce that students
              \textbf{must} put their complete identifying information on
              the front (name, University ID number, TA's name, and
              discussion section time).  You may want to remind them of your
              name ahead of time \smiley.
  
        \item During the quiz walk around and watch the class to ensure that
              no one is looking at anyone else's paper.
  
              If anything suspicious ever occurs, handle it however you
              think best during the quiz, but let me know as soon as the
              discussion section is over (get the names of the student or
              students involved).
  
        \item Count the number of students taking the quiz during each
              section, and \textbf{write it down and save it}.  Count the
              number of quizzes turned in and make sure it agrees.
  
              Every semester there's at least one case where a student who
              has a zero for a quiz claims they took the quiz and their TA
              must have lost it.  Counting the students and quizzes will
              ensure that we know what really happened.
  
        \item If anyone has a question during the quiz try to answer it
              (quietly), but of course don't answer anything in a way which
              gives away the answer.
  
        \item If it ever comes up that anyone asks or thinks there are any
              errors on a quiz, you can tell them that there are no mistakes
              in the code on the quiz (at least there are no intentional
              errors).  \textbf{Unless} a problem says that there may be an
              error-- which problems sometimes do-- students may assume that
              all the code shown on a quiz is correct and valid.
  
        \item Tell everyone they need to remain, in their seats, until you
              collect the quizzes (i.e., don't let anyone leave before the
              end even if they finish early).
  
              It's better to get everyone to stay until the end and then
              just collect all the quizzes then, because if some students
              are getting up and turning in their quiz and packing up their
              things during the quiz, it interferes with your ability to pay
              attention to everyone else, and also is distracting for
              students who are still working.  (It also reduces the time that
              students can tell the quiz questions to their friends in the
              next section in the hallway outside.)  If anyone talks to you
              and says they have an exam across campus right afterwards, you
              can let them leave early on a case by case basis, but unless
              they have a reason like this everyone should stay until the
              end, even if they finish early.
  
              % (\textbf{Note}: this does not apply in the last section of the
              % day.  Those students can leave as they finish,
              % \textbf{provided} that it does not interfere with your ability
              % to pay careful attention to what's going on, and answer
              % questions as needed.)
  
        \item Make sure everyone passes their quiz in immediately when
              you say so (so no one has more time than anyone else).
  
        \item At the meeting we'll talk about how to grade quizzes and
              enter quiz grades.
  
      \end{itemize}
  
      \vspace{-2.5mm}

\end{document}
