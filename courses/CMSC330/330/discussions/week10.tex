\documentclass[12pt]{article}

  \usepackage{330-f12}

  \pagestyle{empty}

\begin{document}

  \header{\course}{Discussion section topics -- Week \#10}{\Term}

  \section{Discussion \#9, Friday, November 2}

    \subsection{More OCaml}

      \subsubsection{Finish material from last time}

        I think that I recall that you all didn't finish all the material from
      last time-- you can go back and finish it now.

      \subsubsection{Mutual recursion}

        Show the students how to write mututally recursive functions, using
      \texttt{and}.  This example is from a text by Ullman; feel free to
      create your own:

        \begin{Verbatim}
        let rec
           take (ls) =
              if ls = [] then []
              else List.hd(ls) :: skip(List.tl(ls))
        and
           skip (ls) =
              if ls = [] then []
              else take(List.tl(ls))
        \end{Verbatim}

        Also show them how to create mutually recursive data types, also
      using \texttt{and}; here is one simple example (feel free to
      create your own):

        \begin{Verbatim}
        type even =
            Zero
          | Even_succ of odd
        and
             odd = Odd_succ of even
        \end{Verbatim}

      \subsubsection{Reading lines until the end of the input}

        Show the students an example of how to read lines from standard input,
      using \texttt{input\_line}, until the end of the input.  This uses
      exceptions (he library exception \texttt{End\_of\_file}).

      \subsubsection{Reading lines from an external file}

        Show the students an example of how to open an external file and read
      lines until the end of the input, a slight extension of what was shown
      before.  One way to do it would look like this.

        Note the use of \texttt{Printf.printf}, which was showed briefly in
      one example in lecture; it could be useful to use it again.

        You might point out that if we wanted to do something with the lines
      read in from the file besides just printing them out, we might want to
      build a list of them in the function and then return it; students
      could think about how to do this.  Or if you have time you could add
      this.

        \begin{Verbatim}[xleftmargin=-5mm]
        let rec read_all_lines filechannel=
          let line= input_line filechannel in
            try
              Printf.printf "Line read: %s\n" line;
              read_all_lines filechannel
            with End_of_file ->
              print_string "Now at end of file.\n"

        let read_file_lines filename=
          let filechannel= open_in filename in
            read_all_lines filechannel
        \end{Verbatim}

      \subsubsection{Command--line arguments}

        It should just take a moment to let the students know how to access
      command--line arguments in OCaml, which is actually an array in OCaml
      (yes, OCaml has arrays); something like:

        \begin{Verbatim}
        let _ =
          for i = 0 to Array.length Sys.argv - 1 do
            Printf.printf "[%i] %s\n" i Sys.argv.(i)
          done
        \end{Verbatim}

    \subsection{If you have extra time}

      If you have any extra time, instead of letting the students go early,
    see if they have any questions and try to answer them if so.  There may
    be questions about the project.

\end{document}
