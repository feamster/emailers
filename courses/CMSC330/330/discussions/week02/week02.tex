\documentclass[12pt]{article}

  %
  % NEXT TIME ADD TO ALL DISCUSSION TOPICS:
  %
  % \section{Notes for teaching TAs}
  % 
  %   \begin{itemize}
  % 
  %     \addtolength{\itemsep}{-1mm}
  % 
  %     \item Don't display these discussion topics themselves using the
  %           projection equipment during class, and don't give students a
  %           copy of them, whether on paper or electronically, even if they
  %           missed discussion for a valid reason.  Students who missed
  %           discussion should be encouraged (are expected) to get notes from
  %           a classmate who was present.
  % 
  %           You can give out \textbf{printed} copies later of any handouts
  %           or quizzes that were originally given out on paper to any
  %           students who need one, but do not give out electronic copies of
  %           anything that was not originally provided electronically.
  % 
  %     \item Please read the topics carefully every week.  Sometimes topics
  %           may be mostly similar from earlier weeks yet have small but
  %           important differences.
  %           % (They also have changes from earlier semesters' versions.)
  % 
  %   \end{itemize}
  % 
  %   \vspace{-2.5mm}
  %

  \usepackage{330-f12}

\begin{document}

  \header{\course}{Discussion section topics -- Week~\#2}{\Term}

  \section{Discussion \#1, Friday, September 7}

    \subsection{Introduction}

      You can introduce yourself, give the course and section number, etc.,
    to make sure students are in the right place at the right time.

    \subsection{Attendance}

      Let the students know that on a regular must attend the discussion
    section that they're registered for.  This is to avoid overcrowding in
    the discussion rooms.  If anyone has a special reason for attending
    another section on a particular day (e.g., an exam in another course,
    car trouble, woke up late, etc.) it's OK if they attend another section
    on that day, if they ask the TA before the section.  But on a routine
    basis they are expected to attend their own section.  Quizzes will be
    given in discussion section and students will only get credit for
    quizzes taken in their own section (unless they have a reason to attend
    another one on the day of the quiz, as mentioned above, and ask the TA
    at the beginning).

    \subsection{Ruby examples}

      In my directory are several Ruby examples that you can show the students
    and go over.  Feel free to add additional examples of your own, that use
    similar features (pretty much sticking to what we've covered so far in
    class), or to modify these if you like.  Copy the programs to the Grace
    machines and show them using the room projection equipment.

      The idea is not to just show the code, but ideally to explain what we
    want to do, and where possible to develop the programs one part at a
    time.  Ask the students how to do the first step, and show it, or type
    it into a new file.  Then ask them what the next step should be and type
    it in, etc.  If they tell you the wrong thing you can ask them if it'll
    work, or why not.  If there are different ways to do something you can
    discuss them.  Or just show one part at a time of the provided examples.

      There is at least one new feature in one of the examples,
    \texttt{eval()}, that you will have to explain.  If you find any other
    things that we didn't explain in class so far, just explain them also.

      Let the students know that they don't need to write the examples down,
    because we will be providing the discussion example code to them.  They
    can just pay attention in class, and can copy the examples to their own
    disk space when they are provided, run them, play around with them, etc.

      You can cover them in any order, but here are some comments about them
    (just in the order that I would cover them).  I think that the hash
    examples are the most important.  Note that in the earlier examples I
    intentionally avoided some things that you might want to use, like
    iterpolation into double--quoted strings.  You can show the examples and
    ask students what could be changed.  For instance, instead of something
    like:

      \begin{Verbatim}[gobble=6]
      print("You entered: ", str, ".\n")
      \end{Verbatim}

      \noindent
      we might more easily do:

      \begin{Verbatim}[gobble=6]
      print("You entered: #{str}.\n")
      \end{Verbatim}

      \noindent
      or

      \begin{Verbatim}[gobble=6]
      puts("You entered: #{str}.")
      \end{Verbatim}

      \vspace{-2.5mm}

      \begin{description}

        \addtolength{\itemsep}{1mm}

        \item[\texttt{simple-input.rb}:] This just illustrates simple input,
              and output, but you can also make sure students understand the
              mechanism of running the script as an executable (the first
              line, and having to use \texttt{chmod u+x}).

              By the way, the way the comment is asking about, to write
              \texttt{str= str.chomp()} without an assignment, is
              \texttt{str.chomp!()} (the mutator methods end in \texttt{!}).

        \item[\texttt{fib.rb}:] Illustrates use of a command--line argument,
              use of an observer method (ending in \texttt{?}),
              \texttt{to\_i()} and \texttt{to\_s()}.  (Why do we need
              \texttt{to\_i()} and \texttt{to\_s()} where they are used?)

        \item[\texttt{hash-example1.rb}:] This example illustrates a very
              common paradigm in Ruby, and scripting languages in general.

              Suppose we have a list of students, along with their
              University ID numbers (a sample input is in the file
              \texttt{hash-example1+2-input}), or you can change it to be
              names and scores on an exam, or anything where names are
              associated with some numbers.  Assume each student has a first
              name and a last name, and a single space separates the three
              fields on each line (the names and the ID).  (After we cover
              regular expressions, it would be very simple to allow
              arbitrary whitespace.)  We want to print the names in order of
              increasing Uninversity ID number (ID, last name , comma, first
              name).

              You might first explain the problem, show the input, and ask
              students to think for a minute about how they would code this
              in Java.  They might come up with various ideas, some of which
              would work and some of which would not, or ideas that would
              work but be a little long or awkward to write.  It's very
              simple and short in Ruby though.

              The idea is to use a hash, where the hash key is each
              student's ID number, and the value is their name (just as a
              string, last name followed by a comma followed by first name).
              Then (this is the very common idiom in scripting languages) we
              get the keys of the hash-- which in Ruby returns an array--
              iterate over that, and access the value corresponding to each
              key.

              You could show it first without the \texttt{sort()}, to see
              that it works, but doesn't print the results in the order we
              want.  The \texttt{Array} \texttt{sort()} method easily fixes
              that though.

        \item[\texttt{hash-example2.rb}:] What if we want to do almost the
              same thing, with the same input except print the output in
              alphabetical order by student name rather than ID?  Just use
              the names as hash keys instead, and the IDs as the hash
              values.

        \item[\texttt{hash-example3.rb}:] Now what if we want to do the same
              thing as the first example, but for multiple sections?  Now
              each line (of the input file \texttt{hash-example3-input}) has
              a section number as well.

              The idea is to create a hash in which the keys are section
              numbers, and the value associated with each section number key
              is another hash just like the hash in the first hash example
              (that associates names with keys of IDs).  Now we need a
              outer code block to iterate over the section numbers, and an
              inner one to iterate over the students in each section.

        \item[\texttt{loops.rb}:] This illustrates different ways of writing
              loops.  The interesting thing is that the loops are each passed
              as input (a string) into the \texttt{show\_loop()} method,
              which uses \texttt{eval()} to treat the string as code and
              execute it.

              Do C or Java have an analogue to \texttt{eval()}?  What is it
              about Ruby that allows it (and similar languages) to have an
              \texttt{eval()} mechanism?

      \end{description}

      \vspace{-2.5mm}

    \subsection{If you have extra time}

      If you have any extra time, instead of letting the students go early,
    see if they have any questions about Ruby, and try to answer them if so.
    Or you can think of any other similar examples ahead of time, or any
    modifications to the above ones, and ask the students to try to write
    the program, or make the modifications, for a few minutes, just using
    pencil and paper.  Then you can ask them to tell you how to write it and
    you can type it in and run it

\end{document}
