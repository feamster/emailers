\documentclass[12pt]{article}

  \usepackage{330-f12}

  \usepackage{wasysym}

\begin{document}

  \header{\course}{Discussion section topics -- Week \#4}{\Term}

  \section{Discussion \#3, Friday, September 21}

    \subsection{More Ruby}
  
      \subsubsection{Backquote substitution}
  
        Another very commonly used feature in scripting languages is backquote
      substitution, in which backquotes are replaced by the output of an
      operating system command, which can therefore be captured and used.
      For example:
  
        \vspace{-2.5mm}
  
        \begin{quote}
  
          \begin{Verbatim}
        str= `ls -l`
        # process str in some way here
          \end{Verbatim}
  
        \end{quote}
  
        \vspace{-1.5mm}
  
        Explain this and come up with an example to illustrate.  It doesn't
      have to use \texttt{ls -l}; any command that produces output that we
      can do something with would be good.
  
      \subsubsection{A few more Ruby regular expression features}
  
        Here's something else useful about extracting parts of strings with
      regular expressions.  Explain and illustrate the \texttt{\$\&},
      \texttt{\$`}, and \texttt{\$'} constructs, that can extract the part
      of a string that matched a regular expression, the part of the string
      before it, and the part of the string after it.
  
        The \texttt{i} regular expression modifier (\texttt{/\ldots/i}), which
      causes case to be ignored, might be useful to mention.
  
      \subsubsection{The Ruby debugger}
  
        Give a brief introduction and illustration of use of the Ruby debugger
      (starting it up, stepping, showing the value of variables, setting
      breakpoints, etc.).  Hopefully this will help students debug their
      programs better (which might save us some work).  Students used
      \texttt{gdb} in the prior course, so they know debugger concepts; just
      show them how the debugger is used in Ruby.
  
      \subsubsection{Example}
  
        If time permits, you could try to develop a slightly larger example.
      Here's one idea that might be cute; feel free to use another one if
      you prefer.
  
        As you may or may not know, September 19 of every year is
      International Talk Like A Pirate Day
      (\texttt{www.talklikeapirate.com}).  Write a program that reads
      lines of text (sentences) until the end of the input, and performs a
      few translations to make them sound like pirate--speak.  That site
      actually has a Pirate--speak--to--English translator, but you can make
      up whatever rules you want just to make things interesting.  Your
      program could for example:
  
      \vspace{-1.5mm}
  
      \begin{itemize}
  
        \addtolength{\itemsep}{-.5mm}
  
        \item Pick several verbs and look for all occurrences of them in the
              input, changing them to the present tense.  But the word has
              to appear as a separate word, not a substring of another word,
              so you have to make sure it's delimited by either whitespace,
              or a period, or by the beginning or the ending of the line.
  
        \item Pick several few words ending in ``ing'' and replace the
              ``ing'' with a quote (only if the ``ing'' is a word ending).
  
        \item Insert ``Arr'' at the beginning of every sentence (at the
              beginning of the line, or after a period).  (We assume that a
              sentence can be longer than one input line, and one sentence
              can end in the middle of an input line and another one begin
              on the same line, after the period.)
  
      \end{itemize}
  
      \vspace{-1.5mm}
  
      For example, ``We are ingeniously learning programming in Ruby.''
  
      \smallskip
  
      could become ``Arr, we be ingeniously learnin' programmin' in Ruby."
  
      \smallskip
  
        (Note that the ``ing'' in ``ingeniously'' wasn't changed, since it
      didn't appear at the end of a word.)
  
        You can probably add more if you are inclined.
  
    \subsection{Grading and returning Quiz \#1}
  
      \begin{itemize*}
  
        \addtolength{\itemsep}{2mm}
  
        \item About grading quizzes:

              \vspace{-1.5mm}

              \begin{itemize}

                \addtolength{\itemsep}{.75mm}

                \item Note: before the quizzes can be returned they have to
                      be graded, and the grades have to be entered into ELMS
                      (www.elms.umd.edu).  Click on ``Control Panel'' and
                      ``Gradebook''.
  
                      To make entering the grades as accurate as
                      possible, please alphabetize each section's quizzes,
                      and click on the column heading ``Section'' in ELMS;
                      that way the students will be in the same order as the
                      quizzes.  (Otherwise it's very easy to type a student's
                      score into the wrong person's row by accident.)  Since
                      some students don't take a quiz, for whatever reason, be
                      sure to enter the each score into the right student's
                      row.  If there are any students who don't take a quiz,
                      just \textbf{leave their entry blank}-- don't enter
                      zeros for them, since that throws off the calculation of
                      the quiz average.
  
                      Note that you can hide columns in the ELMS grade center
                      by clicking on the column heading; this can make it
                      easier to enter scores.

                \item Grade quizzes using red ball-point pen only.  If you
                      grade in black or blue pen it's hard to tell later, if
                      a question comes up, what the students wrote on
                      quizzes and what you wrote in grading.  And if you use
                      a felt tip pen the writing smears and fades if it gets
                      wet (I don't know why but some students seem to get
                      their quizzes and exams wet; maybe they carry them
                      outside when it's raining and they don't have them in
                      anything like a backpack or folder).  Using a red pen
                      also reduces the chance a student could try to alter
                      the grading later.

                \item Mark deductions in the problem (negative points for
                      mistakes), but give the total number of points for
                      each question as a positive number, written near to
                      the number for that question (so students can see how
                      many points they got out of the total for that
                      question).  Then write the total quiz score somewhere.

                \item You don't have to write the correct answers on quizzes
                      when students get things wrong; that would take too
                      long, and it's quicker to just go over the quiz
                      answers once in class.  But you need to make some type
                      of indication about what students get wrong.  In other
                      words, don't just write a point deduction like
                      \pts{-3} without either giving a few words describing
                      what's incorrect, or circling on the description of
                      the question what wasn't done correctly, or X'ing on
                      the student's answer what was wrong, and just write
                      the deduction next to that.

                \item Please grade the quizzes early, so in case questions
                      about grading come up there will be be time to resolve
                      them.

                \item If some students don't give their complete identifying
                      information (University ID number, TA's name, and
                      discussion time) make a note that they need to do so.
                      When exams are given it slows down the exam grading to
                      have to look this information up, so I want students
                      to understand now that they need to put this complete
                      information correctly on quizzes and exams.

                \item I don't want to give anyone a score of zero (or almost
                      zero) if they at least took a quiz.  Hopefully there's
                      no one who will get every question wrong, but if it
                      ever happens that anyone's score is 5 or less then
                      just give them 5 points for at least writing their
                      name.  Please write a short note on the quiz of anyone
                      who did poorly (whatever the lowest range of scores
                      is) encouraging them to come to office hours for help.
                      If there are any quizzes on which anyone does
                      particularly well you're encouraged to write something
                      congratuatory on their quiz (even just a ``Good
                      job!'').

              \end{itemize}

              \vspace{-1.5mm}

        \item Return the quizzes, go over the answers as needed, and answer
              any questions about the solutions.  If there are problems that
              many students got wrong show how to solve them if time
              permits.  (If there's ever not enough time to go over all the
              answers when returning a quiz then just return them and
              encourage students to try to figure out what mistakes they
              made, and ask questions in office hours if needed.  Or you may
              have more time the next discussion to go over the answers.)

        \item Before returning the quizzes, make a strong effort to
              encourage anyone who didn't do that well (whatever the lower
              range of scores turns out to be) to come to office hours for
              extra help or explanation right away, so they can get caught
              up before they are even further behind.

        \item Be sure to hand back quizzes so no one else can see anyone's
              grade.  By University policy and by law, grades are considered
              highly confidential; no student should see any other student's
              grade for anything, even on a quiz.  A simple thing you can do
              to avoid this, as well as to not waste time handing out each
              quiz individually, is to just write everyone's score on the
              back of their quiz.  That way you can spread them face--up on
              the table for students to pick up as they come in or leave,
              without having to hand them out individually, yet the grades
              are still not visible.  Or if you prefer you can hand them out
              in the ten minutes before class, as students come in, so as
              not to use up class time doing it.

        \item The quiz grading key should not be given or shown to
              students.  Students can get an explanation about anything they
              get wrong, including how many points off a certain error was,
              or how many points a part of the quiz was, but should not be
              given the entire grading key, either on paper or electronically.

        \item Keep bringing any unreturned quizzes to subsequent discussion
              sections; sometimes students don't show up to class for a while.
  
      \end{itemize*}
  
      \vspace{-2.5mm}
  
    \subsection{If you have extra time}
  
      If you have any extra time, instead of letting the students go early,
    see if they have any questions and try to answer them if so.  I am
    assuming that the first project will have been assigned a while back by
    this point; you can see if there are questions about it (of course you
    will have had to have read it carefully to answer).  Or, if you think
    you may have extra time, make up a few problems that are similar to
    examples recently done in class, and solve them yourself, or ask
    students how to solve them.

\end{document}
