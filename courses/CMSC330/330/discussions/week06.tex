\documentclass[12pt]{article}

  \usepackage{330-f12}

  \usepackage{wasysym}

\begin{document}

  \header{\course}{Discussion section topics -- Week \#6}{\Term}

  \section{Discussion \#5, Friday, October 5}

    \subsection{A practical and fairly realistic Ruby application}

      A student posted something in Piazza about having to remove his output
    files (files used for output redirection) before running his project
    each time.  When I read this, my first thought was that this would be a
    good application for a shell script, to have you all show the students
    an example in discussion section of a script that would run their
    Project \#1 on all the public inputs, to automate the process.  But
    since we're teaching Ruby, it could be a Ruby program instead.  It's
    debatable whether a shell script would be more practice with Ruby.  It's
    a pretty natural application for a scripting language-- running
    programs, capturing their output, and interacting with the shell.

      What the program can do:

      \vspace{-2mm}

      \begin{itemize}

        \addtolength{\itemsep}{-.5mm}

        \item Iterate over all files in the current directory of the form
              \texttt{public*.input}.  This can be done using either
              \texttt{Dir["public?.input"].sort()each()}, or something like

              \smallskip

              \begin{centering}

                \texttt{`ls public?.input
                        secret??.input`.split().sort().each()}

              \end{centering}

              \medskip

              (Those are backquotes around the ls command.)  It would be
              good to show and explain both if possible.

        \item Print the name of each test.

        \item Use \texttt{system()} to run a command that executes
              \texttt{proj1.rb}, with input redirected from the name of the
              test and output redirected to the file \texttt{my-output}.

        \item Create the corresponding output filename from the input file
              name (for example, \texttt{public3.input} has the
              corresponding output file named \texttt{public3.output}).

        \item Use backquote substitution to run diff, comparing the expected
              output file with the contents of the file \texttt{my-output},
              and capturing the output in a string.

        \item If the result of the above (the output of diff) is just an
              empty string, print ``passed''.

        \item Otherwise print ``failed'' and also print the output produced
              by diff (the result of the backquote substitution).

        \item Use \texttt{File.unlink("my-output")} to remove the
              redirected output file.

      \end{itemize}

      \vspace{-2mm}

      Of course various things could be done in different ways; the above just
    illustrates certain features.  For example, we could test the exit code
    of diff instead of comparing the output to the empty string, but I
    figure if a test failed the student is going to want to see what
    differed, so we may as well capture the output of diff to be able to show
    that.

      Also, this wouldn't handle the situation where the program was supposed
    to be run with different command--line arguments for some inputs.  If
    you have a minute you could talk about how that might be handled.

      If you have any questions how to write the program feel free to mail me
    what you have so far, and your questions, and I will look at it and
    reply.

      After the section, one of you should put your version in the public
    directory on Grace, for the students to see.  Of course use good
    style-- comments would be ideal \smiley.

    \subsection{Writing some small list functions in OCaml}

      Write some OCaml functions like those below, or have the students try to
    write some of them themselves first (or some combination of these).

      Here are rough ideas for types of functions, which you'll have to make
    more specific; you can also use different ones if you like (just avoid
    doing the same ones that will have already been done in lecture).  I
    tried to put them in order of increasing difficulty.

    %   You might want to do one or two functions yourself first, then let the
    % students work on the others themselves.

      Go over and explain your versions of as many of them as time permits.
    After the section, one of you should put your solutions to all of them
    in the public directory on Grace,l for the students to see.  It should
    be in the form of a program with sample calls to the functions.

      Students don't know about closures or currying yet, so functions with
    multiple arguments should be written to take tuples as arguments.  The
    students should have seen higher--order functions in lecture yesterday--
    at least \texttt{map}; not sure about \texttt{fold} (hopefully the
    lecture slides will be posted soon after the previous lecture so you can
    see what was covered).

      \vspace{-1mm}

      \begin{itemize}

        \addtolength{\itemsep}{-1mm}

        \item \label{part:firstpart} Write a (nonrecursive) function that
              takes two lists A and B as parameters (a tuple of the two
              lists), removes the first element from A, adds it to the
              beginning of B as B's new first element, and returns the tuple
              (A,B) (with the ``new'' values of A and B).

        \item Write a (recursive) function that processes a list without
              creating a new list.  For instance, it could be a function
              searching whether a given element is present in a list or not
              (it could return a true or false result, or could return the
              position where the element was found).

        \item Write a (recursive) function that processes a list,
              constructing an accumulating answer (not a list, just a single
              value) based upon the values of the list's elements.  For
              instance, it could be a function which counts the number of
              occurrences of some element in a list.

        \item Write a (recursive) function that processes a list, creating
              some type of result list (a different list) based upon the
              values of the parameter list's elements.  For instance, it
              could create a list containing only the even elements in the
              parameter list, or could replace all occurrences of some
              element in a list with another value and return the result.

        \item Write a (recursive) function that calls the function in
              part~\ref{part:firstpart} above to move all of the elements of
              A to B (prepends A's elements in reverse order to the
              beginning of B).  This is just an example of one function
              calling another.

      \end{itemize}

      \vspace{-1mm}

      Note that this exercise isn't necessarily supposed to represent examples
    of useful programs in OCaml, just to get some initial practice using it.

    \subsection{Return Quiz \#2}

      Return the quizzes and go over the answers as needed, and answer any
    questions about the answers.  If there are problems that many students
    got wrong show how to solve them if time permits.  (If there's ever not
    enough time to go over all the answers when returning a quiz then just
    return them and encourage students to try to figure out what mistakes
    they made, and ask questions in office hours if needed.)

      Before returning the quizzes, if you still have any unreturned Quiz
    \#1s, \textbf{staple them to the student's current quiz}, so if they
    pick that one up they will get them both.  And keep bringing unclaimed
    quizzes to subsequent discussion sections in case the students show up,
    so you can give them back if so.

      Before returning the quizzes, make a strong effort to encourage anyone
    who didn't do that well (whatever the lower range of scores turns out to
    be) to come to office hours for extra help or explanation right away, so
    they can get caught up before they are even further behind.

      If any students didn't put their University ID number, TA's name, or
    discussion time, or any of these were wrong, please write a short note
    telling them they must put all their identifying information correctly.

    \subsection{If you have extra time}
  
      If you have any extra time, instead of letting the students go early,
    see if they have any questions and try to answer them if so.  If the
    second project has been assigned by now (presumably so) there may be
    questions about it.

\end{document}
