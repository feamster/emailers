\documentclass[12pt]{article}

  %
  % Semester-specific note about lecture slides is below.
  %

  \usepackage{330-f12}

  \newcommand{\myepsilon}{\mbox{\Large\ensuremath{\epsilon}}}

\begin{document}

  \header{\course}{Discussion section topics -- Week \#5}{\Term}

  \section{Discussion \#4, Friday, September 28}

    \subsection{Finish material from before}

      Some TAs hadn't finished all the material from last time, so try to
    finish it today if needed
  
    \subsection{Constructions on automata}

    Run through an example of the algorithm to convert an NFA to a DFA.  In
    a separate PDF is one example NFA and the DFA that should result from
    it; of course you could use a different example if you like.  (I am
    showing the NFA and DFA on two separate pages, so you can display them
    using the room PC without students seeing these topics>0 Do the
    construction methodically, identifying what
    $\myepsilon\hspace{-.5mm}-\hspace{-.5mm}closure$ and \emph{move}
    specifically are at each step.  If you think that time permits you could
    also ask students to try to work on it on their own for a few minutes
    first and see what they come up with.

      Then run through the construction to minimize the states in a DFA (just
    come up with a good example DFA that has lots of duplicative states
    yourself).

      I'm not sure how long these might take, so be prepared with a few of
    extra DFAs and NFAs in case.  You might work through the first one of
    each yourself, then get the students to work through another pair
    themselves, going over the results afterwards.

      If that still doesn't seem like it's going to use up all the time, come
    up with a small example regular expression (or two) and run through the
    construction to convert it to a DFA.

      (Note that the lecture slides from this week regarding these
    constructions are not posted on the class webpage as of this writing,
    but they will be after the lectures today.)
  
    \subsection{If you have extra time}
  
      If you have any extra time, instead of letting the students go early,
    see if they have any questions and try to answer them if so.  There may
    be questions about the project.

    \subsection{Quiz \#2}

      Please review the quiz information from the earlier discussion topics
    so everything is clear.

      Stop the above so the students have 15 minutes at the end of the section
    to do the quiz (or earlier or later if we decide that any particular
    quiz should have more time).

      When handing out the quiz, emphasize that students \textbf{must} put
    their complete identifying information on the front (name, University ID
    number, TA's name, and discussion section time).

      Give the quiz at the end of the discussion.  Count the number of
    students taking the quiz during each section, and write it down and save
    it.  Count the number of quizzes turned in and make sure it agrees.

\end{document}
