\documentclass[11pt]{article}

\usepackage{epsf}
\usepackage{epsfig}
\usepackage{url}
\usepackage{6829hw}

\newcommand{\newc}{\newcommand}

\newc{\code}[1]{{\tt #1}}
\newc{\func}[1]{{\em #1\/}}

\newc{\be}{\begin{enumerate}}
\newc{\ee}{\end{enumerate}}

\newc{\bi}{\begin{itemize}}
\newc{\ei}{\end{itemize}}

\newc{\bd}{\begin{description}}
\newc{\ed}{\end{description}}

\newc{\ov}[1]{$\overline{#1}$}
\newc{\instr}{\tt}

\newc{\doublespace}{\renewcommand{\baselinestretch}{1.5}}

\newcommand{\figref}[1]{Figure~\ref{#1}}
\newcommand{\tref}[1]{Table~\ref{#1}}

% Captioned table
\newc{\tbl}[3]{
        \begin{table}[htb]
                \centering
                #1
                \caption{#3}
                \label{#2}
        \end{table}
}

% Input a table.
\newcommand{\dblfig}[3]{
        \begin{figure}[htb]
		\centering
                \input{#1}
                \caption{#3}
                \label{#2}
        \end{figure}
}

\newcommand{\ddblfig}[4]{
        \begin{figure}[htb]
		\hspace{-0.1in}
                \psfig{figure=#1,width=0.45\textwidth}
                \caption{#3}
                \label{#2}
        \end{figure}
}

% Figure with no caption
\newcommand{\nofig}[2]{
        \begin{figure}[htb]
                \centering
                \psfig{figure=#1}
                \label{#2}
        \end{figure}
}

% Whole page figure
\newcommand{\schfig}[3]{
        \begin{figure}[p]
                \centering
                \psfig{figure=#1,height=7in}
                \caption{#3}
                \label{#2}
        \end{figure}
}

% Small figure
\newcommand{\sfig}[3]{
        \begin{figure}[ltb]
                \centering
               \hspace*{\fill}\rule{\linewidth}{.5mm}\hspace*{\fill}\vspace{3mm}
                \psfig{figure=#1,width=0.4\textwidth}
                \caption{#3}
                \label{#2}
               \vspace{3mm}\hspace*{\fill}\rule{\linewidth}{.5mm}\hspace*{\fill}
        \end{figure}
}

% Medium figure
\newcommand{\mfig}[3]{
        \begin{figure}[ltb]
		\centering
               \hspace*{\fill}\rule{\linewidth}{.5mm}\hspace*{\fill}\vspace{1mm}
                \psfig{figure=#1,height=2.5in}
                \caption{#3}
                \label{#2}
               \vspace{0mm}\hspace*{\fill}\rule{\linewidth}{.5mm}\hspace*{\fill}
        \end{figure}
}

\newcommand{\widefig}[4]{
        \begin{figure*}[htb]
                \centering
               \hspace*{\fill}\rule{\linewidth}{.5mm}\hspace*{\fill}\vspace{5mm}
                \psfig{figure=#1,width=#3}
                \caption{#4}
                \label{#2}
               \vspace{5mm}\hspace*{\fill}\rule{\linewidth}{.5mm}\hspace*{\fill}
        \end{figure*}
}

\newcommand{\mcfig}[4]{
        \begin{figure}[htbp]
                \centering
               \hspace*{\fill}\rule{\linewidth}{.5mm}\hspace*{\fill}\vspace{5mm}
                \psfig{figure=#1,width=#3}
                \caption{#4}
                \label{#2}
               \vspace{5mm}\hspace*{\fill}\rule{\linewidth}{.5mm}\hspace*{\fill}
        \end{figure}
}

\newcommand{\docfig}[3]{
        \begin{figure}[htbp]
               \hspace*{\fill}\rule{\linewidth}{.5mm}\hspace*{\fill}\vspace{5mm}
                \centering
                \psfig{figure=#1,width=#3}
                \label{#2}
               \vspace{5mm}\hspace*{\fill}\rule{\linewidth}{.5mm}\hspace*{\fill}
        \end{figure}
}

% Medium-large figure
\newcommand{\mlfig}[3]{
        \begin{figure}[htb]
                \centering
               \hspace*{\fill}\rule{\linewidth}{.5mm}\hspace*{\fill}\vspace{5mm}
                \psfig{figure=#1,height=3.25in}
                \caption{#3}
                \label{#2}
               \vspace{5mm}\hspace*{\fill}\rule{\linewidth}{.5mm}\hspace*{\fill}
        \end{figure}
}

% Large figure
\newcommand{\lfig}[3]{
        \begin{figure}[p]
                \centering
               \hspace*{\fill}\rule{\linewidth}{.5mm}\hspace*{\fill}\vspace{5mm}
                \psfig{figure=#1,height=5in}
                \caption{#3}
                \label{#2}
               \vspace{5mm}\hspace*{\fill}\rule{\linewidth}{.5mm}\hspace*{\fill}
        \end{figure}
}

% 'gg' figures are the double column versions of the 'g' figures above.
\newcommand{\sfigg}[3]{
        \begin{figure*}[htb]
                \centering
               \hspace*{\fill}\rule{\linewidth}{.5mm}\hspace*{\fill}\vspace{5mm}
                \psfig{figure=#1,height=1.5in}
                \caption{#3}
                \label{#2}
               \vspace{5mm}\hspace*{\fill}\rule{\linewidth}{.5mm}\hspace*{\fill}
        \end{figure*}
}

% Medium figure
\newcommand{\mfigg}[3]{
        \begin{figure*}
                \centering
               \hspace*{\fill}\rule{\linewidth}{.5mm}\hspace*{\fill}\vspace{5mm}
                \psfig{figure=#1,width=\linewidth}
                \caption{#3}
                \label{#2}
               \vspace{0mm}\hspace*{\fill}\rule{\linewidth}{.5mm}\hspace*{\fill}
        \end{figure*}
}

% Medium-large figure
\newcommand{\mlfigg}[3]{
        \begin{figure*}[htb]
                \centering
               \hspace*{\fill}\rule{\linewidth}{.5mm}\hspace*{\fill}\vspace{5mm}
                \psfig{figure=#1,height=3.25in}
                \caption{#3}
                \label{#2}
               \vspace{5mm}\hspace*{\fill}\rule{\linewidth}{.5mm}\hspace*{\fill}
        \end{figure*}
}

% Large figure
\newcommand{\lfigg}[3]{
        \begin{figure*}[p]
                \centering
               \hspace*{\fill}\rule{\linewidth}{.5mm}\hspace*{\fill}\vspace{5mm}
                \psfig{figure=#1,height=5in}
                \caption{#3}
                \label{#2}
               \vspace{5mm}\hspace*{\fill}\rule{\linewidth}{.5mm}\hspace*{\fill}
        \end{figure*}
}

% Variable size figure
\newcommand{\vfigg}[4]{
        \begin{figure*}[htb]
                \centering
               \hspace*{\fill}\rule{\linewidth}{.5mm}\hspace*{\fill}\vspace{5mm}
                \psfig{figure=#1,#2}
                \caption{#4}
                \label{#3}
               \vspace{5mm}\hspace*{\fill}\rule{\linewidth}{.5mm}\hspace*{\fill}
        \end{figure*}
}

\newcommand{\vfig}[4]{
        \begin{figure}[ltb]
                \centering
               \hspace*{\fill}\rule{\linewidth}{.5mm}\hspace*{\fill}\vspace{1mm}
                \psfig{figure=#1,#2}
                \caption{#4}
                \label{#3}
               \vspace{1mm}\hspace*{\fill}\rule{\linewidth}{.5mm}\hspace*{\fill}
        \end{figure}
}

\newcommand{\vnlfig}[4]{
        \begin{figure}[htb]
                \centering
               \hspace*{\fill}\rule{\linewidth}{0mm}\hspace*{\fill}\vspace{5mm}
                \psfig{figure=#1,#2}
                \caption{#4}
                \label{#3}
               \vspace{0mm}\hspace*{\fill}\rule{\linewidth}{0mm}\hspace*{\fill}
        \end{figure}
}

\newcommand{\dblvfig}[6]{
        \begin{figure}[htb]
                \centering
                \hspace*{\fill}\rule{\linewidth}{0mm}\hspace*{\fill}\vspace{0.5mm}
                \psfig{figure=#1,#2}
	        \hspace{1in}
                \psfig{figure=#3,#4}
                \caption{#6}
                \label{#5}
               \vspace{2mm}\hspace*{\fill}\rule{\linewidth}{0mm}\hspace*{\fill}
        \end{figure}
}
\newc{\myspacing}{
        \let\oldtextheight=\textheight
        \let\oldtextwidth=\textwidth

        \let\oldtopmargin=\topmargin
        \let\oldheadheight=\headheight
        \let\oldfootheight=\footheight
        \let\oldheadsep=\headsep
        \let\oldoddsidemargin=\oddsidemargin


        \textheight 8.5in
        \textwidth 6in

        \topmargin 0in
        \headheight 0in
        \footheight 1.5in
        \headsep 0in
        \oddsidemargin 0in

}

\newc{\oldspacing}{
        \let\textheight=\oldtextheight 
        \let\textwidth=\oldtextwidth

        \let\topmargin=\oldtopmargin 
        \let\headheight=\oldheadheight 
        \let\footheight=\oldfootheight
        \let\headsep=\oldheadsep
        \let\oddsidemargin=\oldoddsidemargin
}
% Local Variables: 
% mode: latex
% TeX-master: t
% End: 


\begin{document}

\newcounter{listcount}
\newcounter{sublistcount}


\handout{H3}{April 3, 2013}{Instructor: Prof. Nick Feamster}
{College of Computing, Georgia Tech}{Problem Set 3: Transport and
  Application Layers} 

%This problem set has three questions, each with several parts.  Answer
%them as clearly and concisely as possible.  You may discuss ideas with
%others in the class, but your solutions and presentation must be your
%own.  Do not look at anyone else's solutions or copy them from
%anywhere. (Please refer to the Georgia Tech honor code, posted on the
%course Web site).

Turn in your writeup and talk on {\bf April 17, 2013} by 11:59pm.
{\em Please upload your solutions to T-Square.  Other forms of
  submission will not be accepted!}  We will be providing more
information about how to turn in your assignment as the due date
approaches.

\section{Web Caching and Performance}

\begin{enumerate}
\item Explain the difference between Web proxying and caching.  Why
  would you want to run a non-caching Web proxy?  What benefits are
  provided by a caching Web proxy?
\item List three different Web proxies online that you might want to use
  for different purposes. For the ones that have public services
  running, configure your Web browser to use the proxy.  Do you notice
  any difference in performance for any of them?
\item Occasionally, companies, businesses, and governments will re-route
  your traffic through a proxy in a ``transparent'' fashion, without you
  knowing it.  What kind of tests could you run to determine that your
  Web requests were being routed through a Web proxy? (There is no
  single right answer; think about what types of discrepancies proxies
  might introduce.)

\item Go to {\tt http://webpagetest.org//} and create a test for {\tt
  http://nytimes.com}.  Include a snapshot of the report in your
  assignment. Include information about what location your request was
  made from, as well as which browser was used for the test (Note: not
  {\em your} browser, but the Web browser that
  \url{http://webpagetest.org/} uses.  For each of the questions below,
  please {\em explain how you computed your answer}.
\begin{itemize}
\item What is your best guess at how \url{http://webpagetest.org/}
  works?  Suppose you wanted to run this type of test from many
  different locations.  How would you design such a service?
\item What was the total loading time for the page?
\item How many total objects were downloaded?
\item How many DNS lookups were required to load all of the resources
  for this page?
\item Some objects were returned with a 302 error code.  What is a 302
  error code, and why do you think it was returned for certain objects?
\item What is ``time to first byte'', and why is it an important metric?
\item How many distinct locations were the Web page resources downloaded
  from?  For those labeled ``Akamai'', can you find the location of
  those resources by doing a traceroute, IP lookup, etc.?  Where is that
  content relative to where the test was performed?
\item How many unique domains is the browser connected to at any time?
  How many connections does the browser make to each domain at any time?
  How does the number of parallel connections affect page load time?
  What impact might parallel HTTP connections have on ``fairness'' to
  other clients and Web browsers?
\item What determines the order in which the resources for this page are
  fetched and loaded?
\end{itemize}

\item Suppose that a client has an ``origin server'' for a Web site that
  is 300~ms away (such might be the case for a user in a developing
  country), but that a local cache is 10~ms away.  Suppose that the page
  contains static content that can be retrieved in five round trips to
  the server that contains the content over a TCP connection.
\begin{itemize}
\item If all of the static content resides in the local cache, how much
  time would be saved in loading the page?
\item Suppose only 50\% of the static content were in the cache.  How
  much time would be saved to load the content?
\item What types of steps might you perform in designing a Web cache to
  improve the cache ``hit rate''?
\item Suppose you have a site that also has dynamic content.  Describe
  how you would use a ``reverse proxy'' to improve the loading time for
  a Web page.
\end{itemize}
\end{enumerate}




\section{Programming Assignment: Simple HTTP Proxy}

In the previous assignment, you wrote a client and server program to do
a TCP file transfer.  This follow-up assignment builds up on the
previous one to implement a HTTP Proxy that can handle an HTTP GET
request. {\em We will run moss on your code to check for copying, so
  please be sure to write your own code.}

Most HTTP transactions can be broken down to the following sequences
(RFC 2616): 
\begin{enumerate}
\item Client establishes a connection with the server (a TCP connection)
\item The client issues a request, by sending a line of text using HTTP
  method (e.g., GET,POST etc. which were discussed in class), with the
  version of HTTP.
\item The server sends back a response status code, a reason
phrase (description of the response code) and the message body
containing the response.
\item The connection is closed, unless persistent connections are used.
\end{enumerate}
\noindent
{\bf HTTP Proxy:} In contrast to the ordinary client-server model, where
an HTTP client directly communicates with the Web server, in certain
situations (such as in a content distribution network) the Web client
might send a Web request to a proxy.

A proxy server is an intermediary between the web server and the client,
which implies it serves the pages that would be served by another Web
server.  Your task is to implement a Web proxy that can accept and
forward request to the original server. {\em Your server only needs to be
able to handle one HTTP request at a time.}  (Unless you want to do
more, of course, which we will certainly welcome. :-)

The only type of HTTP request you need to support with your proxy is a
GET request. The GET request addressed to a proxy server must have an
absolute (not relative) URI. You should also implement ``Bad Request''
(400) error code for an invalid request, ``Not Found'' (404), and ``Not
Implemented'' 501 error code if the server does not support the
functionality required to fill the request.


You should write your Web proxy in C or Python with the file structure
and Makefile as explained in Socket Programming Assignment.  Do not
hardcode the port number that your proxy is running on.  {\em If you are
  using Python, you are not allowed to use any libraries that implement
  the Web proxy for you.}

You can use your Web browser to test your Web Proxy by setting the Web
proxy settings in the configuration of your Web browser and issuing an
HTTP request from the command line.

To test your proxy, we will enter the request type, absolute URL
(beginning with HTTP), and HTTP protocol with the version.  You can also
use a simple command line tool such as {\tt curl} to fetch a URL
(without fetching all of the embedded objects).  For the test
case, we will attempt to use your proxy to fetch a simple HTML file with
no embedded objects from a local server at Georgia Tech.


{\bf Design question: Extending your proxy.}  Suppose that you want to
extend your Web proxy to handle multiple concurrent HTTP requests
(either from a single browser that opens multiple connections to the
server, or from multiple clients.

\begin{enumerate}
\item Why would you want your proxy to support multiple simultaneous
  clients? (Hint: Think about caching.)
\item Why would you want your proxy to support multiple simultaneous
  requests from the same client?
\item How would you change your proxy code to support multiple
  concurrent HTTP requests?  (Hint: Think about the things discussed in
  the sockets programming lecture, where several options were
  discussed.)  What libraries or function calls would you use? Be as
  specific as you can; if you'd like to write the code to support
  concurrent requests, you're welcome to do so, although you're not
  required to do this.  An explanation of what you {\em would} do will
  suffice. 
\end{enumerate}

\end{document}
