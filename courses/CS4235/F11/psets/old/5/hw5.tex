\documentclass[letterpaper]{article}
\usepackage{amsmath}
\usepackage{amssymb}
%\usepackage{fullpage}

\title{}
\date{}

\begin{document}
\thispagestyle{empty}

\section*{CS 4235 / CS 8803IIS Homework 5}

\noindent {\bf Assigned:} 9 March 2011

\noindent {\bf Due:} 28 March 2011, 5:00pm Atlanta time. Students submitting solutions after that time but by 5:00pm Atlanta time on 30 March will have their scores scaled by 0.8. No solutions will be accepted after 5:00pm on 30 March.

\noindent {\bf Teaming:} This is a group project. Please work in a group of size two or three. You may choose your own group partners.

\bigskip\noindent
Your solution must be your own (and teammates) work. Although you can use
outside sources for information and library code, you:
\begin{itemize}
\item {\bf must} be legally allowed to use any code not written by you, and
\item {\bf must} cite all sources of information and code. A citation should provide
sufficient information for myself or anyone else to find the source that you
used.
\end{itemize}
We may employ the use of automated code analysis and comparison tools as part of
our evaluation. Any suspected academic conduct violations will be referred to the
Dean of Students Office. If you are ever unsure whether or not you are acting in
violation of academic conduct policies, please ask us rather than guessing.

This homework has one programming part worth 200 points. Scores will be posted on T-Square.

\bigskip\noindent
This assignment is programming-intensive. If you do not get an early start, you may
be unable to successfully complete the project. You may implement your solution in
any of C, C++, or Java without prior approval from the TA. You may use a different
language only if the TA gives you permission.

In this project, you will develop an instant messaging system called Chatter. Chatter
has a client-server architecture that allows multiple clients to communicate through
a single shared server. The human user of each client must successfully authenticate
to the server, and all communication between every client and the server is
encrypted to provide confidentiality from eavesdroppers.

The Chatter ``system'' is a single server and one or more clients connected to the
server. You are responsible for implementing the complete system; this will require
both network programming and cryptographic programming. For this project, all
components of the system should execute on Red Hat Enterprise Linux as found on
any of the killerbee systems (e.g.\ killerbee1.cc.gatech.edu).

\subsection*{Chatter server}

The Chatter server is a command-line network server with two purposes:
\begin{enumerate}
\item It authenticates human users to the Chatter system.
\item It securely distributes messages typed by one user to all connected clients.
\end{enumerate}

The server should have a database of known usernames and the cryptographic hash
of each user’s password. You do not need to implement functionality to add or
change existing users and can manually create the database. Note that passwords
should not be stored in cleartext so that accidental disclosure of the database to an
attacker does not allow the attacker to know users’ passwords. Choose a non-reversible
cryptographic hash function, and store only the hash of each password.
An attacker cannot reverse the hashes to determine passwords, but the Chatter
server can still authenticate users by recomputing the hash of a typed password and
verifying that it matches the stored hash in the user database.

When started, the server should begin listening for incoming TCP connections from
Chatter clients. Upon receipt of an incoming connection, the server and client should
use Diffie-Hellman key exchange to establish a shared secret key. Note that the D-H
values chosen by the client and server (g, p, a, and b) should be selected using a
random number generator so that each session will have a different key. As soon as
the key is established, the client and server should begin sending all data encrypted
with a symmetric key protocol using the D-H secret key.

After establishing the secure channel, the server should authenticate the human
user of the Chatter client. The client should send the username and password of the
user across the encrypted channel to the server. The server should recompute the
password hash and then verify the username and hash against the database of
known users. If the verification fails, then the server should disconnect the client.
Otherwise, the server next provides instant messaging service to the client.

An authorized client sends messages to the server and receives messages back from
the server (including messages sent by itself). The server should accept a message
from any connected and authorized client. It should then distribute that message to
all authorized clients, including the client that generated the message. Note that
distribution will require the server to decrypt the message using the secret key
shared with the sender and then to reencrypt the message using each client’s unique
shared secret key. When distributing messages, the server should also send the
username of the client that generated the message.

A message is one line of printable characters ending in a newline. Messages may
have an arbitrary number of characters before the newline, so be sure that your
server and client allow arbitrary length strings.

The server should silently discard messages with unprintable characters.

Clients may come and go during while the server is providing instant messaging
service, so it should be able to authenticate some clients while simultaneously
receiving and distributing messages to other clients already authenticated.

For our testing, the server should support up to four simultaneously connected
users. You may program this support into the server in whatever way is easiest for
you. For example, the server can listen for all connections on one port and then pass
off each incoming connection request to a child process. This approach scales nicely
to a large number of simultaneous connections. Alternatively, the server could listen
on four ports and expect each of the four clients to connect on different ports. This
will allow you to write a simpler single-threaded, single-process server, although it
is less elegant. We suggest printing a message to the screen when the server starts
up instructing us how to connect each of the clients to the server.

Once started, the server should be non-interactive. At your discretion, you can have
the server print out debug messages if you think they will help us test your system,
but we should not need to type commands into the server. We will expect to
terminate the server's execution via Control-C.

The server should gracefully handle exiting clients or lost network connections. The
server should simply remove that client from its list of currently connected clients,
and new clients should still be able to connect to the server and join the system.

\subsection*{Chatter client}

The Chatter client is a command-line interactive program that a human uses to chat
with other participants of the Chatter system. The client should take command-line
arguments that specify how to connect to the Chatter server. This should include the
TCP port at which the server is listening, and may optionally include a computer
name or IP address specifying where the server is executing. Note that our tests will
execute the server and all clients on the same machine, so a machine name
argument is not required.

When started, the client should attempt to connect to the server. If the attempt fails
(perhaps the server is not executing), then the client should print an error message
and exit. If the connection succeeds, then the client should run the Diffie-Hellman
key exchange algorithm with the server to establish the shared session key. After
establishing the key, the client should use symmetric key cryptography for all
subsequent communication with the server.

The client should prompt the user to type a username and password at the
keyboard. The client then sends this information to the server over the encrypted
channel for user authentication. If authentication fails, then the client should print
an error message and exit. Otherwise, the client should enter its instant messaging
mode.

In the instant messaging mode, the client receives input from both the keyboard and
the server. Input from the keyboard is made of messages typed by the human using
the client. A message is a single line of printable text ending with a newline. When
the user finishes typing a message (e.g. presses $\langle$Return$\rangle$), the client should send
that message across the encrypted TCP channel to the server.
Input from the server is comprised of messages and the username of each message's
author. The client should print each message to the screen in the format:
\begin{quote}
$\langle$username$\rangle$: $\langle$message$\rangle$
\end{quote}
without the angle brackets. All messages from the server should be printed,
including the messages that were sent by the same client. Make sure that you can
handle the case where the client receives a message from the server while the
human user was in the middle of typing at the keyboard.

We will expect to terminate a client's execution by pressing Control-C, although you
are free to create a different exit mechanism at your discretion.

The client should gracefully handle an unexpected termination of the server or a lost
network connection. Printing an error message and exiting is a suitable behavior.

\subsection*{Cryptographic algorithms}

You need to choose two cryptographic algorithms for use in the Chatter system:
\begin{enumerate}
\item The hash algorithm used on user passwords.
\item The symmetric-key cryptographic scheme used for data confidentiality.
\end{enumerate}
Select algorithms that you believe appropriate based on our classroom discussions.
You should not need to implement either of these algorithms. It is perfectly suitable
to find and use libraries implementing cryptographic operations.

You may need to massage the data that you would like to encrypt. Algorithms often
expect plaintext data of a specific length, so you may need to break long plaintext
into smaller blocks and pad short messages out to a full block length. When
decrypting a message, your code should remove padding and reassemble blocks as
necessary.

\subsection*{Diffie-Hellman key agreement}

You will need to use D-H key agreement to create the shared secret key used for
subsequent symmetric-key cryptography. The algorithm was covered in class; recall
that it uses the following mathematical operations:
\begin{itemize}
\item Selection of a large prime number.
\item Exponentiation of a large number to a large exponent in modular arithmetic.
\end{itemize}
You are welcome to find and use a library providing this functionality. However,
these operations are simple enough that you may choose to implement them
yourself.

For this project, you should choose unsigned integer values less than $2^{32}$. Beware of
integer overflow: the square of a 32-bit number may require 64 bits to represent. If
implementing your own algorithms, you should make use of the
unsigned long long data type, or use a library for large numbers.

For efficient selection of a large prime, look up the ``Miller-Rabin primality test''
online.

For efficient large number exponentiation, look up ``Exponentiation by squaring''
online.

\subsection*{Network protocols}

Your Chatter system should use TCP for all communication between clients and a
server. The system requires several different types of data to be sent at different
points of execution:
\begin{itemize}
\item D-H parameter exchange data (in cleartext).
\item Data containing a human’s username and password (encrypted).
\item An authentication response from the server (encrypted).
\item Data containing a username and message sent by a human to all participants
in the system (encrypted).
\end{itemize}
Depending on how you implement connection establishment and connection
teardown between a Chatter client and a server, you may have additional protocol
messages not listed above.

You will need to develop the necessary protocols. We suggest creating a list of
different message types, where each type uses a unique identifier in the first byte of
the message. The recipient of data from the network can then read the first byte to
know how to interpret the remaining data.

\subsection*{How to proceed}

This project requires a fairly extensive implementation, so it is important to plan
your solution. I suggest dividing the work among the team members. For
example, one team member could work on the server and another could work on
the client. They could initially ignore the need for encryption simply to get network
communication operating properly. Meanwhile, the third team member could be
developing the necessary cryptography which can then be easily inserted into the
client and server code.

\subsection*{What to turn in}

Submissions will be done electronically on the T-Square “assignments” page. Each
group only needs to submit one solution, but it should list all group members.

Create a README file containing the following information:
\begin{itemize}
\item The names of all group members.
\item A description of what you were able to complete. If you believe that you have
a fully working solution, then this explanation will be brief. Otherwise, you
should explain what we will find working and what we will find broken or
missing when we begin testing.
\item A statement indicating what cryptographic algorithm you used to compute
password hashes.
\item A statement indicating what cryptographic algorithm you used for symmetric
encryption and decryption.
\item A list of the usernames and passwords for the users in your user database.
\item Exact, step-by-step instructions telling us how to build your Chatter server
and client on a College of Computing Red Hat Linux server.
\item Instructions telling us how to run your solution. Please give the exact
command-lines that we need to type.
\item A statement telling us which TCP port numbers are used by your system; or,
instructions stating how we can figure this out at runtime (in case the ports
are chosen differently for each execution). We need this information for part
of our testing: we may use network monitoring tools to verify that the
communication channels between clients and servers carry encrypted traffic.
\item Citations to websites stating where you found any non-standard libraries
used in your solution.
\end{itemize}
Create a zip file or tarball containing the following:
\begin{itemize}
\item All source code that you wrote.
\item All library code that is non-standard (e.g. you downloaded it from the
Internet). Without this code, we will be unable to build your utilities.
\item A compiled version of your server and client all ready for us to run
immediately.
\item A user database file containing usernames and password hashes.
\item The README file described above.
\end{itemize}
Post your zip file or tarball to T-Square. Only one group member needs to post the
solution, but they should list their team members as part of the submission.

\end{document}
