\documentclass[letterpaper]{article}
\usepackage{amsmath}
\usepackage{amssymb}
\usepackage{alltt}
%\usepackage{fullpage}
%\usepackage{times}

\title{}
\date{}

\begin{document}
\thispagestyle{empty}

\section*{CS 4235 / CS 8803IIS Homework 8}

\noindent {\bf Assigned:} 15 April 2011

\noindent {\bf Due:} 22 April 2011, 5:00pm Atlanta time. Students submitting solutions after that time but by 5:00pm Atlanta time on 25 April will have their scores scaled by 0.8. No solutions will be accepted after 5:00pm on 25 April.

\noindent {\bf Teaming:} Work individually.

\bigskip\noindent
Solutions should be typewritten and submitted as a PDF file on T-Square. Be sure to include your name and GTID number on your submission. Scores will be posted on T-Square.

\bigskip\noindent
Although you may use outside sources for information, you:
\begin{itemize}
\item {\bf must not} copy-and-paste text or figures from those sources, and
\item {\bf must} cite the sources. A citation should provide sufficient information for myself or anyone else to find the source that you used.
\end{itemize}
You do not need to cite the textbook or any course materials. If you are unsure whether or not you are using outside material appropriately, please ask me rather than guessing.

This homework has one written part worth 25 points. Please solve the following problems.

\subsection*{Written exercises}

\begin{enumerate}

\item From Chapter 6:
\begin{enumerate}
\item (10 points) \#10
\item (10 points) \#13
\end{enumerate}

\item (5 points) $UNDO$ is a recovery operation for databases. It is a command that obtains information from a transaction log and resets the data within a database to their values before a particular transaction was performed. That is, if $x$ is the original value of a database and $y$ is the incorrectly modified version, we want $UNDO(y)=x$.
\begin{enumerate}
\item This operation is {\em idempotent}. What does this mean?
\item What is the value of $UNDO(x)$?
\end{enumerate}

\end{enumerate}

\end{document}
