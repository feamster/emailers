\documentclass{article}

\begin{document}
\title{Research Statement}
\author{Nick Feamster}
\maketitle

My research focuses on helping people construct more robust distributed
systems.  My work has focused on today's interdomain routing protocol,
Border Gateway Protocol version 4 (BGP), but, more generally, my
research interests involve the analysis and design of large-scale
distributed systems.  

%%%%%%%%%%%%%%%%%%%%%%%%%%%%%%%%%%%%%%%%%%%%%%%%%%%%%%%%%%%%%%%%%%%%%%

\section*{Research Philosophy}

Rather than trying to apply a specific tool to solve various problems, I
search for interesting problems and try to discover (or devise) the
tools for solving those problems.  I'm most inspired by problems that
involve complex, poorly understood systems that affect a large user
community.  Internet routing is an example of such a system.  Internet
routing is essentially a large distributed program where individual
routers exchange reachability information and perform route selection.
The correct operation of Internet routing depends on thousands of
operators each correctly configuring the tens to hundreds of routers in
each of their networks; a misconfiguration, design mistake, or other
mishap can affect the ability of end users to successfully contact
Internet destination.

Working in problem domains with
challenging, underexplored problems has proved .  
Much previous work on Internet routing had characterized various
specific ills of the routing protocols.  Despite all of this previous
work, the large community of operators that actually configures the
deployed routing infrastructure was still experiencing the same trouble
configuring and managing their networks, and the research community had
reached a standstill concerning proposals for routing protocol design
aimed at remedying these various ills.  These problems created the
following opportunities: (1)~addressing the immediate needs of the
network operations community and (2)~a problem domain about which very
little was known at a fundamental level.  The first opportunity excited
me because it was an opportunity for synergy with a community of people
with domain-specific knowledge that could help fuel further research on
applied problems in Internet routing.  The second presented the
opportunity to expose the (as of then unknown) fundamental problems with
today's Internet routing architecture, as well as myriad possibilities
for creative, open-ended protocol design.

To advance the state of the art in interdomain routing, it is crucial to
understand which details matter and which have no bearing on the problem
at hand.  Unlike some more mature areas of computer networking research
(e.g., congestion control), the complexity of today's interdomain
routing infrastructure has, to this point, been poorly understood.  To
this end, much of my research has studied interdomain routing from the
"bottom up".  The models of BGP routing that I developed to help network
operators perform traffic engineering, which appeared at {\em ACM
SIGMETRICS} 2004, require intimate knowledge of how various details of
the protocol interact to affect BGP's route selection process.  We
implemented our model and validated it using the routing and
configuration data from a large tier-1 ISP.  This modeling exercise has
provided a better understanding of which aspects of BGP introduce
unncessary complexity and make the protocol so difficult to model.  More
recently, I developed a tool for detecting faults in BGP configuration,
"rcc", that not only helps network operators with immediate
configuration problems but also allows us to better understand which
aspects of BGP configuration make the protocol so error-prone.  rcc is
widely available, has been presented at the meeting of the North
American Network Operators Group (NANOG), and has been downloaded by
over sixty network operators.

Thinking about "top-down" solutions often requires understanding
protocol details from the bottom up.  Many top-down proposals for
radical changes to network architecture or protocols have no hope of
being deployed because they are not inspired a clear grasp of the
problems in today's architecture, let alone what is causing the
problems.  My work on a "routing logic" was the first attempt to
classify the specific problems that people have observed into a
definition, which recognizes that a routing protocol should, first and
foremost, satisfy two properties: {\em route validity} and {\em path
visibility}.  My "bottom up" work on route prediction for network
engineering and on rcc gave me insight into the aspects of BGP that
caused both route validity and path visibility problems.  This insight
helped me design a substantive "top down" change to the interdomain
routing architecture, the Routing Control Platform (RCP), that fixes
many of BGP's problems in a manner that is both backwards compatible
with today's routing architecture and compatible with an operator
incentives.


%%%%%%%%%%%%%%%%%%%%%%%%%%%%%%%%%%%%%%%%%%%%%%%%%%%%%%%%%%%%%%%%%%%%%%
% get into some details here

\section*{Future Research Areas}

\end{document}