
{\em Which aspects of BGP introduce unnecessary complexity into route
  computation?}  The answer to this question has important implications
  for protocol design, but determining it requires modeling: i.e.,
  understanding which aspects of BGP constitute irrelevant details and
  which are responsible for the problem.  Our analysis identified
  that a route attribute called the multiple exit discriminator (MED)
  and a scaling technique called ``route reflection'' were most
  responsible for introducing unnecessary complexity into BGP route
  selection (these artifacts are responsible for vagaries such as
  persistent oscillation, partitions, etc.).  Our model provides
  theoretical results on the complexity of BGP route computation with
  and without both route reflectors and MED.  Using these insights, we
  were able to propose protocol modifications that eliminate the
  negative side effects imposed by these two artifacts without limiting
  function.

Because interdomain routing facilitates the exchange of routing
information between cooperating, yet competing, ASes, an interdomain
routing protocol must support flexible policy expression.  For example,
the protocol must be able to allow an AS the flexibility to express
different rankings over multiple candidate routes to a destination
(``import policy''), as well as the ability to prevent a given route
from being readvertised to certain neighboring ASes (``export
policy'').  My current research involves developing a model of
interdomain routing that answers the following fundamental question
about policy-based routing:

{\em What are the fundamental tradeoffs between policy flexibility and
  guaranteed stability?}  Previous work has observed that BGP may not
  always converge to a stable route assignment.  Although there has been
  some work that explores how placing certain restrictions on import and
  export policies can guarantee stability, the tradeoff between policy
  flexibility and guaranteed stability is still poorly understood.
  Specifically, restrictions on export policy is (1)~not enforceable,
  since any AS can make a decision to export a route based on bilateral
  contracts and (2)~not verifiable, since any AS may violate such a
  restriction simply by advertising a route.  In light of this, my
  ongoing research is exploring how expressive each AS's {\em local}
  import policy (i.e., rankings over paths) can be, given no
  restrictions on export policy.
