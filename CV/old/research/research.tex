\documentclass[11pt]{article}

\textheight 9 in
\textwidth 6.5 in
\hoffset -1 in
\voffset -0.5 in

\newcommand{\ie}{{\em i.e.}}
\newcommand{\eg}{{\em e.g.}}


\begin{document}
\title{Research Statement}
\author{\Large Nick Feamster}
\date{}
\maketitle

I am interested in designing, building, analyzing, and measuring
networked systems that are composed of multiple autonomous, potentially
untrusted entities.  There are numerous examples of such systems: the
Internet routing infrastructure, 
public wireless networks, peer-to-peer systems, large-scale distributed
computing infrastructures (\eg, grid computing), Web services, content delivery
networks, and enterprise networks.

These systems present three challenges to system designers:
\begin{enumerate}
\itemsep=-4pt
\topsep=-5pt
\item Developing communication protocols that ensure correct,
  predictable, and robust 
operation;
\item Designing practically deployable techniques for secure
operation;
\item Supporting fault diagnosis, troubleshooting, and monitoring.
\end{enumerate}
Some fields of engineering have reached a level of maturity where there
are concrete design principles that ensure a level of correctness and
robustness: for example, it is reasonably well understood how to build
bridges and buildings that withstand earthquakes and high winds.
Engineers can apply such principles to help them design robust, reliable
systems.  Unfortunately, network system designers and engineers lack
such a rubric.

%My research applies measurement, modeling, and analysis to guide the

My research uses four techniques towards the ultimate goal of developing
sound methods for designing and implementing networked systems:
measurement; modeling; design and implementation; and deployment.
Measurement provides evidence and intuition for the severity
of a problem.  Modeling---the process of deriving a simpler
representation that abstracts irrelevant details and concisely describes
the aspects that affect the properties under study---facilitates a more
thorough understanding of a problem's fundamental causes.  Design and
implementation provide the opportunity to apply the intuition gained
from measurement and modeling to make tangible improvements to
real-world systems.  Deployment demonstrates the feasibility and
practicality of an implementation and provides the excitement of seeing
research ideas applied in practice.  In my future work, I intend to
apply these techniques to specific problems in routing, network
security, anonymous communication, anomaly detection, and monitoring in
resource-limited environments.

%% I use the
%% insights gained from measurement and modeling to design and build
%% large-scale networked systems and increase the robustness of existing
%% systems.  In my research, I have incorporated the models I developed
%% into tools that network operators use for daily network operations and
%% management.  I have also used models to design and implement systems
%% ranging from an adaptive, reliable video streaming protocol to a system
%% that circumvents Web censorship.

\section*{Dissertation Work: Robust, Predictable Internet Routing}

My dissertation solves some of the challenges raised above in the
context of Internet routing, which requires that competing, autonomous
networks (``autonomous systems'', or ASes) cooperate to establish global
connectivity.  I have designed and implemented models and tools that
make today's routing infrastructure more robust and easier 
to manage.  I have examined fundamental tradeoffs between
routing stability and expressiveness in generic policy-based routing
protocols.  Finally, I have proposed a new Internet routing architecture
that solves many of the problems we discovered (\ie, through measurement
and analysis) with today's routing infrastructure.

Routing configuration is essentially a complex distributed program.
Each AS independently configures local {\em policies} that control how
routers select and re-advertise routes. These policies implicitly codify
bilateral business relationships between ASes.  Each AS may contain tens
to hundreds of routers, each of which is individually configured with
hundreds to thousands of lines of code. The collection of configurations
within an AS determines whether the routing protocol operates correctly.
Faults in configuration can induce routing failures, such as forwarding
loops, partitions, and instability, that can prevent packets from
reaching their destinations.

%% Network operators need assurances that routing will operate
%% correctly. To maintain the network and achieve high performance, they
%% also need to be able to the outcome (\ie, route assignment) from a
%% particular configuration.  
%% \begin{enumerate}
%% \itemsep=-4pt
%% \item {\em Measurement.}  My investigation of the relationships between
%%   the failures of Internet paths and routing instability revealed that
%%   many Internet path failures result from routing instability, which
%%   motivates the importance of developing robust routing protocols.
%%   Studying traffic characteristics within a large tier-1 ISP allowed me
%%   to determine the the extent to which today's infrastructure can
%%   support network operations tasks (\eg, balancing traffic load).
%% \item {\em Modeling.} Sound system design first requires having a clear
%%   specification of ``correct'' behavior; my work on a {\em routing
%%   logic} provides a correctness specification for routing.  To better
%%   understand the relationships between policy expressiveness and routing
%%   stability, I derived practical constraints on routing policies that
%%   must be satisfied to guarantee routing stability.  To understand and
%%   isolate the sources of complexity in today's routing system, I 
%%   modeled the process by which an AS's routers select routes.
%% \item {\em Design and implementation.}  To make today's configuration
%%   process less likely to induce routing failures, I designed and
%%   implemented {\em rcc} (``router configuration checker''). Using the
%%   routing logic as a specification, {\em rcc} analyzes the set of router
%%   configurations from a single AS and detects faults that could result
%%   in violations of that specification.  The route selection model I
%%   developed provided insight that helped me (1)~implement a tool that
%%   allows a network operator to determine how traffic will flow through
%%   an AS and evaluate the benefits of possible policy changes; and
%%   (2)~design the Routing Control Platform (RCP), which makes routing
%%   much less prone to failures (\eg, loops and partitions).
%% \item {\em Deployment and evaluation.}  {\em rcc} has been downloaded by
%%   over sixty network operators and has successfully identified faults in
%%   the router configurations of several Internet service providers with
%%   nationwide backbone networks.  Studying real-world configurations with
%%   {\em rcc} has allowed me to investigate the nature and extent of
%%   configuration faults in today's networks.  We are currently
%%   investigating the feasibility of deploying RCP in a large backbone
%%   network.
%% \end{enumerate}

My research has applied measurement, modeling, design and
implementation, and deployment to help make today's Internet routing
infrastructure less prone to failure, as well as more predictable.
Network operators need assurances that today's routing protocols will
operate correctly, and they need to know which route each router will
select, given a set of configurations.  My work on a {\em routing logic}
defines a correctness specification for policy-based routing.  Based on
this specification, I developed {\em rcc} (``router configuration
checker''), a tool that analyzes the set of router configurations from a
single AS and detects configuration faults that could induce routing
failures.  {\em rcc} has been used by over sixty network operators and
has successfully identified faults in the configurations of several
Internet service providers with nationwide backbone networks.
Experience with {\em rcc}'s deployment in real-world networks has
provided a better understanding of the nature and extent of
configuration faults that occur in practice.  Additionally, my work on
modeling route selection led to a tool that makes routing more
predictable by helping network operators predict the effects of a
configuration change before deploying it.
%To improve predictability for network management
%tasks, I also modeled how routers within an AS select routes, and I
%implemented a tool that allows a network operator to compute offline the
%routes that each router will select.

With collaborators, I have also applied the above techniques to design
improvements to today's Internet routing protocols.  Ideally, Internet
routing should disseminate loop-free routes and converge to a stable
routing topology, regardless of how each AS configures its local
policies.  We have applied an 
abstract model of today's Internet routing protocol to derive
constraints on local policies that must be satisfied to guarantee that a
policy-based routing protocol will not oscillate.  To guarantee correct
dissemination of loop-free routes, we proposed the Routing Control
Platform (RCP), a system that computes routes on behalf of routers.  By
applying two design principles---(1)~compute consistent routes with
complete routing information and (2)~control interactions between
different routing protocols (\eg, between the inter-AS routing protocol
and an AS's internal routing protocol)---RCP explicitly prevents the
forwarding loops and oscillations that plague today's Internet routing
infrastructure. 

%% Measurement helps system designers understand Internet performance and
%% the extent to which certain vagaries appear in practice.
%% Modeling allows us to understand certain properties of a complex system
%% via a simpler representation that abstracts irrelevant details and
%% concisely describes the aspects that affect the properties under study.
%% I have used measurement and experimentation to demonstrate that routing
%% faults occur and cause observable failures in practice, as well as to
%% quantify the effects of these faults on end-to-end path performance.  I
%% have used modeling to understand the complexities of route
%% selection process and applied this model to implement a tool that allows an
%% operator to compute offline the routes that each router will select,
%% given the router configurations and a snapshot of the available routes.



\section*{Future Research Directions}

I intend to continue working on improving the robustness, security,
and diagnosis capabilities of large-scale systems in which potentially
untrusted entities must cooperate to provide some service.  

\subsection*{Robustness and Predictability}
\parindent=0pt
\parskip=6pt


{\bf Wireless mesh networks.}  While Internet routing is perhaps the
best studied example of a routing system that requires cooperation among
multiple untrusted parties, other domains, such as public wireless
``mesh'' networks, present interesting issues.  These networks are
composed of nodes that are typically owned by different parties (\eg,
homes, businesses) that must cooperate to provide connectivity.  Because
each of these entities may have vastly different criteria for ranking
preferred paths through the network and for carrying traffic over those
paths (\eg, minimizing loss rate, cost, etc.), these wireless networks
may benefit from using policy-based routing protocols.  I intend to
explore routing problems in public wireless networks to see what design
principles can tackle wireless-specific challenges (\eg, contention for
a shared channel, interference, mobility) to achieve robust and
predictable routing.

{\bf Routing stability.}  Policy-based routing protocols provide each
participant remarkable flexibility for implementing complex business
arrangements in local policy; unfortunately, the interactions between
these policies may conflict, resulting in instability.  The tradeoff
between routing policy flexibility and stability is poorly understood
today.  I would like to characterize the minimal set of constraints that
must be imposed on each participant's policies to guarantee global
stability. In the context of Internet routing, I would like to determine
whether those constraints are expressive enough to implement important
operational tasks (\eg, load balancing traffic).  My
experience using game theory and mechanism design to study routing
protocol oscillation, as well as my knowledge of network operations, should
prove useful for solving these problems.

\subsection*{Secure Networked Systems}

{\bf Data plane security.}  Previous work has studied routing protocol
security, but little attention has been paid to security and policy
enforcement in the {\em data plane} (\ie, the path that data packets
actually traverse).  Today's Internet architecture provides scant
support for a network to thwart unwanted packets (\eg, spam, viruses,
denial of service attacks) and essentially no control over the
sequence of ASes that outgoing traffic traverses en route to a
destination.  I plan to design architectural modifications that could
facilitate stronger security and policy enforcement capabilities in the
data plane.

{\bf Anonymous and censorship-resistant communication.}  Governments of
certain countries routinely implement firewalls to restrict communication
to various destinations.  To enable clients behind these firewalls to
access restricted Web content, we designed and implemented Infranet, an
anti-censorship system that embeds requests for blocked content in a
covert channel that appears to the censor as innocuous traffic to
permissible Web sites.  I would like to design
anti-censorship systems like Infranet that are robust to 
untrusted or malicious participants.  More generally, I intend to examine how
incorporating network-layer information can make anonymous communication
systems more resistant to eavesdropping attacks.


\subsection*{Fault Diagnosis and Monitoring}

{\bf Anomaly detection.}  Supporting both fault diagnosis and secure
operation in large-scale networked systems typically requires the
ability to collect, analyze, and audit large quantities of data.  I
intend to explore whether signal processing and clustering techniques
can be useful for performing forensic analysis of spam (\eg, determining
groups of machines that are being controlled by a single sender).  I
have also begun investigating whether signal processing-based anomaly
detection techniques such as principal component analysis can be useful
for detecting routing anomalies.


{\bf Monitoring in resource-limited environments.}  
I previously applied signal processing techniques to design and
implement a video transcoder that allowed video streams that were
originally encoded at very high bitrates to be transmitted over
relatively low-bandwidth wireless links in real-time.  I plan to
investigate how signal processing can reduce communication costs in other
bandwidth and resource-constrained environments.  For example, sensor
networks have strict bandwidth and energy budgets that often require
data streams with high data rates to be processed in the network.  I 
would like to design and implement distributed signal processing
algorithms that help reduce computation and communication overhead in
resource-constrained environments.

%% \begin{enumerate}
%% \itemsep=-1pt
%% \item anti-censorship stuff
%% \item intrinsic robustness of routing protocols
%% \item spam: forensics, architectures, etc.
%% \item architectural modifications/enhancements for data-plane security
%% \item incentives/game theoretic techniques for resolving policy disputes
%%   on a slower timescale
%% \item some stuff with measurement/characterization, and
%%   troubleshooting.  real-time anomaly detection in routing protocols
%%   (perhaps also data plane)
%% \item correctness/security in other routing protocols (wireless?)
%% \end{enumerate}


\end{document}




%% Additionally, given a set of
%% router configurations within an AS, it should be easy to determine the
%% outcome of the AS-wide routing computation.  Many artifacts of today's
%% interdomain routing protocol complicate this task, and my
%% research has focused on (1)~how to make today's routing infrastructure
%% operate in a more predictable and manageable fashion and (2)~how to
%% eliminate the artifacts of today's protocols without limiting useful
%% functions.

%% Researchers and practitioners have noted myriad shortcomings and
%% problems with BGP and proposed reactive point fixes to some of these
%% problems, often at the cost of increasing complexity, feature
%% interaction, and brittleness. In contrast, my work applies proactive
%% approaches for increasing protocol robustness by first understanding the
%% fundamental artifacts that induce these problems and implementing
%% improvements inspired by first principles.  Measurement and modeling
%% play important roles in informing these design changes.

%% Making substantive fixes to the interdomain routing architecture first
%% requires understanding, at a fundamental level, what problems are
%% specific to unfortunate design choices and, specifically, which aspects
%% of the design are inducing the problem.  Similarly, introducing
%% wholesale improvements to today's infrastructure requires a ``wish
%% list'' of features that today's protocol and architecture cannot
%% support, and understanding why it fundamentally cannot support those
%% features.  

%% I have also examined the extent and severity
%% of faults in real-world routing configurations and to quantify the effects
%% of configuration faults on end-to-end path performance.

%%My measurement and modeling work
%% has also shed light on protocol complexities and helped guide the design
%% of future routing protocols by isolating the causes of artifacts in
%% today's protocols.
