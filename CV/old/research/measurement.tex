{\em What are the extent of faults in real-world BGP configurations?  Do
  these configuration faults have the potential to induce routing
  failures?}  Conventional wisdom has long recognized the shortcomings
  of today's router configuration languages---they are ``primitive'',
  ``too low-level'', and so forth.  My research on the router
  configuration checker, {\bf rcc}, has allowed me to study the
  properties of real-world configurations from 17 different operational
  networks on the Internet and understand the extent to which various
  errors appear.  {\bf rcc} performs static analysis on a group of
  router configurations from a single AS and detects possible
  configuration faults.

  Our work on {\bf rcc} has resulted in several significant findings.
  First, we demonstrated that static analysis can be useful for
  detecting configuration faults that affect complex aspects of protocol
  operation that are otherwise difficult to uncover.  For example, by
  examining how the BGP session-level topology within an AS is
  configured, {\bf rcc} can determine whether or not BGP will correctly
  propagate external routes to all routers within the AS, regardless of
  the router that learns the external route (a failure to correctly
  propagate routes can create network partitions).  Second, our study
  indicated the types of configuration faults that are most prevalent in
  practice.  {\bf rcc}'s discovery that internal BGP (the protocol that
  disseminates BGP routes within an AS) has motivated the need for a
  better architecture for disseminating BGP routing information within
  an AS; we have incorporated this insight into our work on designing a
  new interdomain routing architecture.

{\em Can these types of configuration faults be observed dynamically in
  practice?}  A separate question from whether BGP configurations
  actually have faults is whether these faults can actually be observed
  in practice.  We performed a measurement study to study one particular
  configuration fault---the failure to advertise consistent routes
  across all peering points---from the perspective of a tier-1 Internet
  Service Provider (ISP).  Typically, two ISPs establish a bilateral
  ``settlement-free peering'' contracts with a provision that requires
  each ISP to advertise each route on all peering points (thus affording
  the neighboring AS complete flexibility over the peering points to
  which it can direct traffic).  Our study highlighted that, while most
  violations of this provision can be explained by transient protocol
  behavior, some ISPs consistently violate this condition.  This study
  provides further evidence that BGP configuration faults can in fact
  result in routing that does not comply with an ISP's high-level
  policy.

{\em How do routing failures and instability affect the end-to-end
  performance of Internet paths?}  Ultimately, end users and network
  operators are concerned with whether or not {\em data} packets
  ultimately reach their intended destinations along loop-free paths
  with low latencies and loss rates.  Thus, I believe it is important to
  understand the extent to which routing failures are reflected in
  end-to-end path performance.  Our work on correlating BGP routing
  instability with end-to-end path failures, as well as our subsequent
  work on quantifying the effects of routing instability on end-to-end
  path performance, extensively quantifies the relationships between
  ``control plane'' behavior and ``data plane'' performance.  We have
  developed techniques and algorithms to understand the control plane
  phenomena that result in packet loss and have discovered via empirical
  measurements that routing dynamics account for a significant fraction
  of all packet loss.
