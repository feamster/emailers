\documentclass[11pt]{article}

\textheight 9 in
\textwidth 6.5 in
\hoffset -1 in
\voffset -1 in

\newcommand{\ie}{{\em i.e.}}
\newcommand{\eg}{{\em e.g.}}


\begin{document}
\title{Teaching Statement}
\author{\Large Nick Feamster}
\date{}
\maketitle

%% The knowledge that we discover in our scientific pursuits becomes useful
%% only when it can be communicated clearly to others.  Teaching is a
%% cornerstone of science because it imparts knowledge to those who may
%% make more advances than we could have ever imagined or could ever have
%% made ourselves.  

I look forward to the responsibility
and privilege of teaching and mentoring students.
I believe that a teacher has two primary responsibilities: 

\begin{enumerate}
\itemsep=-4pt
\topsep=-5pt
\item Exciting students about problems and helping them discover their interests;
\item Providing students with the necessary resources to succeed in
  pursuing those interests.
\end{enumerate}

A good teacher uses a course not only to impart knowledge, but also to
instill excitement about the material and help students discover new
interests.  As a teaching assistant for MIT's graduate networking
course, I prepared lectures, recitations, and new questions for problem
sets and quizzes.  Lectures and recitations provide a unique opportunity
to present new problems to students and challenge them to think in new
ways.  A good lecture or recitation should engage students by relating
topics that they are not familiar with to those that they know well or
some topic that they can be excited about (\eg, motivating video
transcoding with a wireless digital television application, rather than
speaking exclusively in the abstract).  It should also
challenge students to think on the fly.  Finally, I believe that a
lecture should leave students with a clear intuition for the fundamental
high-level problems and intellectual ideas in some area (which they
could hopefully carry with 
them for years) but still allow a student who is interested
in delving into the topic further with enough information to pursue his
or her interests.

A good advisor helps students discover their interests by recognizing
their specific strengths and guiding them towards interesting open
problems that capitalize on those strengths.  I have found that an
effective way to excite students about a problem is to pique their
interest with a concrete problem, a couple of (sometimes small)
interesting results, and a handful of open questions to think about.  To
this end, I spend time thinking about interesting questions (\eg, Do
spammers steal addresses from other Internet service providers to
untraceably send spam?) and 
performing some initial exploration of these problems so that I can pose
problems to students in terms of concrete examples, rather than simply
describing a problem in the abstract.  I have found this
approach to be successful in helping guide the research of
Winston Wang, a Master's student who addressed some open problems I
posed in the Infranet project and ultimately received an award for his
thesis on the topic; and, more recently, of Mythili Vutukuru, a
first-year Ph.D. student who is
working on open issues related to the Routing Control Platform, based
on the design that we previously proposed.

An advisor should also develop his or her students' research tastes,
encouraging creativity while ensuring that students do not waste time on
irrelevant or uninteresting problems (\eg, problems that are too
short-sighted, emphasize implementation without a research goal, etc.).
One of the 
most valuable things that my advisor, Hari Balakrishnan, did for my
research was to help steer it clear of these types of problems in a
constructive way that encouraged me to refine my ideas; I intend to do
the same for my students.

Teaching also requires providing students with the necessary resources
to succeed once they have discovered their passions.  Perhaps the most
important resource an advisor can provide is the ability to draw
connections between different research areas that do not initially
appear to be related.  For example, my advisor recognized that
information theory could help me design the covert channels in the
Infranet project.  This type of insight requires familiarity with many
fields; I believe that my background in other fields, including game
theory and signal processing, will help me provide the same support to
my students.
%To encourage my students to develop a strong
%foundation of requisite background, I plan to hold informal meetings
%where my students (and myself) can discuss ongoing related research in
%their areas.  I think this exercise will encourage my students to think
%critically about their work and other work in their areas, and will also
%keep my students abreast of ongoing research in many areas of
%networking.  
%Additionally, an advisor should provide professional resources to his or
%her students by introducing them to other researchers in the field who
%might potentially provide additional mentoring and offer different
%perspectives and insight. 
%For example, my advisor gave me the
%opportunity to work with Jennifer Rexford.


One of the most important aspects of teaching undergraduates is
conveying excitement about the material.  I believe that one good way to
do this is to pose a problem in terms of a concrete example or application that
provides solid intuition.  My discrete math professor explained merge sort with
a Tower of Hanoi-style set of rings and combinatorics with card tricks.  I
believe that using concrete examples---whether they involve using cards to
explain combinatorics or video streams to explain the effects of packet loss on
streaming video quality---not only excites students about material, but also
provides them with intuition that they will remember long after their memories
of details have faded.  Coursework should involve problems and projects that
encourage students to both ``get their hands dirty'' and to develop creative
problem solving skills based on their newfound knowledge.  
%The proliferation of computer networks makes it feasible to develop simple
%network measurement experiments (\eg, using ubiquitous tools like {\tt
%traceroute} or even measuring properties of the campus wireless network) that
%a student could execute right from his or her dorm room computer; when
%teaching undergraduate networking, I intend to employ the tools and
%applications that students encounter everyday to 

Given my experience in Internet routing, Internet measurement and
modeling, and network security, I would also be excited to hold graduate
seminars on any of these topics.  Finally, I am very interested in
designing and teaching courses at both the undergraduate and graduate
levels that could be applied to interdisciplinary research that I intend
to pursue (\eg, applying game theory and signal processing to networks).
%.  I would welcome the
%opportunity to game theory and signal processing to undergraduates, as
%well as organizing graduate seminars that explore how these tools have
%been and could be applied to networking problems.
%
\end{document}

%% I have found that teaching often solidifies my own understanding of a
%% topic or problem; the process of continually digging deeper in an effort
%% to explain something to a student (or yourself) sometimes unexpectedly
%% results in new and interesting open research problems.  Each student
%% brings a different fresh perspective to any given problem area.  One of
%% the most rewarding aspects of
