\documentclass{article}
\usepackage{fullpage}
\usepackage{times}
\usepackage{bibunits}
\usepackage{url}
\usepackage[ManyBibs,NoDate]{currvita}




%\textwidth 6.7in
%\oddsidemargin -.11in
%\evensidemargin -.11in
\textwidth 6.6in
\oddsidemargin -.05in
\evensidemargin -.05in
\textheight 9.2in
\topmargin -.1in

\newcounter{bibctr}

\newcommand{\mkbib}[1]{
    \refstepcounter{bibctr}
    \item[\hskip0pt plus1filll \hbox{[\arabic{bibctr}]}]
  \begin{bibunit}[plain]
      \nocite{#1}
      \putbib[ref,rfc,talks]
    \label{bib:#1}
  \end{bibunit}
}
\newcommand{\refbib}[1]{[\ref{bib:#1}]}

\newcommand{\ie}{{\em i.e.}}
\newcommand{\eg}{{\em e.g.}}

\def\up#1{\raise.16ex\hbox{#1}}

\begin{document}
\begin{cv}{}

\noindent {\nf Nick Feamster}\\

\begin{tabular}{p{3.5in}p{4in}}
\noindent 
Assistant Professor & {\small\tt feamster@cc.gatech.edu} \\
College of Commputing & {\small\tt http://www.cc.gatech.edu/\~{}feamster/} \\
Georgia Institute of Technology  & \\
801 Atlantic Drive & \\
Atlanta, GA 30308-0280 \\
\end{tabular}

\begin{cvlist}{Education}
  \item{} {\sc Massachusetts Institute of Technology}
	\hfill {\em Cambridge, MA}
\item
\verb$    $ Ph.D. candidate in Computer Science.  September 2005. \\
\verb$    $ {\it Dissertation:} Proactive Techniques for Correct and Predictable Routing\\
\verb$    $ {\it Advisor:}  Hari Balakrishnan \\
\verb$    $ Minor in Game Theory

\verb$    $ M.Eng. in Computer Science, 2001\\
\verb$    $ {\it Thesis:} Adaptive Delivery of Real-Time Streaming
Video\\
\verb$    $ {\it Advisor:}  Hari Balakrishnan \\
\verb$    $ {\em William A. Martin Memorial Thesis Award} (MIT M.Eng. thesis award)

\verb$    $ S.B. in Electrical Engineering and Computer Science, 2000\\
\verb$    $ Concentration in Economics

\end{cvlist}

\begin{cvlist}{{Research Interests}}
%\item  Design, modeling, analysis, and measurement of network and
%  distributed systems protocols; network management and traffic
%  engineering; and security.
\item Networked computer systems: network protocols, routing, security,
  management, measurement, and censorship circumvention
\end{cvlist}


\begin{cvlist}{Employment History}
 \item [2005--] \textbf{Assistant Professor} \hfill Georgia Tech, {\em
 Atlanta, GA}

My research focuses on networked systems, particularly routing,
network security and management, measurement, and fault detection and
diagnosis. I am interested in designing, building, analyzing, and
measuring networked systems with autonomous (and potentially untrusted)
participants.

  \item[2000--2005] \textbf{Research Assistant} \hfill MIT, {\em Cambridge, MA}

Research assistant at the MIT Computer Science and Artificial
Intelligence Laboratory (CSAIL).  Projects include work on interdomain
routing robustness, circumventing Web censorship (Infranet), and the
Congestion Manager project.  More details are available on Page~5.

\item [2001--2005] \textbf {Technical Intern and Consultant} \hfill AT\&T
  Labs--Research, {\em Florham Park, NJ}

Research on interdomain traffic engineering and modeling.
More details are available on Page~5.
\\

\item [1999] \textbf {Technical Associate} \hfill Bell Laboratories,
  Lucent Technologies, {\em Murray Hill, NJ}

Designed and implemented a JavaBeans-based call filtering/disposition
system which allows end users to easily design a call flow based on
various criteria.

\item [1998--2000] \textbf {Intern} \hfill Hewlett-Packard Laboratories,
  {\em Palo Alto, CA}

Designed and implemented a transcoding algorithm
for real-time conversion of MPEG-2 to H.263 bitstreams.  More details are
available on Page~6.
%Patent application filed.

\item [1997] \textbf{Technical Staff} \hfill LookSmart Ltd., {\em
  San Francisco, CA}

Designed and implemented Web crawler, as well as monitoring and
testing scripts for production search engine system.

\end{cvlist}

\begin{cvlist}{{Teaching Experience}}
\item[2005] Instructor, Georgia Tech CS~7260, Internetworking
  Protocols and Architectures. (27 students) \\
Developed materials for graduate project-based course, including
publicly available lecture notes, problem sets, and programming
assignments. 

\item[2002] Teaching Assistant, MIT Course 6.829, Computer Networks.\\
  Contributed new problems to problem sets and quizzes, gave two lectures,
  and taught recitations covering advanced topics.

\item[2002--2003] M.Eng. research supervisor, MIT.\\
  With Hari Balakrishnan, supervised Winston Wang, whose thesis on an
  implementation of the Infranet anti-censorship system received
  MIT's Charles and Jennifer Johnson Thesis Prize. 

\end{cvlist}

\begin{cvlist}{{Refereed Publications}} 

\item

\hspace*{-1.1in}
{\small {\em Note:} Papers are listed in reverse chronological
order by topic area.} \\ 

\hspace*{-0.75in}{\it\large Internet Routing} 

\mkbib{Feamster2005:allerton}
\mkbib{Feamster2005b}
\mkbib{Feamster2004h}
\vspace*{-0.1in}
{\em Best paper award.}
\mkbib{Caesar2004}
\mkbib{Feamster2004e}
\mkbib{Feamster:fdna2004}
\mkbib{Feamster2004}
\mkbib{Feamster2003f}
\mkbib{Feamster2003e}
\mkbib{Feamster2003b}





\item
\hspace*{-0.75in}{\it\large Internet Measurement} 

\mkbib{Freedman2005}
\mkbib{Feamster2004f} % CCR submission
\mkbib{Feamster2004b}
\mkbib{Feamster2003}
\mkbib{Andersen02}


\item
\hspace*{-0.75in}{\it\large Network Security} 

\mkbib{Feamster2004d}
\mkbib{Feamster2003h}
\mkbib{Feamster2002}
\vspace*{-0.1in}
{\em Best student paper award.}
\mkbib{Fu2001} 
\vspace*{-0.1in}
{\em Best student paper award.}




\item
\hspace*{-0.75in}{\it\large Adaptive Streaming Media Protocols} 

\mkbib{Feamster02-pv}
\mkbib{Feamster01}
\mkbib{Wee99}
\mkbib{Feamster99}




\end{cvlist}

\begin{cvlist}{Submitted Publications and Works-in-progress}


\mkbib{Wang2004}


\mkbib{Feamster2004g} % ton submission



\mkbib{Monteleoni2004}

\end{cvlist}

\begin{cvlist}{{Unrefereed Papers and Technical Reports}}
\mkbib{Feamster01b}
\mkbib{Feamster2002b}
\mkbib{Feamster2003g}
\mkbib{Feamster2004c}
\mkbib{id-path-validation}

\end{cvlist}

\noindent
Invited talks at 
the 
North American Network Operators Group (NANOG),
Cooperative Association for Internet Data Analysis (CAIDA), 
Boston University,
Carnegie Mellon SDI seminar, 
New York University, 
Harvard University,
University Catholique de Louvain (Belgium),
AT\&T Research, 
Hewlett-Packard Laboratories, 
and 
Agilent Technologies.

%% \begin{cvlist}{Other Selected Talks}
%% \mkbib{whatif-nanog}
%% \mkbib{infranet:cmu}
%% \mkbib{infranet:hp}
%% \mkbib{infranet:ksg}
%% \mkbib{caida:leiden}
%% \mkbib{rolex-nanog}
%% \mkbib{rcc:nyu}
%% \mkbib{rcc:att}
%% \mkbib{rcc-nanog-noauth}
%% \mkbib{caida:imdc}
%% \mkbib{caida:e2e}
%% \mkbib{caida:anomaly}
%% \end{cvlist}

\begin{cvlist}{{Awards and Honors}}
\item[2005] Best Paper, 2nd Usenix Symposium on Networked Systems Design
  and Implementation
\item[2004] Cisco URP Grant Recipient
\item[2002--] NSF Graduate Research Fellow
\item[2002] Best Student Paper, 11th Usenix Security Symposium
\item[2001] Best Student Paper, 10th Usenix Security Symposium
\item[2001] MIT William A. Martin Memorial Thesis Award
\item[1999--] Tau Beta Pi Engineering Honor Society
\item[1999--] Eta Kappa Nu Honor Society
\item[1999] Letter of Commendation for Outstanding Performance, MIT Digital
  Design Laboratory
\item[1998--1999] Phi Sigma Kappa Scholarship Award
\item[1997] National Merit Scholar
\item[1997] Rotary Club Scholarship
\item[1996] AP Scholar with Distinction

\end{cvlist}

\begin{cvlist}{{Service and Other Activities}}
\item[2006] Program Committee Co-Chair, {\em Workshop on the Economics of Networked Systems (NetEcon)}
\item[2006] Program Committee, {\em CoNEXT}
\item[2006] Program Committee, {\em IEEE Symposium on Security and Privacy}
\item[2006] Program Committee, {\em IEEE Infocom Student Poster Session}

\item External reviewer for {\em IEEE/ACM Transactions on Networking}, 
{\em SIGCOMM} (2002, 2003, 2004), 
{\em SOSP} (2001, 2003), 
{\em Infocom} (2004, 2006),
{\em HotNets} (2003), 
{\em HotOS} (2001), 
{\em USENIX Security Symposium} (2002), 
{\em  ACM Computer Communication Review}, 
{\em IEEE Network Magazine},  
{\em IEEE Journal on Selected Areas in Communications},
{\em Image Communication} (EURASIP), 
{\em ASPLOS} (2004), 
{\em MobiSys} (2004), 
{\em USENIX} (2005, 2006),
{\em NSDI} (2005, 2006), 
{\em IPTPS} (2005).  \\  

\end{cvlist}


\section*{\df References}
\begin{tabular}{@{}l@{\qquad\qquad}l}
Prof. Hari Balakrishnan &              Prof. M. Frans Kaashoek\\
MIT Computer Science \& AI Lab &  MIT Computer Science \& AI Lab\\
32 Vassar Street, 32G-940 &      32 Vassar Street, 32G-992 \\
Cambridge, MA 02139&    Cambridge, MA 02139\\
(617) 253-8713 &                       (617) 253-7149\\                    
hari@csail.mit.edu &                     kaashoek@csail.mit.edu\\
\noalign{\vskip10pt}
Prof. Jennifer Rexford    &      Prof. Ramesh Johari \\
Princeton University &  Stanford University \\
Department of Computer Science  &    Department of Management Science
and Engineering\\
35 Olden Street, CS 306  &    380 Panama Mall \\
Princeton, NJ 08544 &   Stanford, CA 94305 \\
(609) 258-5182           &   (650) 723-0937 \\
jrex@cs.princeton.edu           &  ramesh.johari@stanford.edu \\
\noalign{\vskip10pt}
Prof. Lixin Gao\\
Department of Electrical and Computer Engineering \\
Knowles Engineering Building  \\
University of Massachusetts\\
Amherst, MA 01002 \\
(413) 545-4548 \\
lgao@ecs.umass.edu \\
\end{tabular}


\newpage
\parindent=0pt
\parskip=10pt
%\begin{cvlist}{Internet Routing}
{\large\df Internet Routing}

    The Internet is composed of more than 17,000 independently operated
    networks, or autonomous systems (ASes), that exchange routing
    information using the Border Gateway Protocol (BGP).  Network
    operators in each AS configure routers to control the routes that
    the routers learn, select, and propagate.  Configuring a network of
    BGP routers is like writing a distributed program where complex
    feature interactions occur both within one router and across
    multiple routers.  This complex process is exacerbated by the number
    of lines of code, by the absence of useful high-level primitives in
    today's router configuration languages, by the diversity in
    vendor-specific configuration languages, and by the number of ways
    in which similar high-level functionality can be expressed in a
    configuration language.  As a result, router configurations tend to
    have faults.  Faults in BGP configuration can cause forwarding
    loops, packet loss, and unintended paths between hosts.  Operators
    must be able to evaluate the effects of a configuration and be
    assured that the configuration is correct before deploying it.  My
    dissertation advances the state of the art in Internet routing by
    devising fault detection and modeling tools for today's Internet
    routing protocols and proposing a new Internet routing architecture
    that alleviates many of the problems we uncovered in our work on
    fault detection and modeling.

    %Traffic shifts,
    %equipment failures, planned maintenance, and topology changes in
    %other parts of the Internet can all degrade performance.  To
    %maintain good performance, network operators must continually
    %reconfigure the routing protocols.  
%%     My dissertation work improves the state of the art by: (1)~designing
%%     and implementing a static analysis tool to detect BGP configuration
%%     faults, (2)~designing algorithms to allow operators to predict the
%%     effects of configuration changes before the configuration is
%%     deployed and (3)~proposing a new interdomain routing architecture
%%     that makes both fault detection and modeling for network engineering
%%     much easier than they are today.


%  \item[2003--] 
{\mf Detecting Faults in BGP Configuration with Static Analysis} \hfill MIT
\vspace*{-0.1in}

    {\bf rcc}, the {\em router configuration
    checker}, detects faults in the BGP configurations of
    routers in an AS using static analysis. {\bf rcc} detects two broad
    classes of faults that affect network reachability: route validity
    faults, where routers may learn routes that do not correspond to
    usable paths, and path visibility faults, where routers may fail to
    learn routes for paths that exist in the network.  {\bf rcc} enables
    network operators to test and debug configurations before deploying
    them in an operational network, improving on the status quo where
    most faults are detected only during operation.  {\bf rcc} has been
    downloaded by more than sixty network operators to date.  
    I presented {\bf rcc} to the North American Network Operators Group
    (NANOG), and the tool has been used by several large backbone
    Internet Service Providers (ISPs)
    to successfully detect faults in deployed configurations.  This
    work was inspired by my work on the {\em routing logic} that I
    presented at the 2003 {\em ACM SIGCOMM Workshop on Future Directions
    in Network Architecture} and appears at the {\em 2nd
    USENIX Symposium on Networked Systems Design and Implementation}.
    We have also studied configuration faults as part of several
    measurement studies.  We presented an algorithm to detect route
    advertisements that violate peering contracts and an empirical study
    of their prevalence at the 2004 {\em ACM Internet Measurement
    Conference}. 


%  \item[2001--] 
{\mf Modeling Internet Routing for Network Engineering} \hfill MIT/AT\&T Labs--Research
\vspace*{-0.1in}

    Since interdomain route selection is distributed, indirectly
    controlled by configurable policies, and influenced by complex
    interactions with {\em intra}domain routing protocols, operators
    cannot predict how a particular BGP configuration would behave in
    practice.  We devised an algorithm that computes the outcome of the
    BGP route selection process for each router in a {\em single} AS,
    given only a static snapshot of the network state, without
    simulating BGP's complex dynamics.  Using data from a large ISP, I
    demonstrated that the algorithm correctly computes BGP routing
    decisions and has a running time that is efficient and accurate
    enough for many tasks, such as traffic engineering and capacity
    planning.  Studying the general properties and computational
    overhead of modeling the route selection process in each of these
    cases provides insight into the unnecessary complexity introduced
    by various aspects of today's interdomain routing architecture.  I
    used these insights to propose improvements to BGP that avert the
    negative side effects of various artifacts without limiting
    functionality.  This work appeared in {\em
    ACM SIGMETRICS} 2004 and has also been submitted to {\em IEEE/ACM
    Transactions on Networking}.

%  \item[2004--] 
{\mf Internet Routing Architecture: Routing Control Platform} \hfill MIT/AT\&T Labs--Research
\vspace*{-0.1in}

   The limitations in today's routing system arise in large part from
   the fully distributed path-selection computation that the IP routers
   in an AS must perform.  We proposed that interdomain routing should be
   separated from today's IP routers, which should simply forward packets (for
   the most part).  Instead, a separate {\em Routing Control Platform (RCP)}
   should select routes on behalf of the IP routers in each AS and exchange
   reachability information with other domains.  RCP could both select routes
   for each router in a domain (\eg, an AS) and exchange routing information
   with RCPs in other domains.  By selecting routes on behalf of {\em all\/}
   routers in a domain, RCP can avoid many internal BGP-related complications
   that plague today's mechanisms for disseminating and computing routes within
   an AS.  RCP facilitates traffic engineering, simpler and less error-prone
   policy expression, more powerful diagnosis and troubleshooting, more rapid
   deployment of protocol modifications and features, enforceable consistency
   of routes, and verifiable correctness properties.  The architectural
   proposal for RCP appeared at the 2004 {\em ACM SIGCOMM Workshop on Future
   Directions in Network Architecture}; the design and implementation of an RCP
   prototype won the best paper award at the {\em 2nd USENIX Symposium
   on Networked Systems Design and Implementation (NSDI)}.

%\end{cvlist}

\pagebreak
%\begin{cvlist}{Internet Measurement}
{\df Internet Measurement}

%  \item[2002--] 
{\mf Understanding End-to-End Internet Path Failures} \hfill MIT
\vspace*{-0.1in}

    Empirical evidence suggests that reactive routing systems, which
    detect and route around faulty paths based on measurements of path
    performance, improve resilience to Internet path failures.  We
    studied {\em why} and under {\em what 
    circumstances} these techniques are effective by
    correlating end-to-end active probes, loss-triggered traceroutes of
    Internet paths, and BGP routing messages.  This work was the first known
    study to correlate routing instability with degradations in {\em
    end-to-end} reachability. We found that most
    path failures last less than fifteen minutes.  Failures that appear
    in the network core correlate better with BGP instability than
    failures that appear close to end hosts.  Surprisingly, there is
    often increased BGP traffic both before and after failures.  Our
    findings suggest that reactive routing is most effective between
    hosts that have multiple connections to the Internet and that
    reactive routing systems could pre-emptively mask about 20\% of
    impending failures by using BGP routing messages to
    predict these failures before they occur.  This work appeared at {\em ACM
    SIGMETRICS} 2003.

    End-to-end path failures are typically attributed to either
    congestion or routing dynamics. Unfortunately, the extent to which
    congestion and routing dynamics cause end-to-end failures, and the
    effect of routing dynamics on end-to-end performance, are poorly
    understood.  In a follow-up study, we used similar techniques to
    find that routing dynamics contribute significantly to end-to-end
    failures and, in particular, routing dynamics are responsible for
    most long-lasting path failures.  The study also finds that
    long-lived end-to-end path failures that involve routing dynamics
    are typically caused by BGP convergence or instability.  This work is
    the first to quantify the impact of routing dynamics on end-to-end
    path availability; it was submitted to {\em ACM SIGMETRICS} 2005.


%\end{cvlist}

%\begin{cvlist}{ Network Security}
{\df Network Security}

%  \item[2001--2003] 
{\mf Infranet: Circumventing Web Censorship} \hfill MIT
\vspace*{-0.1in}
    
    An increasing number of countries and companies routinely block or
    monitor access to parts of the Internet.  To counteract these
    measures, we designed and implemented {\em Infranet}, a system that
    enables clients to surreptitiously retrieve sensitive content via
    cooperating Web servers distributed across the global Internet.
    These Infranet servers provide clients access to censored sites
    while continuing to host normal uncensored content.  Infranet uses a
    tunnel protocol that provides a covert communication channel between
    its clients and servers, modulated over standard HTTP transactions
    that resemble innocuous Web browsing.  In the upstream direction,
    Infranet clients send covert messages to Infranet servers by
    associating meaning to the {\em sequence} of HTTP requests being
    made.  In the downstream direction, Infranet servers return content
    by hiding censored data in uncensored images using steganographic
    techniques.  This work appeared at the {\em 11th USENIX Security
    Symposium}. 

%\end{cvlist}

%\begin{cvlist}
{\df Adaptive Streaming Media Protocols}

%  \item[2000--2001] 
{\mf Reliable, Adaptive Video Streaming} \hfill MIT
\vspace*{-0.1in}

    Video compression exploits redundancy between frames to achieve
    higher compression, but packet loss can be detrimental to
    compressed video with interdependent frames because errors
    potentially propagate across many frames.  In my Master's thesis, I
    quantified the effects of packet loss on the quality of MPEG-4 video,
    developed an analytical model to explain these effects, and
    presented an RTP-compatible protocol, called {\em SR-RTP}, that {\em
    adaptively} delivers higher quality video in the face of packet
    loss.  This work appeared at the {\em 12th International Packet
    Video Workshop} and was later implemented as part of a
    streaming video server for MIT Project Oxygen.
%    The Internet's variable bandwidth and delay make it difficult to
%    achieve high utilization, TCP-friendliness, and a high-quality
%    constant playout rate.  Traditional congestion avoidance
%    schemes such as TCP's ad\-ditive-increase/\-multiplicative-decrease
%    (AIMD) induce variable transmission rates that degrade
%    the perceptual quality of the video stream.  
    We also designed a scheme for performing quality adaptation of
    layered video for a general family of congestion control algorithms
    called {\em binomial congestion control}.
% and showed that a
%    combination of smooth congestion control and receiver-buffered
%    quality adaptation reduces oscillations, increases interactivity,
%    and delivers higher quality video for a given amount of buffering.
    This work appeared at the {\em 11th International Packet Video
    Workshop}.


%  \item[1999--2000] 
{\mf Video Transcoding} \hfill Hewlett-Packard Laboratories
\vspace*{-0.1in}

    We designed and implemented an algorithm that transcoded MPEG video
    input to a lower-bitrate H.263 progressive bitstream, facilitating
    the transmission of a digital television signal over a wireless
    medium.  This algorithm was the first
    to use both spatial and temporal downsampling in an MPEG-2 to H.263
    field to frame transcoder to achieve substantial bitrate reduction.
    The proposed algorithm exploits the properties of the MPEG-2 and
    H.263 compression standards to perform interlaced to progressive
    (field to frame) conversion with spatial downsampling and frame-rate
    reduction in a CPU and memory efficient manner, while
    minimizing picture quality degradation.  This work appeared at the
    {\em IEEE International Conference on Image Processing} in 1999.


%\end{cvlist}

%References available upon request.

\end{cv}

\end{document}
