\documentclass{article}
\usepackage{fullpage}
\usepackage{times}
\usepackage{bibunits}
\usepackage{url}
\usepackage[ManyBibs,NoDate]{currvita}




%\textwidth 6.7in
%\oddsidemargin -.11in
%\evensidemargin -.11in
\textwidth 6.6in
\oddsidemargin -.05in
\evensidemargin -.05in
\textheight 9.2in
\topmargin -.1in

\newcounter{bibctr}

\newcommand{\mkbib}[1]{
    \refstepcounter{bibctr}
    \item[\hskip0pt plus1filll \hbox{[\arabic{bibctr}]}]
  \begin{bibunit}[plain]
      \nocite{#1}
      \putbib[ref,rfc,talks]
    \label{bib:#1}
  \end{bibunit}
}

\newcommand{\mkbiba}[2]{
    \refstepcounter{bibctr}
    \item[\hskip0pt plus1filll \hbox{[\arabic{bibctr}]}]
  \begin{bibunit}[plain]
      \nocite{#1}
      \putbib[ref,rfc,talks]
    \label{bib:#1} \vspace{-0.05in}
    {\em Acceptance rate: #2\%}
  \end{bibunit}
}


\newcommand{\refbib}[1]{[\ref{bib:#1}]}

\newcommand{\ie}{{\em i.e.}}
\newcommand{\eg}{{\em e.g.}}

\def\up#1{\raise.16ex\hbox{#1}}

\begin{document}
\begin{cv}{}

\noindent {\nf Nick Feamster}\\

\begin{tabular}{p{3.25in}p{4in}}
\noindent 
Professor & \url{feamster@cs.princeton.edu} \\
Department of Computer Science & \url{http://www.cs.princeton.edu/~feamster/} \\
Princeton University  & \url{http://connectionmanagement.org/}\\
35 Olden Street &  \url{http://noise-lab.net/}\\
Princeton, NJ 08540 \\
\end{tabular}

\section*{Education}

\begin{center}
\begin{tabular}{p{.5in}lp{2.5in}r}

{\bf Degree} & {\bf Year} & {\bf University} & {\bf Field} \\
\hline \\
{\bf Ph.D.} & 2005 & Massachusetts Institute of Technology & Computer Science
\\
& & Cambridge, MA \\

&& \multicolumn{2}{l}{{\it Dissertation:} Proactive Techniques for
Correct and Predictable Internet Routing}\\ 
&& \multicolumn{2}{l}{{\em Sprowls Honorable Mention for best MIT
Ph.D. dissertation in Computer Science}} \\
&& {\it Advisor:}  Hari Balakrishnan \\
&& Minor in Game Theory \\
\\


{\bf M.Eng.} & 2001 & Massachusetts Institute of Technology & Computer Science
\\
& & Cambridge, MA \\
&& \multicolumn{2}{l}{{\it Dissertation:} Adaptive Delivery of Real-Time
Streaming Video}\\ 
&& {\it Advisor:}  Hari Balakrishnan \\
&& \multicolumn{2}{l}{{\em William A. Martin Memorial Thesis Award} (MIT
M.Eng. thesis award)} \\
\\


{\bf S.B.} & 2000 & Massachusetts Institute of Technology & Electrical
Engineering and Computer Science \\
& & Cambridge, MA \\
\end{tabular}
\end{center}

\section*{Research Interests}
\addcontentsline{toc}{section}{CURRENT FIELDS OF INTEREST}
\label{sec:current}

My research focuses on networked computer systems, with a strong
emphasis on network architecture and protocol design; network security,
management, and measurement; routing; and anti-censorship techniques.
The primary goal of my research is to help network operators run their
networks better, and to enable users of these networks (both public and
private) to experience high availability and good end-to-end
performance.  I have a strong interest in tackling practical problems
using a ``first principles'' approach, designing systems
based on these principles, and implementing and deploying these systems
in practice.  


\section*{Employment History}

\begin{center}
\begin{tabular}{p{2in}p{3in}r}
\textbf{Title} & \textbf{Organization} & \textbf{Years} \\ \hline
{\bf Professor} & Princeton University & {\em
January 2015--Present} \\
{\bf Professor} & Georgia Institute of Technology & {\em
March 2014--December 2014} \\
{\bf Associate Professor} & Georgia Institute of Technology & {\em
March 2011--March 2014} \\
%{\bf Associate Professor} & University of Maryland College Park & {\em
%August 2012--December 2012} \\

{\bf Assistant Professor} & Georgia Institute of Technology & {\em
2006--2011} \\

{\bf Postdoctoral Research Staff} & Princeton University & {\em Fall 2005} \\

{\bf Research Assistant} & Massachusetts Institute of Technology & {\em
2000--2005} \\

{\bf Intern/Consultant} & AT\&T Labs--Research & {\em 2001--2005} \\
{\bf Technical Associate} & Bell Laboratories & {\em 1999} \\
{\bf Intern} & Hewlett-Packard Laboratories & {\em 1999} \\
{\bf Technical Staff} & LookSmart, Ltd. & {\em 1997} \\

\end{tabular}
\end{center}

\newpage
\section*{Honors and Awards}
\label{subsec:mylabel4}




\subsection*{National Awards}

\begin{itemize}
\itemsep=-1pt
\item NSF Presidential Early Career Award for Scientists and Engineers (PECASE) 
\item Technology Review Top Innovators Under 35
\item Alfred P. Sloan Fellowship
\item NSF CAREER Award
\item IBM Faculty Award
\end{itemize}

\subsection*{University Awards}
\begin{itemize}
\itemsep=-1pt
\item John P. Imlay Distinguished Lecture, Georgia Tech 
\item Hesburgh Teaching Fellow
\item Georgia Tech College of Computing Outstanding Junior Faculty Research Award
\item Georgia Tech Sigma Xi Young Faculty Award
\item Georgia Tech Sigma Xi Best Undergraduate Research Advisor
\end{itemize}


\subsection*{Paper and Publication Awards}
\begin{itemize}
\itemsep=-1pt
\item ACM SIGCOMM Community Award, ACM SIGCOMM Internet Measurement Conference
\item Best Paper, Passive and Active Measurement ConferenceBest Student Paper, ACM SIGCOMM 
\item Best Paper, USENIX Symposium on Networked Systems Design and Implementation 
\item Best Student Paper, 11th USENIX Security Symposium
\item Best Student Paper, 10th USENIX Security Symposium
\item PRSA Bronze Anvil Award for Wall Street Journal Editorial Article
\end{itemize}

\subsection*{Thesis Awards}
\begin{itemize}
\itemsep=-1pt
\item George M. Sprowls honorable mention for best Ph.D. thesis in computer science, MIT 
\item MIT William A. Martin Memorial Thesis Award for Best EECS Master’s Thesis
\end{itemize}

\subsection*{Other Honors}

\begin{itemize}
\itemsep=-1pt
\item Panelist for NSF/Discover Magazine Special Issue on “The New Internet”
\item U.S. National Academy of Engineering Frontiers of Engineering Symposium
\item U.S. National Academy of Science Kavli Frontiers of Science
  Symposium
\end{itemize}

\newpage
\section*{Publications}
\label{sec:research}

\subsection*{Theses}
\label{subsec:thesisch}

%\begin{cvlist}{}
\begin{pub}
\mkbib{Feamster2005:phd}
\mkbib{feamster:meng}
\end{pub}
%\end{cvlist}


%%%%%%%%%%%%%%%%%%%%%%%%%%%%%%%%%%%%%%%%%%%%%%%%%%%%%%%%%%%%
% Journal

\subsection*{Journal Publications}
\label{subsec:journals}

\begin{pub}
\mkbib{feamster2014:ccr}
\mkbib{burnett2013:censorship}
\mkbib{kim2013:improving}
\mkbib{Tariq2010:wise}
\mkbib{Sundaresan2012:broadband}
\mkbib{sundaresan2012:accelerating}
\mkbib{Chetty2012:sdn}
\mkbib{Motiwala2012:cost}
\mkbib{Feamster2011:neted}
\mkbib{Yu2011:vlan}
\mkbib{Koponen2011:fii}
\mkbib{Calvert2011:home}
\mkbib{Anwer2010:netfpga}
\mkbib{Feamster2006:policy}
\mkbib{Feamster2006}% ToN paper
\mkbib{Feamster2004f} % CCR submission
\mkbib{Feamster2003e}
\end{pub}

\subsection*{Books and Book Chapters}

\begin{pub}
\mkbib{Lee2008:botnet}
\end{pub}

%%%%%%%%%%%%%%%%%%%%%%%%%%%%%%%%%%%%%%%%%%%%%%%%%%%%%%%%%%%%
% Conference

\subsection*{Conference Publications}


% \hspace*{-0.75in}{\it\large Internet Routing} 

\begin{pub}
\mkbiba{kim2015:nsdi}{19}
\mkbiba{chetty2015:chi}{23}
\mkbiba{sundaresan2015:pam}{27}
\mkbib{chavula2014:africomm}
\mkbiba{narain2014:ccs}{19}
\mkbiba{jones2014:imc}{22}
\mkbiba{gupta2014:sigcomm}{19}
\mkbiba{sundaresan2014:atc}{15}

\mkbiba{gupta2014:pam}{31}
\mkbiba{xing2014:pam}{31}

\mkbiba{chetty2013:dev}{33}
\mkbiba{mundada2013:silverline}{19}

\mkbiba{Hao2013:dns}{23}
\mkbiba{Grover2013:nat}{23}
\mkbiba{Sundaresan2013:webperf}{23}

\mkbiba{Xing2013:poison}{16}
\mkbiba{Valancius2013:pecan}{11}

\mkbiba{Lumezanu2012:spam}{24}
\mkbiba{Katz2012:lifeguard}{14}
\mkbiba{Lumezanu2012:bias}{20}
\mkbiba{Konte2012:pam}{30}

\mkbiba{Kim2011:config}{19}
\mkbiba{Hao2011:dns}{19}
\mkbiba{Valancius2011:tiers}{14}
\mkbiba{Sundaresan2011:bismark}{14}
\mkbib{Ramachandran2011:ceas}

\mkbiba{Anwer2010:switchblade}{12}
\mkbiba{Burnett2010:collage}{15}
\mkbiba{Antonakakis2010:notos}{15}
\mkbiba{Valancius2010:bgpmux}{17}
\mkbiba{Perdisci2010:http}{16}

\mkbiba{Tariq2009:nano}{17}
\mkbiba{Tariq2009:vlan}{22}
\mkbiba{Cunha2009:tomo}{22}
\mkbiba{Hao2009:snare}{15}

\mkbiba{Konte2008:pam}{20}
\vspace*{-0.05in}
{\bf Best paper award.}
%
\mkbiba{Ramachandran2008:flexsample}{17}
%
\mkbiba{Motiwala2008:splicing}{12}
\mkbiba{Tariq2008:wise}{12}
\mkbiba{Andersen2008:aip}{12}

\mkbib{Feamster2008:sigcse}
\mkbiba{Ramachandran2007:spamtracker}{24}
\mkbiba{Khadilkar2007:dhcp}{24}
\mkbiba{Huang2006:bgp}{17}
\mkbiba{Wang2006:bgp}{40}
\mkbiba{Lee2007:mobcast}{47}
\mkbiba{Feamster2006:spam}{12}
\vspace*{-0.05in}
{\bf Best student paper award.}

\mkbiba{Bavier2006}{12}


\mkbib{Feamster2005:allerton}
\mkbiba{Feamster2005b}{11}
\mkbiba{Freedman2005}{24}
\mkbiba{Feamster2004h}{22}
\vspace*{-0.05in}
{\bf Best paper award.}
\mkbiba{Caesar2004}{22}
\mkbiba{Feamster2004b}{25}
\mkbiba{Feamster2004}{12}
\mkbiba{Feamster2003}{12}


\mkbiba{Feamster2002}{17}
\vspace*{-0.05in}
{\bf Best student paper award.}


\mkbiba{Fu2001}{28}
\vspace*{-0.05in}
{\bf Best student paper award.}

%\hspace*{-0.75in}{\it\large Adaptive Streaming Media Protocols} 

\mkbiba{Wee99}{45}
\mkbib{Feamster99}
\end{pub}

%%%%%%%%%%%%%%%%%%%%%%%%%%%%%%%%%%%%%%%%%%%%%%%%%%%%%%%%%%%%
% Workshops


\subsection*{Workshop Publications}

\begin{pub}

\mkbib{schinkler2014:hotnets}
\mkbib{donovan2014:hotnets}
\mkbib{jones2014:foci}

\mkbib{feamster2013:isoc-latency}
\mkbib{chetty2013:isoc-latency}

\mkbiba{anwer2013:slick}{23}
\mkbiba{feamster2013:sdx}{23}

\mkbib{voellmy2012:procera}
\mkbib{martin2012:prio}

\mkbib{Sundaresan2011:labels}
\mkbib{Feamster2011:neted-workshop}
\mkbib{Mundada2011:silverline}

\mkbib{Anwer2010:visa}
\mkbib{Feamster2010:homenets}
\mkbib{Calvert2010:homenets}

\mkbib{Anwer2009:visa}
\mkbib{Nayak2009:resonance}

\mkbib{Bhatia2008:roads}
\mkbib{Valancius2008:mint}
%
\mkbiba{Tariq2008:nano}{20}
%
\mkbib{Yardi2008:wosn}
\mkbib{Ramachandran2008:wosn}

\mkbiba{Motiwala2007:splicing}{18}
\mkbiba{Andersen2006:aip}{18}
\mkbib{Ramachandran2007:bitstore}

\mkbib{Ramachandran2006:recon}
\mkbib{Ramachandran2006:dnsbl}

\mkbib{Feamster2004e}
\mkbib{Feamster:fdna2004}
\mkbib{Feamster2004d}

\mkbib{Feamster2003h}
\mkbib{Feamster2003f}
\mkbib{Feamster2003b}

\mkbiba{Andersen02}{42}
\mkbib{Feamster02-pv}

\mkbib{Feamster01}

\end{pub}


\section*{Software}
\label{subsubsec:softwaredl}

My research group regularly releases software and makes a practice of
releasing source code with most published papers.  As a supplement to
the descriptions below, my research group's Github page is available at:
\url{http://github.com/gtnoise/}.

\subsection*{Ongoing and Maintained Software}

\begin{itemize}
\item {\em Project BISmark: An Application Platform for Home Networks.}
  Project BISmark (Broadband Internet Service Benchmark) is a platform
  for developing network management applications for home networks.  The
  BISmark firmware is based on OpenWrt, an open-source operating system
  for home routers.  Currently, BISmark includes a suite of passive and
  active network measurements that allows a home Internet user to
  continuously monitor various performance metrics, such as upstream and
  downstream throughput, latency, and packet loss.  As of Spring 2013,
  BISmark is deployed in nearly 300 homes around the world in more than
  20 countries.  We are currently working both to expand the deployment
  and to extend the capabilities of the platform, to allow other
  researchers to use the platform for their own measurements.  \\ See
  \url{http://projectbismark.net} for details.

\item {\em MySpeedTest: A Tool for Mobile Performance Measurement.}
  Building on the success of BISmark, my students Sachit Muckaden,
  Abhishek Jain, and I have developed a tool to measure performance from
  mobile cellular handsets.  The application collects a variety of data,
  including latency and throughput measurements to a variety of servers
  around the world, hosted by Measurement Lab.  The application is now
  deployed on more than 4,000 handsets in over 130 countries.  Some of
  the most significant deployments are in developing countries, and we
  are currently collaborating with ResearchICTAfrica, a policy
  organization in Africa, to study the performance of both fixed and
  mobile broadband across the continent.  Software is available at
  \url{http://goo.gl/28tx3}.

\item {\em Bobble: Bursting Online Filter Bubbles.} With students Xinyu
  Xing, Dan Doozan and colleaguge Wenke Lee, I have designed and
  developed Bobble, a Chrome extension that allows users to see how
  their Web searches appear from different vantage points.  The filter
  bubble is a concept developed by Internet activist Eli Pariser in his
  book to describe a phenomenon in which websites use algorithms to
  predict what information a user may like to see based on the user's
  location, search history, etc. As a result, a website may only show
  information which agrees with the user's past viewpoints. A typical
  example is Google's personalized search results. To "pop" the bubbles
  created by Google search (also called de-personalization), our
  research group in the Georgia Tech Information Security Center is
  conducting ground-breaking research and developing software, Filter
  Bubble. Filter Bubble is a chrome extension that uses hundreds of
  nodes to distribute a user's Google search queries world wide each
  time the user performs a Google search. Using Filter Bubble, a user
  can easily see differences between his and others' Google search
  returns.  The plugin has been installed by more than 100 users around
  the world and is available at \url{http://bobble.gtisc.gatech.edu/}.

\item {\em Appu: Measuring Online Privacy Footprints.}  With so many web
  applications and sites in the current time, it's hard for a user to
  keep track of where does her personal information reside. With Appu,
  we aim to make this job easier for the end user. Appu keeps track of
  personal information such as passwords, username, birthdate, address,
  credit card numbers, and social security number so that a user can
  find out all sites that store a particular bit of personal
  information.  In the current beta release, Appu downloads personal
  information from sites where you have account and also tries to
  prevent password reuse across websites by warning users about
  it. Software is available at \url{http://appu.gtnoise.net/}

\item {\em Transit Portal.} We developed software for the
  NSF-Sponsored GENI Project Office that (1) adds facilities
  and functions to the VINI testbed to enable experiments to carry
  traffic from real users; and (2) increases the experimental use of the
  VINI testbed by providing a familiar experiment management
  facility. The deliverables for this project all comprise software for
  supporting external connectivity and flexible, facile experimentation
  on the GENI testbed. The primary deliverables are a BGP session
  multiplexer---a router based on the Quagga software routing suite,
  software support for virtual tunnel and node creation, and integration
  of the above functionality with clearinghouse services developed as
  part of the ProtoGENI project.  \\ See
  \url{http://tp.gtnoise.net/}.

  This project contributes to GENI design and prototyping through BGP
  mux development integration with ISPs; tunnel and topology
  establishment and management; ProtoGENI clearinghouse integration; and
  support for isolation and resource swapout.  With researchers at
  Princeton, we have also built VINI, a large distributed testbed for
  specifying virtual network topologies and experimenting with routing
  protocols and architectures in a controlled, realistic emulation
  environment.  See \url{http://vini-veritas.net/} for details.
%

\item {\em Campus-Wide OpenFlow Deployment: Access and Information Flow
  Control for Enterprise Networks.} Resonance is a system for
  controlling access and information flow in an enterprise network.
  Network operators currently use access control systems that are
  coarse-grained (i.e., it is difficult to apply specialized policy to
  individual users) and static (i.e., it is difficult to quickly change
  the extent of a user's access).  Towards fixing these problems, we
  have developed a system that allows network operators to program
  network policy using a controller that is distinct from the switch
  itself and can be programmed to implement network-wide policy.  We
  have implemented and deployed this system in an operational network
  that spans two buildings on the Georgia Tech campus; the network sees
  regular use, and a deployment in Georgia Tech dormitories or the
  wireless network is planned for the near future.  We first
  demonstrated the function of this network at the 7th GENI Engineering
  Conference in March 2010, and recently demonstrated a version on
  Resonance that facilitates various home network management tasks at
  the 2011 Open Network Summit. See \url
  {http://groups.geni.net/geni/wiki/BGPMux} for details.
%
\item {\em rcc: router configuration checker}.  Static configuration analysis
tool for Border Gateway Protocol (BGP) routing configurations.
Downloaded by over 100 network operators and many large, nationwide
backbone ISPs around the world.  See \url{http://gtnoise.net/rcc/} for
details. 
%

\end{itemize}

\subsection*{Older Projects}

\begin{itemize}

\item {\em NANO: Network Access Neutrality Observatory.} The Network
  Access Neutrality Observatory (NANO) is a system to help users
  determine whether their traffic is being discriminated against by an
  access ISP.  In contrast to existing systems for detecting network
  neutrality violations, NANO makes no assumptions about the mechanism
  for discrimination or the services that the ISP might be discriminated
  against.  NANO has been released in collaboration with Google as part
  of the Measurement Lab project.  A preliminary version of the software
  was released to a small group of users in March 2009 for testing; a
  complete release is available for download at:
  \url{http://gtnoise.net/nano/}.  
%
\item {\em Infranet}.  System for circumventing Web censorship firewalls
  (\eg, those in China, Saudia Arabia, etc.).  Available on
  Sourceforge.  Featured in articles in {\em Technology Review}, {\em
  New Scientist}, and {\em Slashdot}.  See
  \url{http://nms.lcs.mit.edu/projects/infranet/}.
\item {\em The Datapository}.  Originally the ``MIT BGP Monitor'', the
  Datapository is growing to support multiple data feeds (\eg, spam,
  end-to-end measurement probes, traceroutes, Abilene data, etc.).
  Currently used by researchers at Georgia Tech, Carnegie Mellon,
  University of Michigan, Princeton, MIT.  See
  \url{http://www.datapository.net/} for details.
%
\item {\em Secure BGP Implementation}.  Implementation of S-BGP in the
  Quagga software router.  Our implementation was considered by Randy Bush
  and Geoff Huston for their project to develop a certificate
  infrastructure for secure routing protocols.
%
\item {\em SR-RTP}.  Transport protocol for selective retransmission of
  packets in an MPEG video stream.  Incorporated into ``Oxygen TV'' for
  MIT Project Oxygen.  Some ideas incorporated into the OpenDivX
  video transport protocol.
\end{itemize}

\section*{Research Proposals and Grants}
\label{subsec:research}

\def\funding#1#2#3#4#5{\item{{{\bf #1}}
\newline Sponsor: {#2}
\newline Investigator(s): {#3}
\newline Amount: {\sl #4}
\newline {\sl #5}}}

\begin{grants}

\funding{A Software Defined Internet Exchange}{National
  Science Foundation}{N. Feamster (PI),
  J. Rexford, S. Shenker}
{\$500,000 for 3 years.}{Awarded: August 2014} 

\funding{Studying and Improving the Performance of Access
  Networks}{National
  Science Foundation}{N. Feamster (PI), A. Snoeren}
{\$281,364 for 3 years.}{Awarded: August 2014} 

\funding{EPICA: Empowering People to Overcomme Information Controls and
  Attacks}{National
  Science Foundation}{W. Lee (PI), N. Feamster, H. Klein, M. Bailey, M. Chetty}
{\$750,000 for 4 years.}{Awarded: August 2014} 

\funding{An Open Platform for Internet Routing Experiments}{National
  Science Foundation}{N. Feamster, E. Katz-Bassett, D. Choffnes 
  (PI)}{\$361,617 for 3 years.}{Awarded: August 2014} 

\funding{An Open Observatory for the Internet's Last Mile}{National
  Science Foundation}{N. Feamster
  (PI), S. Banerjee, J. Cappos}{\$399,879 for 3 years.}{Awarded: June 2014} 

\funding{Improving the Performance and Security of Home Networks with
  Programmable Home Routers}{Comcast}{N. Feamster (PI)}{\$65,000 for 1
  year}{Awarded: September 2013} 

\funding{Demand Characterization and Management for Access Networks}{Cisco
  Systems}{N. Feamster (PI) and R. Johari}{\$99,766 for 1 year}{Awarded:
  April 2013}

\funding{Personal Information Fusion with In Situ Sensing
  Infrastructure}{National Science Foundation}{N. Feamster
  (PI)}{\$75,000 for 1 year.}{Awarded: July 2012} 

\funding{Characterizing and Exposing Bias in Social and Mainstream
  Media}{National Science Foundation}{N. Feamster (PI)}{\$175,000 for 1
  year.}{Awarded: July 2012} 

\funding{I-Corps: Helping Users and ISPs Manage Home Networks with
  BISmark}{National Science Foundation}{N. Feamster (PI)}{\$50,000 for 1
  year.}{Awarded: June 2012} 

\funding{Optimizing Network Support for Cloud Services: From Short-Term
  Measurements to Long-Term Planning}{National Science Foundation}{N. Feamster (PI), J. Rexford}{
  \$574,996.00 for 4 years.}{Awarded: April 2012}

\funding{Facilitating Free and Open Access to Information on the
  Internet}{National Science Foundation}{R. Dingledine, N. Feamster
  (PI), E. Felten, M. Freedman, 
  H. Klein, W. Lee}{\$1,500,000 for 4 years}{Awarded: March
  2011}

\funding{Measurement Infrastructure for Home Networks}{National Science
  Foundation}{K. Calvert, W.K. Edwards, 
  N. Feamster (PI), R. Grinter}{\$1,200,000 for 4 years}{Awarded: February
  2011}

\funding{Monitoring Free and Open Access to Information on the
  Internet}{Google Focus Grant}{N. Feamster and W. Lee}{\$1,500,000 for
  3 years}{Awarded: February 2011}

\funding{GENI OpenFlow Campus Buildout}{GENI Project Office}{N. Feamster
  (PI), Russ Clark}{\$64,675 for 1 year}{Awarded: October 2010}

\funding{Architecting for Innovation}{National Science
  Foundation}{H. Balakrishnan, N. Feamster, B. Godfrey, N. McKeown,
  J. Rexford, S. Shenker (PI)}{\$200,000 for 1 year}{Awarded: September
  2010}

\funding{Aster*x: Load-Balancing Web Traffic over Wide-Area
  Networks}{National Science Foundation}{N. Feamster 
  (PI), Russ Clark}{\$75,000 for 1 year}{Awarded: August 2010}

\funding{Network-Wide Configuration Testing and Synthesis}{National Science
  Foundation}{N. Feamster (PI), A. Akella}{\$500,000 for 3 years}{Awarded:
  June 2010}


\funding{MEDITA - Multi-layer Enterprise-wide Dynamic Information-flow Tracking \& Assurance}{National Science
  Foundation}{N. Feamster, A. Orso (PI), M. Prvulovic}{\$900,000 for 3
  years}{Awarded: March 2010}

\funding{Campus Network Access and Admission Control with
  Openflow}{National Science Foundation}{N. Feamster (PI), R. Clark}{\$300,000 for 
  3 years}{Awarded: January 2010}

\funding{Studying DNS Traffic Patterns}{Verisign}{N. Feamster}{\$30,000 for 
  1 year}{Awarded: November 2009}

\funding{CIFellowship for Cristian Lumezanu}{National Science
  Foundation}{C. Lumezanu, N. Feamster (PI)}{\$140,000 for 
  1 year}{Awarded: November 2009}

\funding{Military Network Protocol}{DARPA Subcontract}{N. Feamster}{\$37,000 for 
  1 year}{Awarded: November 2009}


\funding{Botnet Attribution and Removal: From Axioms to Theories to
  Practice}{Office of Naval Research}{W. Lee (PI), D. Dagon, J. Giffin,
  N. Feamster, K. Shin, F. Jahanian, M. Bailey, J. Mitchell, G. Vigna,
  C. Kruegel}{\$7,500,000 for 5 years}{Awarded: August 2009}

\funding{Taint-based Information Tracking in Networked Systems}{National
  Science Foundation Trusted Computing Program}{N. Feamster}{\$450,000
  for 3 years}{Awarded: August 2009}

\funding{Towards a Market for Internet Connectivity}{Office of Naval
  Research}{N. Feamster (PI), R. Johari, V. Vazirani}{\$350,000 for 1
  year}{Awarded: March 2009}

\funding{Bringing Experimenters and External Connectivity to GENI}{GENI
  Project Office}{N. Feamster}{\$320,000 for 3 years}{Awarded:
  September 2008}

\funding{Routing Without Recomputation}{Cisco
  Systems}{N. Feamster}{\$96,019 for 1 year}{Awarded: September 2008}

\funding{CLEANSE: Cross-Layer Large-Scale Efficient Analysis of Network
  Activities \\ to Secure the Internet}{National Science
  Foundation Cybertrust Program}{W. Lee (PI), N. Feamster and
  others}{\$1,200,000 for 5 
  years}{Awarded: September 2008} 

\funding{Virtual Center for Network and Security
  Data}{Department of Homeland Security}{N. Feamster}{\$48,000 for 2
  years}{Awarded: March 2008} 


\funding{Sloan Research Fellowship}{Alfred P. Sloan
  Foundation}{N. Feamster}{\$45,000 for 2 years}{Awarded: February 
  2008}

\funding{Enabling Security and Network Management Research for Future
  Networks}{National Science Foundation CRI-IAD
  Program}{N. Feamster (PI), Z. Mao, W. Lee}{\$397,426 for 3 years}{Awarded: February
  2008}

\funding{SMITE: Scalable Monitoring in the Extreme}{DARPA BAA 07-52:
  Scalable Network Monitoring}{N. Feamster (PI), W. Lee}{\$250,000 for 2
  years}{Awarded: January 2008}


\funding{Countering Botnets: Anomaly-Based Detection, Comprehensive
Analysis, \\ and Efficient Mitigation}{Department of Homeland Security
BAA07-09}{W. Lee (PI), N. Feamster, J. Giffin}{\$1,050,730 for 2
years}{Awarded: January 2008}

\funding{Spam Filtering Research}{IBM Faculty Award}{N. Feamster}{\$
  7,500 (unrestricted gift)}{Awarded: June 2007}


\funding{SCAN: Statistical Collaborative Analysis of Networks}{National
  Science Foundation NeTS-NBD Program}{N. Feamster (PI), A. Gray,
  J. Hellerstein, C. Guestrin}{\$ 95,000 for 3 years.}{Awarded:
  June 2007}

\funding{Towards an Accountable Internet Architecture}{National Science Foundation
  CyberTrust Program (Team Proposal)}{D. Andersen,
  H. Balakrishnan, N. Feamster (PI), S. Shenker}{\$ 300,000 for 3 years.}{Awarded:
  May 2007}

\funding{Fish4Phish: Fishing for Phishing in a Large Pond}{AT\&T
  Labs---Research}{N. Feamster (PI), O. Spatscheck, K. van der Merwe}{Funding
  for summer intern.}{Awarded: February 2007}



\funding{Improving Network Operations with a View from the
  Edge.}{National Science Foundation CAREER
  Program}{N. Feamster (PI)}{\$400,000 for 5 years.}{Awarded: January 2007}


\funding{Equipment Donation for Network Operations Research}{Intel
  Corporation}{N. Feamster}{\$30,000}{Awarded: October 2006}

\funding{CABO: Concurrent Architectures are Better than
One} {National Science Foundation NeTS-FIND
Program} {N. Feamster (PI), L. Gao, J. Rexford}{\$ 300,000 for 4
years}{Awarded: June 2006}
\label{funding:nsf-cabo}

\funding{Verification and Modeling of Wide-Area Internet Routing}{Cisco
  Systems University Research Program}{N. Feamster and H. Balakrishnan (PI)}{\$ 
  95,500 for 1 year.}{Awarded: June 2004} 

\end{grants}


\section*{Service}
\label{subsubsec:mylabel4}



\subsection*{Organizing Roles}
\begin{itemize}
\itemsep=-1pt
\item
Program Committee Co-Chair, USENIX Symposium on Networked Systems Design and Implementation (NSDI): 2013
\item
Poster and Demo Committee Co-Chair, ACM SIGCOMM: 2009, 2013
\item
Founder and Chair, USENIX Workshop on Free and Open Communication on the Internet: 2011
\item
Program Committee Co-Chair, CoNext Student Workshop: 2006
\item
Co-organizer, Boston Freedom in Online Communications (BFOC) Day: 2013
\item
Panel Organizer, IEEE Computer and Communications Workshop (CCW): 2011
\item
Program Committee Co-Chair, ACM SIGCOMM Workshop on Hot Topics in Software Defined Networking (HotSDN): 2012
\item
Editor, IEEE Journal on Network and Systems Management
\item
Co-Organizer, DIMACS Workshop Series on Internet Security: 2007
\item
Organizer, Workshop on Internet Routing Evolution and Design: 2006
\item Program Committee Co-Chair, ACM/USENIX Workshop on Networks meet Databases (NetDB): 2007 
\item Program Committee Co-Chair, Workshop on the Economics of Networked Systems (NetEcon): 2006
\end{itemize}

\subsection*{Program Committees}
\begin{itemize}
\itemsep=-1pt
\item
IEEE Symposium on Security and Privacy: 2006, 2010, 2011, 2012, 2013
\item
ACM Conference on Computer and Communication Security (CCS): 2008, 2011
\item
ISOC Network and Distributed Security Symposium (NDSS): 2011, 2014
\item
ACM SIGCOMM: 2008, 2013
\item
ACM/USENIX Symposium on Networked Systems Design and Implementation
(NSDI): 2009, 2012, 2015
\item
USENIX Technical Conference: 2007, 2009
\item
ACM SIGCOMM Internet Measurement Conference: 2006, 2008, 2012, 2013
\item
CoNext: 2006, 2007, 2014

\item
ACM SIGMETRICS: 2008, 2009, 2010
\item
Research Conference on Communication, Information and Internet Policy
(TPRC): 2014, 2015
\item
ACM SIGCOMM Workshop on Hot Topics in Networking (HotNets): 2012
\item
USENIX Workshop on Free and Open Communications on the Internet (FOCI):
 2012--2015
\item USENIX Workshop on Hot Topics in Security (HotSec): 2012
\item First Workshop on Systems and Infrastructure for the Digital Home (HomeSys): 2012
\item ACM SIGCOMM Workshop on Medical Communication (MedCOMM): 2012
\item ACM SIGCOMM COMSNETS: 2012
\item ACM SIGCOMM Workshop on Home Networks (HomeNets): 2011
\item Program Committee USENIX Workshop on Large-Scale Exploits and Emergent Threats: 2009, 2010, 2011
\item ACM SIGCOMM Poster and Demo Session: 2011
\item ACM SIGCOMM Workshop on Internet Network Management: 2006, 2007, 2008, 2009, 2010
\item USENIX Workshop on Hot Topics in Management of Internet, Cloud, and Enterprise Networks and Services (HotICE): 2011
\item ACM SIGCOMM Workshop on Virtualized Infrastructure Systems and Architectures (VISA): 2010
\item ACM SIGCOMM Workshop on Programmable Routers for Extensible Services of Tomorrow (PRESTO): 2008, 2009

\item USENIX International Workshop on Real Overlays \& Distributed Systems: 2008

\item ACM SIGCOMM Workshop on Economics of Networked Systems (NetEcon): 2008

\item International World Wide Web Conference (Security/Privacy Track): 2008
\item IEEE LAN/MAN Workshop: 2008
\item ACM SIGCOMM Student Poster Session: 2007
\item ACM SIGMETRICS Workshop on Mining Internet Data (MineNet): 2007
\item Conference on Email and Anti-Spam (CEAS): 2007
\item USENIX Workshop on Steps to Reduce Unwanted Traffic on the Internet (SRUTI): 2007
\item International World Wide Web Conference (Security/Privacy Track): 2007
\item North American Network Operators Group (NANOG): 2006--2009
\item IEEE Infocom Student Poster Session: 2006

\item IEEE International Conference on Internet Surveillance and Protection: 2006

\end{itemize}

\noindent
External reviewer for {\em IEEE/ACM Transactions on Networking}, {\em
SIGCOMM} (2002, 2003, 2004, 2006, 2007), {\em SOSP} (2001, 2003), {\em
Infocom} (2004, 2006), {\em HotNets} (2003), {\em HotOS} (2001), {\em
USENIX Security Symposium} (2002), {\em ACM Computer Communication
Review}, {\em IEEE Network Magazine}, {\em IEEE Journal on Selected
Areas in Communications}, {\em Image Communication} (EURASIP), {\em
ASPLOS} (2004), {\em MobiSys} (2004), {\em USENIX} (2005, 2006), {\em
NSDI} (2005, 2006), {\em IPTPS} (2005), {\em Workshop on Privacy
  Enhancing Technologies} (2006).  \\



\section*{Teaching}

\subsection*{Courses}
\label{subsec:courses}

\begin{center}
\begin{tabular}{lllrl}
& & & \textbf{Number of \newline } &\\
\textbf{Term} & {\bf Year} & \textbf{Course Number \& Title} &
\textbf{Students} & \textbf{Comments} \\ \hline 
Fall & 2014 & CS~4270/8803 Software Defined Networking & 30 & New Course \\
Fall & 2014 & OMS CS~6250 Computer Networking & 223 & Online MS \\
Fall & 2013 & CS~7001 Introduction to Graduate Studies & 51 &  \\
Spring & 2014 & CS~6250 Computer Networking & 50 \\
Spring & 2014 & OMS CS~6250 Computer Networking & 300 & Online MS \\
Fall & 2013 & CS~8001 Software Defined Networking & 15 & Flipped classroom \\
Fall & 2013 & CS~7001 Introduction to Graduate Studies & 36 & 10k Blog Readers/Month \\
Summer & 2013 & Software Defined Networking (Coursera) & 50,000+ & First
SDN Course \\
Spring & 2013 & CS~3251 Computer Networking & 82 \\
Fall & 2012 & CMSC~330 Programming Languages (Univ. of Maryland) & 100 \\
Fall & 2011 & CS~6250 Computer Networking & 51 \\
Fall & 2011 & CS~4235 Computer Security & 44 \\
Fall & 2010 & CS~6250 Computer Networking & 92 \\
Spring & 2010 & CS~3251 Computer Networking I & 53 \\
Spring & 2010 & CS~8803 NGN Next Generation Networking & 50 & New Course \\
Fall & 2009 & CS~7001 Introduction to Graduate Studies & 39 \\
Spring & 2009 & CS~6262 Network Security & 45 & Updated Syllabus \\
Fall & 2008 & CS~4251 Computer Networking II & 16 & \\
Fall & 2008 & CS~7001 Introduction to Graduate Studies & 44 \\
Spring & 2008 & CS~4251 Computer Networking II & 14 & New Syllabus\\
Fall & 2007 & CS~7001 Introduction to Graduate Studies & 53 \\
Spring & 2007 & CS~7260 Internetworking Protocols and Architectures & 29 \\
Fall & 2006 & CS~7001 Introduction to Graduate Studies & 74 &
New Syllabus \\
Fall & 2006 & CS~8001 Networking Research Seminar & 30 & New Syllabus\\
Fall & 2006 & CS~1100 Freshman Leap Seminar & 15 & \\
Spring & 2006 & CS~7260 Internetworking Protocols and Architectures & 27 &
New Syllabus \\
\end{tabular}
\end{center}

\noindent
{\bf Other Highlights:}
\begin{itemize}
\itemsep=-1pt
\item Created the first-ever course on Software Defined Networking (SDN), and
delivered it on Coursera to over 50,000 enrolled students 
\item Tutorial on network measurement at African Network Operators Group
(AfNOG) in Summer 2013. 
\item Guest lecture on Internet censorship in Georgia Tech CS~4001 in October
2011. 
\item Tutorial on software-defined networking at African Network Operators
Group (AfNOG) in Summer 2011. 
\item Tutorials on BGP Multiplexer at GENI Experimenters Workshop and
  GENI Engineering Conference in Summer 2010. 
\item Tutorial on network security at African Network Operators Group (AfNOG)
in Summer 2010.
\item Tutorial on Internet routing at Simposio Brasileiro de Redes de Computadores (SBRC) in Summer 2008. 
\item Lecture for DIMACS Tutorial on Next-Generation Internet Routing
  Algorithms in August 2007. 
\item Guest lecture for CS~6250 (Advanced Computer Networks) in Fall 2007. 
\item Guest lecture for CS~3251 (Computer Networks I) in Fall 2006. 
\item Multiple guest lectures for CS~4251 (Computer Networks II) in Spring
2006.
\item Guest lecture for MIT Course 6.829 (Computer Networking) in Fall
  2005. 
\end{itemize}


\subsection*{Curriculum Development}
\label{subsec:curriculum}

\paragraph{Coursera and Online Masters Program Development:} I have
developed an online Coursera course for the topic of Software Defined
Networking (SDN), an emerging topic in computer networking that is
reshaping how networks are defined.  In the course, students learn about
the history of SDN and develop hands-on experience with tools to develop
technologies and applications for SDN.  The course had approximately
50,00 students enrolled, and had about 4,000 students actively
participating on the forums and in lectures.  Nearly 1,000 students
successfully completed the programming assignments for the course,
qualifying as a pass ``with distinction''.  The course reviewed
extremely favorably from students, as we have documented here:
\url{http://goo.gl/So7Uis}.  Additionally, the success of the course was
covered on many technical forums, including the Mininet blog
(\url{http://goo.gl/ZqfL1N}), the Packet Pushers blog
(\url{http://goo.gl/okzd44}), and the sFlow blog
(\url{http://goo.gl/qSMCq0}).  Based on its overwhelming success, the
course will be offered again in May 2014.


\paragraph{Georgia Tech Online Masters Program:} 
I served on the committee to develop a Master of Science in
Computer Science degree based on Massive Open Online Course (MOOC)
offerings and took an active role in ensuring the creation of the online
degree at Georgia Tech.  As of October 2013, I am currently working with
Udacity and Georgia Tech to create the first online graduate course in
computer networking, which will launch in January 2014 and become an
integral part of the Georgia Tech online degree program.

\paragraph{CS 7001 Introduction to Graduate Studies:}
With Professor Alex Gray, I have developed a new course syllabus and
structure to CS 7001 around the larger goal of introducing new students
to {\em how to do great research} as soon as their first term at Georgia
Tech.  In contrast with previous terms, where CS 7001 consisted of
faculty ``advertisements'' for their research and projects consisted of
short ``mini-projects'' where little research could be accomplished in a
short time span of 3 weeks, we have improved the syllabus by bringing in
faculty members to talk about research philosophy, exciting new
directions, etc.  We have also given the students the option to do a
research project that is a term-long project in conjunction with CS
8903; our goal is to give students the flexibility to select meaningful
research problems based on their research assistantships while helping
them learn the skills required for writing papers, finding and
evaluating research ideas, and performing other tasks associated with
doing great research.  Alex Gray and I wrote a conference paper on our
development of this course, which appeared at {\em ACM SIGCSE 2008}.  We
have developed a website for the course, \url{greatresearch.org}, which
makes the material generally available for use at other universities and
by other researchers.  {\bf The website and blog have more than 25,000 views
since its launch in mid-August 2013, and it has been receiving about
10,000 views every month.}

\paragraph{CS 6250 Graduate Computer Networking:} In Fall 2011, I redesigned the
graduate computer networking course to focus more on current
technologies and hands-on assignments.  Conventional networking courses
treat today's protocols and mechanisms as fixed artifacts, rather than
as part of a continually evolving system.  To prepare students to think
critically about Internet architecture, Jennifer Rexford and I created a
graduate networking course that combines ``clean slate'' networking
research with hands-on experience in analyzing, building, and extending
real networks.  My goal was to prepare students to create and explore
new architectural ideas, while teaching them the platforms and tools
needed to evaluate their designs in practice.  The course, with
offerings at both Georgia Tech and Princeton, focuses on network
management as a concrete way to explore different ways to split
functionality across the end hosts, network elements, and management
systems.  I have refined the course in Fall~2011 to include more
hands-on assignments ad refactored the course around networking problems
in different types of networks: transit networks, home networks, content
hosting networks, and mobile and wireless networks.  {\bf Our work on
  the course received the best paper at the {\em ACM SIGCOMM Workshop on
    Networking Education (NetEd)} in 2011}.

\paragraph{College of Computing Research Day and Seminar Series:}
In addition to the course itself, to fulfill some of the functions of
the former 7001 course, Alex Gray and I financed and organized a
college-wide seminar series and research day in Fall 2009 and again in
Spring 2011.  Throughout the term, faculty speakers from across the
college gave one-hour talks about their research; we raised money from
Yahoo to support this event.  The research day brings together students
and faculty from around the college to see talks, demonstrations, and
posters from around the college to exchange ideas.

\paragraph{CS 8803 Next-Generation Networking:} I developed a
new graduate course that gives students practical experience with
a variety of tools for next-generation networking, ranging from the
Click software router to the OpenFlow switch framework.  The course also
teaches students about the state of the art in networking
research---students read papers about research and industry trends and
do a course project that incorporates aspects of these new
technologies.  This course relates to the larger nationwide effort on
Global Environment for Network Innovations (GENI), which is building
infrastructure for researchers to provide the next generation of
networking protocols and technologies.



\section*{Ph.D. Students}

\def\student#1#2#3#4#5{\item {\bf #1}. {#2 ~
{\sl #3}. \\
Publications: #4 \\
{\sl #5}}}

\def\studentnopub#1#2#3#4#5{\item {\bf #1}. {#2 ~
{\sl #3} 
{\sl #5}
}}



\subsection*{Graduated}

\begin{itemize}

\student{Shuang Hao}{}{Fall 2007 -
  Fall 2014}{
\refbib{Hao2013:dns},
\refbib{Hao2011:dns},
\refbib{Hao2009:snare}
}{
First job: Postdoc at UC Santa Barbara.
}

\student{Sam Burnett}{}{Fall 2008 -
  Spring 2014}{
\refbib{Burnett2010:collage}, \refbib{Grover2013:nat}
}{ First job: Google.}

\student{Srikanth Sundaresan}{}{Fall 2008 -
  Spring 2014}{
\refbib{Sundaresan2013:webperf},
\refbib{Sundaresan2012:broadband},
\refbib{Sundaresan2011:bismark},
\refbib{Sundaresan2011:labels}
}{ 
First job: Research Scientist at ICSI.
} 



\student{Vytautas Valancius}{}{Summer 2007 -
  Spring 2012}{
\refbib{Valancius2013:pecan},
\refbib{Katz2012:lifeguard},
\refbib{Valancius2011:tiers},
\refbib{Valancius2010:bgpmux},
\refbib{Valancius2008:mint},
\refbib{Bhatia2008:roads}
}{First job: Google. }


\student{Murtaza Motiwala}{}{Fall 2006 -
Spring 2012}
{
\refbib{Motiwala2012:cost},
\refbib{Motiwala2008:splicing},
\refbib{Motiwala2007:splicing}
}
{First job: Google.}


\student{Anirudh Ramachandran}{}{Spring 2006 -
Spring 2011} {  
\refbib{Lee2008:botnet},
\refbib{Ramachandran2011:ceas}
\refbib{Ramachandran2007:spamtracker},
\refbib{Feamster2006:spam},
\refbib{Ramachandran2008:wosn},
\refbib{Ramachandran2007:bitstore},
\refbib{Ramachandran2006:recon},
\refbib{Ramachandran2006:dnsbl}
}{First job: Founded Nouvou, security startup (sold).  Now at Deutche Telekom Labs.}


\student{Mukarram Bin Tariq}{}{Spring 2007 -
  Spring 2010}{
\refbib{Tariq2008:wise},
\refbib{Tariq2009:vlan},
\refbib{Tariq2009:nano},
\refbib{Tariq2008:nano},
\refbib{Tariq2010:wise}
}
{First job: Google.
}

\end{itemize}

\subsection*{Graduating in 2015}

\begin{itemize}

\student{Hyojoon Kim}{}{Fall 2009 - Summer 2015 (Expected)}{
\refbib{Grover2013:nat},
\refbib{Kim2011:config}
}
{
 \vspace*{-1.5em}
}

\student{Bilal Anwer}{}{Fall 2008 - Fall 2015 (Expected)}{
\refbib{Anwer2010:netfpga},
\refbib{Anwer2010:switchblade},
\refbib{Anwer2010:visa},
\refbib{Anwer2009:visa}
}
{
 \vspace*{-1.5em}
} 

%\studentnopub{Robert Lychev}{}{Fall 2008 -
%  Spring 2014 (Expected)}{}{ }

\student{Maria Konte}{}{Fall 2007 - Summer 2015 (Expected)}{
\refbib{Konte2012:pam},
\refbib{Konte2008:pam}
}
{
 \vspace*{-1.5em}
}

\student{Yogesh Mundada}{}{Fall 2007 - Summer 2015 (Expected)}{
\refbib{mundada2013:silverline},
\refbib{Mundada2011:silverline}
}
{
 \vspace*{-1.5em}
}


\end{itemize}


\subsection*{Current}

\begin{itemize}
\student{Arpit Gupta}{}{Fall 2013 - Present}{
\refbib{gupta2014:pam}
}{
 \vspace*{-1.5em}
}
\studentnopub{Ben Jones}{}{Fall 2013 - Present}{}{}
\studentnopub{Sean Donovan}{} {Fall 2013 - Present}{}{}
\studentnopub{Swati Roy}{} {Fall 2013 - Present}{}{}
\student{Sarthak Grover}{} {Fall 2012 - Present}{
\refbib{Grover2013:nat}
}{
 \vspace*{-1.5em}
}
\studentnopub{Muhammad Shahbaz}{}{Fall 2012 - Present}{}{}
\studentnopub{Abhinav Narain}{}{Fall 2011 -  Present}{}{}
\student{Xinyu Xing}{}{Fall 2011 -  Present}{
\refbib{xing2014:pam},
\refbib{Xing2013:poison}
}{
 \vspace*{-1.5em}
}
\end{itemize}


\section*{\df References}
\begin{tabular}{@{}l@{\qquad\qquad}l}
Prof. Hari Balakrishnan &              Prof. Scott Shenker\\
MIT Computer Science \& AI Lab &  University of California, Berkeley\\
32 Vassar Street, 32G-940 &      415 Soda Hall \\
Cambridge, MA 02139&    Berkeley, CA 94709\\
(617) 253-8713 &                       (510) 643-3043\\                    
hari@csail.mit.edu &                     shenker@eecs.berkeley.edu\\
%% \noalign{\vskip10pt}
%% Prof. Jennifer Rexford    &      Prof. Nick McKeown \\
%% Princeton University &  Stanford University \\
%% Department of Computer Science  &    Department of Computer Science
%% and Engineering\\
%% 35 Olden Street, CS 306  &    Gates 340 \\
%% Princeton, NJ 08544 &   Stanford, CA 94305 \\
%% (609) 258-5182           &   (650) 723-3623 \\
%% jrex@cs.princeton.edu           &  nickm@stanford.edu \\
\noalign{\vskip10pt}
Prof. Vern Paxson    &      Prof. Nick McKeown \\
University of California, Berkeley &  Stanford University \\
Department of Computer Science  &    Department of Computer Science
and Engineering\\
737 Soda Hall  &    Gates 340 \\
Berkeley, CA 94709 &   Stanford, CA 94305 \\
(510) 643-4209           &   (650) 723-3623 \\
vern@cs.berkeley.edu           &  nickm@stanford.edu \\
\noalign{\vskip10pt}
Prof. Stefan Savage\\
Department of Computer Science and Engineering \\
EBU3B 3106  \\
University of California, San Diego \\
La Jolla, CA 92093 \\
(858) 822-4895 \\
savage@cs.ucsd.edu \\
\end{tabular}

\if 0
\newpage
\parindent=0pt
\parskip=10pt
%\begin{cvlist}{Internet Routing}
{\large\df Internet Routing}

    The Internet is composed of more than 17,000 independently operated
    networks, or autonomous systems (ASes), that exchange routing
    information using the Border Gateway Protocol (BGP).  Network
    operators in each AS configure routers to control the routes that
    the routers learn, select, and propagate.  Configuring a network of
    BGP routers is like writing a distributed program where complex
    feature interactions occur both within one router and across
    multiple routers.  This complex process is exacerbated by the number
    of lines of code, by the absence of useful high-level primitives in
    today's router configuration languages, by the diversity in
    vendor-specific configuration languages, and by the number of ways
    in which similar high-level functionality can be expressed in a
    configuration language.  As a result, router configurations tend to
    have faults.  Faults in BGP configuration can cause forwarding
    loops, packet loss, and unintended paths between hosts.  Operators
    must be able to evaluate the effects of a configuration and be
    assured that the configuration is correct before deploying it.  My
    dissertation advances the state of the art in Internet routing by
    devising fault detection and modeling tools for today's Internet
    routing protocols and proposing a new Internet routing architecture
    that alleviates many of the problems we uncovered in our work on
    fault detection and modeling.

    %Traffic shifts,
    %equipment failures, planned maintenance, and topology changes in
    %other parts of the Internet can all degrade performance.  To
    %maintain good performance, network operators must continually
    %reconfigure the routing protocols.  
%%     My dissertation work improves the state of the art by: (1)~designing
%%     and implementing a static analysis tool to detect BGP configuration
%%     faults, (2)~designing algorithms to allow operators to predict the
%%     effects of configuration changes before the configuration is
%%     deployed and (3)~proposing a new interdomain routing architecture
%%     that makes both fault detection and modeling for network engineering
%%     much easier than they are today.


%  \item[2003--] 
{\mf Detecting Faults in BGP Configuration with Static Analysis} \hfill MIT
\vspace*{-0.1in}

    {\bf rcc}, the {\em router configuration
    checker}, detects faults in the BGP configurations of
    routers in an AS using static analysis. {\bf rcc} detects two broad
    classes of faults that affect network reachability: route validity
    faults, where routers may learn routes that do not correspond to
    usable paths, and path visibility faults, where routers may fail to
    learn routes for paths that exist in the network.  {\bf rcc} enables
    network operators to test and debug configurations before deploying
    them in an operational network, improving on the status quo where
    most faults are detected only during operation.  {\bf rcc} has been
    downloaded by more than sixty network operators to date.  
    I presented {\bf rcc} to the North American Network Operators Group
    (NANOG), and the tool has been used by several large backbone
    Internet Service Providers (ISPs)
    to successfully detect faults in deployed configurations.  This
    work was inspired by my work on the {\em routing logic} that I
    presented at the 2003 {\em ACM SIGCOMM Workshop on Future Directions
    in Network Architecture} and appears at the {\em 2nd
    USENIX Symposium on Networked Systems Design and Implementation}.
    We have also studied configuration faults as part of several
    measurement studies.  We presented an algorithm to detect route
    advertisements that violate peering contracts and an empirical study
    of their prevalence at the 2004 {\em ACM Internet Measurement
    Conference}. 


%  \item[2001--] 
{\mf Modeling Internet Routing for Network Engineering} \hfill MIT/AT\&T Labs--Research
\vspace*{-0.1in}

    Since interdomain route selection is distributed, indirectly
    controlled by configurable policies, and influenced by complex
    interactions with {\em intra}domain routing protocols, operators
    cannot predict how a particular BGP configuration would behave in
    practice.  We devised an algorithm that computes the outcome of the
    BGP route selection process for each router in a {\em single} AS,
    given only a static snapshot of the network state, without
    simulating BGP's complex dynamics.  Using data from a large ISP, I
    demonstrated that the algorithm correctly computes BGP routing
    decisions and has a running time that is efficient and accurate
    enough for many tasks, such as traffic engineering and capacity
    planning.  Studying the general properties and computational
    overhead of modeling the route selection process in each of these
    cases provides insight into the unnecessary complexity introduced
    by various aspects of today's interdomain routing architecture.  I
    used these insights to propose improvements to BGP that avert the
    negative side effects of various artifacts without limiting
    functionality.  This work appeared in {\em
    ACM SIGMETRICS} 2004 and has also been submitted to {\em IEEE/ACM
    Transactions on Networking}.

%  \item[2004--] 
{\mf Internet Routing Architecture: Routing Control Platform} \hfill MIT/AT\&T Labs--Research
\vspace*{-0.1in}

   The limitations in today's routing system arise in large part from
   the fully distributed path-selection computation that the IP routers
   in an AS must perform.  We proposed that interdomain routing should be
   separated from today's IP routers, which should simply forward packets (for
   the most part).  Instead, a separate {\em Routing Control Platform (RCP)}
   should select routes on behalf of the IP routers in each AS and exchange
   reachability information with other domains.  RCP could both select routes
   for each router in a domain (\eg, an AS) and exchange routing information
   with RCPs in other domains.  By selecting routes on behalf of {\em all\/}
   routers in a domain, RCP can avoid many internal BGP-related complications
   that plague today's mechanisms for disseminating and computing routes within
   an AS.  RCP facilitates traffic engineering, simpler and less error-prone
   policy expression, more powerful diagnosis and troubleshooting, more rapid
   deployment of protocol modifications and features, enforceable consistency
   of routes, and verifiable correctness properties.  The architectural
   proposal for RCP appeared at the 2004 {\em ACM SIGCOMM Workshop on Future
   Directions in Network Architecture}; the design and implementation of an RCP
   prototype won the best paper award at the {\em 2nd USENIX Symposium
   on Networked Systems Design and Implementation (NSDI)}.

%\end{cvlist}

\pagebreak
%\begin{cvlist}{Internet Measurement}
{\df Internet Measurement}

%  \item[2002--] 
{\mf Understanding End-to-End Internet Path Failures} \hfill MIT
\vspace*{-0.1in}

    Empirical evidence suggests that reactive routing systems, which
    detect and route around faulty paths based on measurements of path
    performance, improve resilience to Internet path failures.  We
    studied {\em why} and under {\em what 
    circumstances} these techniques are effective by
    correlating end-to-end active probes, loss-triggered traceroutes of
    Internet paths, and BGP routing messages.  This work was the first known
    study to correlate routing instability with degradations in {\em
    end-to-end} reachability. We found that most
    path failures last less than fifteen minutes.  Failures that appear
    in the network core correlate better with BGP instability than
    failures that appear close to end hosts.  Surprisingly, there is
    often increased BGP traffic both before and after failures.  Our
    findings suggest that reactive routing is most effective between
    hosts that have multiple connections to the Internet and that
    reactive routing systems could pre-emptively mask about 20\% of
    impending failures by using BGP routing messages to
    predict these failures before they occur.  This work appeared at {\em ACM
    SIGMETRICS} 2003.

    End-to-end path failures are typically attributed to either
    congestion or routing dynamics. Unfortunately, the extent to which
    congestion and routing dynamics cause end-to-end failures, and the
    effect of routing dynamics on end-to-end performance, are poorly
    understood.  In a follow-up study, we used similar techniques to
    find that routing dynamics contribute significantly to end-to-end
    failures and, in particular, routing dynamics are responsible for
    most long-lasting path failures.  The study also finds that
    long-lived end-to-end path failures that involve routing dynamics
    are typically caused by BGP convergence or instability.  This work is
    the first to quantify the impact of routing dynamics on end-to-end
    path availability; it was submitted to {\em ACM SIGMETRICS} 2005.


%\end{cvlist}

%\begin{cvlist}{ Network Security}
{\df Network Security}

%  \item[2001--2003] 
{\mf Infranet: Circumventing Web Censorship} \hfill MIT
\vspace*{-0.1in}
    
    An increasing number of countries and companies routinely block or
    monitor access to parts of the Internet.  To counteract these
    measures, we designed and implemented {\em Infranet}, a system that
    enables clients to surreptitiously retrieve sensitive content via
    cooperating Web servers distributed across the global Internet.
    These Infranet servers provide clients access to censored sites
    while continuing to host normal uncensored content.  Infranet uses a
    tunnel protocol that provides a covert communication channel between
    its clients and servers, modulated over standard HTTP transactions
    that resemble innocuous Web browsing.  In the upstream direction,
    Infranet clients send covert messages to Infranet servers by
    associating meaning to the {\em sequence} of HTTP requests being
    made.  In the downstream direction, Infranet servers return content
    by hiding censored data in uncensored images using steganographic
    techniques.  This work appeared at the {\em 11th USENIX Security
    Symposium}. 

%\end{cvlist}

%\begin{cvlist}
{\df Adaptive Streaming Media Protocols}

%  \item[2000--2001] 
{\mf Reliable, Adaptive Video Streaming} \hfill MIT
\vspace*{-0.1in}

    Video compression exploits redundancy between frames to achieve
    higher compression, but packet loss can be detrimental to
    compressed video with interdependent frames because errors
    potentially propagate across many frames.  In my Master's thesis, I
    quantified the effects of packet loss on the quality of MPEG-4 video,
    developed an analytical model to explain these effects, and
    presented an RTP-compatible protocol, called {\em SR-RTP}, that {\em
    adaptively} delivers higher quality video in the face of packet
    loss.  This work appeared at the {\em 12th International Packet
    Video Workshop} and was later implemented as part of a
    streaming video server for MIT Project Oxygen.
%    The Internet's variable bandwidth and delay make it difficult to
%    achieve high utilization, TCP-friendliness, and a high-quality
%    constant playout rate.  Traditional congestion avoidance
%    schemes such as TCP's ad\-ditive-increase/\-multiplicative-decrease
%    (AIMD) induce variable transmission rates that degrade
%    the perceptual quality of the video stream.  
    We also designed a scheme for performing quality adaptation of
    layered video for a general family of congestion control algorithms
    called {\em binomial congestion control}.
% and showed that a
%    combination of smooth congestion control and receiver-buffered
%    quality adaptation reduces oscillations, increases interactivity,
%    and delivers higher quality video for a given amount of buffering.
    This work appeared at the {\em 11th International Packet Video
    Workshop}.


%  \item[1999--2000] 
{\mf Video Transcoding} \hfill Hewlett-Packard Laboratories
\vspace*{-0.1in}

    We designed and implemented an algorithm that transcoded MPEG video
    input to a lower-bitrate H.263 progressive bitstream, facilitating
    the transmission of a digital television signal over a wireless
    medium.  This algorithm was the first
    to use both spatial and temporal downsampling in an MPEG-2 to H.263
    field to frame transcoder to achieve substantial bitrate reduction.
    The proposed algorithm exploits the properties of the MPEG-2 and
    H.263 compression standards to perform interlaced to progressive
    (field to frame) conversion with spatial downsampling and frame-rate
    reduction in a CPU and memory efficient manner, while
    minimizing picture quality degradation.  This work appeared at the
    {\em IEEE International Conference on Image Processing} in 1999.
\fi

%\end{cvlist}

%References available upon request.

\end{cv}

\end{document}
