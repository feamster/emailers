% evaluation.
% update the following
\newpage 

\setcounter{page}{1}
\lfoot{\fancyplain{Printed: \today}{Nick Feamster}}
\cfoot{\fancyplain{Revised: \today}{Annual Evaluation 2009}}
\rfoot{Page \thepage}

\setcounter{page}{1}
\begin{center}
{\Large\textbf{Accomplishments for 2008 \\ April 2009}}\\[.1in]
{\large\textbf{Nick Feamster}}\\[.05in]
feamster@cc.gatech.edu \\
\url{http://www.cc.gatech.edu/~feamster/} \\[.1in]
\end{center}

\begin{sloppypar}

\section*{Research}

In this section, I outline my research accomplishments.  I divide this
work into several distinct areas: recognition and publication,
software/systems artifacts, and funding.  Overall, this has been a good
year, both for attracting funding and for having published papers in
top-tier conferences.  The year was highlighted by the Sloan and PECASE
Awards, and three publications in {\em SIGCOMM} (as many as I have had
in all previous years combined).  My research group also began releasing
software for its respective papers and projects, a milestone which I see
as very important for long-term impact.  Several of the algorithms we
have developed for spam filtering are also making their way into
commercial products and defense technology.

Over the pasy year by research group grew significantly.  The biggest
challenge this year will be maintaining high-quality research output and
ensuring that all students received adequate mentoring to develop their
careers.  This year, my first student will graduate.  In the coming year
or two, I aim to make sure that my more senior students are
well-prepared for the job market. 

\subsection*{Recognition/Publication}

This year I received several awards recognizing my contributions to
networking research, including the Sloan Reseach Fellowship and the NSF
Presidential Early Career Award for Scientists and Engineers (PECASE).
Additionally, our group had the following major conference paper
acceptances last year:

\begin{itemize}
\itemsep=-1pt
\item Path splicing ,a scalable multipath routing scheme, was accepted
at {\em SIGCOMM 2008}.
\item AIP, an architecture for Internet accountability, was accepted at
at {\em SIGCOMM 2008}.
\item WISE, a system for predicting ``what if'' network configuration
scenarios, was accepted at {\em SIGCOMM 2008}.
\item NANO, a system for detecting network neutrality violations, was
acepted at {\em HotNets 2008}.
\item FlexSample, a traffic sampling algorithm for allowing operators to
focus traffic sampling on specific subpopulations, was accepted at {\em
ACM Internet Measurement Confernece 2008}.
\end{itemize}
\noindent
My group also had a strong showing at the Sigcomm workshops and the
CoNext workshops, with several publications at each of these venues.

This was a strong year for research publications; this goal was one of
the major things that I set out to achieve from last year where I was
successful.  Part of the reason for this wave of successful publications
was that the previous year was slightly weaker, so I had several papers
in the pipeline that were already mature.  We have another ten papers in
submission in 2009 already; it is difficulty to attain the same amount
of success as last year, but I expect that 2009 will also see a handful
of strong conference publications.

One particular weakness I see is that the research group has not had any
success at either {\em NSDI} or {\em SOSP}.  I attribute this partially
to the fact that it can sometimes take several years to build complete,
working systems that are ``publication ready'' at these conferences.
This year, however, I hope to have a handful of submissions to {\em
NSDI}, coinciding with the completion of several software systems (see
below).

\subsection*{Funding}

It was another strong year for funding, with five new grants awarded,
including a large Cybertrust proposal (co-PI, with Wenke
Lee)\footnote{Although there are many co-PIs on this grant, I would
classify myself as the ``lead co-PI''; Wenke and I wrote the grant
almost singlehandedly.}, a large grant from the Office of Naval Research
(for which I am the lead PI), and two single-PI grants: one from Cisco
(thus achieving my goal from last year), one from the Office of Naval
Research, and one from the GENI Project Office.  My funding successes
this year diversified from NSF funding and have also been instrumental
in showing my ability to lead large single-PI projects (notably, the
GENI project).

\begin{itemize}
\itemsep=-1pt
\item Co-PI for new large NSF Cybertrust grant.
\item Lead PI on new Office of Naval Research project to develop a new
Interdomain routing protocol based around Internet connectivity markets.
\item PI on project from the GENI project office to develop topology
creation mechanisms and technologies for bringing external connectivity
to network testbeds.
\end{itemize}
\noindent
Given this level of funding success, I intend to spend much of next
year focusing on paper submissions, while still submitting where
appropriate, and to ensure a steady stream of funding for my research
group.

\subsection*{Software and Systems}

This year has been a relatively strong year for software development.
Several projects which had previously existed only in paper form have
taken shape in the form of software systems.  Notably, we have released
a preliminary version of {\em NANO}, a tool to help users determine
whether their ISP is discriminating against them.  We are testing a
preliminary version of this tool with a small group of users.  We are
planning a larger release for early summer 2009.  

We have also begun developing and testing a new anti-censorship system
called Collage.  This system relies on the large amount of
user-generated content that is currently being published on the Internet
as ``cover traffic'' to allow senders and receivers to exchange censored
content.  We submitted a design and simulation-based evaluation of
Collage to SIGCOMM and are currently working on an implementation of the
system, which we plan to release in summer 2009.

We have also continued software implementations associated with several
other projects, including the path splicing and spam filtering projects.
Implementations are currently preliminary.  We intend to spend a large
portion of spring and summer on these implementations and others for
possible submissions later this year (more details under ``goals''
below).  I should note that progress on software development for both of
these systems has been painstakingly slow.  Adding several students to
these projects (notably an undergraduate student, Megan Elmore, and a
masters student, Yogesh Mundada) has increased the pace of software
development on path splicing, but the systems have yet to be deployed.
I am making a push to get a preliminary version of this software
released by Summer 2009, for an eventual submission to {\em NSDI} in
October 2009.

\begin{itemize}
\itemsep=-1pt
\item Released a preliminary version of NANO, a tool to help users
determine when their traffic is being discriminated against by access
ISPs.
\item Released software related to the GENI prototyping effort.
\item Began building a testbed for network experimentation with
programmable switches, for greater interaction with the Georgia Tech
campus network office of information technology on various research
projects. 
\end{itemize}

\section*{Service}

I have continued my service both internally to the department, and
externally on program committees and organizing committees.  In addition
to serving on program committees, I have given various tutorials, and
have helped organize sessions at NSF PI meetings, as outlined below. 

In the past year, I have done more than my share of service on program
committees, mostly because I was being asked to serve on very good
program committees. I served on the program committees of all major
networking and some security conferences: {\em SIGCOMM 2008}, {\em
SIGMETRICS 2009}, {\em IMC 2008}, {\em NSDI 2009}, {\em CCS 2008}, and
{\em CoNext 2008}.  This year, I have already been asked to serve on
{\em NSDI 2010}.  I am also the poster/demo co-chair of {\em SIGCOMM
2009}, so I am taking more of a leadership role on these committees as
well.

Additionally, I am continuing in my role as an organizer of the DIMACS
workshop series on secure Internet routing.

\begin{itemize}
\itemsep=-1pt
\item Co-Chair of the demo and poster committee for {\em SIGCOMM 2009}. 
\item Program committee member of all major networking conferences, and
  also some major security conferences.
\item Lead organizer for the DIMACS workshop series on secure Internet
routing. 
\end{itemize}

Internally, I am serving on the faculty recruiting committee and the
search committee for the new College of Computing dean.  I am also
involved, with Alex Gray, as part of an effort to re-design the
Ph.D. process.  At the college retreat in the fall, I presented a new
model for the Ph.D. program entitled ``cross threads''.  This program
re-design has been thrown into a somewhat uncertain state with the dean
transition, but I hope to resume this process in the coming year.

\section*{Education}


My work with Alex Gray on the redesign of a course on ``Introduction to
Ph.D. Research'' appeared in {\em ACM SIGCSE} in March 2008.  Based on
this writeup, he and I re-vamped the syllabus for this course in Fall
2008. The course now has a complete, coherent syllabus organized around
themes ranging from research creativity to more practical matters (\eg,
time management).  Alex and I are continuing to make improvements to the
course.  Now that the course content has solidified somewhat, we hope to
turn our attention towards turning our course notes into a book.

I undertook the teaching of two new courses: CS~4251 (undergraduate
computer networking) and CS~6262 (graduate network security).  In both
of these courses, I updated course content---from lectures to problem
sets to projects---and added content to each of these courses of a more
hands-on, operational flavor.  Many of the problem sets, project ideas,
and lecture materials are in use in courses at other universities as
well (e.g., Princeton). Some of the improvements to each of these
courses include:
\begin{itemize}
\itemsep=-1pt
\item Updated CS~6262 project list to include more research-related
projects involving network security and operations.
\item Updated CS~6262 syllabus to include research papers and projects
related to current work and developments in the area.
\item Updated problem sets in CS~4251 to include more hands-on
assignments (e.g., experiments on Emulab).
\end{itemize}
\noindent
Many of the improvements to these courses are also detailed in my CV.  


%%%%%%%%%%%%%%%%%%%%%%%%%%%%%%%%%%%%%%%%%%%%%%%%%%%%%%%%%%%%

\newpage
\begin{center}
{\Large\textbf{Goals for 2009}}\\[.1in]
{\large\textbf{Nick Feamster}}\\[.05in]
feamster@cc.gatech.edu \\
\url{http://www.cc.gatech.edu/~feamster/} \\[.1in]
\end{center}

\section*{Research}

Many of my conceptual goals and future directions are outlined in the
future directions section of my research statement.  Over the next few
years, I will seek to develop agendas along the lines of accessibility,
accountability, and availability.  One of my main goals for this year is
to begin developing and deploying secure network architectures from the
ground up, based on a refactoring of control and enabled by the advent
of programmable network devices (e.g., OpenFlow-enabled switches).
Along these lines, we will work with TSO and OIT to develop a campus
testbed for this research that can also be used in courses.

I also intend to begin considering principled approaches to some of the
problems I have been working on for the past several years.  For
example, much of my work on network monitoring to date has focused on
using off-the-shelf machine learning and statistica inference algorithms
to solve network operations and security problems.  In the coming year,
I hope to begin exploring how redesigning network protocols could
improve inference.  I also intend to develop my work on Internet
censorship with a release of a working anti-censorship system.

% improved visibility for research
% ultimate deployment of systems

I have two logistical goals for this year.  First, I plan an increased
focus on the {\em development and deployment of real-world systems}.
Much of our research has matured to the point where the ideas,
algorithms, concepts, and prototypes can be transferred to practice.
There are several opportunities for tech transfer, deployment, and
possibly even commercialization, which I discuss in more detail
below. Second, I plan to spend time {\em improving the visibility} of
the work coming out of my research group, with a Web site describing
active projects and a blog discussing current research issues.

\subsection*{Network Security}

Our work on spam filtering has matured; this year should constitute the
final push to get the last wave of systems and papers deployed on spam
filtering.  In the past year, we developed SNARE (Spatio-Temporal
Network-level Automated Reputation Engine), a supervised learning
algorithm that differentiates spammers from legitimate email senders
based solely on properties that can be gleaned from network-level
traffic (and often from a single packet!).  This work has been rejected
a couple of times---from IMC and from NDSS---but it is in resubmission
to USENIX Security.  I am hoping that this work is accepted, since the
student, Shuang Hao, could certainly use a confidence boost with a paper
acceptance at a top-tier conference.  If it is not accepted to USENIX
Security, we will likely resubmit the work to CCS.  From a practical
standpoint, the work is mature, and is seeing adoption by major spam
filtering companies (e.g., Cisco/Ironport) and mail service provider
(e.g., Yahoo).  The student who led this work, Shuang Hao, is going to
Yahoo for the summer to work on deployment of some of these ideas into
Yahoo's spam filtering system.

The other aspect of our spam filtering work is the development and
deployment of a real-time sender reputation blacklist, based on the
SNARE algorithms and the SpamTracker algorithms (previously published at
{\em ACM CCS}).  The system development has been progressing somewhat
slowly---I had hoped that this deployment would have been complete by
now, but some aspects of the system (in particular, the integration of
the algorithms) has stalled somewhat.  We spent time last summer
developing the platform for serving a real-time blacklist, but the
system is still not complete.  We submitted to NSDI, and the paper was
rejected. We are aiming for publication at a security conference, or
ultimately a submission to NSDI, this year.  In my opinion, this work
needs to be completed in order for Anirudh to graduate.

There are also two new directions that we are starting along these
lines: a taint-tracking system called Pedigree, and a dynamic access
control system that may ultimately form the basis of a new architecture
for the Georgia Tech authentication and access control system.  With the
advent of programmable switches (specifically OpenFlow), we have a
unique opportunity to develop and deploy the algorithms and systems that
result from our research on the Georgia Tech campus network.  We have
submitted a com,plete design of this system to the SIGCOMM {\em WREN}
workshop on enterprise networks.  This year, I will take steps towards
enabling this ultimate deployment, with help from Ron Hutchins at OIT.
I am also planning to submit a full conference paper based on this
topic.

% deployment of spamspotter - old!
% - get the paper published
% - yahoo
% pedigree - deployment on testbed that we're building 
% - implementation and deployment.  commercialization?
% - nsdi submission
% implementation/deployment of dynamic access control
% - nsdi submission


\begin{itemize}
\itemsep=-1pt
\item Deploy SpamSpotter, the real-time DNS-based blacklist for spammer
reputation. 
\item Work with students to transfer algorithms developed in the SNARE
and SpamTracker work to real-world spam filtering systems and mail
service providers (e.g., Yahoo).
\item Develop a prototype implementation of Pedigree and submit this
work to {\em NSDI}.  Explore possible commercialization opportunities
along these lines.
\item Continue the development of the dynamic access control work,
including deployment and testing on the testbed that we are developing.
Submit a version of this work to {\em NSDI}.
\end{itemize}

\subsection*{Virtual Networks and Network Architecture}


This year marked the start of a virtual network testbed at Georgia Tech,
as part of the larger GENI effort.  As part of our GENI contract,
``Bringing Experimenters and External Connectivity to GENI'', we have
developed a topology creation service, as well as a mechanism for
bringing external connectivity to the campus testbed (this will be the
same testbed as the one mentioned above, on which we perform the
OpenFlow-related work).  The second aspect of the project, bringing
external connectivity to the testbed, is also progressing.  We have
worked with Georgia Tech's office of IT to establish a direct, layer two
connection to our research cluster, as well as an external BGP session.
This capability will allow us to essentially operate our own
experimental autonomous system within the campus network.  I intend to
use this platform for continued research on virtual networks, as well as
a teaching tool for more ``hands on'' networking and systems assignments
(see the Education section below, as well).

From a prototyping standpoint, my goal is to complete the implementation
of the BGP session multiplexer, which will allow virtual networks in
this testbed to have global connectivity.  We will also integrate our
framework with Emulab, at last achieving the goal of having a virtual
network testbed that is integrated with Emulab's ``ProtoGENI'' front
end.

One of the major challenges of this work has been to generate
publishable research.  Along these lines, however, I have come up with a
new concept---Cloud Networking---that we will develop, using the
underlying framework we have built as the platform.  The concept is as
follows: many providers of distributed services and other enterprises
need networks, but they do not have the desire or capability to maintain
their own network infrastructure.  The platform we are developing,
however, could serve as a platform by which these providers might be
able to ``rent'' parts of their network from a shared physical
infrastructure, in much the same way that today user's can rent CPU
cycles from cloud computing clusters.  We will spend the summer
developing this idea in conjunction with the completion of the BGP
multiplexer and topology creation service for a possible submission to
{\em NSDI}.

Finally, continuing on the themes we began with the VINI work, we will
continue to develop technologies for accelerating packet forwarding in
virtual networks.  Specifically, we have been developing techniques for
accelerating packet forwarding in virtual networks using the NetFPGA
platform.  We have submitted a preliminary version of this work to the
SIGCOMM VISA workshop (on virtualization); we are planning a full
conference paper submission to NSDI this fall.

% completion of BGP session multiplexer
% deployment of FPGA-based virtualization platform, submission to NSDI
% NSDI submission of ``cloud computing for networking''

\begin{itemize}
\itemsep=-1pt
\item Completion of the BGP session multiplexer and topology creation
service for the GENI project.
\item Integration of these technical pieces into a ``cloud networking''
platform, with planned submission to {\em NSDI}.
\item Deployment, testing, and integration of hardware-accelerated
virtualization, with planned submission to {\em NSDI}.
\end{itemize}

\subsection*{Secure, Scalable Network Monitoring}

Our work on secure, scalable network monitoring has become more focused,
along two lines: (1)~monitoring for spam, botnets, and phishing attacks
on high-speed networks; (2)~developing mechanisms for secure, in-band
network troubleshooting.  The former is maturing and is based largely on
our exiting spam work; the latter is a resurrection of some work from a
couple of years ago.

We have been working with BBN on a DARPA contract to integrate our spam
filtering algorithms (described above) into a hardware-based system that
can be deployed on military networks and monitor traffic at very
high speeds (upwards of 10~Gbps).  It is unclear whether there are
interesting research problems as part of this project, but there are two
possibilities.  The first possibility is that monitoring at high speeds
will require intelligent sampling; if necessary, we could build on our
earlier traffic sampling work (FlexSample, which appeared at {\em IMC
2008}) to help develop better sampling algorithms for spam and botnet
detection at high speed.  The second possibility is that the hardware
implementation of the algorithms will give rise to new research
problems; I am sending one student to BBN for the summer to work on these
hardware implementations in hopes that we can get a conference
publication from the work.

One a related note, another student, Yiyi Huang, has been working on
developing streaming algorithms to detect botnets in real time (i.e., a
real-time version of BotMiner).  The work is in submission to USENIX
Security; if the work is not published there, our goal is to send this
work to another venue where it is more likely to be published (e.g.,
RAID).

\begin{itemize}
\itemsep=-1pt
\item Deploy monitoring algorithms at BBN.  Aim for
a conference publication based on this work.
\item Get Yiyi's work on real-time botnet detection published at a first
or second-tier conference.
\end{itemize}


\subsection*{New Directions}

% programmable networks - OIT/GENI involvement
% cross-domain troubleshooting
% social networks - ?
% contract enforcement

One of the main directions I see with ample opportunity for new
directions and impact is the refactoring of network security and
availability based on programmable layer-two switching technology (i.e.,
OpenFlow-enabled switches).  I plan to spend much of my effort in the
coming year exploring how this technology might be used in three
specific scenarios: (1)~information flow control and tracking in
networks (specifically, the Pedigree work); (2)~dynamic access control
in enterprise networks (specifically, the design, implementation, and
deployment of a new network access control framework for the Georgia
Tech campus network); (3)~improving capacity and availability in data
center networks, using technologyies that we have previously developed
(specifically, path splicing).

Another direction that I continue to explore is how better market design
can improve the connectivity and efficiency of Internet routing.  We
submitted a paper to SIGCOMM that proposes a market for Internet
transit.  I plan to continue to explore this area, both from an economic
perspective and in terms of figuring out possible implementation and
deployment strategies.  

%%%%%%%%%%%%%%%%%%%%%%%%%%%%%%%%%%%%%%%%%%%%%%%%%%%%%%%%%%%%

\section*{Education}

% integration of testbeds as a teaching tool
% develop course on next-gen network management, etc.
% solidify course content 
% better mentoring of students

With much of my work on CS~7001 completed, I would like to turn my
attention to solidifying this course content, in the form of a monogram
or book, so that the course content can be used as a model for others
(both others teaching the course here at Georgia Tech and other
computer science departments).   

I would also like to turn some of my attention towards developing more
hands-on material for networking, systems, and security courses, based
on the network testbed technology that we are developing.  Specifically,
the deployment of the campus network testbed as part of the GENI project
may make Georgia Tech the only campus in the world where students can
experiment with programmable network devices {\em and} have external
connectivity (i.e., they will essentially be able to act as network
operators within their own AS).  I would like to design a component of
some networking course whereby students can use this testbed to learn
about network security, operations, and design, as well as to develop
and test new network protocols and architectures.  

Finally, with my group growing large, I intend to pay particular
attention to my mentoring of students.  I will ensure that I meet my
students regularly, and give them rigorous feedback on their papers and
talks, so that these students develop a culture of high-quality output
that is seen at other top-ranked schools.

\begin{itemize}
\itemsep=-1pt
\item Integrate the emerging network testbed as a teaching tool in
networking, security, and systems courses. 
\item Solidify CS~7001 course content as a monogram or short book.
\item Provide better mentoring of students: more timely and rigorous
feedback on papers, talks, etc. 
\end{itemize}

%%%%%%%%%%%%%%%%%%%%%%%%%%%%%%%%%%%%%%%%%%%%%%%%%%%%%%%%%%%%

\section*{Service}

I have been somewhat over-extended with service in the past couple of
years, so I don't plan to do anything overly ambitous as far as service.
One thing I might like to do is to organize a network operations
workshop in conjunction with one of the NANOG meetings.  I will also
continue my activity as organizer of the DIMACS workshops on securing
Internet routing.  As organizer of the networking seminar in the fall, I
plan to make a concerted effort to bring in a series external speakers
on network security.  In my opinion, the seminar could be much more
lively with a better selection of external visitors and speakers.

I will remain active on program committees, though I plan to continue to
narrow my focus so that I am only serving on the PCs of the top
conferences.  I also hope to, and plan to, remain active in faculty
recruiting in the coming years.

% sigcomm poster/demo chair
% possible routing/network management workshop in fall?
% seminar with external speakers
% faculty recruiting committee

\begin{itemize}
\itemsep=-1pt
\item Do a good job as SIGCOMM demo/poster co-chair.
\item Organize a network operations workshop to be held in conjunction
with NANOG.
\item Organize a network security seminar at Georgia Tech with external
speakers. 
\item Continue service on various program committees.
\item Continue service on the faculty recruiting committee.
\end{itemize}


%%%%%%%%%%%%%%%%%%%%%%%%%%%%%%%%%%%%%%%%%%%%%%%%%%%%%%%%%%%%

\section*{Enablers}


I think at this point I am well-equipped to succeed on the above goals.
One possible exception is that I woul like to develop more hands-on
networking assignments (which, apart from being one of my own goals, is
a committment I have made to the GPO), but, as I am not teaching a
networking course next year, I may not be able to spend adequate time
doing this.  I may simply develop a few example problems for inclusion
in networking courses that I teach in future years.

Over the past year, I have encountered significant obstacles in our
finance office, which have consumed a large fraction of my time.  On one
end, I have had to track down missing awards.  On the submission side of
things, operations became so bad that I have ended up doing the
administrative parts of three different grants already this year.  As
someone who needs significant funds to keep my research group going, I
would like much more streamlined support from our financial office.

Our group is also experiencing technical growing pains.  While TSO is
helpful for some things (and, in particular, Debbie Davis has been
incredibly supportive and helpful in our testbed deployment---please do
whatever we can to keep her!), they are not helpful in maintaining basic
research infrastructure (which we have had to do ourselves for years).
I need some way to pay a system administrator affordably; I would like
to be able to do this without having to pay for system administration
twice.  In the past, Russ Poole has mentioned the possibility of
providing partial support for a technical staff member that could
support my research group; I would like to pursue this possibility in
earnest, and could use the department's support in doing so.

Finally, my students could benefit from lab space where they could
interact more directly with students across areas (e.g., with network
security).  Department support for construction of the Internet War Room
will help significantly with this effort.

% retaining teaching relief
% a more cohesive lab space - Internet War Room
% better support from the financial office
% better technical support
% retaining money for a postdoc

\begin{itemize}
\itemsep=-1pt
\item Better support from the financial office.
\item Better support from TSO; if possible, financial support to hire
our own system administrator.
\item Opportunity to spend time developing course materials.
\item A lab space that is more conducive to cross-area research (e.g.,
the Internet War Room).
\end{itemize}


\end{sloppypar}
