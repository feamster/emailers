\newpage \setcounter{page}{1}
\lfoot[Printed: \today]{Nick Feamster}
\cfoot{\fancyplain{Revised: June 12, 2013}{Research Statement}}
\rfoot{Page \thepage~of 11}


\begin{center}
{\Large\textbf{Research Statement}}\\[0.1in] {\large\textbf{Nick
Feamster}}\\
%feamster@cc.gatech.edu \\
%http://www.cc.gatech.edu/$\sim$feamster/ \\[.1in] 
\end{center}

\section*{Summary}

{\bf My research focuses on developing tools, algorithms, and protocols
that make the network easier to manage, more secure, and more
available.}

Nobody notices when the network works well, but everyone suffers when it
doesn't.  Thus, communications networks should be both secure and
available.  Network {\em security} has many facets, ranging from the
ability to stop ``unwanted traffic'' (e.g., spam and denial-of-service
attacks) to the ability to trace back attacks to their perpetrators
(``accountability'').  {\em Availability} means that the network must
provide good performance for users whenever they want to use
it---unfortunately, the increasing complexity of the network, coupled
with hardware faults, software bugs, misconfigurations, and malice, make
it difficult to achieve this goal.  Unfortunately, these two important
goals have also been among the most evasive.  Breakthroughs require not
only extensive domain knowledge, but also the ability to techniques from
a wide range of areas, ranging from economics to machine learning.  My
work combines domain knowledge, extensive interactions with network
operators, techniques from a wide range of disciplines, and---perhaps
most importantly---the competence and tenacity to implement and deploy
these systems in practice.  This unique combination has allowed me to
build one of the few networking research groups in the world that
interacts directly with network operators to deploy fundamentally new
systems and technologies in real-world networks.

I discover interesting and challenging practical problems through
frequent discussions and meetings with network operators and people in
industry.  I then tackle these problems from first principles, develop
new methods, and transfer these solutions back to practice in the form
of working systems.  I have tackled a variety of problems in network
operations, ranging from real-time network diagnosis to stemming
unwanted traffic like spam to architectures for fast failure recovery.
Many people---most notably, operators ``in the trenches''---are also
working on these problems.  Unfortunately, many of the people who have
the domain knowledge that best equip them to solve these problems are
busy with day-to-day operations, putting out fires as they arise but
rarely taking time to think about fundamental changes to the network
that might eradicate these problems.  My research fills this niche.  I
first devise methods to understand the nature of the problem in
practice.  I tackle domain-specific problems with tools and techniques
from other disciplines---ranging from machine learning to economics to
program analysis---whose principles might provide insights into a new,
previously undiscovered solution.  I then devise a new approach or
solution, and I transfer it to practice through implementation and
deployment of real-world systems.

My research in this broad area is currently focusing on several themes:
(1)~Internet censorship and open access; (2)~home and access networks;
and (3)~software defined networking.  These themes, which I have been
developing in the past several years since receiving tenure, build on
the broader research themes I have developed on network security and
operations.  I first survey the new leadership roles that I have assumed
in research, teaching, and service.  Then, I discuss each of the new
research themes I have developed since tenure and the impact that they
have had on both other researchers and on industry.

%% My work in this broad area follows four themes: (1)~making the network
%% more secure; (2)~improving network availability and performance by
%% making the network easier to operate and manage; (3)~designing platforms
%% for virtual networks that facilitate technical innovation in both
%% network security and operations; (4)~making performance and reachability
%% problems more transparent to users.  
%% The first two themes involve developing solutions that make the network
%% more robust and resilient in the face of faults, misconfiguration, and
%% malice; the third theme provides an avenue to evaluate and deploy these
%% solutions in practice.  The fourth theme concerns my ongoing and planned
%% work concerning emerging problems and threats in the battle for
%% information and influence on the Internet.

\section*{Highlights as an Associate Professor}

\paragraph{Research.}  Since I received tenure, I have expanded my
research along three themes: Internet censorship and information
control, home and access networks, and software defined networking.  My
work in these areas over the past three years produced four {\em
SIGCOMM} papers, five {\em SIGCOMM Internet Measurement Conference}
papers, one {\em SIGMETRICS} paper, one {\em USENIX Security} paper, and
more than \$5M in research funding.  My work in home networking has
already received several awards, such as the IRTF Advanced Networking
Research Prize and a Research Highlight in {\em Communications of the
ACM}.  Our home networking work is now being commercialized, and our
technologies and Huawei is now attempting to license our innovations in
Software Defined Networking (SDN) from Georgia Tech.  According to
Google Scholar, as of October 2013, {\bf my h-index is 41 and my
citation count is over 6,300}, comfortably placing me in the top-tier of
highly cited senior computer science researchers worldwide.

I briefly provide some specific highlights of my accomplishments in the
past three years.

\begin{itemize}
\item {\em Home and Access Networks.}  We began our research studying
the performance of home networks and mobile networks in June 2010, when
we began studying the performance of DSL networks in France.  Upon
realizing that accurate measurements would require deploying
infrastructure in the home router itself, we began developing BISmark
(Broadband Internet Service Benchmark), custom router firmware which has
now been deployed in more than 300 home networks in nearly 30 countries
around the world.  We also developed a version of this software that
runs on Android phones and has been installed by more than 4,000 users
in 130 countries.  The testbed that we have developed is the first of
its kind to study access and cellular networks, and our work
characterizing broadband Internet performance has already produced two
{\em SIGCOMM} papers, a {\em Communications of the ACM} journal article,
and several other workshop papers.  The work has won three prestiguous
awards: the IETF Advanced Networking Research Prize, an ACM {\em
Communications of the ACM} research highlight, and the {\em ACM SIGCOMM
Community Award Paper} at the {\em SIGCOMM Internet Measurement
Conference}, for recognition of research that contributes broadly to the
research community.  Our research has also garnered more than \$2M in
funding from various funding agencies, as well as initial seed money for
commercialization.
%
\item {\em Censorship and Information Manipulation.} Although my first
work on censorship circumvention dates back to my work on the Infranet
system in 2002, we started this work again on a system called Collage,
which appeared in the {\em USENIX Security Symposium} in 2012 and has
received attention from multiple news outlets, including {\em The
Economist}, {\em Slashdot}, and {\em Ars Technica}.  Recently, I founded
the {\em USENIX Workshop on Free and Open Communications on the
Internet} and I successfully led a new large \$3M NSF project (awarded 2012)
on censorship measurement and circumvention.  In 2011, I also received
a \$1.5M Google Focused Research Award (with co-PI Wenke Lee) on
Internet Transparency.  This work has also continued, in collaboration
with Wenke Lee, and has expanded to include {\em profile
pollution attacks}, whereby an attacker can affect the content that a
user sees (\eg, search results, recommended products) by polluting the
users profile with cross-site scripting attacks.
% 
\item {\em Software Defined Networking.} Our work on software defined
networking over the past three years has led to one {\em SIGCOMM} paper,
a journal article in {\em IEEE Network Magazine}, and several workshop
papers in the {\em ACM SIGCOMM Workshop on Hot Topics in Software
Defined Networking (HotSDN)}, a workshop that I co-founded in 2012.
Additionally, our work on event-driven network control is being licensed
by Huawei; I have presented our work on event-driven network control at
invited keynote presentations at the {\em Open Network Summit} (and
industry forum that draws more than one thousand attendees), the {\em
Internet Research Task Force}, and the {\em IEEE Conference on Network
and Service Management}.  This research has also appeared in many popular
technical publications such as {\em Ars Technica}.
\end{itemize}

\paragraph{Teaching and Outreach.}
I have begun to disseminate the results of my research and teaching
efforts through online media that can reach a much greater set of
people.  I have developed two blogs---one about my own research and one
about research methods.  I have also developed the first Massive Open
Online Course (MOOC) on Software Defined Networking, which is currently
being offered to a course of 33,000 students.  I briefly describe these
significant outreach efforts below:
\begin{itemize}
\item {\em Research Methods Blog.} I have turned the 
notes for CS~7001 (the Instruction to the Ph.D. course that we designed)
into an online book, at \url{http://greatresearch.org}.  As of October
2013, the site has more than 25,000 in a little more than two months,
and receives about 10,000 page views every month.
%.
\item {\em Coursera Course.} To help a larger number of people learn
about the history of Software Defined Networking (SDN), I have developed
a Massive Open Online Course (MOOC) on SDN, which is currently being
offered to more than 50,000 students.  About 4,000 students remained
actively engaged in watching videos and participating in forums, and
nearly 1,000 successfully completed all of the programming assignments.
I have been making video lectures on a variety of topics ranging from
the history of SDN to its uses and applications, and I intend to use
these videos as part of a ``flipped classroom'' seminar for students at
Georgia Tech this fall.
%
\item {\em Research Blog.} I have been blogging about various networking topics
at \url{http://connectionmanagement.org}; each post receives more
than 150 views.  I also received two awards: The 2012 Hesburgh Teaching
Fellows award from Georgia Tech, and the Bronze Anvil journalism award
for an article I wrote on Internet censorship that appeared in the {\em
Wall Street Journal} in 2011.  This blog has had nearly 10,000 views and
has more than 100 regular readers.
%
\end{itemize}

\paragraph{Service.}  My external service has included acting as program
committee co-chair for a major top-tier conference in my area ({\em
USENIX Symposium on Networked Systems Design and Implementation (NSDI)}, the
poster and demo co-chair for the other major conference {\em ACM
SIGCOMM}.  I also founded two workshops in areas where I have focused my
research recently: the {\em USENIX Workshop on Free and Open
Communications on the Internet} (co-founded with Wenke Lee) and the {\em
ACM SIGCOMM Worskhop on Hot Topics in Software Defined Networking
(HotSDN)}.  I elaborate on these events and others where I hold
significant leadership roles below:
\begin{itemize}
\item {\em NSDI PC Co-Chair, SIGCOMM Poster/Demo Co-Chair.} One of my
most significant recent service accomplishments was to serve 
as the program committee co-chair for {\em USENIX Networked Systems
Design and Implementation} in 2013.  The conference is the premier
conference for networked systems research and just completed with the
largest attendance in history (about 260 registered attendees).
\item {\em Founder of two new workshops.} I founded the {\em ACM SIGCOMM Workshop on
Hot Topics in Networking (HotSDN)}, which had about 150 participants in
August 2012 and about 80 submitted papers; and the {\em USENIX Workshop
on Free and Open Communications on the Internet (FOCI)}, which is now a
regular workshop at the {\em USENIX Security Symposium} and regularly
gets about 50 attendees.
\item {\em Founder and Co-Chair of IRTF Working Group.}  I was recently selected as one of the
founding co-chairs of the Software Defined Networking Research Group at
the Internet Research Task Force (IRTF), a division of the Internet
Engineering Task Force (IETF).
\end{itemize}
\noindent
In addition to external service, in recent years, I have continued my
service to Georgia Tech by co-leading the CS Ph.D. visit weekend
committee (including organizing a ``College of Computing Research Day''
for many of the years we have had the visit day, and serving on the
committee to design the Master of Science degree based on massive open
online courses (MOOCs).

\noindent
In the remainder of the research statement, I outline my accomplishments
and research themes in more detail.  I first survey three research
themes that are new since receiving tenure; I then discuss my more
established research themes and my accomplishments in those areas.
Finally, I discuss my plans moving forward.


\section*{New Theme 1: Home and Access Networks}

Access networks (\ie, cellular networks and home broadband networks) are
proliferating: Over 90\% of US households now have Internet access, and
networks have become an essential part of every home.  Video streaming
already accounts for over 60\% of the peak download bandwidth for the
Internet;
%http://www.sandvine.com/news/pr_detail.asp?ID=340];
remote learning is flourishing, with Khan Academy alone delivering over
86 million videos; and within five years, Forrester
Research expects 63 million Americans to telecommute from
home. Bandwidth to the home is also growing
rapidly: Huge investments by industry and government mean that over 60\%
of US homes have broadband access.  Inside the
home, 55\% of traffic is delivered to game consoles, set-top boxes,
smart TVs, and mobile devices.  Further, cellular
networks have become the predominant mode of Internet access for many
people: For example, in Brazil, Russia, India, China, and Indonesia,
there are 610 million Internet users, but 1.8 billion mobile-phone
connections.

Towards providing better {\em transparency} to users concerning their
Internet service, I am developing objective, independent third-party
services for users that help them both determine whether their Internet
service provider or government is restricting access to certain content
or services or degrading service for particular applications and gain
access to information that they might not otherwise have access to.  My
research on Internet transparency is focusing on three areas: (1)~the
{\em performance} that they receive from their ISP; (2)~{\em
connectivity} to various Internet destinations; (3)~the {\em
information} that they can discover via search engines and social media.

To provide users better information about the performance that they are
receiving, I started Project BISmark (\url{http://projectbismark.net})
in 2010; BISmark is a software platform for home routers.  We have
already used BISmark to develop a network measurement suite for access
Internet service providers; our first paper on BISmark appeared in {\em
ACM SIGCOMM} in 2011.  With collaborators in programming languages and
human-computer interaction, I am now exploring ways to use BISmark to
simplify the management of home networks by applying some of the same
network management principles that we have learned in our studies of
transit and enterprise networks.

\fp {\bf Impact.} 
Our results from the initial BISmark study influenced the design and
implementation of the performance measurements used by the Federal
Communications Commission's study of broadband connectivity across the
United States.  The project has been featured in {\em Ars Technica} and
{\em GigaOm} and has received over 20,000 signups from interested users.
We have currently deployed BISmark routers in about 250 home networks
around the world; it is also currently deployed on Google's Measurement
Lab.  To transition some of the technologies we are developing in
research to practice, I participated in Georgia Tech's venture program,
Flashpoint, to scale our efforts to a larger number of users and learn
more about the problems faced by ISPs, content providers, and consumers.
We also received an NSF Innovation Corps grant and Georgia Research
Alliance grant to help us commercialize this technology.

More recently, we have been expanding our work on BISmark across
developing countries and across a broader range of devices.  For
example, we recently completed a study with Research ICT Africa (RIA) to
characterize fixed and mobile performance across South Africa; we are in
the process of expanding this study to other countries in Africa.
Second, we have developed a home network performance troubleshooting
tool that helps users identify whether performance bottlenecks are
within their home network or in the Internet service provider (ISP)
network.  The Federal Communications Commission (FCC) has recently
agreed to back the deployment of our software in 4,000 home networks
across the United States, and Comcast has also recently agreed to a
trial deployment of this software. 

Beyond the impact of the technology itself in industry, I have been
developing the BISmark platform as an educational tool.  In Summer 2011,
I hosted a BISmark ``summer camp'' at Georgia Tech to help students
become familiar with programming network applications on the OpenWrt
router platform; the week-long event was attended by about twenty
students and faculty members from across the United States, France, and
Italy.  I have incorporated much of the material into the graduate
networking course at Georgia Tech, to give students hands on experience
with developing and deploying a variety of network measurement tools.
Through these activities, I aim to provide students both concrete
exposure to problems and concepts in networking and a platform on which
they can innovate.

\fp {\bf Most Cited Publication (81 Citations)}
\begin{quote}
S. Sundaresan, W. de Donato, N. Feamster, R. Teixeira, S. Crawford, A. Pescape 
``Broadband Internet Performance: A View From the Gateway''
{\em ACM SIGCOMM}, Toronto, Ontario, Canada. August 2011. {\bf Winner of
the IETF Advanced Networking Research Prize.  Selected for
Communications of the ACM Research Highlights.}
\end{quote}

\fp {\bf Representative Publication}
\begin{quote}
S. Sundaresan, W. de Donato, N. Feamster, R. Teixeira, S. Crawford, A. Pescape 
``Measuring Home Broadband Performance''
{\em Communications of the ACM}, Volume 55, Number 9.  September 2012.
\end{quote}


\section*{New Theme 2: Internet Censorship and Information Manipulation}

Free and open access to information and communications on the Internet
is at risk: the Open Net Initiative reports that nearly 60 countries
censor some access to information on the Internet.  Similarly, ISPs can
degrade network performance for certain subsets of users for some or all
services.  For example, some ISPs have been found to routinely block or
throttle certain application traffic (e.g., BitTorrent); additionally,
studies of access network performance in the United Kingdom and France
have revealed that the level of performance that users achieve in their
homes is sometimes as little as half of the rates that ISPs advertise to
their users.  Although it may not be feasible to always guarantee open,
unfettered access to information, users should know when their access to
information has been obstructed, restricted, or tampered with.

\paragraph{Circumvention tools.}
I am actively developing techniques that help users gain access
to information that they might not otherwise see as a result of
overt censorship.  Ten years ago, I developed Infranet, a tool to
circumvent Internet censorship that was both robust to blocking attempts
and deniable---meaning that an adversary could not easily detect that a
user was engaged in activities to circumvent censorship; the work won
the Best Student Paper Award at the {\em USENIX Security Symposium} in
2002.  Recent developments, such as the rise of user-generated content,
have made it easier to deploy censorship circumvention systems, since
sites that host user-generated content can be used as covert ``drop
sites'' for messages; based on this insight, we designed and implemented
Collage, a tool that allows users to circumvent censorship firewalls by
building covert channels into user-generated content.  Collage was
presented at the {\em USENIX Security Symposium} in 2010; it has been
downloaded hundreds of times and appeared in various news outlets
including {\em Ars Technica}, {\em GigaOm}, and {\em Slashdot}.

\paragraph{Soft censorship and information manipulation.}
One of the growing threats to free and open access to information in the
coming years will be the emergence of ``soft'' forms of censorship, such
as intentional performance degradation, the spread of propaganda through
social media, and selective filtering or placement of search results.
To defend against these threats, I have begun developing techniques to
identify propagandistic behavior in social media and to allow users to
compare their search results with a baseline set of search results
assembled through crowdsourced measurements.  We have developed tools
such as Bobble (\url {http://bobble.gtisc.gatech.edu/}) and Appu
(\url{http://appu.gtnoise.net/}), both of which now have large groups of
users, to help users track online information manipulation and privacy.
Our work on search poisoning, whereby an attacker can affect the search
results that a user sees by polluting a user's search history through
cross-site request forgery (XSRF) attacks.  This work will appear in the
2013 {\em USENIX Security Symposium}.

\fp {\bf Most Cited Publication (107 citations)}
\begin{quote}
N. Feamster, M. Balazinska, G. Harfst, H. Balakrishnan, D. Karger.
``Infranet: Circumventing Web Censorship and Surveillance''
{\em Proceedings of the 11th USENIX Security Symposium}, San Francisco,
CA, August 2002.  {\bf Best Student Paper Award.}
\end{quote}

\fp {\bf Representative Publication}
\begin{quote}
S. Burnett, N. Feamster, S. Vempala
``Chipping Away at Censorship Firewalls with User-Generated Content''
{\em USENIX Security Symposium}, Washington, DC. August 2010. 
\end{quote}


\section*{New Theme 3: Software Defined Networking}

In 2002, Larry Peterson, Scott Shenker, and
Jon Turner argued that networking research had ``ossified'', because
researchers faced a huge deployment hurdle for deploying their research
in production environments, and also because the large stakeholders had
little incentive to allow disruptive innovation to take place.  Their
argument was essentially that, by ``letting a thousand flowers bloom'',
multiple networking technologies could be deployed in parallel, thereby
providing researchers a path to innovation.  The main research challenge
was how to design and implement a virtual network infrastructure that
supported this philosophy.  

Towards solving this challenge, I began working on network
virtualization during my postdoc at Princeton.  Network virtualization
allows multiple networks to operate in parallel on the same physical
infrastructure.  Although this concept is not new (commonly used Virtual
Private Networks, or ``VPNs'', come to mind as a prominent real-world
example of network virtualization), virtualizing all aspects of the
network infrastructure---in particular, both the links {\em and} the
routers themselves---holds great promise for enabling innovation.
Jennifer Rexford and I wanted to implement a new network protocol we had
designed at the end of my graduate career. Our plan was to use
PlanetLab---a large testbed with virtualized servers distributed around
the world---to do it.  Unfortunately, we quickly realized that PlanetLab
did not have the necessary functions to instantiate test {\em networks};
in particular, PlanetLab offered no functions for building virtual
routers and links, and also had no support for forwarding traffic at
high rates for virtual routers (e.g., every packet needed to be copied
several times at each node, significantly slowing the packet forwarding
rates).  These shortcomings caused us to pursue a larger project to
build such a testbed that would support the kinds of experiments that we
wanted to run.  With Andy Bavier and Larry Peterson, we built a Virtual
Network Infrastructure (VINI), a testbed that allows researchers to
build virtual networks.  This work appeared in {\em ACM SIGCOMM} in
2006.  The concepts behind virtual programmable networks, in concert
with some of our earlier work on the Routing Control Platform (RCP)
ultimately led to the advent of Software Defined Networking~(SDN)

Since this initial work, I have focused on three aspects of software
defined networking: (1)~providing Internet connectivity and routing
control to software defined networks; (2)~designing very fast packet forwarding
technologies for software defined networks; (3)~designing better
languages and control models for software defined networks.

A virtual network---either an experiment or a distributed ``cloud''
service---typically needs connectivity to the rest of the Internet so
that users can actually exchange traffic with it.  To provide such
connectivity, and to give each virtual network direct control over how
user traffic reaches it, I designed, implemented and deployed the
Transit Portal, a software-defined controller for interdomain routing
that provides individual virtual networks the illusion of having a
direct, physical upstream connection to multiple Internet service
providers.  This work appeared in {\em USENIX Annual Technical
Conference} in June 2010.  We performed several research projects to
follow up on this work, which used the Transit Portal to improve both
the reliability and performance of cloud-hosted Internet services.  This
follow-up work has appeared in {\em ACM SIGCOMM} in 2012, and {\em ACM
SIGMETRICS} in 2013.

The Transit Portal is also a cornerstone of the larger nationwide GENI effort
(featured here, for example: \url{http://www.geni.net/?p=1682}).  Our
work on designing faster packet forwarding technologies for virtual
networks started with the Trellis project, which moved packet forwarding
for virtual networks into the kernel; although this work resulted only
in a workshop publication, the software itself was adopted by University
of Utah's Emulab, the most prominent emulation-based testbed for
networking research.  Our current efforts have focused on accelerating
packet forwarding further by supporting custom packet forwarding for
virtual networks in Field Programmable Gate Arrays (FPGAs); our work on
SwitchBlade, a platform for rapidly developing and deploying custom
forwarding engines in hardware for virtual networks, appeared at {\em
ACM SIGCOMM} in August 2010.

Finally, I have been developing new control models and languages to
support event-based control for software defined networks.  I have
focused on how better control models and languages can help solve three
problems in network management: (1)~enabling frequent changes to network
conditions and state; (2)~providing support for network configuration in
a high-level language (including developing one of the first formal
languages for software defined networks, Procera); and (3)~providing
better visibility and control over tasks for performing network
diagnosis and troubleshooting.  With my students, I built and deployed
software defined networks in campus and home networks to demonstrate how
SDN can improve common network management tasks.  An early version of
this work appears in the February 2013 issue of {\em IEEE Network
Magazine}.

{\bf Impact.}  The impact of this work thus far has been to support
network experimentation for researchers; many other virtual network
technologies and platforms have built on this work. Our work on virtual
networks has been over nearly 500 times (the VINI paper has been cited
more than 300 times, and our work describing a network architecture
based around network virtualization has been cited over 200 times).

The Transit Portal is currently deployed in six locations---including a
recent deployment in the Amsterdam Internet Exchange in May 2013---and I
am using it in my courses to provide students with hands-on experience
configuring networks of routers and connecting them to real BGP-speaking
routers on the Internet.  The course I have developed that uses this
technology is likely serves as the first course where students can
directly configure networks of routers that are connected to the global
Internet.  The Transit Portal is also being actively used in research
and has supported many other research projects, including several
projects at the University of Souther California and the University of
Washington that have resulted in multiple independent research papers
that have appeared at {\em ACM SIGCOMM}.

\fp {\bf Most Cited Publication (448 citations)}
\begin{quote}
A. Bavier, N. Feamster, M. Huang, L. Peterson, J. Rexford. ``In VINI
Veritas: Realistic and Controlled Network Experimentation". {\em ACM
SIGCOMM}, August 2006. Pisa, Italy.
\end{quote}

\fp {\bf Representative Publication}
\begin{quote}
V. Valancius, B. Ravi, N. Feamster, A. Snoeren
``Quantifying the Benefits of Joint Content and Network Routing''
{\em ACM SIGMETRICS}
Pittsburgh, PA. June 2013.
\end{quote}
%% \begin{quote}
%% B. Anwer, M. Tariq, M. Motiwala, N. Feamster. ``SwitchBlade: A Platform
%% for Rapid Deployment of Network Protocols on Programmable Hardware''
%% {\em ACM SIGCOMM}, New Delhi, India, August 2010.
%% \end{quote}


\newpage


\section*{Established Theme 1: Network Operations}

A well-established of my work is {\em network operations}, which is what
I call the field of designing networks so that they are easier to run
and manage.  Much of my work in this area has focused on fault detection
and troubleshooting.  Prior to my dissertation work, operators relied on
detecting problems with networks ``at runtime'' on a live network.  My
dissertation work demonstrated that, in fact, many routing problems
could be detected simply by examining the configuration of the routing
protocols, before the configuration is even deployed.  I applied
techniques from static program analysis to routing configuration to help
network operators catch mistakes and predict dynamic network behavior
before the configurations are deployed on a live network, preventing
costly and catastrophic network downtime.

Beyond predicting behavior and proactively detecting configuration
faults, operators must understand the network's behavior {\em as it is
running} (e.g., to detect equipment failures, attacks, or unplanned
shifts in network traffic).  Unfortunately, operators are drowning in
heterogeneous data.  To help operators better understand network faults
``at runtime'', I have applied unsupervised learning techniques to
Internet routing data to help them efficiently mine the data for network
events that require corrective action.  This work appeared in {\em ACM
SIGMETRICS} in 2007.  My work has also applied statistical inference
techniques to help network operators determine the answers to ``what
if'' configuration questions in content distribution networks; we
developed a system called ``WISE'' (What-If Scenario Evaluator) to help
network operators determine the effects of configuration changes on
network response time.  A paper on this system appeared at {\em ACM
SIGCOMM} in 2008 and is now used by operators and network designers at
Google.  A more mature version of this work that also describes
deployment experiences at Google is in submission to {\em IEEE/ACM
Transactions on Networking}.

Users of communications networks also face the potential of intentional
performance degradation or manipulation by Internet Service Providers
(ISPs); these problems are popularly referred to as ``network neutrality
violations''.  This transparency can help users determine whether their
network is the cause of performance degradation, or whether performance
problems that they are seeing are due to some other cause.  With
students, I designed, built, and deployed the {\em Network Neutrality
Access Observatory (NANO)}, a system that aggregates measurements from
end systems to help users and operators of edge networks infer when
transit networks may be discriminating against certain types of traffic.
This work appeared in {\em ACM SIGCOMM CoNext} in 2009, and we have
deployed the system on Google's Measurement Lab
(\url{http://www.measurementlab.net/}).  More recently, we have been
looking at methods for helping users diagnose general problems with
access network performance and examining which factors have the most
influence on access network performance.

I have developed new network protocols and architectures that improve
availability and accountability in communications networks in the face
of both faults and malice.  Networks face the continual threat of
failures and attacks that disrupt end-to-end connectivity.  Prior to my
work, one promising approach to improving connectivity involved routing
traffic along multiple paths between two endpoints (``multipath
routing''); despite the promise of this approach, previous approaches
encountered two significant challenges: First, previous approaches for
disseminating information about multiple paths through the network did
not scale to large networks.  Second, end systems had no way to signal
to the network that an end-to-end path had failed or was providing
inadequate performance.  My research applied a new perspective to this
problem: rather than simply routing traffic on one of multiple paths to
a destination, allow traffic to switch paths at intermediate points en
route to the destination, and allow end systems to signal to the network
when it should attempt to use a different path to the destination using
a small number of bits that can be carried in the traffic itself.  This
system, called {\em path splicing}, provides up to an exponential
improvement in reliability for only a linear increase in the amount of
state that each router in the network must store.

{\bf New research since tenure.} Since receiving tenure, I have
continued to work on tools and protocols that help operators configure
their networks better.  To better understand how network operators make
changes to network configurations, we performed a study of the evolution
of network configuration over five years across two campus networks and
have clustered these changes into common tasks, with an eye towards
raising the level of abstraction for network configuration.  This work
appeared in the {\em ACM SIGCOMM Internet Measurement Conference} in
2011.  

Second, we have been actively developing systems on top of the Internet
routing infrastructure to help network operators optimize the
performance of their applications that run in the network.  We developed
a system called PECAN which jointly optimizes content routing (\ie, the
mapping of clients to service replicas) and network routing (\ie, the
network-level paths between clients and their respective replicas).  We
discovered that jointly optimizing network and content routing can
significantly improve performance over simply performing each operation
independently; our results will appear in {\em ACM SIGMETRICS} 2013.

Finally, we have performed a data-driven econometric analysis that
showed how a tiered pricing model can yield both higher profit margins
for Internet service providers and greater consumer surplus for users.
These results appeared in {\em ACM SIGCOMM} in 2011.


{\bf Impact.} The foundation of this research theme comes from a system
I built called called ``rcc'' (router configuration checker).  This
system was the centerpiece of my doctoral dissertation and has had
significant impact in both research and industry.  The work received the
Best Paper Award at {\em ACM/USENIX Networked Systems Design and
Implementation (NSDI)} in 2005 and has been used by hundreds of Internet
Service Providers (ISPs) around the world to check their network
configurations for errors.

The path splicing work resulted in a Sigma Xi undergraduate research
award for Megan Elmore.  The work was funded by Cisco, and they have
considered the possibility of extending their existing multiple routing
configuration (MRC) function to support path splicing.  A more likely
deployment scenario, however, may be the incorporation of path splicing
into a network where network elements are more programmable.  We have
published an open-source implementation of path splicing on several
programmable networking platforms.

Our work on tiered pricing was covered extensively in the media,
including in {\em The Economist}.

\fp {\bf Most Cited Publication (235 Citations)}
\begin{quote}
N. Feamster and H. Balakrishnan ``Detecting BGP Configuration Faults
with Static Analysis'' {\em 2nd Symposium on Networked Systems Design
and Implementation (NSDI)}, Boston, MA, May 2005. {\bf Best Paper Award.}
\end{quote}

\fp {\bf Representative Publication}
\begin{quote}
M. Tariq, A. Zeitoun, V. Valancius, N. Feamster, and
M. Ammar. ``Answering What-if Deployment and Configuration Questions
with 'WISE': Techniques and Deployment Experience". {\em IEEE/ACM
Transactions on Networking}, January 2013. (also appeared in {\em
SIGCOMM} in August 2008)
\end{quote}


\fp {\bf New Representative Publication (Since Tenure)}
\begin{quote}
V. Valancius, C. Lumezanu, N. Feamster, R. Johari, V. Vazirani
``How Many Tiers? Pricing in the Internet Transit Market''
{\em ACM SIGCOMM}
Toronto, Ontario, Canada. August 2011. {\bf Appeared in multiple popular
news venues, including {\it The Economist}}.
\end{quote}


\section*{Established Theme 2: Network Security}

\fp My research explores the role that communications
networks---in particular the network layer---can play in improving
computer and communications security.  This line of research began with
my arrival at Georgia Tech in 2006. A cornerstone of this
research is a system that was published in August 2009 called ``SNARE''
(Spatio-temporal Network-level Automated Reputation Engine). This work
appeared at the {\em USENIX Security Symposium}, a top-tier security
conference. The main idea behind SNARE---and the key insight behind my
research in spam filtering---is that spammers have different sending
behavior than legitimate senders.  Filters can distinguish
spammers from legitimate senders by examining their {\em sending
behavior} (i.e., how they send traffic), rather than what is in the
messages themselves.

Prior to my research, conventional spam filters attempted to distinguish
spam from legitimate email by looking at message contents: that is, they
would look at the words or language used in the messages themselves and
try to detect spam based on what the message said.  This approach has
become increasingly untenable, since spammers have begun to embed their
messages in all sorts of media, ranging from images to PDFs to audio
files to spreadsheets---by the time developers perfected their content
filters for one type of medium, spammers moved onto the next.  My line
of work has taken an entirely different, but complementary approach: I
look at features of the senders' {\em behavior} (e.g., the time of day
they are sending, whether there are other ``nearby'' senders on the
network, whether and how the sizes of the messages of the senders vary
over time) to distinguish spamming behavior from legitimate email use.
The method is harder for spammers to evade, it is more flexible because
it can be deployed anywhere in the network, and it can work at much
higher traffic rates than conventional approaches.  This idea was first
laid out in the initial award paper at {\em SIGCOMM} and finally
realized in the SNARE paper from August 2009 at {\em USENIX Security}.

I have also worked on sweeping changes to the Internet architecture that
could improve {\em accountability}, thus making it more difficult for
malicious parties to operate unfettered in the first place.  The current
Internet architecture provides little to no accountability. Malicious
end systems can conceal the source of their traffic (``spoofing''), and
edge networks can provide false information about their reachability to
various Internet destinations (``route hijacking''); both of these
attacks make it difficult to track down perpetrators of attacks.
Current approaches to solving these problems require manual
configuration and operator vigilance, which make them weak and
error-prone.  Towards building networks that are inherently accountable,
I have developed the Accountable Internet Protocol (AIP).  One of my
contributions to the design was to make the addresses in this protocol
self-certifying, which forms the cornerstone of the basic design.  I
also demonstrated how to apply AIP to secure BGP, the Internet's
interdomain routing protocol.

{\bf New research since tenure.} I have continued my research in network
security by studying how attackers use the underlying Internet
infrastructure to achieve {\em agility}.  In particular, we performed a
study that explored the initial DNS behavior of spammers that appeared
in the {\em ACM SIGCOMM Internet Measurement Conference} in 2011. We
also performed a second study that explored how attackers use the
Internet's interdomain routing protocol, Border Gateway Protocol~(BGP)
to evade detection when sending spam and performing other malicious
activities.  That study appeared in the {\em Passive and Active
Measurement Conference} in 2011.  Finally, we are exploring how to
prevent data leaks from cloud-based Web applications, even when the
applications themselves may be compromised.  We have one preliminary
paper in the {\em USENIX Workshop on Hot Topics in Cloud Computing
(HotCloud '11)}.


\fp {\bf Impact.}
My research in network security has had impact in research, in industry,
and on the national level.  My research on this topic has earned the
Presidential Early Career Award for Scientists and Engineers (PECASE), a
Sloan fellowship, and the Best Paper Award at {\em ACM SIGCOMM} (the
premier computer networking conference).  Aspects of my work have also
been incorporated into commercial spam filtering products and Web mail
clients at companies including Yahoo, Cisco/Ironport, and McAfee, as
well as a project for the Department of Defense on high-speed network
monitoring.  My paper on understanding the network-level behavior of
spammers---which won the Best Student Paper award at {\em SIGCOMM} in
2006---has been cited over 300 times since its initial publication in
August 2006---it spawned a variety of high-impact follow-on work,
including looking at network-level behavior not only to develop better
spam filters, but also to detect botnets more effectively and defend
against phishing attacks, click fraud, and other serious threats to the
Internet infrastructure.  I have also been working on similar approaches
to help detect and dismantle the Internet's scam hosting infrastructure
(e.g., Web sites that attempt to steal user passwords, money, and so
forth).  My initial paper on this topic (``Dynamics of Online
Scam-Hosting Infrastructure'') won the Best Paper award at the {\em
Passive and Active Measurement} conference in April 2009.

My work on SNARE has also garnered significant attention in industry.
This work was featured in {\em Technology Review} and on Slashdot (a
popular, high-traffic site for news in information technology).  Several
companies including Yahoo have incorporated the network-level features
that SNARE identifies into its spam filters, and companies that develop
spam filtering appliances, such as McAfee, are also using these features
to improve the accuracy and performance of their spam filtering
appliances.

AIP appeared in {\em ACM SIGCOMM} in 2008; an early version of the
design also appeared in {\em ACM Workshop on Hot Topics in Networking
(HotNets)}.  I am incorporating a version of this technology into a
working system and transferring them to practice.  I am working with BBN
on a DARPA project that will ultimately result in incorporating AIP's
mechanisms into a military network protocol that allows attribution of
traffic to sources (the details may ultimately be classified). 

My impact on the broader field of cybersecurity goes beyond my own
research.  I am also having impact in the national arena in several
ways.  Last year, I was involved in setting the nation's agenda for
cyber security, through multiple additional activities.  First, I led a
community-wide effort to develop a ``wish list'' document that describes
the security community's needs for access to better data---ranging from
network traffic data, to data about our country's infrastructure.  This
report was ultimately delivered to Tom Kalil, the deputy director for
policy in the Office of Science and Technology Policy.  Second, with
program managers Karl Levitt and Lenore Zuck at NSF, I organized a
community-wide, multi-agency workshop on ``Security-Driven
Architectures''.  The workshop included participants from computer
science, with an eye towards setting a research agenda for developing
more holistic approaches to computer security that consider {\em all}
aspects of computer and communications systems, rather than just a
single piece (like the network).  Finally, my work on developing
next-generation Internet protocols to improve accountability (which
could eradicate spam in the first place), based on work that appeared at
{\em ACM SIGCOMM} in 2008, was included in reports for the
National Cyber Leap Year.

Our recent work on DNS and BGP reputation has been patented and
implemented by Verisign, and is currently in use in several of their
security products.  Our recent work on detecting fraudulent voting on
webmail messages has been implemented and deployed in Yahoo's webmail system.

\fp {\bf Most Cited Publication (461 Citations)}
\begin{quote}
A. Ramachandran and N. Feamster. ``Understanding the Network-Level
Behavior of Spammers". {\em ACM SIGCOMM}, August 2006. Pisa, Italy. {\bf
Best Student Paper Award.}
\end{quote}


\fp {\bf Representative Publication}
\begin{quote}
S. Hao, N. Syed, N. Feamster, A. Gray and S. Krasser. "Detecting
Spammers with SNARE: Spatio-temporal Network-level Automatic Reputation
Engine". {\em USENIX Security Symposium}, August 2009. Montreal, Quebec,
Canada. 
\end{quote}

\fp {\bf New Representative Publication (Since Tenure)}
\begin{quote}
S. Hao, N. Feamster, R. Pandrangi
``Monitoring the Initial DNS Behavior of Spammers''
{\em ACM SIGCOMM Internet Measurement Conference},
November 2011. Berlin, Germany. {\bf Resulted in two Verisign patents.}
\end{quote}






%%%%%%%%%%%%%%%%%%%%%%%%%%%%%%%%%%%%%%%%%%%%%%%%%%%%%%%%%%%%
% next steps
% secure foundations: openflow, pedigree
% economic foundations: MINT
% open access: collage
% fusing machine learning+networks

