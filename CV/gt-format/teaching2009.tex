\newpage \setcounter{page}{1}
\lfoot[Printed: \today]{Nick Feamster}
\cfoot{\fancyplain{Revised: January 15, 2009}{Teaching Statement}}
\rfoot{Page \thepage~of 1}

\begin{center}
{\Large\textbf{Teaching Statement}}\\[0.1in] {\large\textbf{Nick
Feamster}}\\
%feamster@cc.gatech.edu \\
%http://www.cc.gatech.edu/$\sim$feamster/ \\[.1in] 
\end{center}

I believe that students learn best by doing.  Hands-on experience and
real-world examples can not only make classes more exciting, but they
also provide memorable examples and analogies that help students learn.
Our ability to understand abstract concepts is much more limited than
our ability to process concrete examples that relate to familiar
concepts and ideas.  Some of my most memorable classroom experiences
were lectures that were concrete and participatory: For example, in game
theory, I found the abstract concept of common knowledge much easier to
understand after playing interactive inference games with other
classmates in lecture.  Similarly, in networking, route hijacks or
spoofing become much more real when students can actually see network
traffic going where it isn't supposed to go.  Spamming and phishing
attacks become much more real with concrete examples that are appearing
in the ``real world'' on a daily basis.  Routing protocols make more
sense when students can configure routers, fail links on an experimental
network, and watch the behavior of network traffic over that network.
Unfortunately, students often take networking classes without ever
configuring a router or sending traffic over a network; this must
change.

As such, my first goal in teaching both networking and security classes
is to connect textbook material with (1)~real-world examples; and
(2)~hands-on experience.  I have made a sustained effort to imbue these
elements into every class I have taught.  At both the undergraduate and
graduate levels, I have incorporated course material that familiarizes
students with the state of the art in network design, implementation, and
experimentation; for example, I have students run experiments on network
testbeds like Emulab and VINI, with real software routers that they can
configure.  Where appropriate, I have also allowed students to shape the
course themselves by bringing in real-world examples, from which I will
design a lecture.  For example, in my undergraduate networking course, I
maintained a wiki where students could post relevant ``current events''
in networking and vote on which topics they would like to see covered.
Based on that input, I incorporated new material into the
syllabus---teaching not just the current event itself, but also the
relevant foundational material.  While this takes more effort than
dusting off old notes, I find that it not only keeps the students
engaged, but it also helps me keep abreast of what is happening both in
industry and in research.  In the future, I plan to integrate my
expertise and research in network testbeds and network operations to
give students even more real-world experience.

My second goal is to elevate course material so that students are not
just learning mechanics of protocols and systems, but also gaining a
deeper understanding of the {\em concepts} that underlie their design.
My reasoning here is two-fold.  First, I believe that the traditional
classroom lecture is``going the way of the blackboard'', to paraphrase a
recent New York Times article.  With so many computing and
communications tools for aggregating and processing information these
days, it is hard for me to see how a conventional lecture will continue
to be the most efficient way to convey textbook information.  In my
lectures, I try to go beyond what is taught in the textbook---rather
than teaching only mechanics of protocols, for example, I ask students
about design rationales, and whether they would make the same choices
today, given the changing roles of communications networks.  Second,
given that networking is still maturing as a field, there is a tendency
to focus on protocol details (particularly in textbooks); these details
may change out from under us very quickly.  Ten years from now, I would
like someone who took my class to be able to say that the concepts they
learned have remained applicable.  To this end, I try to organize the
topics I teach around higher-level concepts (\eg, randomization,
caching, tree formation, identity), so that even as protocol details
continue to change, the concepts that they take away remain valuable
assets.

I have a passion for teaching students how to think.  Along these lines,
I have been given the opportunity (with Alex Gray) to design a graduate
course that focuses on teaching first-year graduate students how to do
research.  The course includes topics ranging from paper reading to
fellowship applications to generating (and executing!)  research
ideas. To me, the value of the course is evident from student response:
Students at various points along their Ph.D.---even more senior
students---sit in on lectures.  We presented a paper on the course at
{\em SIGCSE} in 2008, and other computer science departments (\eg, Duke,
Princeton) are now looking at the course to see if they can adopt some
of the material or possibly even have a similar course.
