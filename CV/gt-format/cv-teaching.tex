% cv-teaching.tex
\section{TEACHING}
\label{sec:teaching}

\subsection{Courses Taught}
\label{subsec:courses}

\begin{center}
\begin{tabular}{lllrl}

& & \textbf{Number of \newline } &\\
\textbf{Term} & {\bf Year} & \textbf{Course Number \& Title} &
\textbf{Students} & \textbf{Comments} \\ \hline \\[\tabitemskip]
Fall & 2013 & CS~8001 Software Defined Networking & 15 & Flipped classroom \\
Fall & 2013 & CS~7001 Introduction to Graduate Studies & 36 & New blog \\


Summer & 2013 & Software Defined Networking (Coursera) & 50,000+ & First
SDN Course \\
Spring & 2013 & CS~3251 Computer Networking & 82 \\
Fall & 2012 & CMSC~330 Programming Languages & 100 \\
Fall & 2011 & CS~6250 Computer Networking & 51 \\
Fall & 2011 & CS~4235 Computer Security & 44 \\
Fall & 2010 & CS~6250 Computer Networking & 92 \\
Spring & 2010 & CS~3251 Computer Networking I & 53 \\
Spring & 2010 & CS~8803 NGN Next Generation Networking & 50 & New Course \\
Fall & 2009 & CS~7001 Introduction to Graduate Studies & 39 \\
Spring & 2009 & CS~6262 Network Security & 45 & Updated Syllabus \\
Fall & 2008 & CS~4251 Computer Networking II & 16 & \\
Fall & 2008 & CS~7001 Introduction to Graduate Studies & 44 \\
Spring & 2008 & CS~4251 Computer Networking II & 14 & New Syllabus\\
Fall & 2007 & CS~7001 Introduction to Graduate Studies & 53 \\
Spring & 2007 & CS~7260 Internetworking Protocols and Architectures & 29 \\
Fall & 2006 & CS~7001 Introduction to Graduate Studies & 74 &
New Syllabus \\
Fall & 2006 & CS~8001 Networking Research Seminar & 30 & New Syllabus\\
Fall & 2006 & CS~1100 Freshman Leap Seminar & 15 & \\
Spring & 2006 & CS~7260 Internetworking Protocols and Architectures & 27 &
New Syllabus \\
%Spring 2006 & CS~4251 Computer Networks II & -- & Multiple Guest Lectures \\
%Fall 2005 & CS~7001 & -- & Guest Lecture \\
\end{tabular}
\end{center}

\noindent
{\bf Other Highlights:}
\begin{itemize}
\itemsep=-1pt
\item Created the first-ever course on Software Defined Networking (SDN), and
delivered it on Coursera to over 50,000 enrolled students 
\item Tutorial on network measurement at African Network Operators Group
(AfNOG) in Summer 2013. 
\item Guest lecture on Internet censorship in Georgia Tech CS~4001 in October
2011. 
\item Tutorial on software-defined networking at African Network Operators
Group (AfNOG) in Summer 2011. 
\item Tutorials on BGP Multiplexer at GENI Experimenters Workshop and
  GENI Engineering Conference in Summer 2010. 
\item Tutorial on network security at African Network Operators Group (AfNOG)
in Summer 2010.
\item Tutorial on Internet routing at Simposio Brasileiro de Redes de Computadores (SBRC) in Summer 2008. 
\item Lecture for DIMACS Tutorial on Next-Generation Internet Routing
  Algorithms in August 2007. 
\item Guest lecture for CS~6250 (Advanced Computer Networks) in Fall 2007. 
\item Guest lecture for CS~3251 (Computer Networks I) in Fall 2006. 
\item Multiple guest lectures for CS~4251 (Computer Networks II) in Spring
2006.
\item Guest lecture for MIT Course 6.829 (Computer Networking) in Fall
  2005. 
\end{itemize}


\subsection{Curriculum Development}
\label{subsec:curriculum}

\subsubsection{Creation of New Content}

\paragraph{Coursera and Online Masters Program Development:} I have
developed an online Coursera course for the topic of Software Defined
Networking (SDN), an emerging topic in computer networking that is
reshaping how networks are defined.  In the course, students learn about
the history of SDN and develop hands-on experience with tools to develop
technologies and applications for SDN.  The course had approximately
50,00 students enrolled, and had about 4,000 students actively
participating on the forums and in lectures.  Nearly 1,000 students
successfully completed the programming assignments for the course,
qualifying as a pass ``with distinction''.  The course reviewed
extremely favorably from students, as we have documented here:
\url{http://goo.gl/So7Uis}.  Additionally, the success of the course was
covered on many technical forums, including the Mininet blog
(\url{http://goo.gl/ZqfL1N}), the Packet Pushers blog
(\url{http://goo.gl/okzd44}), and the sFlow blog
(\url{http://goo.gl/qSMCq0}).  Based on its overwhelming success, the
course will be offered again in May 2014.


\paragraph{Georgia Tech Online Masters Program:} 
I served on the committee to develop a Master of Science in
Computer Science degree based on Massive Open Online Course (MOOC)
offerings and took an active role in ensuring the creation of the online
degree at Georgia Tech.  As of October 2013, I am currently working with
Udacity and Georgia Tech to create the first online graduate course in
computer networking, which will launch in January 2014 and become an
integral part of the Georgia Tech online degree program.

\paragraph{CS 7001 Introduction to Graduate Studies:}
With Professor Alex Gray, I have developed a new course syllabus and
structure to CS 7001 around the larger goal of introducing new students
to {\em how to do great research} as soon as their first term at Georgia
Tech.  In contrast with previous terms, where CS 7001 consisted of
faculty ``advertisements'' for their research and projects consisted of
short ``mini-projects'' where little research could be accomplished in a
short time span of 3 weeks, we have improved the syllabus by bringing in
faculty members to talk about research philosophy, exciting new
directions, etc.  We have also given the students the option to do a
research project that is a term-long project in conjunction with CS
8903; our goal is to give students the flexibility to select meaningful
research problems based on their research assistantships while helping
them learn the skills required for writing papers, finding and
evaluating research ideas, and performing other tasks associated with
doing great research.  Alex Gray and I wrote a conference paper on our
development of this course, which appeared at {\em ACM SIGCSE 2008}.  We
have developed a website for the course, \url{greatresearch.org}, which
makes the material generally available for use at other universities and
by other researchers.  {\bf The website and blog have more than 25,000 views
since its launch in mid-August 2013, and it has been receiving about
10,000 views every month.}

\paragraph{CS 6250 Graduate Computer Networking:} In Fall 2011, I redesigned the
graduate computer networking course to focus more on current
technologies and hands-on assignments.  Conventional networking courses
treat today's protocols and mechanisms as fixed artifacts, rather than
as part of a continually evolving system.  To prepare students to think
critically about Internet architecture, Jennifer Rexford and I created a
graduate networking course that combines ``clean slate'' networking
research with hands-on experience in analyzing, building, and extending
real networks.  My goal was to prepare students to create and explore
new architectural ideas, while teaching them the platforms and tools
needed to evaluate their designs in practice.  The course, with
offerings at both Georgia Tech and Princeton, focuses on network
management as a concrete way to explore different ways to split
functionality across the end hosts, network elements, and management
systems.  I have refined the course in Fall~2011 to include more
hands-on assignments ad refactored the course around networking problems
in different types of networks: transit networks, home networks, content
hosting networks, and mobile and wireless networks.  {\bf Our work on
  the course received the best paper at the {\em ACM SIGCOMM Workshop on
    Networking Education (NetEd)} in 2011}.

\paragraph{College of Computing Research Day and Seminar Series:}
In addition to the course itself, to fulfill some of the functions of
the former 7001 course, Alex Gray and I financed and organized a
college-wide seminar series and research day in Fall 2009 and again in
Spring 2011.  Throughout the term, faculty speakers from across the
college gave one-hour talks about their research; we raised money from
Yahoo to support this event.  The research day brings together students
and faculty from around the college to see talks, demonstrations, and
posters from around the college to exchange ideas.

\paragraph{CS 8803 Next-Generation Networking:} I developed a
new graduate course that gives students practical experience with
a variety of tools for next-generation networking, ranging from the
Click software router to the OpenFlow switch framework.  The course also
teaches students about the state of the art in networking
research---students read papers about research and industry trends and
do a course project that incorporates aspects of these new
technologies.  This course relates to the larger nationwide effort on
Global Environment for Network Innovations (GENI), which is building
infrastructure for researchers to provide the next generation of
networking protocols and technologies.

\subsubsection{Revision of Existing Content}

\paragraph{CS 4251 Computer Networking II:}
{\em Spring/Fall 2008.} I developed new hands-on assignments to give
students experience with real-world networking tools and software (\eg,
Emulab, Quagga, Click).  I also revamped the course around various
high-level themes in networking, including layering, resource sharing,
and tree formation (routing and forwarding, etc.).  Finally, I developed over
20 new lectures for the syllabus, as well as new problem sets which can
be re-used for future offerings of the course.

\paragraph{CS 6262 Network Security:}
{\em Spring 2009.} I updated the syllabus to
include recent network attacks (\eg, spam, botnets, reflection attacks,
etc.) and also to integrate more hands-on assignments.  Updating problem
sets and project lists.

\paragraph{CS 7260 Internetworking Architectures and Protocols:}
{\em Spring 2006.}  I developed a new project-based graduate course
with substantial programming assignments using a wide variety of
state-of-the-art networking software tools and platforms (\eg, rcc,
PlanetLab, scriptroute, NetFlow, etc.).  I contributed questions to a
larger bank of questions also used in graduate-level networking courses
at Carnegie Mellon and MIT.  Finally, I developed 24 new lectures, many
based on current ``hot topics'' in computer networking (\eg, spam,
botnets, traffic anomaly detection, etc.)

{\em Spring 2007.}  I developed two new course modules: (1)~sound
techniques for network measurement; and (2)~evaluation platforms
(Emulab, VINI, etc.).  I designed new problem sets on these topics.
With faculty at Carnegie Mellon, I instituted the use of a
cross-institutional online forum for paper discussion.  Students read
papers from the CS 7260 syllabus and commented on papers before class to
help stimulate paper discussion; students also read the discussion blog
and could comment on papers being discussed in networking classes at
other schools.

\medskip

%\bigskip

\def\student#1#2#3#4#5{\item[#1]{#2
\newline{\sl #3}
\internalonly{\newline Publications: {\sl #4}}
\newline {\sl #5}}}

\def\studentnopub#1#2#3#4#5{\item[#1]{#2
\newline {\sl #3}
\newline {\sl #5}
}}

\def\studentnonote#1#2#3#4{\item[#1]{#2
\newline{\sl #3}
\newline Publications: {\sl #4}}}
%\newline {\sl #4}

\def\studentempty#1#2#3{\item[#1]{#2
\newline{\sl #3}}}

\subsection{Individual Student Guidance}
\label{subsec:individual}

\subsubsection{Postdocs and Research Scientists Supervised}

\begin{description}
\studentnopub{Dave Levin}{Department of Computer Science}{Fall 2012 -}{}
{Various networking systems research topics, including wireless
  networking and software defined networking.}

\studentnopub{Stephen Woodrow}{College of Computing}{Spring 2011 -
  Summer 2012}{}{Research on the performance of Internet access networks
  and development for the BISmark project. \\ {\bf Stephen is now at
    Stripe, a bay area startup.}}

\studentnopub{Nazanin Magharei}{College of Computing}{Spring 2011 -
  Spring 2012}{}{Research on the performance of Internet access
  networks.} \\ {\bf Nazanin is now a full-time employee at Cisco Systems.}

\student{Cristian Lumezanu}{College of Computing}{Fall 2009 -
  Fall 2011}{
\refbib{Lumezanu2012:spam},
\refbib{Lumezanu2012:bias}
\refbib{Valancius2011:tiers},
}{Research on Internet measurement and economics. \\ {\bf Cristian is
    now a researcher at NEC Research Labs.}}

\end{description}

\subsubsection{Ph.D. Students Supervised}

\begin{description}

\studentnopub{Arpit Gupta}{Department of Computer Science} {Fall
    2013 - Present}{}{}

\studentnopub{Ben Jones}{Department of Computer Science} {Fall
    2013 - Present}{}{}

\studentnopub{Sean Donovan}{Department of Computer Science} {Fall
    2013 - Present}{}{}

\studentnopub{Swati Roy}{Department of Computer Science} {Fall
    2013 - Present}{}{}

\studentnopub{Sarthak Grover}{Department of Computer Science} {Fall
    2012 - Present}{}{}

\studentnopub{Muhammad Shahbaz}{Department of Computer Science}{Fall
    2012 - Present}{}{}

\studentnopub{Abhinav Narain}{Department of Computer Science}{Fall 2011 -
  Present}{}{Research on performance of home wireless networks.}

\student{Xinyu Xing}{College of Computing}{Fall 2011 -
  Present}{
\refbib{Xing2013:poison}
}{Research on anti-censorship.}

\student{Hyojoon Kim}{College of Computing}{Fall 2009 - Present}{
\refbib{Kim2011:config},
\refbib{Kim:sigcomm2011:cap}
}
{Research on programmable networks and network configuration.}
\\ {\em
    Passed qualifier.}

\student{Bilal Anwer}{College of Computing}{Fall 2008 - Present}{
\refbib{Anwer2012:bubble},
\refbib{Anwer2010:netfpga},
\refbib{Anwer2010:switchblade},
\refbib{Anwer2010:visa},
\refbib{Anwer2009:visa}
}
{Research on support for hardware forwarding in virtual network 
  environments.} 
\\ {\em
    Passed qualifier.}

\student{Yogesh Mundada}{College of Computing}{Fall 2007 - Present}{
\refbib{Mundada2011:silverline},
\refbib{Bhatia2008:trellis},
\refbib{Mundada:nsdi2008:trinity}
}
{Research on a data-leak prevention system for Web
  applications.  Development of experiment specification for VINI and
  integration of VINI with Emulab.}
\\ {\em
    Passed qualifier.}

\studentnopub{Robert Lychev}{College of Computing}{Fall 2008 -
  Present}{}{Research on contract enforcement for transit markets. \\ {\em
    Passed qualifier.}}

\student{Sam Burnett}{College of Computing}{Fall 2008 -
  Present}{
\refbib{Burnett2010:collage},
\refbib{Burnett:sigcomm2010:collage},
\refbib{Burnett:nsdi2009:collage} 
}{Design and implementation of anti-censorship systems. \\ {\em 
    Passed qualifier.}}

\student{Srikanth Sundaresan}{College of Computing}{Fall 2008 -
  Present}{
\refbib{Sundaresan2012:broadband},
\refbib{Sundaresan2011:bismark},
\refbib{Sundaresan2011:labels},
\refbib{Sundaresan:sigcomm2012:bismark},
\refbib{Sundaresan:sigcomm2010:sculpte},
\refbib{Sundaresan2012:keystone},
\refbib{Sundaresan2012:web}
}{Research on access network performance and online traffic
  engineering. \\ {\em 
    Passed thesis proposal.
}} 

\student{Shuang Hao}{College of Computing}{Fall 2007 -
  Present}{
\refbib{Hao2011:dns}
\refbib{Hao2009:snare}
}{Research on botnet detection, network monitoring, and spam
  filtering. \\ {\em Passed qualifier.}}

\student{Maria Konte}{College of Computing}{Fall 2007 - Present}{
\refbib{Konte2012:pam},
\refbib{Konte2008:pam},
\refbib{Konte:sigcomm2011:bgp}
}
{Measurement study of fast-flux networks. \\ {\em Passed
    qualifier.}}

\student{Vytautas Valancius}{College of Computing}{Summer 2007 -
  Spring 2012}{
\refbib{Katz2012:lifeguard}
\refbib{Valancius2011:tiers},
\refbib{Valancius2010:bgpmux},
\refbib{Valancius2008:mint},
\refbib{Bhatia2008:roads},
\refbib{Valancius2008:bgpmux-tr},
\refbib{Bhatia2008:trellis},
\refbib{Valancius2007:presto},
\refbib{Valancius:sigcomm2010:tp}
\refbib{Mundada:nsdi2008:trinity},
\refbib{Valancius2007:bgpmux-conext}
}{Research on interdomain routing and network virtualization. \\ {\bf
    Graduated, now at Google.}}


\student{Murtaza Motiwala}{College of Computing}{Fall 2006 -
Spring 2012}
{
\refbib{Motiwala2012:cost},
\refbib{Motiwala2008:splicing},
\refbib{Motiwala2007:splicing},
\refbib{Bhatia2008:trellis},
\refbib{Motiwala2007:presto},
\refbib{Motiwala2006:orchid-wired},
\refbib{Mundada:nsdi2008:trinity}
}
{Research on (1)~in-band
troubleshooting and (2)~scalable network architectures for path
diversity, including {\em path splicing}. \\ {\bf Graduated, now at Google.}}



\student{Mukarram Bin Tariq}{College of Computing}{Spring 2007 -
  Spring 2010}{
\refbib{Tariq2008:wise},
\refbib{Tariq2009:vlan},
\refbib{Tariq2009:nano},
\refbib{Tariq2008:nano},
\refbib{Tariq2010:wise},
\refbib{Ramachandran2008:pedigree}
}
{Co-advised with Mostafa Ammar. Research on statistical inference
  methods for network planning and troubleshooting problems.  Mukarram's
  dissertation work is now part of operational systems at Google.
  \\ {\bf Graduated. Now at Google in the network
    monitoring group.}} 

\student{Anirudh Ramachandran}{College of Computing}{Spring 2006 -
Present} {  
\refbib{Lee2008:botnet},
\refbib{Ramachandran2011:ceas}
\refbib{Ramachandran2007:spamtracker},
\refbib{Feamster2006:spam},
\refbib{Ramachandran2008:wosn},
\refbib{Ramachandran2007:bitstore},
\refbib{Ramachandran2006:recon},
\refbib{Ramachandran2006:dnsbl},
\refbib{Ramachandran2008:fish4phish},
\refbib{Ramachandran2008:pedigree},
\refbib{Ramachandran:sigcomm2008:pedigree},
\refbib{Ramachandran:nsdi2006:spam}
}{Research on network-level behavior of
spammers and passive botnet detection. \\ {\bf Graduated; winner of the
  Georgia Tech Dissertation Award.  Now the
  founder of a data-leak prevention startup, Nouvou.}}



\end{description}

\subsubsection{Masters Students Supervised}

\begin{description}

\studentnopub{Sachit Muckaden}{College of Computing}{Spring 2013}{}
{Mobile performance measurement, development of MySpeedTest, and
  Android-based mobile performance measurement tool that is now in
  widespread use across 130 countries.}

\student{Swati Roy}{Electrical and Computer Engineering}{Fall
  2012-Spring 2013}{\refbib{roy2013:sigcomm}}
{Study of correlated lantecy spikes in home networks.}

\studentnopub{Abhishek Jain}{College of Computing}{Fall 2011 - Fall 2012}{}
{Design and implementation of {\tt networkdashboard.org}, a front end
  Web interface for network data gathered from home networks.}


\student{Ankur Nayak}{College of Computing}{Spring 2009 - Spring 2010}{
\refbib{Nayak2009:resonance}
}{Dynamic access control with programmable switches.}

\studentnopub{Umayr Hassan}{College of Computing}{Fall 2008 - Spring 2010}{}
{Research on the design of a market for Internet transit, and on home
  network configuration. \\ {\em Umayr now works full-time at Bloomberg.}}

\student{Nadeem Syed}{College of Computing}{Spring 2007 - Spring 2008} {
  \refbib{Syed2007:spamsvm-nips} } {Co-advised with Alex Gray.
  Developing and implementing new machine learning techniques for fast
  disruption detection. \\ {\em Nadeem is in the MBA program at Georgia
    Tech.}}

\student{Kaushik Bhandakar}{College of Computing}{Spring 2007 -
  Summer 2008}{
\refbib{Ramachandran:sigcomm2008:pedigree}
}
{Experiments for VINI performance benchmarking; implementation and
  prototyping for the ``Pedigree'' packet provenance project; research
  on incentives in BitTorrent. \\ {\em Kaushik now works full-time at Google.}}

\studentnopub{Samantha Lo}{Hong Kong Polytechnic University}{Spring
  2007}{}{Research on market-based network architectures and inbound
  traffic engineering. \\ {\em Samantha is now a Ph.D. student at Georgia Tech.}}


\student{Manas Khadilkar}{College of Computing}{Fall 2006 - Spring 2007}
{\refbib{Khadilkar2007:dhcp}}{Research on efficient settings of lease
times for DHCP address allocation.  Algorithm in development, to be used
on the Georgia Tech campus network for optimizing lease time
settings. \\ {\em Manas now works full-time for Expedia.}}

\studentnopub{Han Lu}{College of Computing}{Fall 2006 - Spring 2007}
{}{Research on spam traffic patterns by IP address space.}

\studentnopub{Chris Kelly}{College of Computing}{Fall 2006 - Fall 2007}
{}{Developing new software features for the Campus-Wide Performance
  Monitoring and Recovery (CPR) project. \\ {\em Chris now works full-time
    for SugarCRM, an Atlanta-based startup.}}

\student{Yiyi Huang}{College of Computing}{Spring 2006 - Fall 2009}
{
\refbib{Huang2006:bgp},
\refbib{Huang2008:doppler}
}{Co-advised with Jim Xu. Research on fast, distributed network anomaly
  detection.\\ {\em Yiyi now works full-time at Microsoft.}}

\student{Winston Wang}{M.I.T. EECS}{Fall 2002 - Spring 2003}
	{\refbib{Feamster2003h}}{Thesis on an implementation of the
	Infranet anti-censorship system received MIT's Charles and
	Jennifer Johnson Thesis Prize.}



\end{description}


\subsubsection{Undergraduate Students Supervised}

\begin{description}

\student{Jake Martin}{College of Computing}{Spring 2012}
{
\refbib{martin2012:prio}
}{Research on prioritizing application traffic in home networks}

\student{Michael Dandy}{College of Computing}{Summer 2012 - Fall 2012}{
\refbib{Rotsos2012:staggercast}
}
{Design of usage cap management interface for the BISmark platform.
  Study of incentive schemes for encouraging users to shift traffic
  demands to off-peak hours.}

\studentnopub{Andrew Kahn}{Department of Computer Science, Northwestern
  University}{Summer 2012}{} 
{Design and implementation of a captive portal system for the BISmark
  home router platform.}

\studentnopub{Alex Wong}{College of Computing}{Summer 2012}{}
{Design and implementation of {\tt networkdashboard.org}, a front end
  Web interface for network data gathered from home networks.}

\studentnopub{Alfred Roberts}{College of Computing}{Fall 2011 - Present}{}
{Design and implementation of {\tt networkdashboard.org}, a front end
  Web interface for network data gathered from home networks.}


\student{Alex Reimers}{College of Computing}{Spring 2009}{
\refbib{Nayak2009:resonance}
}{Worked on dynamic access control (a replacement of Georgia Tech OIT's
  current authentication system) with programmable switches.  \\ Alex now
  works full time at BigSwitch, an OpenFlow-based startup.}


\student{Megan Elmore}{College of Computing}{Fall 2007 - Spring 2009} {
  \refbib{Motiwala2008:splicing} }{Experiments for interdomain path
  splicing; design and implementation of the path splicing prototype.
  Work received 2nd prize in 2008 Georgia Tech College of Computing
  undergraduate research competition. Megan was also the winner of the
  2009 College of Computing Undergraduate Research Award, and the 2009
  Sigma Xi Best Undergraduate Researcher Award. {\em Megan is now a
    Ph.D. student at Stanford University.}}

\studentnopub{Hongyi Hu}{M.I.T. EECS}{Spring 2005 - Fall
  2005}{}{Extensions to the {\em rcc} router configuration checker tool
  for static configuration analysis of internal routing protocol
  configurations.}
\end{description}


\subsubsection{Special Projects}

\begin{description}

\studentnopub{Dan Doozan}{College of Computing}{Fall 2011 - Summer 2012}{}
{Research on anti-censorship and filter bubbles.}

\studentnopub{Mona Chitnis}{College of Computing}{Spring 2010}{}
{Research on OpenFlow network architectures.}

\studentnopub{Sravanthi Gondhi}{College of Computing}{Spring 2010}{}
{Research on online traffic engineering.}

\studentnopub{Shruti Gupta}{College of Computing}{Spring 2010}{}
{Research on online traffic engineering.}

\studentnopub{Utkarsh Shrivastava}{College of Computing}{Spring 2010}{}
{Research on network-level behavior of spammers.}

\studentnopub{Pooja Rajanna}{College of Computing}{Spring 2010}{}
{Research on network-level behavior of spammers.}


\studentnopub{Luxmi Saha}{College of Computing}{Spring 2010}{}
{Research on data-center scheduling algorithms.}


\studentnopub{Dongchan Kim}{College of Computing}{Fall 2009}{}
{Research on spam filtering.}

\studentnopub{Sonali Batra}{College of Computing}{Summer 2009}{}
{Research on anti-phishing techniques.}


\studentnopub{Radhika Partharathy}{College of Computing}{Fall 2008}{}
{Research on anti-phishing techniques.}


\studentnopub{Sagar Mehta}{College of Computing}{Fall 2006 - Spring 2008}{}
{Research on anti-phishing techniques.}

\studentnopub{Bhairav Dutia}{College of Computing}{Fall 2006}
{}{Research on anti-censorhip techniques and countermeasures.}

\studentnopub{Megan Benoit}{College of Computing}{Fall 2006}{}
{Research on spammers' email address harvesting practices.}



\studentnopub{Amit Khanna}{College of Computing}{Fall 2006}
{}{Implemented Secure BGP (S-BGP) in the Quagga software router.
Software publically available and operators are using the codebase for
ongoing work on certificates for secure routing.}



\studentnopub{Daniel Mentz}{College of Computing}{Spring 2006}
{}{Research on campus network security troubleshooting.}

\studentnopub{Buddy Moore}{College of Computing}{Summer
  2006}{}{Implemented distributed version of the Infranet
  anti-censorship software.  Publicly available.}





\end{description}


%\subsection{Teaching Honors and Awards}

