
\documentclass{gtcv}

\usepackage{ncntrsbk}
%ncntrsbk helvetic times palatino
%\usepackage{mathptm}
%\usepackage{amsfonts}

\usepackage{fancyhdr}
%\thispagestyle{fancy} %
\pagestyle{fancyplain} %
\lhead{} \rhead{} \chead{}

\lfoot{\fancyplain{Irfan Aziz Essa}{Irfan Aziz Essa}}
\cfoot{\fancyplain{References}{References}} \rfoot{Page \thepage}
\renewcommand{\headrulewidth}{0pt}
\renewcommand{\footrulewidth}{.5pt}
\renewcommand{\plainfootrulewidth}{1pt}
\addtolength{\headheight}{3pt}


%\title{\vspace{-1.1in}\Large\bf Irfan Aziz Essa \vspace{-1em}}
\title{\vspace{-1.0in}Irfan Aziz Essa}
%\title{\vspace{-1.0in}\name}
\author{Assistant Professor \\ College of Computing \\ Georgia Institute of Technology \\
Atlanta, GA  30332-0280, USA}
\date{}



\begin{document}

\maketitle

\setlength{\parskip}{1pt}
\section*{LETTER WRITERS for IRFAN A. ESSA}

\begin{description}

\item[Prof. Alexander (Sandy) Pentland]~\note{Need to Contact and Update the
BIO}\\
Massachusetts Institute of Technology\\
Media Laboratory\\
20 Ames Street, Room \#E15-383\\
Cambridge, MA 02139, USA\\
sandy@media.mit.edu\\
http://sandy.www.media.mit.edu/people/sandy\\
Tel:\\
Fax:\\

\medskip
Position:  Academic Head, MIT Media Laboratory and the Toshiba
Professor of Media Arts and Sciences at the Massachusetts
Institute of Technology. Also, External Director, Center for
Future Health, Strong Hospital, University of Rochester.

Area of research: computer vision, computer graphics, and
perceptual user interfaces.

Contact Reason:  Pentland is one of the preeminent researchers in
computer vision and novel interfaces.  He was my PhD advisor and
perhaps knows my technical abilities the best.

\smallskip
Biographical Sketch:

\item[Prof. Takeo Kanade]~\note{Need to Contact and Update the
BIO}\\
Carnegie Mellon University \\
Robotics Institute \\
5000 Forbes Avenue \\
Pittsburgh, PA 15213, USA\\
tk@cs.cmu.edu  \\
http://www.cs.cmu.edu/$\sim$tk/\\
Phone: (412) 268-3016 \\
Fax: 412-268-5570

\medskip

Position: U. A. Helen Whitaker University Professor at School of
Computer Science and Robotics Institute at the Carnegie Mellon
University.  He has also served as the the Director of the
Robotics Institute.

Contact Reason: Kanade is a preeminent leader in computer vision
research and his work crosses many of the areas that I have worked
in.

\smallskip
Biographical Sketch: Dr. Kanade received his Ph.D. in Electrical
Engineering from Kyoto University, Japan, in 1974. After being on
the faculty of the Department of Information Science, Kyoto
University, he joined the Computer Science Department and Robotics
Institute in 1980. He became an Associate Professor in 1982, a
Full Professor in 1985, the U. A. and Helen Whitaker Professor in
1993, and a University Professor in 1998. He was the Director of
the Robotics Institute from 1992 to Spring 2001. He served as the
founding Chairman (1989 - 93) of the Robotics Ph. D. Program at
CMU, probably the first of its kind in the world.

Dr. Kanade has worked in many areas of robotics, including
manipulators, sensors, computer vision, multimedia applications,
and autonomous robots, with more than 200 papers on these topics.
He has been the founding editor of the International Journal of
Computer Vision.

Dr. Kanade's professional honors include: election to the National
Academy of Engineering, a Fellow of IEEE, a Fellow of ACM, a
Fellow of American Association of Artificial Intelligence; several
awards including C \& C Award, the Joseph Engelberger Award,
Yokogawa Prize, JARA Award, Otto Franc Award, and Marr Prize
Award.



\item[Prof. Eric Grimson]~\note{Contacted 6/18/01 and Update the
BIO}\\
Massachusetts Institute of Technology\\
AI Laboratory\\
545 Technology Square \\
Cambridge, MA 02139, USA \\
welg@ai.mit.edu \\
http://www.ai.mit.edu/people/welg/welg.html \\
Tel: (617) 253-5346\\
Fax: (617) 258-6287

\medskip
Position: Eric Grimson is a Professor of Computer Science and
Engineering at the Massachusetts Institute of Technology, and
holds the Bernard Gordon Chair of Medical Engineering.

Contact Reason:


\smallskip
Biographical Sketch: Eric Grimson is a Professor of Computer
Science and Engineering at the Massachusetts Institute of
Technology, and holds the Bernard Gordon Chair of Medical
Engineering. He also holds a joint appointment as a Lecturer on
Radiology at Harvard Medical School and at Brigham and Women's
Hospital. He received a B.Sc. (Hons) in Mathematics and Physics
from the University of Regina in 1975 and a Ph.D. in Mathematics
from MIT in 1980. Prof. Grimson currently heads the Computer
Vision Group of MIT's Artificial Intelligence Laboratory, which
has pioneered state of the art systems for object recognition,
image database indexing, image guided surgery, target recognition,
site modeling and many other areas of computer vision. Recently,
his group has been active in applying vision techniques in
medicine: for image guided surgery, minimally invasive surgery and
telemedicine.


\item[Prof. Larry Davis]~\note{Contacted 6/18/01. Update the BIO}\\
University of Maryland at College Park \\
Department of Computer Science\\
A.V. Williams Building \\
College Park, MD 20742, USA\\
lsd@cs.umd.edu\\
http://www.umiacs.umd.edu/$\sim$lsd/
 Tel: (301) 405-2662
 Fax: (301) 405-6707

\medskip
Position: Larry S. Davis is a Professor and Chairman of Computer
Science Department at U of Maryland. He is also a member of the
Institute for Advanced Computer Studies. He is affiliated with the
Computer Vision Laboratory of the Center for Automation Research,
for which he served as the head from 1981-1986.

Area of research:

Contact Reason:

\smallskip
Biographical Sketch:

\item[Prof. Demetri Terzopoulos]~\note{Need to Contact and Update the BIO}\\
Media Research Laboratory \\
New York University\\
719 Broadway, 12th Floor\\
New York, NY 10003, USA\\
dt@cs.nyu.edu \\
http://www.mrl.nyu.edu/$\sim$dt/ \\
Tel: (212) 998-3477\\
Fax: (212) 995-4122

\medskip
Position:  Professor of Computer Science and Mathematics at the
Computer Science Department and The Courant Institute of
Mathematical Sciences, New York University.

Area of research: computer vision and graphics (including
animation)

Contact Reason:  Terzopoulos is one of the preeminent researchers
in computer animation.

\smallskip
Biographical Sketch: Demetri Terzopoulos received MEng and BEng
degrees from McGill University in 1980 and 1978, respectively, and
a PhD degree from MIT in 1984. During 1984--85 he was a research
scientist at the MIT AI Lab. From 1985 to 1992, he was affiliated
with Schlumberger Inc. in Palo Alto and Austin, serving as Program
Leader. He joined the faculty at the University of Toronto in
1989.

Professor Terzopoulos has published more than 150 scientific
articles, primarily in computer vision and graphics, but also in
computer-aided design, medical imaging, artificial intelligence,
and artificial life, including the recent edited volumes Real-Time
Computer Vision (Cambridge University Press, 1994) and Animation
and Simulation (Springer-Verlag, 1995).

In 1989, Professor Terzopoulos was appointed Fellow of the
Canadian Institute for Advanced Research. In 1996, he was awarded
a Steacie Fellowship by NSERC. His other awards include an award
from the American Association for Artificial Intelligence in 1987
for his work on deformable models in vision, an award from the
IEEE in 1987 for his work on active contours (snakes), and awards
from the Canadian Academy of Multimedia Arts and Sciences in 1994
and from Ars Electronica in 1995 for his work on artificial
animals for computer animation and virtual reality. He is a
founding member of the editorial boards of the journals Medical
Image Analysis, Graphical Models and Image Processing, and the
Journal of Visualization and Computer Animation, and is a member
of the IEEE, NYAS, and Sigma Xi.

\item[Prof. Edward H. Adelson]~\note{Needs updating}
Massachusetts Institute of Technology\\
Department of Brain and Cognitive Science, \\
3 Cambridge Center, MIT Room \# NE20-444, \\
Cambridge, MA 02139, USA\\
adelson@psyche.mit.edu\\
http://www-bcs.mit.edu/people/adelson/adelson.html\\
Tel: (617) 253-0645\\
Fax: (617) 253-8335\\

\medskip
Position:

Area of research:

Contact Reason:

\smallskip
Biographical Sketch:


\item[Prof. Jitendra Malik]~\note{Update the BIO}\\
University of California \\
Computer Science Division \\
725 Soda Hall\\
Berkeley, CA 94720-1776 \\
malik@cs.berkeley.edu\\
http://www.cs.berkeley.edu/$\sim$malik/\\
Phone: (510) 642-7597 \\
Fax: (510) 642-5775 \\



\medskip
Position: Professor of Computer Science, University of California
at Berkeley. Editor-in-Chief of the International Journal of
Computer Vision

Area of research:

Contact Reason:

\smallskip
Biographical Sketch: Jitendra Malik was born in Mathura, India in
1960. He received the B.Tech degree in Electrical Engineering from
Indian Institute of Technology, Kanpur in 1980 and the PhD degree
in Computer Science from Stanford University in 1985. In January
1986, he joined the faculty of the Computer Science Division,
Department of EECS, University of California at Berkeley, where he
is currently a Professor. During 1995-1998 he also served as
Vice-Chair for Graduate Matters. He is a member of the Cognitive
Science and Vision Science groups at UC Berkeley.

His research interests are in computer vision and computational
modeling of human vision. His work spans a range of topics in
vision including image segmentation and grouping, texture,
stereopsis and object recognition,. His research has found
applications in image based modeling and rendering, content based
image querying, and intelligent vehicle highway systems. He has
authored or co-authored more than 90 research papers on these
topics.

He received the gold medal for the best graduating student in
Electrical Engineering from IIT Kanpur in 1980, a Presidential
Young Investigator Award in 1989, and the Rosenbaum fellowship for
the Computer Vision Programme at the Newton Institute of
Mathematical Sciences, University of Cambridge in 1993. He
received the Diane S. McEntyre Award for Excellence in Teaching
from the Computer Science Division, University of California at
Berkeley, in 2000. He is an Editor-in-Chief of the International
Journal of Computer Vision.


\end{description}

\centerline{Additional}

\begin{description}

\item[Jake Aggarwal]~\\
The University of Texas Austin \\
Computer \& Vision Research Center \\
Dept. of Electrical and Computer Engineering\\
Austin, TX 78712-1084, USA \\
aggarwal@ece.utexas.edu\\
Tel: 512-471-3259 \\
Fax: 512-471-5532


\item[Pietro Perona]~\\
Caltech \\
Department of Electrical Engineering\\
1200 East California Boulevard, \\
MS 136-93 \\
Pasadena, CA 91125 USA \\
perona@caltech.edu \\
http://www.vision.caltech.edu/html-files/Perona.html \\
Tel:(626) 395-4867 \\
Fax: (626) 795-8649

\item[P. Anandan]~\\
Microsoft Research\\
Vision Technology\\
One Microsoft Way \\
Redmond, WA 98052-6399 USA\\
anandan@microsoft.com\\

\item[Norman Badler]~\note{Need to Contact and Update the BIO}\\
Computer and Information Science Department \\
University of Pennsylvania \\
200 South 33rd Street \\
Philadelphia, PA 19104-6389 \\
badler@central.cis.upenn.edu \\
215-898-5862 \\
215-573-7453 (fax)

\medskip
Position: Director, Center for Human Modeling and Simulation and
   Professor, Computer and Information Science Department

Area of research: computer graphics modeling, animation, and
rendering techniques for synthetic humans

Contact Reason:

\smallskip
Biographical Sketch: Dr. Norman I. Badler is a Professor of
Computer and Information Science at the University of Pennsylvania
and has been on that faculty since 1974. Active in computer
graphics since 1968 with more than 100 technical papers, his
research focuses on human figure modeling, manipulation, and
animation. He is the originator of the Jack software system, now
used at about 100 active commercial, government, and University
sites worldwide. His expertise includes real-time 3-D graphics,
intuitive user interfaces, complex object modeling, and animation
systems. Badler received the BA degree in Creative Studies
Mathematics from the University of California at Santa Barbara in
1970, the MSc in Mathematics in 1971, and the Ph.D. in Computer
Science in 1975, both from the University of Toronto. He is
co-editor of the Journal Graphical Models and Image Processing and
co-author of the book Simulating Humans published by Oxford
University Press. He also directs the Center for Human Modeling
and Simulation with three full time staff members and about 40
students.

\item[Dimitris N. Metaxes]~\\
U. Penn (moving to Rutgers).

\item[Ken Perlin]~\\
NYU.

\end{description}



\end{document}
