% evaluation.
% update the following
\newpage 

\setcounter{page}{1}
\lfoot{\fancyplain{Printed: \today}{Nick Feamster}}
\cfoot{\fancyplain{Revised: \today}{Annual Evaluation 2007}}
\rfoot{Page \thepage}

\setcounter{page}{1}
\begin{center}
{\Large\textbf{Accomplishments for 2006 \\ February 2007}}\\[.1in]
{\large\textbf{Nick Feamster}}\\[.05in]
feamster@cc.gatech.edu \\
\url{http://www.cc.gatech.edu/~feamster/} \\[.1in]
\end{center}

\begin{sloppypar}

\section*{Research}

In this section, I outline my research accomplishments.  I divide this
work into several distinct areas: recognition and publication,
software/systems artifacts, and funding.

\subsection*{Recognition/Publication}

We were very lucky this year to receive the award paper at {\em
SIGCOMM}, which propelled me and my student, Anirudh Ramachandran, into
the limelight as ``experts'' in spam research and has subsequently
helped us further this research agenda (with data, funding, etc.).

\begin{itemize}
\itemsep=-1pt
\item Spam filtering work received Best Student Paper award at {\em
  SIGCOMM 2006}.
\item Two papers accepted to {\em SIGCOMM 2006}, one paper accepted to
  {\em SIGMETRICS 2007}. 
\item Two journal papers accepted to {\em IEEE Transactions on
  Networking}, for publication in 2007.
\end{itemize}

\subsection*{Funding}

This was a good year for NSF funding; I received two NSF awards (out of
three proposals submitted).  Commercial funding has been slower to
acquire: my proposals to Cisco were rejected in two separate rounds. 

\begin{itemize}
\itemsep=-1pt
\item PI for two new NSF awards (CAREER award and NeTS-FIND award).
\item Established relationships with Ironport (spam filtering
  company, recently acquired by Cisco) and Outblaze (large email hosting
  provider). 
\item Established new relationships at Cisco for hopeful funding in 2007.
\end{itemize}

\subsection*{Software and Systems}

My main accomplishments in developing new software has been in work on
the VINI network testbed, jointly with my student, Murtaza Motiwala, and
researchers at Princeton University.  This year, we got the VINI testbed
up and running with different types of software routers and ran the
first successful set of ``network architecture'' experiments on the
testbed.

\begin{itemize}
\itemsep=-1pt
\item Created a framework for experimental setup, now an integral part of
  the VINI codebase.
\item Augmented the testbed to run different software routers (XORP,
  Quagga, etc.).
\item Used VINI as an evaluation platform for modifications to the
  network-layer protocol to enable network troubleshooting.
\end{itemize}

\section*{Service}

Most of my ``unusual'' service this year was external, so I will focus
on that in my writeup.

My hallmark achievement in service this year was organizing and hosting
the {\em Workshop on Internet Routing Evolution and Design (WIRED)} at
Georgia Tech.  This workshop (website available at
\url{http://www.wired2006.org/}) brought together 30 of the top
researchers in Internet routing from around the world for a two-day
workshop at Georgia Tech.  Although the workshop had several
co-organizers, I took the lead role in organizing the workshop, from
inviting researchers, to organizing sessions, to managing workshop
logistics.  Several researchers personally thanked me, saying that they
were indebted to me for organizing the workshop and that it was very
good that I organized the workshop.  I am not certain whether I want to
spend the time to do this annually, but I was happy to resurrect it and
to provide a service both to the community and to younger researchers
(i.e., students) in Internet routing.

This year was also good in providing me the opportunity to have several
leadership roles in service.  In 2006, I was fortunate to co-chair two
workshops---the {\em First Workshop on the Economics of Networked
Systems (NetEcon)} (with Jeff Chase) and the {\em CoNext Student
Workshop} (with Renata Teixeira).

\begin{itemize}
\itemsep=-1pt
\item Lead organizer of the {\em Workshop on Internet Routing Evolution
  and Design}.
\item Co-chair of two workshops.
\item Appointed to {\em NANOG} Program Committee.
\end{itemize}

\section*{Education}

My main accomplishments in education have been the re-vamping of two
courses, CS~7260 (Internet Architectures and Protocols) and CS~7001
(Introduction to Graduate Studies).  In CS~7260, my contributions were
as follows:
\begin{itemize}
\itemsep=-1pt
\item Developed a new project-based graduate course
with substantial programming assignments using a wide variety of
state-of-the-art networking software tools and platforms (\eg, rcc,
PlanetLab, scriptroute, NetFlow, etc.).  
\item Contributed questions to a
larger bank of questions also used in graduate-level networking courses
at Carnegie Mellon and MIT.  
\item Developed 24 new lectures, many based on
current ``hot topics'' in computer networking (\eg, spam, botnets,
traffic anomaly detection, etc.)
\end{itemize}
\noindent
This year, I have also continued to add new content to this course, parts
of which are detailed in my CV.  

My CV also details my contributions to CS~7001; most notably, with Alex
Gray, I have undertaken a complete redesign of this course to help
students get more involved with research from the moment they arrive in
the Ph.D. program at Georgia Tech.  Our focus on assignments has been to
encourage interdisciplinary thinking in the assignments, by getting
groups of students from different areas to try to apply research ideas
and concepts across areas.

\newpage
\begin{center}
{\Large\textbf{Goals for 2007}}\\[.1in]
{\large\textbf{Nick Feamster}}\\[.05in]
feamster@cc.gatech.edu \\
\url{http://www.cc.gatech.edu/~feamster/} \\[.1in]
\end{center}

\section*{Research: Network Operations}

I will continue to develop my research program around the broader theme
of {\em network operations}, developing tools, techniques, and
algorithms that make a network operator's job of running the network
easier.  Specifically, my high-level goals for my research program this
year involve developing a firmer position as one of the leading
researchers in the following two areas: (1)~distributed network
monitoring and (2)~network virtualization.  These two broad areas
because roughly correspond to my two funded NSF grants.  I will briefly
discuss my specific goals in each of these areas.

\subsection*{Distributed Network Monitoring}

Many network tasks become easier when an operator has multiple vantage
points from which to view a particular problem or phenomenon.  The NANOG
mailing list is rife with examples where network operators send ``SOS''
mails when a problem arises, to see whether other operators witness the
same problem.  Similarly, within a single network, an operator can
monitor the network from many independent vantage points to collect
information about network conditions.  In both of these cases, operators
do not suffer from a lack of data, but rather an inability to {\em mine}
the data for interesting features.  

In my work this year, I hope to write two large conference papers on
topics relating to mining network data for anomaly and fault diagnosis.
As part of this work, my research will leverage machine learning
techniques, which I believe could bring much to bear in solving many of
these problems.  While others have tackled network diagnosis from
various angles in the past, few have aplied a rigorous machine learning
approach to these problems, and none have built working systems that
apply these techniques to solve real problems.  My goal for this year in
distributed network monitoring research is to do both of these (i.e.,
applying machine learning algorithms to distributed diagnosis and
actually {\em building} these algorithms into real, working systems).

\begin{itemize}
\itemsep=-1pt
\item Learn about statistical inference and machine learning techniques
  and study how they can be aplied to networking problems.
\item Write two conference papers on new distributed network monitoring
  algorithms for network fault detection.
\item Write one conference paper on using distributed network monitoring
  for predictive spam filtering.
\end{itemize}

Finally, I have been working on developing a network data repository,
with Dave Andersen at CMU.  I plan to continue building this
infrastructure, and I would like to send a paper to {\em NSDI} on
network measurement infrastructure for supporting sound Internet
measurement strategies.

\subsection*{Network Virtualization}

I plan to continue two research threads as part of the Cabo
(``Concurrent Architectures are Better than One'') project: in-band
network diagnosis and using network virtualization to improve
availability.  The first problem relates to improving {\em
accountability} in the network; it does not directly relate to using
network virtualization as a network architecture, but it uses the VINI
testbed as a deployment and evaluation platform.  With Murtaza Motiwala,
I have built a preliminary prototype of an in-band troubleshooting
protocol; we would like to deploy this as a long-running service on the
VINI testbed, which involves designing ways to attract real users to the
service.

The second research project area involves exploring how network
virtualization can be used to improve availability (i.e., the fraction
of time that groups of hosts can use paths in the network to communicate
with one another).  As part of this research thread, I plan to work on
three specific research project areas: path splicing, a market-based
routing architecture, and multi-layer network architectures.  The first
project involves developing new a forwarding paradigm that significantly
improves path diversity in networks by running multiple routing
protocols in parallel.  We already have algorithms designed for this
purpose, so the main task now is to implement evaluate these protocols
on the VINI testbed.  My goal for the market-based architecture work is
to design and implement a version of this routing protocol by the end of
the year, which involves research in both economics and protocol design.
My work on investigating multi-layer network architectures is much more
preliminary, so I have no specific problems that I am tackling at this
point.

\begin{itemize}
\itemsep=-1pt
\item Implement path splicing and deploy it on VINI.  Write the path
  splicing paper and get it accepted to a major conference (i.e., {\em
  SIGCOMM}, {\em NSDI}, or {\em SOSP/OSDI}).
\item Submit two papers on market-based network architectures: one to
  the {\em JSAC} special issue on economics in networks (deadline in
  August) and one to {\em SIGCOMM}.
\item Write at least one paper on multi-layer network architectures.  
\end{itemize}

A related problem area is better understanding the use of various
evaluation techniques (i.e., simulation, emulation, wide-area
deployment) to evaluate new protocols and architectures.  I would like
to conduct some research that can help the community better understand
when each of these evaluation techniques is an appropriate evaluation
method. 

\section*{Education}

My educational goals this year will once again focus around CS~7260 and
CS~7001.  With respect to CS~7260, I am working with my advisor, Hari
Balakrishnan, on writing a graduate-level networking textbook from our
lecture notes.  As part of this effort, I have been writing new problem
set questions for CS~7260 that span a wide range of areas and testing
these in my class.  As mentioned in my CV, I have also been developing
new course modules (e.g., on network measurement and performance
evaluation).  

\begin{itemize}
\itemsep=-1pt
\item Complete book chapters on network security and Internet routing
  (presuming book writing continues this year).
\item Develop new problem set questions and lectures for CS~7260
  focusing on network performance evaluation and measurement.
\end{itemize}

As far as CS~7001 is concerned, I would like to continue developing
projects and assignments for this course with Alex Gray, further
developing our themes of teaching students how to find research ideas
and perform interdisciplinary research.  Our ultimate goal is to submit
a SIGCSE paper on this introductory graduate course, though, since
the deadline is September, it us unclear whether we will be able to
submit such a paper this year or whether it will have to wait for
another year. 

\begin{itemize}
\itemsep=-1pt
\item Continue developing CS~7001 assignments on developing research
  ideas and performing interdisciplinary research.
\end{itemize}

\section*{Service}

I tend to take service opportunities that make sense as they come my
way, so I don't have specific goals as far as service on specific
program committees, etc.  However, I see two opportunities for service
within the context of committees I am already serving on and research I
am doing: (1)~using my position on the NANOG program committee to better
bridge the gap between researchers and network operations; an (2)~using
VINI's need for real users as an opportunity to extend network
connectivity to under-served communities.  

My position as the only academic faculty member on the NANOG program
committee firmly emphasizes position as the leader in the field of
operationally relevant network research.  I would like to use this
position to bring operators and network researchers closer together.  In
this regard, I have one specific goal:
\begin{itemize}
\item Establish a ``Research BOF'' session at NANOG, and specifically
  invite university researchers (faculty and students) to give
  workshop-style talks at this BOF session.  The next NANOG (in June)
  will be in Redmond, WA, which presents a unique opportunity to invite
  many of the top networking researchers from Washington and Microsoft
  to participate.  In a sense, I view this activity as bringing the
  previously mentioned {\em WIRED} workshop to NANOG, rather than
  enticing the operators come to {\em WIRED}.
\end{itemize}

I am less clear on specific steps to take for the second goal above, but
it is clear that VINI needs real user traffic, and one thought for how
to get real user traffic on VINI is to install wireless access points in
places where there is otherwise no Internet connectivity and tunneling
this back to VINI.  This situation seems like a perfect opportunity to
deploy Internet connectivity in under-served communities, although this
goal is admittedly somewhat ambitious, and I'm not certain it is
possible.  I would need to collaborate with an expert in wireless
networking to make this happen.

\section*{Enablers}

I will briefly list some of the areas where I believe I will need
assistance to make some of these goals happen.  There may be other
factors that didn't come to mind.

\begin{itemize}
\itemsep=-1pt
\item Help attracting the best networking students to work with me.  (I
  sense some competition for these students in our discussions this
  year, but, of the networking faculty, I am the most vulnerable to the
  consequences of ``starvation'' of good students.).  

  I intend to make efforts to attract the best students to my group
  (e.g., making a trip to India, giving talks at other universities,
  getting my research in the news, etc.), but I sense I may need
  protection from certain other faculty members and appropriate
  ``advertisements'' of my work to other students and faculty when
  appropriate.

\item The ability to ``cash in'' on my teaching relief sometime in the
  next two years.
\end{itemize}

\end{sloppypar}