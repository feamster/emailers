% evaluation.
% update the following
\newpage \setcounter{page}{1}
\lfoot{\fancyplain{Printed: \today}{Nick Feamster}}
\cfoot{\fancyplain{Revised: \today}{Annual Evaluation 2006}}
\rfoot{Page \thepage}

\begin{center}

{\Large\textbf{Research Threads and Goals \\ May 2006}}\\[.1in]
{\large\textbf{Nick Feamster}}\\[.05in]
feamster@cc.gatech.edu \\
\url{http://www.cc.gatech.edu/~feamster/} \\[.1in]

\end{center}

\subsection*{Research Threads}

\noindent
{\bf Tools and Techniques for Robust and Secure Network Operations.}
Network operators face a daunting challenge of having to run their
networks at peak performance in the face of constantly changing
conditions and continual threats.  Unfortunately, the complexity of
network configuration, the large number of heterogeneous network
devices, and continually changing network conditions make it difficult
for operators to achieve high availability and performance while
satisfying the high-level policies (\eg, security policies) for
operations of their networks.  Today, network operators are unable to
correctly configure their networks to achieve the tasks they would like
to accomplish, they are unable to protect their networks from emerging
threads such as botnets (and botnet ``applications'', such as spam,
click fraud, etc.), and they are unable to troubleshoot their network
configurations when problems do arise.  By and large, network operators
debug faults and anomalies manually; these operators do not have
efficient, automated tools and techniques to help operators perform
proactive, predictive analysis (\eg, the effects of a change to the
network configuration, etc.).  

These problems in proactive fault detection, network security, and
reactive troubleshooting harm network availability and consume a large
amount of operator time.  Furthermore, these tasks contribute
significant costs to running networks; ISPs are becoming more
competitive, and as the cost of bandwidth drops, they must reduce the
considerable cost (in time and money) of running their networks.  One
thrust of my research is to develop tools and techniques to help network
operators (of both service provider and enterprise networks) run their
networks better: specifically, my research aims to help network
operators (1)~attain assurances of the correctness, security, and
robustness of network configurations; (2)~locate the source of problems
when the network does not behave as expected.

\noindent
{\bf Inherently Correct and Robust Network Protocols and Architectures.}
Today's Internet architecture imposes constraints that can make network
operations tasks staggeringly difficult.  My first research thread helps
network operators run their networks in the context of today's Internet
routing architecture; my second thread, on the other hand, is based
around the following question: How should the Internet architecture be
changed to facilitate better network operations?  In this thread, I
construe network operations somewhat broadly; in addition to tackling
problems such as helping network operators perform traffic engineering,
improve network security, and provision and deploy new services.  This
research thread incorporates work both on network architecture and
protocol design, and comprises two main subthreads:
\begin{enumerate}
\item {\em Improving the fairness and efficiency of Internet routing
through market-based routing protocol design.} Over the past fifteen
years, the Internet has become increasingly commercialized and
federated, but the Internet's routing protocols have not evolved along
with the Internet's economic landscape.  As a result, the mechanisms
that network operators need to implement business policies have been
retrofitted into the existing protocols, which were never designed to
support such functions.This proposal posits that, to satisfy the above
design goals while keeping pace with the Internet's rapidly changing
economic landscape, {\bf a future Internet routing architecture must
decouple business policy from the dissemination of network topology
information.}  This property implies that end hosts may see {\em
multiple} paths to any given destination, so it necessitates that end
hosts that send traffic (sources) must have some control over the paths
that their traffic takes en route to other end hosts (destinations).
Although this characteristic implies a departure from destination-based
routing, there is a broad spectrum of routing architectures where
sources can retain some control over the path that traffic takes to a
destination without having to specify the entire path.  We call this
large class of routing architectures {\em source-assisted routing}.

Developing such an architecture requires not only an understanding of
the design space but also an understanding of the economics and the {\em
market} formed by service providers (who are selling reachability in the
form of path segments) and end users (who are purchasing reachability in
the form of complete paths to other end hosts), in which the overall
goal is to maximize the overall efficiency seen by both parties.  A key
issue that must be explored is the role of {\em settlements}, which
provide the right incentives to both end users and providers.

\item {\em Designing, implementing, and deploying inherently robust
communication networks.}  Although network operators can immediately
benefit from the tools and techniques that are being produced as a
result of my first research thread, networking research should strive to
design communication networks that inherently offer robust communication
channels between communicating parties.  

Communication networks should facilitate open communication between any
two parties, but, in recent years, opressive governments and
organizations routinely limit their users' access to various Internet
destinations (\eg, China routinely blocks access to external news
sources, telecmmunication companies have attempted to block voice over
IP traffic, etc.).  My research involves designing and implementing
systems that help these groups of users gain open acccess to
information and achieve open communications, even in the face of
opressive parties.

In addition to designing tools and techniques to help network operators
run their networks better today, within the contraints of the current
Internet infrastructure, I am also designing and implementing new
network architectures and systems that could help the network be more
inherently robust and secure.  I am investigating how a network
architecture that separates infrastructure providers (the entities that
deploy and maintain the network equipment) from service providers (the
entities that deploy network protocols and end-to-end services). This
separation allows the future Internet to support multiple simultaneous
{\em virtual networks\/}, each of which could offer multiple end-to-end
network services.  I am also planning to investigate how such an
architecture could improve Internet security by allowing malicious or
untrusted traffic to be sandboxed onto a separate, ``dirty'' virtual
network (analogous to previous research on sandboxing with virtual
machines).  Indeed, malicious traffic might even be quarantined and
observed on an Internet-wide basis to help network operators perform
forensics (\eg, tracking botnets).
\end{enumerate}

\noindent 

\subsection*{Objectives for the Coming Year and Current Progress}

My two main goals for the coming year are (1)~to establish a coherent
research program around the above-mentioned threads; and (2)~to begin
building a research group around this research theme.  I will discuss
each of these goals in more detail in the following paragraphs.

\subsubsection*{Goal \#1: Developing a research program}

My first goal for this year is to establish a coherent research theme
around network operations.  Actionable goals for establishing this
research theme include establishing visibility for my work in this area
(\ie, by submitting papers to top networking conferences and giving
presentations to the network operations community) and obtaining funding
for my research in this area.

{\bf Progress: Paper submissions.}  As of April 2006, I have made some
progress towards achieving these actionable goals.  First, I submitted
two papers to {\em ACM SIGCOMM} based on network operations: one on an
analysis of spamming behavior, towards the goal of ultimately designing
better techniques and network architectures for mitigating
spam~[\refbib{Feamster2006:spam}]; and another on the design and
implementation of a virtual network infrastructure
(VINI)~[\refbib{Bavier2006}].  Our work on studying the network-level
behavior of spammers has already received attention at the North
American Network Operator's Group (NANOG) (see
talk~\refbib{nanog36:bgp-spam}).

{\bf Progress: Grant proposals.}  Based on these initial pieces of work,
and a work that has recently been accepted (pending minor revisions) to
{\em ACM/IEEE Transactions on Networks}~[\refbib{Feamster2006:policy}],
I submitted two collaborative grant proposals to the NSF NeTS ``Future
Internet Design'' (FIND) area (see grants \ref{funding:nsf-cabo} and
\ref{funding:nsf-routing}).  Based on this initial work on the study of
spam and botnets, I have submitted a grant to Cisco asking for money to
develop techniques to help network operators discover botnet activity in
their networks.  Finally, I submitted a second grant to the Cisco URP
program to extend the static configuration analysis techniques in my
Ph.D. thesis to include fault detection capabilities for secure network
services (\eg, BGP/MPLS layer 3 VPNs, packet filters, etc.) (see
grant~\ref{funding:cisco-rsec}).

{\bf Action items for remainder of 2006.} The major goal for this year,
and, in particular, this summer, will be to submit a grant proposal to
the NSF CAREER program.  This grant proposal is due at the end of July,
leaving approximately 14 weeks until the proposal is due.  In the next
six weeks (\ie, until the beginning of June), I plan to develop a theme
for this proposal based broadly around network operations, which will
incorporate material from the two grant proposals that I sent to Cisco
this spring.  I will spend June and July fleshing out the details of the
proposal itself.

The remaining major conference deadlines for this year are {\em ACM
Internet Measurement Conference}, {\em ACM SIGMETRICS} and {\em USENIX
NSDI}.  My current plan is to send some preliminary characterization
studies that will help in the design of our botnet detection algorithms
to the {\em ACM Internet Measurement Conference}.  With Jim Xu and Yiyi
Huang, I may also sent a short paper on a taxonomy of routing dynamics,
based on some initial work that is related to Cisco
URP~\ref{funding:cisco-rsec}; the current plan is to design and
implement some anomaly detection and troubleshooting algorithms based on
this taxonomy for submission to {\em ACM SIGMETRICS}.  For the upcoming
{\em NSDI} deadline (October~2006), I would like to send some work on
one or more of the projects related to grants~\ref{funding:nsf-cabo}
and~\ref{funding:nsf-mri}.

Other possible conference and workshop deadlines to which smaller pieces
of this work may be submitted are {\em IEEE Infocom}, {\em HotNets}, and
various workshops at {\em ACM SIGCOMM}.

\subsubsection*{Goal \#2: Building a research group}

By the end of this year, I would like to build my research group around
the theme of network operations and security.  A possible name for this
group, reflecting its focus on network operations, security, and future
network architectures, would be the ``Network Operations and Systems
Group'', or something to this effect.  Aside from developing the
research themes of the group by submitting papers and obtaining funding
for this research will be the following goals:

\begin{enumerate}
\item Attracting strong students and research staff.
\item Holding regular group meetings to allow members to
exchange ideas and discuss papers.
\end{enumerate}
My philosophy for forming this group is that having such a structure
will provide scope for work within our group, keeping students abreast
of current reseach in this area and allowing their research to develop
under this broad theme.  Having regular group meetings will also provide
a cohesiveness that I want my students to experience.  Such meetings
will be particularly important until the new College of Computing
building provides regular interaction with students.

As far as attracting students, my goal is to add roughly two new
Ph.D. students per year, three in exceptional circumstances (\ie, if the
pool is strong).  With the exception of the project related to spam and
botnets (papers~\refbib{Feamster2006:spam},
~\refbib{Ramachandran2006:dnsbl}), none of the remaining projects above
have Georgia Tech students participating.  My goal is to have graduate
students working on many of the projects above.

{\bf Progress.}  My main focus this spring has been to attract the
strongest possible students to my research program at Georgia Tech.  I
have recruited Murtaza Motiwala and Manos Antonakakis to Georgia Tech,
who have both indicated an interest in working with me.  Manos has
indicated an interest in working on projects related to network
security, although his interest in wireless networking may prove to be
difficult; Murtaza has also expressed interest in working on projects
related to a virtual network infrastructure.  At this point, the main
challenge appears to be to find enough good students to ramp up on the
projects that I would like to get going.  Part of my strategy for
solving this problem without falling prey to taking too many
Ph.D. students is to post ads on the UROC forum, etc., to try to attract
some undergraduates to work on various problems (\eg, anti-censorship
systems).

\subsubsection*{Goal \#3: Following through on dissertation work}

Another goal for the remainder of this year will be to follow through on
my dissertation work in two ways: sending thesis chapters to journals,
and transferring technology developed as part of my thesis to industry.
The first part of this process is already well underway: two chapters
from my dissertation have been accepted for publication in {\em IEEE/ACM
Transactions on Networking}, and a third is in submission.  The only
remaining technical chapter from my thesis, on a tool for static
configuration analysis of BGP configuration, has not yet been submitted
to a journal.  At this point, I have no plans to send this work to a
journal, although I could certainly change that; the main roadblock in
this case would be the development of a significant fraction of new
material to turn it into a journal paper.

I have been working on the second half of this follow-through for quite
some time now, in various ways.  The latest effort to get some of the
contributions from my thesis used in some lasting way is to work with
Cisco to integrate some of this technology (and follow-on work) into
their management products as part of ongoing work (see grant
proposal~\ref{funding:cisco-rsec}, which would be the basis for this
collaboration). 
