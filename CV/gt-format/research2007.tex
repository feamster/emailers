\newpage \setcounter{page}{1}
\lfoot{\fancyplain{Printed: \today}{Nick Feamster}}
\cfoot{\fancyplain{Revised: March 8, 2007}{Research Statement}}
\rfoot{Page \thepage}

\begin{center}

{\Large\textbf{Research Statement}}\\[.1in] {\large\textbf{Nick
Feamster}}\\[.05in] feamster@cc.gatech.edu \\
http://www.cc.gatech.edu/$\sim$feamster/ \\[.1in] \end{center}


\section*{Overview}

The goal of my research is to help network operators run their networks
better, and to enable users of these networks (both public and private)
to experience high availability and good end-to-end performance; I call
this area "network operations".  I tackle practical problems using a
"first principles" approach, design systems based on these principles,
and implement and deploy these systems in practice.

My research comprises two complementary approaches: (1) designing and
deploying real-world systems, tools, and algorithms that are immediately
applicable to today's networks and (2) developing fundamental network
primitives and architectures that could dramatically improve network
operations, even if the deployment cost is substantial.  Deploying
"bottom up", real-world solutions to network operations problems gives
me a unique perspective on real-world network operations and an
appreciation for details (e.g., this bottom-up allowed me to build
strong domain expertise in both BGP routing and spam).  Great strides in
networking research are often inherently "top down", but such
fundamental work must be informed by real-world problems and details. My
two-fold approach allows my research strike a balance between
operational relevance and long-term impact.

My current research applies this approach to two areas: (1) distributed
algorithms and systems for network monitoring and security; and (2)
routing protocols for improving network connectivity in federated,
competitive environments.

\section*{Network Monitoring and Security}

Users of computer networks demand high availability and good performance
in the face of continually changing network conditions.  Most
communications networks are kept afloat by the vigilance of network
operators, who tune network configurations in response to changes in
available resources, failures of network elements, fluctuations in
traffic demands, or the onset of malicious or otherwise unwanted traffic
(e.g., spam).

Responsiveness to changing network conditions requires increasingly
sophisticated detection and mitigation strategies as networks become
saddled with more features and subject to new classes of unforeseen
threats, and as users' demands from the network become increasingly
stringent.  Network monitoring has long been a critical piece of this
puzzle: mitigating network disruptions depends, first and foremost, on
tools and techniques to quickly and accurately detect these disruptions
and determine their causes.

Network monitoring must be accurate and robust.  Accurate monitoring
techniques should detect disruptions when they occur (with a negligible
number of false alarms), and, to the extent possible, identify the cause
of the disruption (e.g., the faulty network element, the source of
unwanted traffic).  Robust monitoring should be able to detect
disruptions when measurements may be noisy, incomplete, or when
attackers are actively trying to disguise their presence.  Network
monitoring strategies are most accurate when they are distributed; that
is, when they draw upon observations from a large number of vantage
points.

Unfortunately, fast distributed detection of network anomalies is
challenging. Information about network conditions is voluminous and
noisy.  Network operations needs better techniques for quickly
detecting---and diagnosing---network faults and anomalies, which
requires fundamental advances both in data mining algorithms and in
distributed systems for processing this data.  My research in network
monitoring and security focuses on both the application of data mining
algorithms (and the development of new algorithms), and, perhaps most
importantly, the implementation and evaluation of these algorithms in
real-world networks.  I am applying this method to spam filtering and to
network fault diagnosis.

\subsection*{Problem 1: Spam Filtering}

Spam filtering today is, by and large, reactive: network operators
observe spammers and generate content filters and IP-based blacklists
based on this observed activity.  Content-based filtering is rapidly
becoming ineffective as spammers develop increasingly sophisticated
techniques (e.g., image-based spam) to evade these filters.  To make
matters worse, recent trends in spam activity suggest that reactive
techniques such as IP-based blacklisting are becoming ineffective as an
increasing fraction of spam is being sent from "fresh" IP addresses for
which operators have little or no information about the reputation.

Next-generation spam filters must be proactive and predictive.  My
research focuses on developing techniques that can help network
operators identify spam based solely on network-level patterns.  I call
this approach "behavioral blacklisting".  Our research involves both
developing new spam filtering techniques and algorithms and evaluating
these algorithms on real networks.

\subsection*{Problem 2: Network Fault Diagnosis}

Network faults and disruptions---changes in network conditions that are
caused by underlying failures of routing protocols or network
equipment---have a significant impact on network performance and
availability.  Operators today have myriad datasets (e.g, traffic
statistics, SNMP, routing data, "syslogs") at their disposal to monitor
for network disruptions.  All of these datasets have proven difficult to
use for extracting actionable events from "background noise".  Indeed,
network data is voluminous, complex and noisy; network operators are not
at a loss for network data; rather, they lack efficient algorithms and
systems for analyzing and processing this data to quickly detect network
faults while at the same time maintaining a low overall false alarm
rate.

My research focuses on developing algorithms and systems to help network
operators quickly detect and locate network faults.  I am primarily
focused on developing techniques for network-wide data analysis;
techniques that incorporate independent observations of the same
phenomenon (e.g., a network fault) from multiple vantage points can
provide clues as to the severity, location, and cause of the fault.
Network-wide monitoring, however, mandates both new algorithms for
efficiently extracting actionable network disruptions from distributed,
voluminous, heterogeneous data and a system for efficiently processing
this data so that disruptions can be detected scalably and quickly.  I
am working on algorithms and systems for distributed network fault
detection that I hope to ultimately deploy in operational networks, from
campus networks such as Georgia Tech to possibly even large transit
networks.

\section*{Improving Network Connectivity}

One of the current Internet's biggest pitfalls is that today's end hosts
cannot make efficient use of available connectivity that exists in the
underlying physical network.  This inability to leverage the underlying
rich connectivity creates both inefficient routing (e.g., unnecessarily
long paths) and fragility (as evidenced by the continual de-peering
debacles, fiber cuts, etc., all of which cause significant disruptions
to connectivity).  These inefficiencies arise from two characteristics
of today's routing protocols: (1) single-path routing and (2)
restrictive, bilateral business policies.  Both of these characteristics
prevent end hosts from using available network capacity to a
destination.  In my research, I am investigating how new routing
protocols can scalably achieve sufficient path diversity by making use
of multiple parallel paths and how a new routing framework could create
a more efficient market for connectivity.  Although I am exploring both
of these areas in the context of "future Internet architectures", I am
also exploring how facets of these designs can be deployed in today's
Internet routing fabric.

\subsection*{Problem 1: Scalable Path Diversity}

To make the best use of available capacity, and to gracefully degrade in
the event of network faults, routing protocols should expose multiple
routes for each destination to each end host.  Unfortunately, scalably
providing such path diversity has proven difficult in practice, since
computing and storing additional routes at each node implies (often
prohibitive) increases in complexity and state.  I am developing
solutions that use network virtualization---a primitive that allows
multiple virtual networks to operate in parallel on a single, shared
physical infrastructure---to scalably offer path diversity to end hosts.

I am designing and implementing these architectures on VINI (virtual
network infrastructure), a distributed testbed for new network protocols
and architectures that I am developing with my students, as well as
colleagues at Princeton University.  VINI is aligned with a larger
effort at the National Science Foundation (specifically, the GENI and
FIND research programs) to help network researchers and operators
design, deploy, and evaluate new network protocols and architectures in
realistic settings.

\subsection*{Problem 2: Efficient Markets for Connectivity}

Much of the inefficiency in today's interdomain routing (i.e., routing
between the Internet's thousands of independently operated, competing
networks) results from the fact that interdomain connectivity is cobbled
together from bilateral contracts between pairs of networks.  This
restrictive contracting model not only "hides" a significant amount of
connectivity from end hosts but it also introduces fragility and
inefficiency on end-to-end paths.  I am designing (and ultimately plan
to implement) a new routing architecture that is centered around a small
number of aggregators, which serve as clearinghouses for long-haul
connectivity between "edge" networks.  This AGregator-Oriented
Architecture (AGORA) should create both more efficient end-to-end paths
and a way for end hosts to transfer value for these more efficient paths
to the networks that offer this connectivity.
