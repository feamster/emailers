\newpage \setcounter{page}{1}
\lfoot[Printed: \today]{Nick Feamster}
\cfoot{\fancyplain{}{Revised: June 12, 2013}}
\rfoot{Page \thepage~of 1}

\vspace*{-0.5in}
\begin{center}
{\Large\textbf{Teaching Statement}}\\[0.1in] {\large\textbf{Nick
Feamster}}\\
%feamster@cc.gatech.edu \\
%http://www.cc.gatech.edu/$\sim$feamster/ \\[.1in] 
\end{center}
I believe that students learn best by doing.  Hands-on experience and
real-world exercises not only make classes more exciting, but they
also provide memorable examples and analogies.  Our ability to
understand abstract concepts is more limited than our ability to process
concrete examples that relate to familiar ideas.  Some of
my own memorable classroom experiences were lectures with concrete
examples; I apply a similar approach in my courses, tying abstract
concepts to relevant concrete examples and hands-on experience.  For
example, ``route hijacking'' is more concrete when students can
actually see network traffic going where it isn't supposed to go.
%Routing protocols make more sense when students can configure routers,
%induce failures on an experimental network, and watch the behavior of
%traffic over that network.

To this end, my first goal in teaching both networking and security
classes is to connect textbook material with (1)~real-world examples;
and (2)~hands-on experience.  At both the undergraduate and graduate
levels, I have incorporated course material that familiarizes students
with the state of the art in network design, implementation, and
experimentation; for example, I have students run experiments on network
testbeds like Emulab, VINI, and BISmark, with real software routers that
they can configure.  Where appropriate, I have also allowed students to
shape the course themselves by bringing in real-world examples, from
which I will design a lecture.  For example, in my undergraduate
networking course, I maintain a wiki where students could post relevant
``current events'' in networking and vote on which topics they would
like to see covered.  Based on that input, I incorporate new material
into the syllabus---teaching not just the current event itself, but also
the relevant foundational material.  I also bring my own current events
to lecture and relate them to the textbook being covered.  This takes
more effort than dusting off old notes, but it keeps students engaged
and helps me stay abreast of what is happening both in industry and in
research.

My second goal is to elevate course material so that students are not
just learning mechanics of protocols and systems, but also gaining a
deeper understanding of the {\em concepts} that underlie their design.
My reasoning here is two-fold.  First, I believe that the traditional
classroom lecture is ``going the way of the blackboard''.  With so many
computing and communications tools for aggregating and processing
information, conventional lectures are no longer always the most
efficient way to convey textbook information.  In my lectures, I try to
go beyond what is taught in the textbook---rather than teaching only
mechanics, I ask students about design rationales, and whether they
would make the same choices today, given the changing roles of
communications networks.  Given that networking is still
maturing as a field, there is a tendency to focus on protocol details,
which may change over time.  Ten years from now, I would like someone
who took my class to be able to say that the concepts they learned have
remained applicable.  
%I try to organize topics around higher-level
%concepts (\eg, randomization, caching, tree formation, identity), so
%that even as protocol details continue to change, the concepts that they
%learn remain valuable.

I enjoy teaching students how to think.  I have designed
a graduate course that focuses on teaching first-year graduate students
how to do research.  The course includes topics ranging from paper
reading to fellowship applications to generating (and executing)
research ideas. The value of the course is evident from student
response: Students at various points along their Ph.D.---even more
senior students---sit in on lectures.  We presented a paper on the
course at {\em SIGCSE} in 2008, and other computer science departments
(\eg, Cornell, Duke, Princeton) are now starting to adopt some of the
material, and we have developed a blog (\url{http://greatresearch.org/})
to accompany the course.

I have a strong mentoring and advising record.  My graduate students
have received best paper awards at top conferences (including {\em
SIGCOMM}), and my undergraduate students I have advised have garnered
publications in top conferences and have gone to the best computer
science graduate schools in the country.  One of my most rewarding
advising experiences was with undergraduate student Megan Elmore.  Under
my advisement, she received the Sigma Xi Undergraduate Research Award,
the CoC undergraduate research award, and an NSF Graduate Fellowship;
she is now a Ph.D. student at Stanford University. I have received
multiple ``Thank a Teacher'' awards at Georgia Tech for outstanding
teaching performance.  I also won a Hesburgh Teaching Fellow award at
Georgia Tech in 2012, which is awarded to only about ten instructors
across campus all year.

In my ongoing work in education, I am experimenting with massive open
online courses (MOOCs), as well as using technology such as video
lectures in the classroom to create a ``flipped classroom'' experience.  
