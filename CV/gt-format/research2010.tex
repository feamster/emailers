\newpage \setcounter{page}{1}
\lfoot[Printed: \today]{Nick Feamster}
\cfoot{\fancyplain{Revised: March 23, 2010}{Research Statement}}
\rfoot{Page \thepage~of 6}

\begin{center}
{\Large\textbf{Research Statement}}\\[0.1in] {\large\textbf{Nick
Feamster}}\\
%feamster@cc.gatech.edu \\
%http://www.cc.gatech.edu/$\sim$feamster/ \\[.1in] 
\end{center}


\section*{Summary}

My research focuses on developing tools, algorithms, and protocols that
make the network easier to manage, more secure, and more available.

Nobody notices when the network works well, but everyone suffers when it
doesn't.  Thus, communications networks should be both secure and
available.  Network {\em security} has many facets, ranging from the
ability to stop ``unwanted traffic'' (e.g., spam and denial-of-service
attacks) to the ability to trace back attacks to their perpetrators
(``accountability'').  {\em Availability} means that the network must
provide good performance for users whenever they want to use
it---unfortunately, the increasing complexity of the network, coupled
with hardware faults, software bugs, misconfigurations, and malice, make
it difficult to achieve this goal.  Unfortunately, these two important
goals have also been among the most evasive.  Breakthroughs require not
only extensive domain knowledge, but also the ability to techniques from
a wide range of areas, ranging from economics to machine learning.  My
work combines domain knowledge, extensive interactions with network
operators, techniques from a wide range of disciplines, and---perhaps
most importantly---the competence and tenacity to implement and deploy
these systems in practice.  This unique combination has allowed me to
build one of the few networking research groups in the world that
interacts directly with network operators to deploy fundamentally new
systems and technologies in real-world networks.

I discover interesting and challenging practical problems through
frequent discussions and meetings with network operators and people in
industry.  I then tackle these problems from first principles, develop
new methods, and transfer these solutions back to practice in the form
of working systems.  I have tackled a variety of problems in network
operations, ranging from real-time network diagnosis to stemming
unwanted traffic like spam to architectures for fast failure recovery.
Many people---most notably, operators ``in the trenches''---are also
working on these problems.  Unfortunately, many of the people who have
the domain knowledge that best equip them to solve these problems are
busy with day-to-day operations, putting out fires as they arise but
rarely taking time to think about fundamental changes to the network
that might eradicate these problems.  My research fills this niche.  I
first devise methods to understand the nature of the problem in
practice.  I tackle domain-specific problems with tools and techniques
from other disciplines---ranging from machine learning to economics to
program analysis---whose principles might provide insights into a new,
previously undiscovered solution.  I then devise a new approach or
solution, and I transfer it to practice through implementation and
deployment of real-world systems.

My work in this broad area follows three specific themes: (1)~making the
network more secure; (2)~improving network availability and performance
by making the network easier to operate and manage; (3)~designing
platforms for virtual networks that facilitate technical innovation in
both network security and operations.  The first two themes involve
developing solutions that make the network more robust and resilient in
the face of faults, misconfiguration, and malice; the third theme
provides an avenue to evaluate and deploy these solutions in practice.

\section*{Theme 1: Network Security}

\fp {\bf Summary of work.}  My research explores the role that communications
networks---in particular the network layer---can play in improving
computer and communications security.  This line of research began with
my arrival at Georgia Tech in 2006. A cornerstone of this
research is a system that was published in August 2009 called ``SNARE''
(Spatio-temporal Network-level Automated Reputation Engine). This work
appeared at the {\em USENIX Security Symposium}, a top-tier security
conference. The main idea behind SNARE---and the key insight behind my
research in spam filtering---is that spammers have different sending
behavior than legitimate senders.  Therefore, filters can distinguish
spammers from legitimate senders by examining their {\em sending
behavior} (i.e., how they send traffic), rather than what is in the
messages themselves.

Prior to my research, conventional spam filters attempted to distinguish
spam from legitimate email by looking at message contents: that is, they
would look at the words or language used in the messages themselves and
try to detect spam based on what the message said.  This approach has
become increasingly untenable, since spammers have begun to embed their
messages in all sorts of media, ranging from images to PDFs to audio
files to spreadsheets---by the time developers perfected their content
filters for one type of medium, spammers moved onto the next.  My line
of work has taken an entirely different, but complementary approach: I
look at features of the senders' {\em behavior} (e.g., the time of day
they are sending, whether there are other ``nearby'' senders on the
network, whether and how the sizes of the messages of the senders vary
over time) to distinguish spamming behavior from legitimate email use.
The method is harder for spammers to evade, it is more flexible because
it can be deployed anywhere in the network, and it can work at much
higher traffic rates than conventional approaches.  This idea was first
laid out in the initial award paper at {\em SIGCOMM} and finally
realized in the SNARE paper from August 2009 at {\em USENIX Security}.

I have also worked on sweeping changes to the Internet architecture that
could improve {\em accountability}, thus making it more difficult for
malicious parties to operate unfettered in the first place.  The current
Internet architecture provides little to no accountability. Malicious
end systems can conceal the source of their traffic (``spoofing''), and
edge networks can provide false information about their reachability to
various Internet destinations (``route hijacking''); both of these
attacks make it difficult to track down perpetrators of attacks.
Current approaches to solving these problems require manual
configuration and operator vigilance, which make them weak and
error-prone.  Towards building networks that are inherently accountable,
I have developed the Accountable Internet Protocol (AIP).  One of my
contributions to the design was to make the addresses in this protocol
self-certifying, which forms the cornerstone of the basic design.  I
also demonstrated how to apply AIP to secure BGP, the Internet's
interdomain routing protocol.


\fp {\bf Impact.}
My research in network security has had impact in research, in industry,
and on the national level.  My research on this topic has earned the
Presidential Early Career Award for Scientists and Engineers (PECASE), a
Sloan fellowship, and the Best Paper Award at {\em ACM SIGCOMM} (the
premier computer networking conference).  Aspects of my work have also
been incorporated into commercial spam filtering products and Web mail
clients at companies including Yahoo, Cisco/Ironport, and McAfee, as
well as a project for the Department of Defense on high-speed network
monitoring.  My paper on understanding the network-level behavior of
spammers---which won the Best Student Paper award at {\em SIGCOMM} in
2006---has been cited over 200 times since its initial publication in
August 2006---it spawned a variety of high-impact follow-on work,
including looking at network-level behavior not only to develop better
spam filters, but also to detect botnets more effectively and defend
against phishing attacks, click fraud, and other serious threats to the
Internet infrastructure.  I have also been working on similar approaches
to help detect and dismantle the Internet's scam hosting infrastructure
(e.g., Web sites that attempt to steal user passwords, money, and so
forth).  My initial paper on this topic (``Dynamics of Online
Scam-Hosting Infrastructure'') won the Best Paper award at the {\em
Passive and Active Measurement} conference in April 2009.

My work on SNARE has also garnered significant attention in industry.
This work was featured in {\em Technology Review} and on Slashdot (a
popular, high-traffic site for news in information technology).  Several
companies including Yahoo have incorporated the network-level features
that SNARE identifies into its spam filters, and companies that develop
spam filtering appliances, such as McAfee, are also using these features
to improve the accuracy and performance of their spam filtering
appliances.

AIP appeared in {\em ACM SIGCOMM} in 2008; an early version of the
design also appeared in {\em ACM Workshop on Hot Topics in Networking
(HotNets)}.  I am incorporating a version of this technology into a
working system and transferring them to practice.  I am working with BBN
on a DARPA project that will ultimately result in incorporating AIP's
mechanisms into a military network protocol that allows attribution of
traffic to sources (the details may ultimately be classified). 

My impact on the broader field of cybersecurity goes beyond my own
research.  I am also having impact in the national arena in several
ways.  Last year, I was involved in setting the nation's agenda for
cyber security, through multiple additional activities.  First, with
leadership from Jeannette Wing at NSF, I led a community-wide effort to
develop a ``wish list'' document that describes the security community's
needs for access to better data---ranging from network traffic data, to
data about our country's infrastructure.  This report was ultimately
delivered to Tom Kalil, the deputy director for policy in the Office of
Science and Technology Policy.  Second, with program managers Karl
Levitt and Lenore Zuck at NSF, I organized a community-wide,
multi-agency workshop on ``Security-Driven Architectures''.  The
workshop included participants from all areas of computer science
(ranging from theory and cryptography to networking to systems to
hardware), with an eye towards setting a research agenda for developing
more holistic approaches to computer security that consider {\em all}
aspects of computer and communications systems, rather than just a
single piece (like the network).  Finally, my work on developing
next-generation Internet protocols to improve accountability (which
could eradicate spam in the first place), based on work that appeared at
{\em ACM SIGCOMM} in 2008, were also included in reports for the
National Cyber Leap Year.

\fp {\bf Representative Publication}
\begin{quote}
S. Hao, N. Syed, N. Feamster, A. Gray and S. Krasser. "Detecting
Spammers with SNARE: Spatio-temporal Network-level Automatic Reputation
Engine". {\em USENIX Security Symposium}, August 2009. Montreal, Quebec,
Canada. 
\end{quote}

\fp {\bf Most Cited Publication (236 Citations)}
\begin{quote}
A. Ramachandran and N. Feamster. ``Understanding the Network-Level
Behavior of Spammers". {\em ACM SIGCOMM}, August 2006. Pisa, Italy. {\bf
Best Student Paper Award.}
\end{quote}



\section*{Theme 2: Network Operations}

{\bf Summary of work.}  A second major theme of my work is {\em network
operations}, which is what I call the field of designing networks so
that they are easier to run and manage.  Much of my work in this area
has focused on fault detection and troubleshooting.  Prior to my
dissertation work, operators relied on detecting problems with networks
``at runtime'' on a live network.  My dissertation work demonstrated
that, in fact, many routing problems could be detected simply by
examining the configuration of the routing protocols, before the
configuration is even deployed.  I applied techniques from static
program analysis to routing configuration to help network operators
catch mistakes and predict dynamic network behavior before the
configurations are deployed on a live network, preventing costly and
catastrophic network downtime.

Beyond predicting behavior and proactively detecting configuration
faults, operators must understand the network's behavior {\em as it is
running} (e.g., to detect equipment failures, attacks, or unplanned
shifts in network traffic).  Unfortunately, operators are drowning in
heterogeneous data.  To help operators better understand network faults
``at runtime'', I have applied unsupervised learning techniques to
Internet routing data to help them efficiently mine the data for network
events that require corrective action.  This work appeared in {\em ACM
SIGMETRICS} in 2007.  My work has also applied statistical inference
techniques to help network operators determine the answers to ``what
if'' configuration questions in content distribution networks; we
developed a system called ``WISE'' (What-If Scenario Evaluator) to help
network operators determine the effects of configuration changes on
network response time.  A paper on this system appeared at {\em ACM
SIGCOMM} in 2008 and is now used by operators and network designers at
Google.  A more mature version of this work that also describes
deployment experiences at Google is in submission to {\em IEEE/ACM
Transactions on Networking}.

Users of communications networks also face the potential of intentional
performance degradation or manipulation by Internet Service Providers
(ISPs); these problems are popularly referred to as ``network neutrality
violations''.  This transparency can help users determine whether their
network is the cause of performance degradation, or whether performance
problems that they are seeing are due to some other cause.  With
students, I designed, built, and deployed the {\em Network Neutrality
Access Observatory (NANO)}, a system that aggregates measurements from
end systems to help users and operators of edge networks infer when
transit networks may be discriminating against certain types of traffic.
This work appeared in {\em ACM SIGCOMM CoNext} in 2009, and we have
deployed the system on Google's Measurement Lab
(\url{http://www.measurementlab.net/}).  More recently, we have been
looking at methods for helping users diagnose general problems with
access network performance and examining which factors have the most
influence on access network performance.

I have developed new network protocols and architectures that improve
availability and accountability in communications networks in the face
of both faults and malice.  Networks face the continual threat of
failures and attacks that disrupt end-to-end connectivity.  Prior to my
work, one promising approach to improving connectivity involved routing
traffic along multiple paths between two endpoints (``multipath
routing''); despite the promise of this approach, previous approaches
encountered two significant challenges: First, previous approaches for
disseminating information about multiple paths through the network did
not scale to large networks.  Second, end systems had no way to signal
to the network that an end-to-end path had failed or was providing
inadequate performance.  My research applied a new perspective to this
problem: rather than simply routing traffic on one of multiple paths to
a destination, allow traffic to switch paths at intermediate points en
route to the destination, and allow end systems to signal to the network
when it should attempt to use a different path to the destination using
a small number of bits that can be carried in the traffic itself.  This
system, called {\em path splicing}, provides up to an exponential
improvement in reliability for only a linear increase in the amount of
state that each router in the network must store.


{\bf Impact.} The foundation of this research theme comes from a system
I built called called ``rcc'' (router configuration checker).  This
system was the centerpiece of my doctoral dissertation and has had
significant impact in both research and industry.  The work received the
Best Paper Award at {\em ACM/USENIX Networked Systems Design and
Implementation (NSDI)} in 2005 and has been used by hundreds of Internet
Service Providers (ISPs) around the world to check their network
configurations for errors.

The NANO project is among the most visible of my research projects at
Georgia Tech: The project page receives about 2,000 unique visitors
every month, and, between January and March 2010, about 300 users have
downloaded the code.  The system is deployed on Google's Measurement
Lab, and we are exploring a version of NANO that could be incorporated
into the Google Toolbar with Google's Broadband Access Transparency
(BAT) team.

The path splicing work resulted in a Sigma Xi undergraduate research
award for Megan Elmore.  The work was funded by Cisco, and they have
considered the possibility of extending their existing multiple routing
configuration (MRC) function to support path splicing.  A more likely
deployment scenario, however, may be the incorporation of path splicing
into a network where network elements are more programmable (I discuss
the promise of programmable networking in ``Future Challenges'' below.)
We have published an open-source implementation of path splicing on
several programmable networking platforms.

\fp {\bf Representative Publication}
\begin{quote}
M. Tariq, A. Zeitoun, V. Valancius, N. Feamster, and
M. Ammar. ``Answering What-if Deployment and Configuration Questions with
'WISE'". {\em ACM SIGCOMM}, August 2008. Seattle, WA.
\end{quote}

\newpage
\fp {\bf Most Cited Publication (137 Citations)}
\begin{quote}
N. Feamster and H. Balakrishnan ``Detecting BGP Configuration Faults
with Static Analysis'' {\em 2nd Symposium on Networked Systems Design
and Implementation (NSDI)}, Boston, MA, May 2005. {\bf Best Paper Award.}
\end{quote}

\section*{Theme 3: Virtual Networking}

{\em Network virtualization} allows multiple networks to operate in
parallel on the same physical infrastructure.  Although this concept is
not new (commonly used Virtual Private Networks, or ``VPNs'', come to
mind as a prominent real-world example of network virtualization),
virtualizing all aspects of the network infrastructure---in particular,
both the links {\em and} the routers themselves---holds great promise
for enabling innovation.  In 2002, Larry Peterson, Scott Shenker, and
Jon Turner argued that networking research had ``ossified'', because
researchers faced a huge deployment hurdle for deploying their research
in production environments, and also because the large stakeholders had
little incentive to allow disruptive innovation to take place.  Their
argument was essentially that, by ``letting a thousand flowers bloom'',
multiple networking technologies could be deployed in parallel, thereby
providing researchers a path to innovation.  The main research challenge
was how to design and implement a virtual network infrastructure that
supported this philosophy.  

Towards solving this challenge, I began working on network
virtualization during my postdoc at Princeton.  Jennifer Rexford and I
wanted to implement a new network protocol we had designed at the end of
my graduate career. Our plan was to use PlanetLab---a large testbed with
virtualized servers distributed around the world---to do it.
Unfortunately, we quickly realized that PlanetLab did not have the
necessary functions to instantiate test {\em networks}; in particular,
PlanetLab offered no functions for building virtual routers and links,
and also had no support for forwarding traffic at high rates for virtual
routers (e.g., every packet needed to be copied several times at each
node, significantly slowing the packet forwarding rates).  These
shortcomings caused us to pursue a larger project to build such a
testbed that would support the kinds of experiments that we wanted to
run.  With Andy Bavier and Larry Peterson, we built a Virtual Network
Infrastructure (VINI), a testbed that allows researchers to build
virtual networks.  This work appeared in {\em ACM SIGCOMM} in 2006.
Although we still strive for more widespread adoption, the testbed is
regularly used by several research groups around the country.

Since this initial work, I have focused on two aspects of network
virtualization: (1)~providing Internet connectivity and routing control
to virtual networks; (2)~designing very fast packet forwarding
technologies for virtual networks.  A virtual network---either an
experiment or a distributed ``cloud'' service---typically needs
connectivity to the rest of the Internet so that users can actually
exchange traffic with it.  To provide such connectivity, and to give
each virtual network direct control over how user traffic reaches it, I
designed, implemented and deployed the Transit Portal.  This work will
appear in {\em USENIX Annual Technical Conference} in June 2010; it is
also a cornerstone of the larger nationwide GENI effort (featured here,
for example: \url{http://www.geni.net/?p=1682}).  Our work on designing
faster packet forwarding technologies for virtual networks started with
the Trellis project, which moved packet forwarding for virtual networks
into the kernel; although this work resulted only in a workshop
publication, the software itself was adopted by University of Utah's
Emulab, the most prominent emulation-based testbed for networking
research.  Our current efforts have focused on accelerating packet
forwarding further by supporting custom packet forwarding for virtual
networks in Field Programmable Gate Arrays (FPGAs); our work on
SwitchBlade, a platform for rapidly developing and deploying custom
forwarding engines in hardware for virtual networks, will appear at {\em
ACM SIGCOMM} in August 2010.

{\bf Impact.}  The impact of this work thus far has been to support
network experimentation for researchers; many other virtual network
technologies and platforms have built on this work. Our work on virtual
networks has been cited nearly 300 times (the VINI paper has been cited
200 times, and our work describing a network architecture based around
network virtualization has been cited over 100 times).

The Transit Portal is currently deployed in five locations, and I am
using it in my courses to provide students with hands-on experience
configuring networks of routers and connecting them to real BGP-speaking
routers on the Internet.  The course I have developed that uses this
technology is likely serves as the first course where students can
directly configure networks of routers that are connected to the global
Internet.

\fp {\bf Representative Publication}
\begin{quote}
B. Anwer, M. Tariq, M. Motiwala, N. Feamster. ``SwitchBlade: A Platform
for Rapid Deployment of Network Protocols on Programmable Hardware''
{\em ACM SIGCOMM}, New Delhi, India, August 2010.
\end{quote}

\fp {\bf Most Cited Publication (200 citations)}
\begin{quote}
A. Bavier, N. Feamster, M. Huang, L. Peterson, J. Rexford. ``In VINI
Veritas: Realisitic and Controlled Network Experimentation". {\em ACM
SIGCOMM}, August 2006. Pisa, Italy.
\end{quote}




%%%%%%%%%%%%%%%%%%%%%%%%%%%%%%%%%%%%%%%%%%%%%%%%%%%%%%%%%%%%
% next steps
% secure foundations: openflow, pedigree
% economic foundations: MINT
% open access: collage
% fusing machine learning+networks

\section*{Future Challenges}

%My research has focused on reactive approaches to network operations
%problems, treating the network itself as an existing artifact and
%``bolting on'' tools and algorithms to existing infrastructure.  
In my future work, I plan to take a proactive approach to network
security and operations by developing protocols from the ground up that
make networks fundamentally more secure and easier to manage.  Networks
are large, distributed systems that operators ``program'' with network
configuration.  Unfortunately, this configuration still takes place at
the granularity of individual network devices and relies on low-level,
mechanistic, and unintuitive commands.  Improving network security and
reliability ultimately requires raising this level of abstraction, which
will most likely require developing protocols and languages that allow
network operators to configure networks from a much higher layer of
abstraction than they do today.  It also entails fundamental changes to
both the network devices and protocols.  To achieve this goal, I see
opportunities to draw on techniques from other disciplines.  For
example, my most recently awarded NSF grant proposes applying
programming language and software engineering techniques to help
operators configure networks.

Along these lines, my long-term goal is to {\em design networks that
permit a range of policies but still have provable correctness and
security properties}.  In recent years, I have been redesigning network
protocols so that operators can more easily control which users have
access to which parts of the network (``access control''); I have also
been building host and network support to prevent unauthorized access
and data leaks (``information flow control'').  My goal is to make it
easier for operators to express---and verify---these policies in an
operational network.  I have made some early progress on these problems:
we have built a system called Resonance that allows operators to express
access control policies at a higher level of abstraction and may
ultimately replace the current campus access control system. (A working
version of this system was demonstrated at the GENI Engineering
Conference in March 2010.)  We are also building a system called
Pedigree, which allows network operators to prevent certain types of
data leaks.  Finally, I have begun to design new protocols and systems
to make home networks easier to configure and more secure.  

%% I plan to deploy and test these new protocols and systems in production
%% networks.  Historically, network researchers have faced significant
%% deployment hurdles: short of convincing a major router vendor to adopt
%% new technology, there was little attempt for practical impact.  Two
%% trends are changing the landscape for the better.  First, network
%% management---and, in particular network security---were once ignored,
%% but these topics have come to the forefront as the network's insecurity
%% and instability are coming to blows with users' demands for high
%% availability and security.  This shift in focus is particularly evident
%% in enterprise networks, where operational expenditures are a significant
%% fraction of the cost of running the network and security is of utmost
%% concern.  Second, network devices are becoming increasingly more
%% programmable; many vendors have taken steps towards ``opening up the
%% box'' to give operators (and researchers) can have more control and
%% opportunities to innovate.  One prominent example of this is the
%% OpenFlow project, which allows network devices to be controlled with
%% software programs that are not resident on the switch itself; other
%% example of this refactoring of network intelligence into software are
%% Cisco's AXP platform, Juniper's recent decision to expose its software
%% API, and standardization efforts in the Internet Engineering Task Force
%% (IETF) working group.

%% With the help of Georgia Tech's Office of Information Technology, and in
%% conjunction with a funded GENI project, I have begun some of this work
%% already.  We now have the first OpenFlow-based campus network that
%% routes production traffic in the country outside of Stanford University
%% (where the protocol was first proposed).  We have deployed a separate
%% research network on the Georgia Tech campus with switches that include
%% programmable hardware and are controllable from separate software
%% systems, and we are using this technology in a Next-Generation
%% Networking class that I developed and taught in Spring 2010; the course
%% had almost 50 students.

My continued efforts will not only help advance network operations, but
also to help shape the future of networking research towards more
``hands on'', operationally relevant problems.  My approach will not
only define new areas for networking research, but will also take better
advantage of Georgia Tech's unique facilities and research-minded
network operators.  I aim to make Georgia Tech the top destination for
performing experimental, systems-focused research on network operations
and security.

