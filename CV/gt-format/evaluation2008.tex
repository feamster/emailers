% evaluation.
% update the following
\newpage 

\setcounter{page}{1}
\lfoot{\fancyplain{Printed: \today}{Nick Feamster}}
\cfoot{\fancyplain{Revised: \today}{Annual Evaluation 2008}}
\rfoot{Page \thepage}

\setcounter{page}{1}
\begin{center}
{\Large\textbf{Accomplishments for 2007 \\ February 2008}}\\[.1in]
{\large\textbf{Nick Feamster}}\\[.05in]
feamster@cc.gatech.edu \\
\url{http://www.cc.gatech.edu/~feamster/} \\[.1in]
\end{center}

\begin{sloppypar}

\section*{Research}

In this section, I outline my research accomplishments.  I divide this
work into several distinct areas: recognition and publication,
software/systems artifacts, and funding.  Overall, this was a good year
for ``ramping up'' my lab and amassing resources.  The biggest challenge
this year will be to turn this input into a large quantity of high
quality research output (i.e., papers in major conferences, software, etc.).

\subsection*{Recognition/Publication}

This year I received several awards recognizing my contributions to
networking research, including the IBM Faculty Award and the Sloan
Reseach Fellowship.  Additionally, our group had the following major
conference paper acceptances last year:

\begin{itemize}
\itemsep=-1pt
\item Paper on a new spam filtering technique called ``behavioral
  blacklisting'' accepted to {\em ACM Conference on Computer and
  Communications Security (CCS) 2007}.
\item Paper on detecting network disruptions by monitoring BGP routing
  dynamics accepted to {\em SIGMETRICS 2007}. 
\item Two papers accepted to {\em HotNets 2007}: one paper on path
  splicing, and a second paper on improving Internet accountability with
  a new addressing format.
\end{itemize}

\noindent
Personally, I think this year was weak for publications, even though I
certainly had some.  Notably absent is a paper at either {\em SIGCOMM}
or {\em NSDI}. This is not for a lack of submitting (we sent two
submissions to both conferences); however, this year I hope to send
higher quality submissions and hopefully also have better luck!  In
2008, we already have 7 conference paper submissions in the pipline
(five to {\em SIGCOMM}), and several more planned.  One of this year's
goals will be increased paper output.

\subsection*{Funding}

The funding success roughly paralleled that of last year: it was a good
year for NSF funding; I received three NSF awards, one from the
CyberTrust program, one from Networking Broadly Defined, and one
Computing Research Infrastructure (CRI) grant.  I also received two new
grants, one from the Department of Homeland Security and another from
DARPA.  Commercial funding has been slower to acquire: my proposals to
Cisco are still in the pipeline, due to administrative issues.  However,
we have started a project with SecureComputing and have acquired funding
for one student for this project (through GTISC).  I also received the
Sloan Fellowship, which comes with funding support.

\begin{itemize}
\itemsep=-1pt
\item PI for three new NSF awards (CyberTrust, NeTS-NBD, and CRI).
\item Co-PI on new DARPA and DHS awards (the former is on scalable
  network monitoring; the latter is on monitoring network traffic for
  botnets).
\item Received contract from Secure Computing to support one graduate
  student for one term (more as the project continues).
\item Maintained relationships with Ironport (spam filtering
  company, recently acquired by Cisco) and Outblaze (large email hosting
  provider), for hopeful future funding (this should be more likely now
  that Cisco has acquired Ironport).
\end{itemize}

\noindent
Given this level of funding success, I will probably spend much of next
year focusing on paper submissions, versus writing more grant
proposals.  I do plan to submit to NSF where appropriate, and also to
try to (finally) get some funding from Cisco (mainly to start a
collaboration with them).

\subsection*{Software and Systems}

My main focus for software development has been in work on the VINI
network testbed, jointly with my students, Murtaza Motiwala and Vytautas
Valancius, and researchers at Princeton University.  This year, we got a
new version of the VINI kernel up and running, which forwards packets an
order of magnitude faster than the previous user-space kernel.  We
submitted a paper to NSDI on these improvements, which was rejected.  We
plan to add more functionality to the implementation (e.g., integration
with Emulab) to try to improve the paper's chances for upcoming
conference submissions.

We have also begun software implementations associated with several
other projects, including the path splicing and spam filtering
projects.  Implementations are currently preliminary.  We intend to
spend a large portion of spring and summer on these implementations and
others for possible submissions later this year (more details under
``goals'' below).

\begin{itemize}
\itemsep=-1pt
\item Sent two students, Vytautas Valancius and Murtaza Motiwala, to
  Princeton for summer to work with researchers at Princeton on VINI
  testbed.  Additions to VINI kernel and support for experimental
  specification.  (Unsuccessful NSDI submission on this topic.)
\item Preliminary implementation of ``SpamTracker'', currently being
  tested at SecureComputing.
\end{itemize}

\section*{Service}

I have continued my service both internally to the department, and
externally on program committees and organizing committees.  In addition
to serving on program committees, I have given various tutorials, and
have helped organize sessions at NSF PI meetings, as outlined below.  I
have also kept some aspect of my external service practical by
continuing to serve on the NANOG program committee.  

In particular, I have been asked to serve on the program committees of
all major conferences this year: {\em SIGCOMM 2008}, {\em SIGMETRICS
2008}, {\em IMC 2008}, {\em NSDI 2009}, {\em CCS 2008}, and {\em CoNext
2008}.  Additionally, I am organizing a DIMACS workshop on Internet
security this March.  I also gave a DIMACS tutorial last August on
algorithmic foundations for the next-generation Internet in August 2007.
Due to the large number of program committees I am serving on this year,
I have tried to avoid further committments with respect to conference
organization, etc.

\begin{itemize}
\itemsep=-1pt
\item Lead organizer of the {\em DIMACS Workshop on Internet Security},
  March 2008.
\item Program committee member of all major networking conferences, and
  also some major security conferences.
\item Continuing member of {\em NANOG} Program Committee.
\end{itemize}

Internally, I am serving on the faculty recruiting committee and the
Ph.D. recruiting committee for Spring 2008.  I am also involved, with
Alex Gray, as part of a large-scale effort to re-invent the entire
Ph.D. process.  I expect service on this committee to take a significant
amount of effort as we gather input for making improvements to the program.

\section*{Education}

My most significant educational accomplishment this year was the
complete re-vamping of CS~7001 with Alex Gray.  Our contributions, in
the form of a first-of-its-kind course on ``Introduction to
Ph.D. Research'' will appear in {\em ACM SIGCSE} in March 2008.  Alex
and I are continuing to make improvements to the course, and we
eventually hope to turn our course notes into a book.

I also continued to develop material for CS~7260.  Additional
improvements were as follows:
\begin{itemize}
\itemsep=-1pt
\item Added a wiki discussion page for discussion of research papers.
\item Added recorded lectures to the course home page.
\item Added several new lectures, including material on network
  measurement and performance analysis.
\item Added new problem set questions, including some new ``hands on''
  assignments with Emulab and PlanetLab.
\end{itemize}
\noindent
I have also continued to add new content to this course, parts
of which are detailed in my CV.  

\newpage
\begin{center}
{\Large\textbf{Goals for 2008}}\\[.1in]
{\large\textbf{Nick Feamster}}\\[.05in]
feamster@cc.gatech.edu \\
\url{http://www.cc.gatech.edu/~feamster/} \\[.1in]
\end{center}

\section*{Research}

I will continue to develop my research program around the broader theme
of {\em network operations}, developing tools, techniques, and
algorithms that make a network operator's job of running the network
easier.  I think my research agenda is taking shape in terms of three
larger research areas: (1)~network security; (2)~scalable networking
monitoring; (3)~network virtualization/path splicing.  I outline my
research goals for each of these projects below.

I feel as though my research in these areas is beginning to mature, and
that now will be the time when we can start to implement and deploy
real-world systems on some of these ideas (as well as begin to write
systems papers on these ideas).  In particular, we have several projects
that are ripe for implementation work.  I outline these projects and the
associated goals for this year below.  

\subsection*{Network Security}

My main goals in this area relate to software development and further
developments on two projects: (1)~SpamTracker; and (2)~Pedigree; as well
as further development and measurement studies of botnets and spammers
that will help us develop better systems for stopping unwanted traffic.

We had an initial paper on SpamTracker in {\em CCS 2008}, but the work
is far from done.  In particular, we must explore a much wider variety
of clusteirng algorithms and network features to incorporate into the
spam filtering software.  As part of this task, I am working with Alex
Gray and a co-advised student, Nadeem Syed, to develop better learning
algorithms for spam filtering based on network-level behavior.  I am
also working with a new student, Shuang Hao, to better understand the
dynamics of spamming bots.  I hope to publish several papers this year
in conferences such as {\em Internet Measurement Conference}, {\em CCS},
etc. on the dynamics of the hosting infrastructure for spamming,
phishing, and other malicious behavior.  (Shuang and I have already
submitted a preliminary paper to the {\em USENIX LEET} workshop on
estimating botnet population from attack traffic.)

On the implementation front, it is imperative for us to have an
implementation of SpamTracker running not just at SecureComputing, but
also at Georgia Tech as an independent blacklisting service.  This
implementation will allow us to gather more query data and also to
evaluate the effectiveness of SpamTracker's algorithms in practice.
Finally, we have designed and implemented a new system to curtail
unwanted traffic called Pedigree, which relies on instrumenting end
hosts with a trusted traffic tagger: network elements (e.g., routers,
firewalls) can read these tags to determine the provenance of network
traffic.  One of the goals for this summer is to implement Pedigree for
deployment and ultimate paper submission to NSDI.

\begin{itemize}
\itemsep=-1pt
\item Submit one paper to {\em IMC} and one paper to {\em CCS} on
  analysis of spamming and botnet dynamics, with an eye towards
  developing better algorithms for SpamTracker.
\item Develop prototype implementation of Pedigree for possible {\em
  NSDI} submission.
\item Implement and deploy SpamTracker at Georgia Tech.
\end{itemize}


\subsection*{Scalable Network Monitoring and Troubleshooting}

Work on scalable network monitoring thus far has fallen under two
distinct camps: (1)~monitoring networks for disruption with active and
passive monitoring; (2)~discovering network dependencies through
both configuration analysis and passive analysis of network traffic for
troubleshooting.  I plan to continue these tow lines of research, as
well as to continue a third line of research on scalable monitoring of
traffic on high-speed links, which I started in 2006 but temporarily
shelved last year.  I briefly outline the goals for each of these
thrusts below.  

\paragraph{Disruption monitoring.}  With my student Yiyi Huang, I have
started a project with Thomson Labs to develope a scalable monitoring
system for network disruptions based on active network measurements.
The analysis thus far has been preliminary, however, and we still have
no working system.  Although we submitted a paper to {\em SIGCOMM}, I do
not have high hopes for the paper.  The student, in particular,
performed somewhat sloppy work and did not take ownership of the
project.  This year, I hope to finish this project with an
implementation of our proposed system on the Georgia Tech campus
network; I also hope to ultimately get the work accepted to a top-tier
conference (presuming that this work does not get accepted at SIGCOMM). 

Murtaza Motiwala and I also started a project on in-band network
troubleshooting in 2006, but temporarily shelved the project to work on
path splicing.  We intend to write two papers on this topic: one related
to the implementation of Orchid (the in-band troubleshooting scheme)
itself (including its implementation on a NetFPGA), and a second paper
on the security analysis of Orchid.

\begin{itemize}
\itemsep=-1pt
\item Complete work with Thomson: deploy system on Georgia Tech campus
  network
\item Submit the paper on the resulting system to a top-tier networking
  conference. 
\end{itemize}

\paragraph{Network dependency analysis and troubleshooting.}  My student
Mukarram Tariq spent the summer at Google and devised some new
algorithms for learning network dependencies from passive traffic
analysis for evaluating what-if scenarios in a content distribution
network.  We have extended these ideas to learn dependencies for the
purposes of network troubleshooting.  We plan to finish the design and
implementation of this system, deploy it on the Georgia Tech campus
network, and submit a paper on the proposed system to NSDI.

\begin{itemize}
\itemsep=-1pt
\item Get Mukarram's work on what-if scenario evaluation accepted to a
  top-tier networking conference (\eg, SIGCOMM, NSDI).
\item Submit two additional papers to {\em NSDI}: one on discovering
  dependencies with configuration analysis and one on discovering
  dependencies from passive traffic analysis.
\end{itemize}

\paragraph{High-speed traffic monitoring.} We recently received a DARPA
grant on scalable network monitoring.  We previously had work on traffic
sampling for monitoring small flows on a high-speed network link and
have recently extended this work to support flexible monitoring for a
variety of applications (mostly security-related).  We hope to submit
the first version of this work to the {\em Internet Measurement
Conference} and also to deploy the system on the campus network.
Finally, we plan to integrate this system with anomaly detection systems
that have trouble analyzing all traffic on high-speed links (e.g.,
Wenke Lee's BotHunter and BotSniffer systems).  We are also developing
new monitoring algorithms, with Yiyi Huang, that adjust sampling rates
based on the presumed state of the hosts on the network (e.g., whether
hosts are infected); we plan to submit this work to {\em SIGMETRICS}.

\begin{itemize}
\itemsep=-1pt
\item Submit one paper to {\em IMC} on FlexSample (new
  framework for high-speed traffic monitoring).
\item Submit one paper to a top security conference on applications of
  (or variations of) FlexSample for network security-related
  applications.
\end{itemize}

\vspace*{0.1in}
\noindent
Finally, I have been working on developing a network data repository,
with Dave Andersen at CMU.  I plan to continue building this
infrastructure, and I would like to send a paper to {\em NSDI} on
network measurement infrastructure for supporting sound Internet
measurement strategies.

\subsection*{Path Splicing/Network Virtualization}

The path splicing work is well underway, with a full paper on the work
already submitted to {\em SIGCOMM}; we have high hopes for this paper's
acceptance!  If the paper is not accepted, we will revise and send to
NSDI.  In parallel, we are working on a prototype implementation of path
splicing in VINI and expect to have this completed in the next few
months.  In either case, we will likely write a follow-up paper on
the implementation and applications of path splicing for submission to
{\em NSDI}. 

\begin{itemize}
\itemsep=-1pt
\item Implementation of path splicing on VINI.  Possible implementation
  in hardware.
\item Submit a second follow-up paper describing the path splicing
  implementation and deployment, for submission to {\em NSDI}.  
\end{itemize}

We are also working on a related project involving hardware acceleration
of virtual networks on the NetFPGA platform.  The main idea is to
provide hardware support for packet forwarding in virtual networks (as
opposed to in user space or in software in the kernel).  The challenge
is to figure out how to provide this forwarding support in hardware,
which is not readily virtualizable.  We intend to develop a 
design and implementation for hardware virtualization of network
forwarding and submit this work, first to the SIGCOMM workshops and then
to a major networking conference.

\begin{itemize}
\itemsep=-1pt
\item Develop an implementation of virtualized forwarding tables in
  NetFPGA hardware.  
\item Submit a workshop paper and ultimately a conference paper to a
  top-tier conference (e.g., NSDI) on this work.
\end{itemize}

\subsection*{New Directions}

Social networks is a burgeoning area in computer communications
networks.  I intend to begin a small foray into this area with some
research problems aligned around a broad theme: how can social networks
be leveraged to bootstrap trust in communications networks?  We have
three related projects in this area: (1)~using social networks to
develop a new type of CAPTCHA metchanism that establishes not only
whether a user is human but also whether the human is who he or she
claims to be; (2)~using social networks as a platform for distributed,
collaborative network monitoring; (3)~using social networks as a
mechanism for passing capabilities/authentication tokens out-of-band.
We intend to develop at least preliminary workshop publications on these
topics and ultimately develop related systems for publication this year
or early next year.

\section*{Education}

My primary goal for development in education this year will be to
solidify some course content for eventual inclusion in textbooks.  On
this front, I will be developing material for eventual inclusion in two
books: (1)~a book on ``how to be a Ph.D. student'' with Alex Gray; (2)~a
book on computer networking for graduate students, including hands-on
assignments, etc. (co-authored with my advisor, Hari Balakrishnan).  I
do not expect to finish either of these books this year, but I plan to
make significant headway.  There is a small chance that I can finish the
book with Alex Gray this year.


% tutorials and book chapters
In addition to the above books, I will contribute a book chapter on
algorithmic foundations for the next-generation Internet as a follow-up
to the DIMACS workshop that I organized this past August.  Additionally,
I will give at least two tutorials: one at the technical university in
Brazil, and one at the Internet2 symposium in late April.  Tentatively
on the docket, pending approval, are a tutorial on VINI at {\em SIGCOMM}
and a tutorial in Africa about practical mechanisms for reducing
unwanted traffic.

\begin{itemize}
\itemsep=-1pt
\item Book chapters for CS~7001 book on preparation for Ph.D. research
  (perhaps completion of monograph).
\item Book chapters for book on hands-on networking experience for
  graduate students, based on CS~7260 material.
\item Book chapter on algorithmic foundations for next-generation
  Internet routing protocols
\item Tutorials: DIMACS, Internet2, and possibly SIGCOMM, AfNOG
\end{itemize}

\section*{Service}

My primary external service this year will be on the major program
committees, as mentioned above.  My primary internal service will be on
the faculty recruiting committee, the Ph.D. recruiting committee, and
the committee to reinvent the Ph.D. curriculum.  I also consider my
development of CS~7001 as service to the department.  There is a small
possibility that I will attempt to organize another workshop on Internet
routing evolution and design, as I did in the fall of 2006.

\begin{itemize}
\itemsep=-1pt
\item Program committee service: SIGCOMM, NSDI, SIGMETRICS, IMC, CCS
\item Faculty recruiting committee
\item Ph.D. recruiting committee
\item CS~7001 development and Ph.D. curriculum development
\end{itemize}

\section*{Enablers}

The most difficult part of keeping this many research projects in the
air is to limit distractions and obstacles.  We encountered several
barriers this year to creating a lab identity, such as refusing to offer
my lab a subdomain of cc.gatech.edu.  These decisions are extremely
stifling to external visibility, which ultimately hurt student
recruiting, etc.  In the future, we must figure out how to lower these
barriers, or to come up with creative workarounds.

Ultimately, I think the biggest threat to my research may be a lack of
single focus.  I am not sure exactly how to handle this.  It seems that
ultimately the tenure case will be based on one or two ``big''
contributions.  I could use some guidance in helping better define my
research ``persona''.  I think I will need assistance in the following
areas:
\begin{itemize}
\itemsep=-1pt
\item Help attracting the best networking and systems students to
  continue working with me.

  I plan to make efforts to attract the best students to my group (e.g.,
  making a trip to India, giving talks at other universities, getting my
  research in the news, etc.), but I will need continued protection from
  other faculty members, administrative barriers, and appropriate
  ``advertisements'' of my work to other students and faculty when
  appropriate.  An organizational move to avoid some of these obstacles,
  I think, would be a move away from ``research groups'', which I think
  makes good sense in any case.

\item The ability to ``cash in'' on my teaching relief sometime before
  tenure.
\end{itemize}

\end{sloppypar}
