% cv-research.tex
\section{RESEARCH AND CREATIVE SCHOLARSHIP}
\label{sec:research}

\subsection{Theses}
\label{subsec:thesisch}

%\begin{cvlist}{}
\begin{pub}
\mkbib{Feamster2005:phd}
\mkbib{feamster:meng}
\end{pub}
%\end{cvlist}


%%%%%%%%%%%%%%%%%%%%%%%%%%%%%%%%%%%%%%%%%%%%%%%%%%%%%%%%%%%%
% Journal

\subsection{Journal Publications}
\label{subsec:journals}


\begin{pub}
\mkbib{burnett2013:censorship}
\mkbib{kim2013:improving}
\mkbib{Tariq2010:wise}
\mkbib{Sundaresan2012:broadband}
\mkbib{sundaresan2012:accelerating}
\mkbib{Chetty2012:sdn}
\mkbib{Motiwala2012:cost}
\mkbib{Feamster2011:neted}
\mkbib{Yu2011:vlan}
\mkbib{Koponen2011:fii}
\mkbib{Calvert2011:home}
\mkbib{Anwer2010:netfpga}
\mkbib{Feamster2006:policy}
\mkbib{Feamster2006}% ToN paper
\mkbib{Feamster2004f} % CCR submission
\mkbib{Feamster2003e}
\end{pub}

%% \subsection{Published Books and Parts of Books (Refereed)}
%% \label{subsec:books}

%% \subsubsection{Chapters in Books}
%% \label{subsubsec:chapters}

%% \subsection{Edited Proceedings and Collections}
%% \label{subsec:edited}

\subsection{Books and Book Chapters}

\begin{pub}
\mkbib{Lee2008:botnet}
\end{pub}

%%%%%%%%%%%%%%%%%%%%%%%%%%%%%%%%%%%%%%%%%%%%%%%%%%%%%%%%%%%%
% Conference

\subsection{Refereed Conference Publications}


% \hspace*{-0.75in}{\it\large Internet Routing} 

\begin{pub}

\mkbiba{chetty2013:dev}{33}
\mkbiba{mundada2013:silverline}{19}

\mkbiba{Hao2013:dns}{23}
\mkbiba{Grover2013:nat}{23}
\mkbiba{Sundaresan2013:webperf}{23}

\mkbiba{Xing2013:poison}{16}
\mkbiba{Valancius2013:pecan}{11}

\mkbiba{Lumezanu2012:spam}{24}
\mkbiba{Katz2012:lifeguard}{14}
\mkbiba{Lumezanu2012:bias}{20}
\mkbiba{Konte2012:pam}{30}

\mkbiba{Kim2011:config}{19}
\mkbiba{Hao2011:dns}{19}
\mkbiba{Valancius2011:tiers}{14}
\mkbiba{Sundaresan2011:bismark}{14}
\mkbib{Ramachandran2011:ceas}

\mkbiba{Anwer2010:switchblade}{12}
\mkbiba{Burnett2010:collage}{15}
\mkbiba{Antonakakis2010:notos}{15}
\mkbiba{Valancius2010:bgpmux}{17}
\mkbiba{Perdisci2010:http}{16}

\mkbiba{Tariq2009:nano}{17}
\mkbiba{Tariq2009:vlan}{22}
\mkbiba{Cunha2009:tomo}{22}
\mkbiba{Hao2009:snare}{15}

\mkbiba{Konte2008:pam}{20}
\vspace*{-0.05in}
{\bf Best paper award.}
%
\mkbiba{Ramachandran2008:flexsample}{17}
%
\mkbiba{Motiwala2008:splicing}{12}
\mkbiba{Tariq2008:wise}{12}
\mkbiba{Andersen2008:aip}{12}

\mkbib{Feamster2008:sigcse}
\mkbiba{Ramachandran2007:spamtracker}{24}
\mkbiba{Khadilkar2007:dhcp}{24}
\mkbiba{Huang2006:bgp}{17}
\mkbiba{Wang2006:bgp}{40}
\mkbiba{Lee2007:mobcast}{47}
\mkbiba{Feamster2006:spam}{12}
\vspace*{-0.05in}
{\bf Best student paper award.}

\mkbiba{Bavier2006}{12}


\mkbib{Feamster2005:allerton}
\mkbiba{Feamster2005b}{11}
\mkbiba{Freedman2005}{24}
\mkbiba{Feamster2004h}{22}
\vspace*{-0.05in}
{\bf Best paper award.}
\mkbiba{Caesar2004}{22}
\mkbiba{Feamster2004b}{25}
\mkbiba{Feamster2004}{12}
\mkbiba{Feamster2003}{12}


\mkbiba{Feamster2002}{17}
\vspace*{-0.05in}
{\bf Best student paper award.}


\mkbiba{Fu2001}{28}
\vspace*{-0.05in}
{\bf Best student paper award.}

%\hspace*{-0.75in}{\it\large Adaptive Streaming Media Protocols} 

\mkbiba{Wee99}{45}
\mkbib{Feamster99}
\end{pub}

%%%%%%%%%%%%%%%%%%%%%%%%%%%%%%%%%%%%%%%%%%%%%%%%%%%%%%%%%%%%
% Workshops


\subsection{Workshop Publications}

\begin{pub}

\mkbib{feamster2013:isoc-latency}
\mkbib{chetty2013:isoc-latency}

\mkbiba{anwer2013:slick}{23}
\mkbiba{feamster2013:sdx}{23}

\mkbib{voellmy2012:procera}
\mkbib{martin2012:prio}

\mkbib{Sundaresan2011:labels}
\mkbib{Feamster2011:neted-workshop}
\mkbib{Mundada2011:silverline}

\mkbib{Anwer2010:visa}
\mkbib{Feamster2010:homenets}
\mkbib{Calvert2010:homenets}

\mkbib{Anwer2009:visa}
\mkbib{Nayak2009:resonance}

\mkbib{Bhatia2008:roads}
\mkbib{Valancius2008:mint}
%
\mkbiba{Tariq2008:nano}{20}
%
\mkbib{Yardi2008:wosn}\nw
\mkbib{Ramachandran2008:wosn}\nw

\mkbiba{Motiwala2007:splicing}{18}
\mkbiba{Andersen2006:aip}{18}
\mkbib{Ramachandran2007:bitstore}

\mkbib{Ramachandran2006:recon}
\mkbib{Ramachandran2006:dnsbl}

\mkbib{Feamster2004e}
\mkbib{Feamster:fdna2004}
\mkbib{Feamster2004d}

\mkbib{Feamster2003h}
\mkbib{Feamster2003f}
\mkbib{Feamster2003b}

\mkbiba{Andersen02}{42}
\mkbib{Feamster02-pv}

\mkbib{Feamster01}

\end{pub}

\subsection{Other}
\label{subsec:other}

\subsubsection{Submitted Journal Papers}
\label{subsubsec:submitted-journal}

\internalnote{
\begin{pub}
\mkbib{Feamster2011:cacm}
\mkbib{Feamster2006:rl}
\end{pub}
}

\subsubsection{Submitted Conference and Workshop Papers}
\label{subsubsec:submitted-conf}

\internalnote{
\begin{pub}
\mkbib{xing2013:bobble}
\mkbib{grover2013:nat}
\mkbib{kim2013:config}
\mkbib{hao2013:dnsreg}

\mkbib{mundada2013:silverline}
\mkbib{chetty2013:za}

\mkbib{Anwer2012:bubble}
\mkbib{Sundaresan2012:web}
\mkbib{Rotsos2012:staggercast}
\mkbib{Sundaresan2012:keystone}
%\mkbib{Mundada2010:flowflex}
\end{pub}
}

\subsubsection{Other Technical Reports, Unrefereed Papers, and Drafts in
Preparation}
\label{subsubsec:technical-reports}

\begin{pub}

%sculpte is GT-CS-10-16
%the sigm submission is GT-CS-10-04

\mkbib{Sundaresan2013:wtf}
\mkbib{Sundaresan2010:sculpte}
\mkbib{Sundaresan2009:grenouille}


% http://smartech.gatech.edu/handle/1853/25463

%\mkbib{Valancius2009:mint}
%\mkbib{Lychev2009:contract}
%\mkbib{Ramachandran2009:spamspotter}
\mkbib{Ramachandran2008:fish4phish}
\mkbib{Huang2008:doppler}


% in preparation
\mkbib{Valancius2008:bgpmux-tr}
\mkbib{Ramachandran2008:pedigree}
%\mkbib{Hao2008:botnets}
\mkbib{Bhatia2008:trellis}
%\mkbib{Motiwala2007:orchid}
\mkbib{Feamster2007:presto}
\mkbib{Motiwala2007:presto}
\mkbib{Valancius2007:presto}
\mkbib{Feamster2007:cabo-ccr}
\mkbib{Feamster2006:diagnosis-wired}
\mkbib{Motiwala2006:orchid-wired}
\mkbib{Andersen:datapository2006-tr}
\mkbib{Feamster2004c}
\mkbib{Feamster2003g}
\mkbib{Feamster2002b}
\mkbib{Feamster01b}
\mkbib{id-path-validation}
\end{pub}

\subsubsection{Software}
\label{subsubsec:softwaredl}

My research group regularly releases software and makes a practice of
releasing source code with most published papers.  As a supplement to
the descriptions below, my research group's Github page is available at:
\url{http://github.com/gtnoise/}.

\begin{pub}
\item {\em Project BISmark: An Application Platform for Home Networks.}
  Project BISmark (Broadband Internet Service Benchmark) is a platform
  for developing network management applications for home networks.  The
  BISmark firmware is based on OpenWrt, an open-source operating system
  for home routers.  Currently, BISmark includes a suite of passive and
  active network measurements that allows a home Internet user to
  continuously monitor various performance metrics, such as upstream and
  downstream throughput, latency, and packet loss.  As of Spring 2013,
  BISmark is deployed in nearly 300 homes around the world in more than
  20 countries.  We are currently working both to expand the deployment
  and to extend the capabilities of the platform, to allow other
  researchers to use the platform for their own measurements.  \\ See
  \url{http://projectbismark.net} for details.

\item {\em MySpeedTest: A Tool for Mobile Performance Measurement.}
  Building on the success of BISmark, my students Sachit Muckaden,
  Abhishek Jain, and I have developed a tool to measure performance from
  mobile cellular handsets.  The application collects a variety of data,
  including latency and throughput measurements to a variety of servers
  around the world, hosted by Measurement Lab.  The application is now
  deployed on more than 4,000 handsets in over 130 countries.  Some of
  the most significant deployments are in developing countries, and we
  are currently collaborating with ResearchICTAfrica, a policy
  organization in Africa, to study the performance of both fixed and
  mobile broadband across the continent.  Software is available at
  \url{http://goo.gl/28tx3}.

\item {\em Bobble: Bursting Online Filter Bubbles.} With students Xinyu
  Xing, Dan Doozan and colleaguge Wenke Lee, I have designed and
  developed Bobble, a Chrome extension that allows users to see how
  their Web searches appear from different vantage points.  The filter
  bubble is a concept developed by Internet activist Eli Pariser in his
  book to describe a phenomenon in which websites use algorithms to
  predict what information a user may like to see based on the user's
  location, search history, etc. As a result, a website may only show
  information which agrees with the user's past viewpoints. A typical
  example is Google's personalized search results. To "pop" the bubbles
  created by Google search (also called de-personalization), our
  research group in the Georgia Tech Information Security Center is
  conducting ground-breaking research and developing software, Filter
  Bubble. Filter Bubble is a chrome extension that uses hundreds of
  nodes to distribute a user's Google search queries world wide each
  time the user performs a Google search. Using Filter Bubble, a user
  can easily see differences between his and others' Google search
  returns.  The plugin has been installed by more than 100 users around
  the world and is available at \url{http://bobble.gtisc.gatech.edu/}.

\item {\em Appu: Measuring Online Privacy Footprints.}  With so many web
  applications and sites in the current time, it's hard for a user to
  keep track of where does her personal information reside. With Appu,
  we aim to make this job easier for the end user. Appu keeps track of
  personal information such as passwords, username, birthdate, address,
  credit card numbers, and social security number so that a user can
  find out all sites that store a particular bit of personal
  information.  In the current beta release, Appu downloads personal
  information from sites where you have account and also tries to
  prevent password reuse across websites by warning users about
  it. Software is available at \url{http://appu.gtnoise.net/}

\item {\em Campus-Wide OpenFlow Deployment: Access and Information Flow
  Control for Enterprise Networks.} Resonance is a system for
  controlling access and information flow in an enterprise network.
  Network operators currently use access control systems that are
  coarse-grained (i.e., it is difficult to apply specialized policy to
  individual users) and static (i.e., it is difficult to quickly change
  the extent of a user's access).  Towards fixing these problems, we
  have developed a system that allows network operators to program
  network policy using a controller that is distinct from the switch
  itself and can be programmed to implement network-wide policy.  We
  have implemented and deployed this system in an operational network
  that spans two buildings on the Georgia Tech campus; the network sees
  regular use, and a deployment in Georgia Tech dormitories or the
  wireless network is planned for the near future.  We first
  demonstrated the function of this network at the 7th GENI Engineering
  Conference in March 2010, and recently demonstrated a version on
  Resonance that facilitates various home network management tasks at
  the 2011 Open Network Summit. \\ See \url
  {http://groups.geni.net/geni/wiki/BGPMux} for details.
%
\item {\em NANO: Network Access Neutrality Observatory.} The Network
  Access Neutrality Observatory (NANO) is a system to help users
  determine whether their traffic is being discriminated against by an
  access ISP.  In contrast to existing systems for detecting network
  neutrality violations, NANO makes no assumptions about the mechanism
  for discrimination or the services that the ISP might be discriminated
  against.  NANO has been released in collaboration with Google as part
  of the Measurement Lab project.  A preliminary version of the software
  was released to a small group of users in March 2009 for testing; a
  complete release is available for download at:
  \url{http://gtnoise.net/nano/}.  
%
\item {\em Implementation of GENI Prototype: Virtual Networks and BGP
  Session Multiplexer.} In the process of developing software for the
  NSF-Sponsored GENI Project Office.  This project (1) adds facilities
  and functions to the VINI testbed to enable experiments to carry
  traffic from real users; and (2) increases the experimental use of the
  VINI testbed by providing a familiar experiment management
  facility. The deliverables for this project all comprise software for
  supporting external connectivity and flexible, facile experimentation
  on the GENI testbed. The primary deliverables are a BGP session
  multiplexer---a router based on the Quagga software routing suite,
  software support for virtual tunnel and node creation, and integration
  of the above functionality with clearinghouse services developed as
  part of the ProtoGENI project.  \\ See
  \url{http://groups.geni.net/geni/wiki/BGPMux}.

  This project contributes to GENI design and prototyping through BGP
  mux development integration with ISPs; tunnel and topology
  establishment and management; ProtoGENI clearinghouse integration; and
  support for isolation and resource swapout.  With researchers at
  Princeton, we have also built VINI, a large distributed testbed for
  specifying virtual network topologies and experimenting with routing
  protocols and architectures in a controlled, realistic emulation
  environment.  See \url{http://vini-veritas.net/} for details.
%
\item {\em rcc: router configuration checker}.  Static configuration analysis
tool for Border Gateway Protocol (BGP) routing configurations.
Downloaded by over 100 network operators and many large, nationwide
backbone ISPs around the world.  See \url{http://gtnoise.net/rcc/} for
details. 
%
\item {\em Infranet}.  System for circumventing Web censorship firewalls
  (\eg, those in China, Saudia Arabia, etc.).  Available on
  Sourceforge.  Featured in articles in {\em Technology Review}, {\em
  New Scientist}, and {\em Slashdot}.  See
  \url{http://nms.lcs.mit.edu/projects/infranet/}.
\item {\em The Datapository}.  Originally the ``MIT BGP Monitor'', the
  Datapository is growing to support multiple data feeds (\eg, spam,
  end-to-end measurement probes, traceroutes, Abilene data, etc.).
  Currently used by researchers at Georgia Tech, Carnegie Mellon,
  University of Michigan, Princeton, MIT.  See
  \url{http://www.datapository.net/} for details.
%
\item {\em Secure BGP Implementation}.  Implementation of S-BGP in the
  Quagga software router.  Our implementation may be used by Randy Bush
  and Geoff Huston in their project to develop a certificate
  infrastructure for secure routing protocols.
%
\item {\em SR-RTP}.  Transport protocol for selective retransmission of
  packets in an MPEG video stream.  Incorporated into ``Oxygen TV'' for
  MIT Project Oxygen.  Some ideas incorporated into the OpenDivX
  video transport protocol.
\end{pub}

\subsubsection{Conference Posters and Demos}
\label{subsubsec:posters}

\begin{pub}

\mkbib{roy2013:sigcomm}
\mkbib{kim2012:coronet}

\mkbib{Sundaresan:sigcomm2012:bismark}

\mkbib{Kim:sigcomm2011:cap}
\mkbib{Konte:sigcomm2011:bgp}

%\mkbib{Valancius:nyce2008:mint}

\mkbib{Valancius:sigcomm2010:tp}
\mkbib{Burnett:sigcomm2010:collage}
\mkbib{Sundaresan:sigcomm2010:sculpte}


\mkbib{Ramachandran:sigcomm2009:pedigree}
\mkbib{Valancius:sigcomm2009:bgpmux}
\mkbib{Motiwala:sigcomm2009:splicing}


\mkbib{Burnett:nsdi2009:collage}
\mkbib{Ramachandran:sigcomm2008:pedigree}
\mkbib{Mundada:nsdi2008:trinity}
\mkbib{Valancius2007:bgpmux-conext}
\mkbib{Syed2007:spamsvm-nips}
\mkbib{Motiwala:nsdi2007:orchid}
\mkbib{Ramachandran:nsdi2006:spam}
\mkbib{Feamster:nsdi2004:rcc}
\end{pub}


\subsection{Research Proposals and Grants (Principal Investigator)}
\label{subsec:research}


\def\funding#1#2#3#4#5{\item{{{\bf #1}}
\newline Sponsor: {#2}
\newline Investigator(s): {#3}
\newline Amount: {\sl #4}
\newline {\sl #5}}}


\subsubsection*{1.~Approved and Funded}
\label{subsubsec:approved}
\setcounter{subsubsection}{1}

\begin{grants}

%EAGER: Collaborative: Aster*x: Load-Balancing Web Traffic over Wide-Area Networks  (NSF 1050234)
%GENI OPEN FLOW CAMPUS BUILDOUT (36566P0) - not sure about project number


\funding{Improving the Performance and Security of Home Networks with Programmable Home Routers
}{N. Feamster (PI)}{\$65,000 for 1 year}{Awarded:
  September 2013}\nw

\funding{Demand Characterization and Management for Access Networks}{Cisco
  Systems}{N. Feamster (PI) and R. Johari}{\$99,766 for 1 year}{Awarded:
  April 2013}\nw

\funding{Personal Information Fusion with In Situ Sensing
  Infrastructure}{National Science Foundation}{N. Feamster
  (PI)}{\$75,000 for 1 year.}{Awarded: July 2012} 

\funding{Characterizing and Exposing Bias in Social and Mainstream
  Media}{National Science Foundation}{N. Feamster (PI)}{\$175,000 for 1
  year.}{Awarded: July 2012} 

\funding{I-Corps: Helping Users and ISPs Manage Home Networks with
  BISmark}{National Science Foundation}{N. Feamster (PI)}{\$50,000 for 1
  year.}{Awarded: June 2012} 

\funding{Optimizing Network Support for Cloud Services: From Short-Term
  Measurements to Long-Term Planning}{National Science Foundation}{N. Feamster (PI), J. Rexford}{
  \$574,996.00 for 4 years.}{Awarded: April 2012}

\funding{Facilitating Free and Open Access to Information on the
  Internet}{National Science Foundation}{R. Dingledine, N. Feamster
  (PI), E. Felten, M. Freedman, 
  H. Klein, W. Lee}{\$1,500,000 for 4 years}{Awarded: March
  2011}

\funding{Measurement Infrastructure for Home Networks}{National Science
  Foundation}{K. Calvert, W.K. Edwards, 
  N. Feamster (PI), R. Grinter}{\$1,200,000 for 4 years}{Awarded: February
  2011}

\funding{Monitoring Free and Open Access to Information on the
  Internet}{Google Focus Grant}{N. Feamster and W. Lee}{\$1,500,000 for
  3 years}{Awarded: February 2011}

\funding{GENI OpenFlow Campus Buildout}{GENI Project Office}{N. Feamster
  (PI), Russ Clark}{\$64,675 for 1 year}{Awarded: October 2010}

\funding{Architecting for Innovation}{National Science
  Foundation}{H. Balakrishnan, N. Feamster, B. Godfrey, N. McKeown,
  J. Rexford, S. Shenker (PI)}{\$200,000 for 1 year}{Awarded: September
  2010}

\funding{Aster*x: Load-Balancing Web Traffic over Wide-Area
  Networks}{National Science Foundation}{N. Feamster 
  (PI), Russ Clark}{\$75,000 for 1 year}{Awarded: August 2010}

\funding{Network-Wide Configuration Testing and Synthesis}{National Science
  Foundation}{N. Feamster (PI), A. Akella}{\$500,000 for 3 years}{Awarded:
  June 2010}


\funding{MEDITA - Multi-layer Enterprise-wide Dynamic Information-flow Tracking \& Assurance}{National Science
  Foundation}{N. Feamster, A. Orso (PI), M. Prvulovic}{\$900,000 for 3
  years}{Awarded: March 2010}

\funding{Campus Network Access and Admission Control with
  Openflow}{National Science Foundation}{N. Feamster (PI), R. Clark}{\$300,000 for 
  3 years}{Awarded: January 2010}\nw

\funding{Studying DNS Traffic Patterns}{Verisign}{N. Feamster}{\$30,000 for 
  1 year}{Awarded: November 2009}\nw

\funding{CIFellowship for Cristian Lumezanu}{National Science
  Foundation}{C. Lumezanu, N. Feamster (PI)}{\$140,000 for 
  1 year}{Awarded: November 2009}\nw

\funding{Military Network Protocol}{DARPA Subcontract}{N. Feamster}{\$37,000 for 
  1 year}{Awarded: November 2009}\nw


\funding{Botnet Attribution and Removal: From Axioms to Theories to
  Practice}{Office of Naval Research}{W. Lee (PI), D. Dagon, J. Giffin,
  N. Feamster, K. Shin, F. Jahanian, M. Bailey, J. Mitchell, G. Vigna,
  C. Kruegel}{\$7,500,000 for 5 years}{Awarded: August 2009}\nw

\funding{Taint-based Information Tracking in Networked Systems}{National
  Science Foundation Trusted Computing Program}{N. Feamster}{\$450,000
  for 3 years}{Awarded: August 2009}\nw

\funding{Towards a Market for Internet Connectivity}{Office of Naval
  Research}{N. Feamster (PI), R. Johari, V. Vazirani}{\$350,000 for 1
  year}{Awarded: March 2009}\nw

\funding{Bringing Experimenters and External Connectivity to GENI}{GENI
  Project Office}{N. Feamster}{\$320,000 for 3 years}{Awarded:
  September 2008}\nw

\funding{Routing Without Recomputation}{Cisco
  Systems}{N. Feamster}{\$96,019 for 1 year}{Awarded: September 2008}\nw

\funding{CLEANSE: Cross-Layer Large-Scale Efficient Analysis of Network
  Activities \\ to Secure the Internet}{National Science
  Foundation Cybertrust Program}{W. Lee (PI), N. Feamster and
  others}{\$1,200,000 for 5 
  years}{Awarded: September 2008} \nw

\funding{Virtual Center for Network and Security
  Data}{Department of Homeland Security}{N. Feamster}{\$48,000 for 2
  years}{Awarded: March 2008} \nw


\funding{Sloan Research Fellowship}{Alfred P. Sloan
  Foundation}{N. Feamster}{\$45,000 for 2 years}{Awarded: February 
  2008}

\funding{Enabling Security and Network Management Research for Future
  Networks}{National Science Foundation CRI-IAD
  Program}{N. Feamster (PI), Z. Mao, W. Lee}{\$397,426 for 3 years}{Awarded: February
  2008}

\funding{SMITE: Scalable Monitoring in the Extreme}{DARPA BAA 07-52:
  Scalable Network Monitoring}{N. Feamster (PI), W. Lee}{\$250,000 for 2
  years}{Awarded: January 2008}


\funding{Countering Botnets: Anomaly-Based Detection, Comprehensive
Analysis, \\ and Efficient Mitigation}{Department of Homeland Security
BAA07-09}{W. Lee (PI), N. Feamster, J. Giffin}{\$1,050,730 for 2
years}{Awarded: January 2008}

\funding{Spam Filtering Research}{IBM Faculty Award}{N. Feamster}{\$
  7,500 (unrestricted gift)}{Awarded: June 2007}


\funding{SCAN: Statistical Collaborative Analysis of Networks}{National
  Science Foundation NeTS-NBD Program}{N. Feamster (PI), A. Gray,
  J. Hellerstein, C. Guestrin}{\$ 95,000 for 3 years.}{Awarded:
  June 2007}

\funding{Towards an Accountable Internet Architecture}{National Science Foundation
  CyberTrust Program (Team Proposal)}{D. Andersen,
  H. Balakrishnan, N. Feamster (PI), S. Shenker}{\$ 300,000 for 3 years.}{Awarded:
  May 2007}

\funding{Fish4Phish: Fishing for Phishing in a Large Pond}{AT\&T
  Labs---Research}{N. Feamster (PI), O. Spatscheck, K. van der Merwe}{Funding
  for summer intern.}{Awarded: February 2007}



\funding{Improving Network Operations with a View from the
  Edge.}{National Science Foundation CAREER
  Program}{N. Feamster (PI)}{\$400,000 for 5 years.}{Awarded: January 2007}


\funding{Equipment Donation for Network Operations Research}{Intel
  Corporation}{N. Feamster}{\$30,000}{Awarded: October 2006}

\funding{CABO: Concurrent Architectures are Better than
One} {National Science Foundation NeTS-FIND
Program} {N. Feamster (PI), L. Gao, J. Rexford}{\$ 300,000 for 4
years}{Awarded: June 2006}
\label{funding:nsf-cabo}

\funding{Verification and Modeling of Wide-Area Internet Routing}{Cisco
  Systems University Research Program}{N. Feamster and H. Balakrishnan (PI)}{\$ 
  95,500 for 1 year.}{Awarded: June 2004} 


\end{grants}

\subsubsection*{2.~Pending}
\label{subsubsec:pendingdentification}
\setcounter{subsubsection}{2}

\internalnote{
\begin{grants}

\funding{MRI: Development of an Open Observatory for the Internet's Last
  Mile}{National Science Foundation}{N. Feamster (PI), W. Lee,
  S. Banerjee, D. Levin, N. Spring, R. Clark}{\$4,000,000 for 5
  years}{Submitted: February 2013}   

\funding{TWC SBE: Frontier: Collaborative: EPICS: Empowering People to Overcome Internet Controls
}{National Science
  Foundation}{W. Lee (PI), N. Feamster, H. Klein, M. Bailey, A. Snoeren,
A. Lupia, N. Valentino, A. Halderman, G. Vigna, C. Kreugel}{\$5,500,000 for
  5 years}{Submitted: January 2013}   


\funding{NeTS: Small: Collaborative Research: Studying and Improving the
Performance of Access Networks
}{National Science Foundation}{N. Feamster (PI), D. Levin}{\$250,000 for
  3 years}{Submitted: December 2012}   


\funding{TWC: Medium: Collaborative: Exploiting Structure for Proactive
  Protection against Web-based Malware}{National Science
  Foundation}{N. Feamster (PI), V. Sekar}{\$240,000 for
  3 years}{Submitted: November 2012}   







\end{grants}
}

\internalonly{
\subsubsection*{3.~Not Funded}
\label{subsubsec:mylabel1}
\setcounter{subsubsection}{3}
}

\request{

\begin{grants}

\funding{Demand Characterization and Management for Access
  Networks}{Google Research Program}{N. Feamster,
  R. Johari}{\$99,767 for 1 year}{Submitted: June 2012} 

\funding{Understanding and Managing Wireless in the Home}{Cisco
  University Research Program}{N. Feamster,
  A. Snoeren}{\$100,000 for 1 year}{Submitted: October 2011} 

\funding{MABADA: Monitoring and Analysing BGP and DNS
  Agility}{Department of Homeland Security, BAA 11-02}{N. Feamster (PI),
  M. Dacier, W. Lee}{\$3,000,000 for 3 years}{Submitted: July 2011}


\funding{Proactive Identification of Internet Threats Through
  Large-Scale DNS Traffic Analysis}{Department of Homeland Security, BAA
  11-02}{M. Antonakakis (PI), N. Feamster, F. Monrose}{\$3,000,000 for 3
  years}{Submitted: July 2011} 


\funding{Providing Visibility into Home Networks for Research and
  Operations}{National Science Foundation CNS Division}{K. Calvert,
  W. K. Edwards, N. Feamster, 
  R. Grinter}{\$1,200,000 for 4 years}{Submitted:    
  September 2010}

\funding{Don't Configure the Network, Program It!}{National Science
  Foundation CNS Division}{N. Feamster}{\$80,000 for 1 year}{Submitted:   
  July 2010}

\funding{Automatic Configuration Generation for Data Center
  Configurations}{Yahoo Faculty Research and Engagement
  Program}{N. Feamster}{\$25,000 for 1 year}{Submitted: 
  May 2010}

\funding{Managing the Cost of Network Traffic}{Yahoo Faculty Research
  and Engagement Program}{N. Feamster}{\$25,000 for 1 year}{Submitted:
  May 2010}


\funding{Wide-Area Networking for Distributed Services}{National Science
  Foundation}{N. Feamster, J. Rexford}{\$500,000 for 3 years}{Submitted:
  December 2009}

\funding{Networking Protocols with Practical, Provably Secure Cryptography}{National Science
  Foundation}{N. Feamster, A. Boldyreva}{\$500,000 for 3 years}{Submitted:
  December 2009}

\funding{Networks that Learn}{National Science Foundation CDI
  Program}{N. Feamster, A. Gray}{\$817,940 for 3 years}{Submitted:
  December 2008}\nw


\funding{Discovering Temporal Patterns in Network Traffic}{National
  Science Foundation III Program}{A. Gray, N. Feamster}{\$1,200,000
  for 3 years}{Submitted: October 2008}\nw

\funding{Outsourcing Network Security with Programmable Networks}{National
  Science Foundation Trusted Computing Program}{N. Feamster, W. Lee,
  A. Gray}{\$1,194,608 for 4 years}{Submitted: October 2008}\nw


\funding{A Trust Layer for Network Communications}{National Science
  Foundation NeTS-ANET Program}{X. Yang, N. Feamster, Z. Mao}{\$660,283
  for 3 years}{Submitted: March 2008} 


 \funding{Synthesizing Data and Control-Plane Monitoring to Contain BGP
  Attacks}{Department of Homeland Security}{Z.M. Mao, N. Feamster,
  F. Jahanian}{\$241,423 for 3 years}{Submitted: June 2007}

 \funding{Spam Processing, Archiving, and Monitoring Community Facility
  (SPAM Commons)}{National Science Foundation CRI-CRD Program}{C. Pu,
  N. Feamster, M. Ahamad}{\$992,382 for 3 years}{Submitted: August 2007}

 \funding{AGORA: AGregator-Oriented Architecture}{National Science
  Foundation NeTS-FIND Program}{N. Feamster,
  R. Johari, V. Vazirani}{\$ 618,582 for 3 years.}{Submitted: January
  2007, March 2008}

\funding{Providing Configuration Assurance with Network-Wide
  Analysis}{DARPA/ARO NICECAP Program.}{N. Feamster, R. Clark}{White
  paper.}{Submitted: May 2006} 
\label{funding:darpa-nicecap}

\funding{Efficiency of Routing in a Federated Internet}{National
  Science Foundation NeTS-FIND Program.}{N. Feamster, R. Johari,
  V. Vazirani}{\$  677,095 for 3 years.}{Submitted: March 2006}
\label{funding:nsf-routing}

\funding{Network-Wide Analysis for Secure, Correct Network
  Operations}{Cisco Systems University Research
  Program}{N. Feamster}{\$ 95,396 for 1 year.}{Submitted: August 2006,
  March 2006.} 
\label{funding:cisco-rsec}

\funding{Detecting Botnets to Stem Spam and Phrustrate
  Phishing}{Cisco Systems University Research Program}{N. Feamster}{\$
  95,396 for 1 year.}{Submitted: August 2006, March 2006} 
\label{funding:cisco-botnet}


%\funding{AGORA: Aggregator-Oriented Archiecture}{National Science
%  Foundation NeTS-FIND Program}{N. Feamster, R. Johari,
%  V. Vazirani}{\$675,195 for 3 years}{Submitted: March 2008} 



\end{grants}
}

\subsection{Research Proposals and Grants (Contributor)}
\label{subsec:mylabel3}

\subsubsection*{1.~Approved and Funded}
\label{subsubsec:approved-contrib}
\setcounter{subsubsection}{1}

\internalnote{
\begin{grants}
\funding{Development of a shared network measurement storage and
  analysis infrastructure}{National Science Foundation Major Research
  Infrastructure (MRI) }{D. Andersen, D. Song, C. Wang, H. Zhang}{\$
  101,488.96}{User and developer for proposed infrastructure;
  possible equipment money to Georgia Tech . \\ Submitted: February
  2006.}
\label{funding:nsf-mri}
\end{grants}
}


\subsubsection*{2.~Pending}
\label{subsubsec:pendingdentification-contrib}
\setcounter{subsubsection}{2}



\internalonly{
\subsubsection*{3.~Not Funded}
\label{subsubsec:notfund-contrib}
\setcounter{subsubsection}{3}
}

\request{
\begin{grants}
\funding{The World is WAM (Wireless and Mobile)}{National Science
  Foundation Computing Research Infrastructure (CRI-IAD) Program
  }{G. Abowd, M. Ahamad, T. Balch, B. MacIntyre, C. Isbell, K. Edwards,
  I. Essa, N. Feamster, G. Riley, W. Lee, R. Lipton, L. Liu, M. Sanders,
  K. Ramachandran, R. Clark, S. Pande, R. Sivakumar, T. Starner,
  E. Zegura}{\$1,999,422}{Submitted: November 2006.} 
\label{funding:nsf-cri}
\end{grants}
}


\subsection{Other}

\begin{grants}
 \funding{In-Band, Bottom-Up Support for Network
  Accountability}{N. Feamster, W. Lee, M. Ahamad}{DARPA
  Strategic Technology Office}{White paper / Request for
  Information.}{Submitted: February 2007} 

 \funding{Towards an Accountable Internet Architecture}{D. Andersen,
  H. Balakrishnan, N. Feamster, S. Shenker}{DARPA Strategic Technology
  Office}{White paper / Request for Information.}{Submitted: February
  2007}
\end{grants}


\subsection{Research Honors and Awards}
\label{subsec:mylabel4}

\begin{cvlist}{}
\item[2013] {\bf ACM SIGCOMM Community Award Paper}, {\em ACM SIGCOMM Internet
  Measurement Conference}
\item[2012] {\bf Hesburgh Teaching Fellow}
\item[2012] PRSA Bronze Anvil Award for {\em Wall Street Journal}
  Editorial Article
\item[2011] Selected Participant for U.S. National Academy of
  Engineering Frontiers of Engineering Symposium
\item[2010] John P. Imlay Distinguished Lecture, Georgia Tech
\item[2010] Panelist for NSF/{\em Discover Magazine}
  Special Issue on ``The New Internet''
\item[2010] {\bf {\em Technology Review} Top Innovators Under 35}
\item[2010] Georgia Tech Sigma Xi Young Faculty Award
\item[2010] Selected Participant for U.S. National Academy of Science Kavli Frontiers of Science Symposium
\item[2009] Best Paper, {\em Passive and Active Measurement Conference}\nw
\item[2009] Georgia Tech Sigma Xi Best Undergraduate Research Advisor\nw
\item[2008] {\bf NSF Presidential Early Career Award for Scientists and
  Engineers (PECASE)}\nw
\item[2008] {\bf Alfred P. Sloan Fellowship}
\item[2008] Georgia Tech College of Computing Outstanding Junior Faculty
  Research Award
\item[2007] IBM Faculty Award
\item[2007] NSF CAREER Award
\item[2006] Best Student Paper (Advisor), {\em ACM SIGCOMM} (Premier
  Networking Conference) 
\item[2006] George M. Sprowls honorable mention for best Ph.D. thesis in
  computer science, MIT
\item[2005] Best Paper, {\em 2nd USENIX Symposium on Networked Systems Design
  and Implementation}
%\item[2004] Cisco URP Grant Recipient
\item[2002] Best Student Paper, {\em 11th USENIX Security Symposium}
\item[2001] Best Student Paper, {\em 10th USENIX Security Symposium}
\item[2002--2005] NSF Graduate Research Fellow
\item[2001] MIT William A. Martin Memorial Thesis Award for Best EECS
  Master's Thesis
%\item[1999--] Tau Beta Pi Engineering Honor Society
%\item[1999--] Eta Kappa Nu Honor Society
%\item[1999] Letter of Commendation for Outstanding Performance, MIT Digital
%  Design Laboratory
%\item[1998--1999] Phi Sigma Kappa Scholarship Award
%\item[1997] National Merit Scholar
%\item[1997] Rotary Club Scholarship
%\item[1996] AP Scholar with Distinction
\end{cvlist}

