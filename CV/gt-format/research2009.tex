
\newpage \setcounter{page}{1}
\lfoot[Printed: \today]{Nick Feamster}
\cfoot{\fancyplain{Revised: January 15, 2009}{Research Statement}}
\rfoot{Page \thepage~of 4}

\begin{center}
{\Large\textbf{Research Statement}}\\[0.1in] {\large\textbf{Nick
Feamster}}\\
%feamster@cc.gatech.edu \\
%http://www.cc.gatech.edu/$\sim$feamster/ \\[.1in] 
\end{center}


\section*{Summary}

{\bf My research focuses on building available, secure communications
 networks in the face of failures, misconfiguration, and malice.}  My
 approach to this problem is three-fold: (1)~Design and implement
 approaches to improve the network's inherent robustness, to help
 prevent downtime in the case of network faults; (2)~Design and
 implement techniques to help network operators reduce unwanted and
 malicious traffic (\eg, spam and phishing attacks); (3)~Design and
 implement tools and techniques that help network operators restore
 network connectivity when failures do occur.

Two of the Internet's most pressing problems, availability and security,
have been around since its inception.  They are the most difficult to
solve because they require making the network easier to manage and
operate.  Breakthroughs require domain knowledge, techniques from a wide
range of areas, and implementation of the resulting solutions in working
systems that are ultimately deployed in practice.  My unique approach
and expertise will allow me to continue to make important contributions
to these areas.

I draw inspiration for problems from practice, apply a principled
approach to develop solutions, and transfer these solutions back to
practice.  I discover interesting and challenging practical problems
through frequent discussions and meetings with network operators and
people in industry; in particular, I search for real-world network
operations problems that present an opportunity for making breakthroughs
by applying a principled approach.  I then tackle these problems by
applying first principles, developing new methods, and transferring
these solutions back to practice in the form of working systems.

\section*{Contributions and Impact}

Today's network operators face a major challenge: spam, phishing, denial
of service attacks, and even the increasing size and complexity of the
network itself have made the network increasingly difficult to
manage. At the same time, everyday users are becoming more dependent on
the Internet: even minutes of downtime can result in inconvenience or
loss for users and companies.  Nobody notices when the network works
well, but everyone suffers when it doesn't.  My research focuses on
developing tools and algorithms that make the network easier to manage,
more secure, and more available.

\fp {\bf Approach: Principled approach to network operations problems.}
I have tackled a variety of problems in network operations ranging from
real-time network diagnosis to stemming unwanted traffic like spam to
architectures for fast failure recovery.  Many people---most notably,
operators ``in the trenches''---are also working on these problems.
Unfortunately, many of the people who have the domain knowledge that
best equip them to solve these problems are busy with day-to-day
operations, putting out fires as they arise but rarely taking time to
think about fundamental changes to the network that might eradicate
these problems.  My research fills this gap.  I first devise methods to
understand the nature of the problem in practice.  I then tackle
domain-specific problems with tools and techniques from other
disciplines---ranging from machine learning to economics to program
analysis---whose principles might provide insights into a new,
previously undiscovered solution.  I then devise a new approach or
solution, and I transfer it to practice through implementation and
deployment of real-world systems.


%AIP, Splicing
\fp {\bf Theme 1: Designing inherently robust and resilient networks.}
I have developed new network protocols and architectures that improve
availability and accountability in communications networks in the face
of both faults and malice.  Networks face the continual threat of
failures and attacks that disrupt end-to-end connectivity.  Prior to my
work, one promising approach to improving connectivity involved routing
traffic along multiple paths between two endpoints (``multipath
routing''); despite the promise of this approach, previous approaches
encountered two significant challenges: First, previous approaches for
disseminating information about multiple paths through the network did
not scale to large networks.  Second, end systems had no way to signal
to the network that an end-to-end path had failed or was providing
inadequate performance.  My research applied a new perspective to this
problem: rather than simply routing traffic on one of multiple paths to
a destination, allow traffic to switch paths at intermediate points en
route to the destination, and allow end systems to signal to the network
when it should attempt to use a different path to the destination using
a small number of bits that can be carried in the traffic itself.  This
system, called {\em path splicing}, provides up to an exponential
improvement in reliability for only a linear increase in the amount of
state that each router in the network must store.

When networks do not perform as expected, network operators need ways to
determine the underlying cause for an attack or performance degradation.
In other words, the network should afford some level of {\em
accountability}.  The current Internet architecture provides little to
no accountability whatsoever: Malicious end systems can conceal the
source of their traffic (``spoofing''), and edge networks can provide
false information about their reachability to various Internet
destinations (``route hijacking''); both of these attacks make it
difficult to track down perpetrators of attacks.  Current approaches to
solving these problems require manual configuration and operator
vigilance, which make them weak and error-prone.  Towards building
networks that are inherently accountable, I have developed the
Accountable Internet Protocol (AIP).  One of my contributions to the
design was to make the addresses in this protocol self-certifying, which
forms the cornerstone of the basic design.  I also demonstrated how to
apply AIP to secure BGP, the Internet's interdomain routing protocol.


{\em Impact:} Both Path Splicing and AIP appeared in {\em ACM SIGCOMM}
in 2008, the premier conference on communications networks; early
versions of both papers also appeared in {\em ACM Workshop on Hot Topics
in Networking (HotNets)}.  I am in the process of incorporating both of
these technologies into working systems and transferring them to
practice.  I am working with BBN on a DARPA proposal that will
ultimately result in incorporating AIP's mechanisms into a military
network protocol that allows attribution of traffic to sources (the
details may ultimately be classified).  I am working with several people
at Cisco on incorporating path splicing into Cisco router functionality,
and we are also working on a prototype implementation in existing
software routers.


\fp {\bf Theme 2: Defending against unwanted traffic.}
I have applied my research approach to defend the network against
unwanted traffic, such as spam.  Prior to my work, nearly all previous
approaches to filtering spam relied in some way on analyzing an email's
contents to determine the legitimacy of the message.  This approach,
content filtering, continues to improve dramatically, and many people
are working on tuning content filters.  Nevertheless, the approach has a
fundamental shortcoming: email content is malleable, meaning that it is
very easy for a spammer to alter the content of an email to evade a
filter without changing the actual meaning of the message.  To keep pace
with continually changing content, both operators and implementors of
spam filters must continually tune their filters in a game of catch-up.
For example, last year saw a rise in spam whose main content was
transported in images; when content filters incorporated optical
character recognition for these image-based messages, spammers switched
to portable document format (PDF) messages.  My research has taken a
complementary approach: rather than classifying an email message based
on {\em what} is in the message, classify the message based on {\em how
it is sent} (e.g., what country it coming from, when it was sent).  In
other words, examine the {\em network-level behavior} of the email
sender and classify the spam based on whether the observed sending
behavior likely corresponds to a legitimate sender or a spammer.

{\em Impact:} Based on my study of network-level behavior of spammers,
which received the Best Student Paper Award at {\em ACM SIGCOMM} in
2006, I have developed a new class of blacklisting techniques, called
{\em behavioral blacklisting}, which leverages machine learning
algorithms to classify email senders as spammers based on previously
observed spamming patterns.  I am working with both Cisco/Ironport and
Secure Computing, vendors of several large spam filtering appliances, to
deploy and evaluate these algorithms in practice.  I sit on the board of
Anchor Intelligence, who develop solutions for detecting click fraud;
this company is also applying some of our ideas in studying
network-level behavior for automatically detecting click fraud.

% rcc, BGP, 
\fp {\bf Theme 3: Improving fault detection and diagnosis.}
I have also developed several new techniques that improve Internet
availability by helping operators run their networks better.  Much of my
work has focused on fault detection and troubleshooting.  Prior to my
work, operators relied on detecting problems with networks ``at
runtime'' on a live network.  My work demonstrated that, in fact, many
routing problems could be detected simply by examining the configuration
of the routing protocols, before the configuration is even deployed.  I
applied techniques from static program analysis to routing configuration
to help network operators catch mistakes and predict dynamic network
behavior before the configurations are deployed on a live network,
preventing costly and catastrophic network downtime.

Beyond predicting behavior and proactively detecting configuration
faults, operators need to understand the network's behavior as it is
running (e.g., to detect equipment failures, attacks, or unplanned
shifts in network traffic).  Unfortunately, operators are drowning in a
sea of heterogeneous data, none of which intuitively points them to the
true source of the problem.  To help operators better understand network
faults ``at runtime'', I have applied unsupervised learning techniques
to Internet routing data to help them efficiently mine the data for
network events that require corrective action.  More recently, I have
begun developing a system that collects and aggregates measurements from
end systems to help users and operators of edge networks infer when
transit networks may be discriminating against certain types of traffic.

% NANO

{\em Impact:} One cornerstone of this work is a system called ``rcc''
(router configuration checker), which received the Best Paper Award at
{\em ACM/USENIX Networked Systems Design and Implementation (NSDI)} in
2005 and has been used by hundreds of Internet Service Providers (ISPs)
around the world to check their network configurations for errors.  We
are applying these detection algorithms to routing data in several large
ISPs and enterprise networks.  Finally, we are working with Google to
design, test, and deploy a system for detecting violations of network
neutrality.



%%%%%%%%%%%%%%%%%%%%%%%%%%%%%%%%%%%%%%%%%%%%%%%%%%%%%%%%%%%%
% next steps
% secure foundations: openflow, pedigree
% economic foundations: MINT
% open access: collage
% fusing machine learning+networks

\pagebreak
\section*{Future Directions and Vision}

My previous work has focused heavily on reactive approaches to network
operations problems.  I intend to continue my existing themes, but take
a proactive, foundational approach with opportunities for broader
impact.  I will build networked systems for improving availability,
accessibility and accountability.

\fp {\bf Accountability: Foundations for inherent
robustness and security.} Today's network does not provide any
mechanisms for directly implementing security policies.  As such,
security is typically an afterthought, embodied by middleboxes,
authentications, alert systems, and intrusion detection systems. The
interaction between these many ``moving parts'' creates a system that is
brittle and unresponsive in the face of various security threats.  I am
exploring how could security policies be integrated into the network
fabric itself, with the support of trusted computing platforms,
programmable network elements, and distributed observation and
inference.  A cornerstone of this project will involve deployment on the
Georgia Tech campus network.

\fp {\bf Accessibility: Circumventing censorship.}  My previous work has
focused on eliminating unwanted traffic; the flip side of this coin
involves ensuring that parties that want to communicate with one another
can do so, even in the face of censorship.  Oppressive regimes (and even
some democratic governments) are increasingly restricting access to
content on the Internet.  Although some systems today enable anonymous
communication, these systems do not typically provide users with the
deniability that would prevent them from arousing the suspicion of their
governments.  I am currently working on systems that leverage
user-generated content as a conduit for the unfettered exchange of
information on the Internet.

\fp {\bf Availability: Networks that learn.}
The network's complexity and dynamism underlying processes make machine
learning algorithms a natural tool for detecting, diagnosing, and
mitigating disruptions.  In my previous work on diagnosis, I have found
that directly applying existing algorithms is difficult: Today's
networks were not designed to collect to expose data that facilitates
inference by machine learning algorithms, and today's machine learning
algorithms are not tailored for network data, which is high-volume,
distributed, and rapidly evolving.  I aim to design a {\em network that
learns}. In contrast to existing approaches, which simply patch the
existing network with point solutions (e.g., middleboxes, firewalls,
spam filtering appliances), I plan to: (1)~re-design network protocols
so that they provide a richer set of information to inference engines;
(2)~explore how new learning algorithms that better exploit
domain-specific information provided by these new protocols could
improve detection and diagnosis.

\fp {\bf My vision for networking research.}
Network researchers have historically faced the daunting hurdle of
deployment: short of convincing a major router vendor to adopt new
technology, there was little attempt for practical impact.  I believe
that two trends have changed this landscape.  First, network
management---and, in particular network security---were once ignored,
but these topics have come to the forefront as the network's insecurity
and instability are coming to blows with users' demands for high
availability and security.  This shift in focus is particularly evident
in enterprise networks, where operational expenditures are a significant
fraction of the cost of running the network and security is of utmost
concern.  Second, network devices are becoming increasingly more
programmable; most vendors have taken steps towards ``opening up the
box'' to give operators (and researchers) can have more control and
opportunities to innovate.

My future research agenda will leverage these two trends.  I intend to
shift my research focus slightly, to have an emphasis more on
longer-term projects that involve the deployment of real-world networked
systems.  With the help of the office of information technology, I have
begun some of this work already, with the deployment of a separate
research network on the Georgia Tech campus.  I believe that my
continued efforts will not only help advance network operations in the
areas that I have mentioned, but also to help shape the future of
networking research towards more ``hands on'', operationally relevant
problems.  I believe this approach will not only define new areas for
networking research, but, due to Georgia Tech's unique facilities and
research-minded network operators, it may also help make Georgia Tech
the top destination for performing this type of experimental,
systems-focused research on network operations and security.
